% Copyright 2007 by Mark Wibrow
%
% This file may be distributed and/or modified
%
% 1. under the LaTeX Project Public License and/or
% 2. under the GNU Public License.
%
% See the file doc/generic/pgf/licenses/LICENSE for more details.

% This file defines the mathematical functions and operators.
%
% Version 0.0 08/03/2007

% Adding/redefining extra operators/functions:
%
% Each operator/function XXX has two forms:
%
%
% \pgfmathXXX#1...   a public version which evaluates any
%                    arguments passed to it and passes the
%                    results on to...
%
% \pgfmathXXX@#1...  a non-public version which performs 
%                    required calculation on arguments which
%                    must have already been evaluated (i.e.
%                    *without* dimensions).
% 
% If a function XXX is to be included in the parser, it is 
% recommended, for consistency, that where possible, the 
% pgfmathparser file should define the macro \pgfmath@parseXXX.
% The parser should (ideally) then call \pgfmathXXX@.
%
% It is recommend that the pgfmath versions of the pgf dimension
% and count registers be used, i.e., \pgfmath@x for \pgfmath@x, 
% \c@pgfmath@counta for c@pgfmath@counta, and so on. These are currently
% \let to their pgf equivalents, but it may be necessary to change 
% this.
%
% It is also recommened that all calculations (where necessary)
% take place within a TeX group. \pgfmath@returnone#1 makes and
% expanded version of #1 global and stores this in \pgfmathresult 
% after the group is ended.
%
% Copyright 2007 by Mark Wibrow
%
% This file may be distributed and/or modified
%
% 1. under the LaTeX Project Public License and/or
% 2. under the GNU Public License.
%
% See the file doc/generic/pgf/licenses/LICENSE for more details.

% This file defines the trigonometric functions/operations.
%
% Much of this file is based on ideas and code (particularly 
% \pgfcoremath.code.tex) due to Till Tantau
%
% Version 0.0 08/03/2007

% \pgfmathpi
%
\def\pgfmathpi{\edef\pgfmathresult{\pgfmath@pi}}
\def\pgfmath@pi{3.14159}

% \pgfmathradians
% 
% Convert #1 from radians to degrees (accurate to 1 deg).
%
\def\pgfmathradians#1{%
	\pgfmathparse{#1}%
	\pgfmathradians@{\pgfmathresult}}
\def\pgfmathradians@#1{%
	\begingroup%
		\expandafter\pgfmath@x#1pt\relax%
		\pgfmath@x57.29577\pgfmath@x\relax% 57.29577 = 360/(2*pi)
		\pgfmath@returnone\pgfmath@x%
	\endgroup%
}%

% \pgfmathdeg
% 
% Convert #1 from radians to degrees.
%
\def\pgfmathdeg#1{%
	\pgfmathparse{#1}%
	\pgfmathdeg@{\pgfmathresult}}
\def\pgfmathdeg@#1{%
	\begingroup%
		\expandafter\pgfmath@x#1pt\relax%
		\pgfmath@x57.29577\pgfmath@x\relax% 57.29577 = 360/(2*pi)
		\pgfmath@returnone\pgfmath@x%
	\endgroup%
}%

% \pgfmathrad
% 
% Convert #1 from degrees to radians.
%
\def\pgfmathrad#1{%
	\pgfmathparse{#1}%
	\pgfmathrad@{\pgfmathresult}}
\def\pgfmathrad@#1{%
	\begingroup%
		\expandafter\pgfmath@x#1pt\relax%
		\pgfmath@x3.14159\pgfmath@x% 
		\divide\pgfmath@x180\relax%
		\pgfmath@returnone\pgfmath@x%
	\endgroup%
}%

% \pgfmathsin
% 
% Calculate the sine of #1 (in degrees).
%
\def\pgfmathsin#1{%
	\pgfmathparse{#1}%
	\pgfmathsin@{\pgfmathresult}}
\def\pgfmathsin@#1{%
	% Let #1 = a.b
	% Then sin(#1) \approx (1-b)*cos(a) + b*cos(a+1)
	%
	\begingroup%
		\pgfmath@x#1pt\relax%
		\advance\pgfmath@x-90pt\relax%
		\afterassignment\pgfmath@gobbletilpgfmath@
		\expandafter\c@pgfmath@counta\the\pgfmath@x\relax\pgfmath@
		\divide\c@pgfmath@counta360\relax%
		\multiply\c@pgfmath@counta-360\relax%
		\advance\pgfmath@x\c@pgfmath@counta pt\relax%
		\ifdim\pgfmath@x<0pt\relax\pgfmath@x-\pgfmath@x\fi%
		\ifdim\pgfmath@x<180pt\relax%
		\else
			\pgfmath@x-\pgfmath@x%
			\advance\pgfmath@x360pt\relax%
		\fi%
		% Now 0 <= x < 179. So split x into integer and decimal...
		%
		\afterassignment\pgfmath@xa%
		\expandafter\c@pgfmath@counta\the\pgfmath@x\relax%
		%
		% ...if #1 is an integer, don't do anything fancy.
		\ifdim\pgfmath@xa=0pt%
			\expandafter\pgfmath@x\csname pgfmath@cos@\the\c@pgfmath@counta\endcsname pt\relax%
		\else%
			\pgfmath@x-\pgfmath@xa%
			\advance\pgfmath@x1pt\relax%
			\expandafter\pgfmath@x\csname pgfmath@cos@\the\c@pgfmath@counta\endcsname\pgfmath@x%
			\advance\c@pgfmath@counta1\relax%
			\expandafter\advance\expandafter\pgfmath@x\csname pgfmath@cos@\the\c@pgfmath@counta\endcsname\pgfmath@xa%
		\fi%
		\pgfmath@returnone\pgfmath@x%
	\endgroup
}

% \pgfmathcos
% 
% Calculate the cosine of #1 (in degrees).
%
\def\pgfmathcos#1{%
	\pgfmathparse{#1}%
	\expandafter\pgfmathcos@\expandafter{\pgfmathresult}}
\def\pgfmathcos@#1{%
	% Let x = a.b
	% Then cos(x) \approx (1-b)*cos(a) + b*cos(a+1)
	%
	\begingroup%
		\pgfmath@x#1pt\relax%
		\afterassignment\pgfmath@gobbletilpgfmath@%
		\expandafter\c@pgfmath@counta\the\pgfmath@x\relax\pgfmath@% 
		\divide\c@pgfmath@counta360\relax%
		\multiply\c@pgfmath@counta-360\relax%
		\advance\pgfmath@x\c@pgfmath@counta pt\relax%
		\ifdim\pgfmath@x<0pt\relax\pgfmath@x-\pgfmath@x\fi%
		\ifdim\pgfmath@x<180pt\relax%
		\else%
			\pgfmath@x-\pgfmath@x%
			\advance\pgfmath@x360pt\relax%
		\fi%
		% Now 0 <= x < 179. So split x into integer and decimal...
		%
		\afterassignment\pgfmath@xa%
		\expandafter\c@pgfmath@counta\the\pgfmath@x\relax%
		%
		% ...if #1 is an integer, don't do anything fancy.
		\ifdim\pgfmath@xa=0pt%
			\expandafter\pgfmath@x\csname pgfmath@cos@\the\c@pgfmath@counta\endcsname pt%
		\else%
			\pgfmath@x-\pgfmath@xa%
			\advance\pgfmath@x1pt\relax%
			\expandafter\pgfmath@x\csname pgfmath@cos@\the\c@pgfmath@counta\endcsname\pgfmath@x%
			\advance\c@pgfmath@counta1\relax%
			\ifnum\c@pgfmath@counta=181\relax\c@pgfmath@counta179\relax\fi%
			\expandafter\advance\expandafter\pgfmath@x\csname pgfmath@cos@\the\c@pgfmath@counta\endcsname\pgfmath@xa%
		\fi%
		\pgfmath@returnone\pgfmath@x%
	\endgroup%
}


% \pgfmathsincos
% 
% Calculate the sin and cosine of #1 (in degrees).
%
\def\pgfmathsincos#1{%
	\pgfmathparse{#1}%
	\expandafter\pgfmathcos@\expandafter{\pgfmathresult}}
\def\pgfmathsincos@#1{%
	\pgfmathsin@{#1}\edef\pgfmathresulty{\pgfmathresult}%
	\pgfmathcos@{#1}\edef\pgfmathresultx{\pgfmathresult}%
}
% \pgfmathtan
% 
% Calculate the cotangent of #1 (in degrees).
%
\def\pgfmathtan#1{%
	\pgfmathparse{#1}%
	\pgfmathtan@{\pgfmathresult}}
\def\pgfmathtan@#1{%
	\begingroup%
		\pgfmathcos@{#1}%
		\expandafter\pgfmathreciprocal@\expandafter{\pgfmathresult}%
		\edef\pgfmath@tantemp{\pgfmathresult}%
		\pgfmathsin@{#1}%
		\pgfmath@x\pgfmathresult pt\relax%
		\pgfmath@x\pgfmath@tantemp\pgfmath@x%
		% Adjust to 4 decimal places. This gets rid of some annoyingly tiny errors.
		\afterassignment\pgfmath@x%
		\expandafter\c@pgfmath@counta\the\pgfmath@x\relax%
		\ifdim\pgfmath@x<.0001pt\relax%
			\pgfmath@x0pt\relax%
		\fi%
		\advance\pgfmath@x\c@pgfmath@counta pt%
		\pgfmath@returnone\pgfmath@x%
	\endgroup%
}


% \pgfmathasin
%
% The asin of #1
%
\def\pgfmathasin#1{%
	\pgfmathparse{#1}%
	\expandafter\pgfmathasin@\expandafter{\pgfmathresult}}
\def\pgfmathasin@#1{%
	\begingroup%
		\pgfmath@x#1pt\relax%
		\pgfmath@xa\pgfmath@x%
		\ifdim\pgfmath@x<0pt\relax%
			\pgfmath@x-\pgfmath@x%
		\fi%
		\pgfmath@x1000\pgfmath@x%
		\afterassignment\pgfmath@gobbletilpgfmath@%
		\expandafter\c@pgfmath@counta\the\pgfmath@x\relax\pgfmath@%
		\expandafter\pgfmath@x\csname pgfmath@asin@\the\c@pgfmath@counta\endcsname pt\relax%
		\ifdim\pgfmath@xa<0pt\relax%
			\pgfmath@x-\pgfmath@x%
		\fi%
		\pgfmath@returnone\pgfmath@x%
	\endgroup%
}

% \pgfmathacos
%
% The acos of #1
%	
\def\pgfmathacos#1{%
	\pgfmathparse{#1}%
	\expandafter\pgfmathacos@\expandafter{\pgfmathresult}}
\def\pgfmathacos@#1{%
	\begingroup%
		\pgfmath@x#1pt\relax%
		\pgfmath@xa\pgfmath@x%
		\ifdim\pgfmath@x<0pt\relax%
			\pgfmath@x-\pgfmath@x%
		\fi%
		\pgfmath@x1000\pgfmath@x%
		\afterassignment\pgfmath@gobbletilpgfmath@%
		\expandafter\c@pgfmath@counta\the\pgfmath@x\relax\pgfmath@%
		\expandafter\pgfmath@x\csname pgfmath@acos@\the\c@pgfmath@counta\endcsname pt\relax%
		\ifdim\pgfmath@xa<0pt\relax%
			\pgfmath@x-\pgfmath@x%
		\fi%
		\pgfmath@returnone\pgfmath@x%
	\endgroup%
}

% \pgfmathasin
%
% The atan of #1
%
\def\pgfmathatan#1{%
	\pgfmathparse{#1}%
	\expandafter\pgfmathatan@\expandafter{\pgfmathresult}}
\def\pgfmathatan@#1{%
	\begingroup%
		\pgfmath@x#1pt\relax%
		\pgfmath@xa\pgfmath@x%
		\ifdim\pgfmath@x<0pt\relax%
			\pgfmath@x-\pgfmath@x%
		\fi%
		\pgfmath@xb\pgfmath@x%
		\ifdim\pgfmath@x>1pt\relax%
			\edef\pgfmath@temp{\pgfmath@tonumber{\pgfmath@x}}%
			\pgfmathreciprocal@{\pgfmath@temp}%
			\pgfmath@x\pgfmathresult pt\relax%
		\fi%
		\pgfmath@x1000\pgfmath@x%
		\afterassignment\pgfmath@gobbletilpgfmath@%
		\expandafter\c@pgfmath@counta\the\pgfmath@x\relax\pgfmath@%
		\ifdim\pgfmath@xb>1pt\relax%
			\expandafter\pgfmath@x\expandafter-\csname pgfmath@atan@\the\c@pgfmath@counta\endcsname pt\relax%
			\advance\pgfmath@x90pt%
		\else%
			\expandafter\pgfmath@x\csname pgfmath@atan@\the\c@pgfmath@counta\endcsname pt\relax%
		\fi%
		\ifdim\pgfmath@xa<0pt\relax%
			\pgfmath@x-\pgfmath@x%
		\fi%
		\pgfmath@returnone\pgfmath@x%
	\endgroup%
}

	
\def\pgfmath@def#1#2#3{\expandafter\def\csname pgfmath@#1@#2\endcsname{#3}}
\pgfmath@def{cos}{0}{1.00000}		\pgfmath@def{cos}{1}{0.99985}
\pgfmath@def{cos}{2}{0.99939}		\pgfmath@def{cos}{3}{0.99863}
\pgfmath@def{cos}{4}{0.99756}		\pgfmath@def{cos}{5}{0.99619}
\pgfmath@def{cos}{6}{0.99452}		\pgfmath@def{cos}{7}{0.99255}
\pgfmath@def{cos}{8}{0.99027}		\pgfmath@def{cos}{9}{0.98769}
\pgfmath@def{cos}{10}{0.98481}		\pgfmath@def{cos}{11}{0.98163}
\pgfmath@def{cos}{12}{0.97815}		\pgfmath@def{cos}{13}{0.97437}
\pgfmath@def{cos}{14}{0.97030}		\pgfmath@def{cos}{15}{0.96593}
\pgfmath@def{cos}{16}{0.96126}		\pgfmath@def{cos}{17}{0.95630}
\pgfmath@def{cos}{18}{0.95106}		\pgfmath@def{cos}{19}{0.94552}
\pgfmath@def{cos}{20}{0.93969}		\pgfmath@def{cos}{21}{0.93358}
\pgfmath@def{cos}{22}{0.92718}		\pgfmath@def{cos}{23}{0.92050}
\pgfmath@def{cos}{24}{0.91355}		\pgfmath@def{cos}{25}{0.90631}
\pgfmath@def{cos}{26}{0.89879}		\pgfmath@def{cos}{27}{0.89101}
\pgfmath@def{cos}{28}{0.88295}		\pgfmath@def{cos}{29}{0.87462}
\pgfmath@def{cos}{30}{0.86603}		\pgfmath@def{cos}{31}{0.85717}
\pgfmath@def{cos}{32}{0.84805}		\pgfmath@def{cos}{33}{0.83867}
\pgfmath@def{cos}{34}{0.82904}		\pgfmath@def{cos}{35}{0.81915}
\pgfmath@def{cos}{36}{0.80902}		\pgfmath@def{cos}{37}{0.79864}
\pgfmath@def{cos}{38}{0.78801}		\pgfmath@def{cos}{39}{0.77715}
\pgfmath@def{cos}{40}{0.76604}		\pgfmath@def{cos}{41}{0.75471}
\pgfmath@def{cos}{42}{0.74314}		\pgfmath@def{cos}{43}{0.73135}
\pgfmath@def{cos}{44}{0.71934}		\pgfmath@def{cos}{45}{0.70711}
\pgfmath@def{cos}{46}{0.69466}		\pgfmath@def{cos}{47}{0.68200}
\pgfmath@def{cos}{48}{0.66913}		\pgfmath@def{cos}{49}{0.65606}
\pgfmath@def{cos}{50}{0.64279}		\pgfmath@def{cos}{51}{0.62932}
\pgfmath@def{cos}{52}{0.61566}		\pgfmath@def{cos}{53}{0.60182}
\pgfmath@def{cos}{54}{0.58779}		\pgfmath@def{cos}{55}{0.57358}
\pgfmath@def{cos}{56}{0.55919}		\pgfmath@def{cos}{57}{0.54464}
\pgfmath@def{cos}{58}{0.52992}		\pgfmath@def{cos}{59}{0.51504}
\pgfmath@def{cos}{60}{0.50000}		\pgfmath@def{cos}{61}{0.48481}
\pgfmath@def{cos}{62}{0.46947}		\pgfmath@def{cos}{63}{0.45399}
\pgfmath@def{cos}{64}{0.43837}		\pgfmath@def{cos}{65}{0.42262}
\pgfmath@def{cos}{66}{0.40674}		\pgfmath@def{cos}{67}{0.39073}
\pgfmath@def{cos}{68}{0.37461}		\pgfmath@def{cos}{69}{0.35837}
\pgfmath@def{cos}{70}{0.34202}		\pgfmath@def{cos}{71}{0.32557}
\pgfmath@def{cos}{72}{0.30902}		\pgfmath@def{cos}{73}{0.29237}
\pgfmath@def{cos}{74}{0.27564}		\pgfmath@def{cos}{75}{0.25882}
\pgfmath@def{cos}{76}{0.24192}		\pgfmath@def{cos}{77}{0.22495}
\pgfmath@def{cos}{78}{0.20791}		\pgfmath@def{cos}{79}{0.19081}
\pgfmath@def{cos}{80}{0.17365}		\pgfmath@def{cos}{81}{0.15643}
\pgfmath@def{cos}{82}{0.13917}		\pgfmath@def{cos}{83}{0.12187}
\pgfmath@def{cos}{84}{0.10453}		\pgfmath@def{cos}{85}{0.08716}
\pgfmath@def{cos}{86}{0.06976}		\pgfmath@def{cos}{87}{0.05234}
\pgfmath@def{cos}{88}{0.03490}		\pgfmath@def{cos}{89}{0.01745}
\pgfmath@def{cos}{90}{0.00000}		\pgfmath@def{cos}{91}{-0.01745}
\pgfmath@def{cos}{92}{-0.03490}		\pgfmath@def{cos}{93}{-0.05234}
\pgfmath@def{cos}{94}{-0.06976}		\pgfmath@def{cos}{95}{-0.08716}
\pgfmath@def{cos}{96}{-0.10453}		\pgfmath@def{cos}{97}{-0.12187}
\pgfmath@def{cos}{98}{-0.13917}		\pgfmath@def{cos}{99}{-0.15643}
\pgfmath@def{cos}{100}{-0.17365}		\pgfmath@def{cos}{101}{-0.19081}
\pgfmath@def{cos}{102}{-0.20791}		\pgfmath@def{cos}{103}{-0.22495}
\pgfmath@def{cos}{104}{-0.24192}		\pgfmath@def{cos}{105}{-0.25882}
\pgfmath@def{cos}{106}{-0.27564}		\pgfmath@def{cos}{107}{-0.29237}
\pgfmath@def{cos}{108}{-0.30902}		\pgfmath@def{cos}{109}{-0.32557}
\pgfmath@def{cos}{110}{-0.34202}		\pgfmath@def{cos}{111}{-0.35837}
\pgfmath@def{cos}{112}{-0.37461}		\pgfmath@def{cos}{113}{-0.39073}
\pgfmath@def{cos}{114}{-0.40674}		\pgfmath@def{cos}{115}{-0.42262}
\pgfmath@def{cos}{116}{-0.43837}		\pgfmath@def{cos}{117}{-0.45399}
\pgfmath@def{cos}{118}{-0.46947}		\pgfmath@def{cos}{119}{-0.48481}
\pgfmath@def{cos}{120}{-0.50000}		\pgfmath@def{cos}{121}{-0.51504}
\pgfmath@def{cos}{122}{-0.52992}		\pgfmath@def{cos}{123}{-0.54464}
\pgfmath@def{cos}{124}{-0.55919}		\pgfmath@def{cos}{125}{-0.57358}
\pgfmath@def{cos}{126}{-0.58779}		\pgfmath@def{cos}{127}{-0.60182}
\pgfmath@def{cos}{128}{-0.61566}		\pgfmath@def{cos}{129}{-0.62932}
\pgfmath@def{cos}{130}{-0.64279}		\pgfmath@def{cos}{131}{-0.65606}
\pgfmath@def{cos}{132}{-0.66913}		\pgfmath@def{cos}{133}{-0.68200}
\pgfmath@def{cos}{134}{-0.69466}		\pgfmath@def{cos}{135}{-0.70711}
\pgfmath@def{cos}{136}{-0.71934}		\pgfmath@def{cos}{137}{-0.73135}
\pgfmath@def{cos}{138}{-0.74314}		\pgfmath@def{cos}{139}{-0.75471}
\pgfmath@def{cos}{140}{-0.76604}		\pgfmath@def{cos}{141}{-0.77715}
\pgfmath@def{cos}{142}{-0.78801}		\pgfmath@def{cos}{143}{-0.79864}
\pgfmath@def{cos}{144}{-0.80902}		\pgfmath@def{cos}{145}{-0.81915}
\pgfmath@def{cos}{146}{-0.82904}		\pgfmath@def{cos}{147}{-0.83867}
\pgfmath@def{cos}{148}{-0.84805}		\pgfmath@def{cos}{149}{-0.85717}
\pgfmath@def{cos}{150}{-0.86603}		\pgfmath@def{cos}{151}{-0.87462}
\pgfmath@def{cos}{152}{-0.88295}		\pgfmath@def{cos}{153}{-0.89101}
\pgfmath@def{cos}{154}{-0.89879}		\pgfmath@def{cos}{155}{-0.90631}
\pgfmath@def{cos}{156}{-0.91355}		\pgfmath@def{cos}{157}{-0.92050}
\pgfmath@def{cos}{158}{-0.92718}		\pgfmath@def{cos}{159}{-0.93358}
\pgfmath@def{cos}{160}{-0.93969}		\pgfmath@def{cos}{161}{-0.94552}
\pgfmath@def{cos}{162}{-0.95106}		\pgfmath@def{cos}{163}{-0.95630}
\pgfmath@def{cos}{164}{-0.96126}		\pgfmath@def{cos}{165}{-0.96593}
\pgfmath@def{cos}{166}{-0.97030}		\pgfmath@def{cos}{167}{-0.97437}
\pgfmath@def{cos}{168}{-0.97815}		\pgfmath@def{cos}{169}{-0.98163}
\pgfmath@def{cos}{170}{-0.98481}		\pgfmath@def{cos}{171}{-0.98769}
\pgfmath@def{cos}{172}{-0.99027}		\pgfmath@def{cos}{173}{-0.99255}
\pgfmath@def{cos}{174}{-0.99452}		\pgfmath@def{cos}{175}{-0.99619}
\pgfmath@def{cos}{176}{-0.99756}		\pgfmath@def{cos}{177}{-0.99863}
\pgfmath@def{cos}{178}{-0.99939}		\pgfmath@def{cos}{179}{-0.99985}
\pgfmath@def{cos}{180}{-1.00000}	   \pgfmath@def{cos}{181}{-0.99985}

\pgfmath@def{cosfrac}{0}{1.00000}		\pgfmath@def{cosfrac}{1}{0.99995}
\pgfmath@def{cosfrac}{2}{0.99980}		\pgfmath@def{cosfrac}{3}{0.99955}
\pgfmath@def{cosfrac}{4}{0.99920}		\pgfmath@def{cosfrac}{5}{0.99875}
\pgfmath@def{cosfrac}{6}{0.99820}		\pgfmath@def{cosfrac}{7}{0.99755}
\pgfmath@def{cosfrac}{8}{0.99681}		\pgfmath@def{cosfrac}{9}{0.99597}
\pgfmath@def{cosfrac}{10}{0.99503}		\pgfmath@def{cosfrac}{11}{0.99400}
\pgfmath@def{cosfrac}{12}{0.99287}		\pgfmath@def{cosfrac}{13}{0.99165}
\pgfmath@def{cosfrac}{14}{0.99034}		\pgfmath@def{cosfrac}{15}{0.98893}
\pgfmath@def{cosfrac}{16}{0.98744}		\pgfmath@def{cosfrac}{17}{0.98585}
\pgfmath@def{cosfrac}{18}{0.98418}		\pgfmath@def{cosfrac}{19}{0.98242}
\pgfmath@def{cosfrac}{20}{0.98058}		\pgfmath@def{cosfrac}{21}{0.97865}
\pgfmath@def{cosfrac}{22}{0.97664}		\pgfmath@def{cosfrac}{23}{0.97455}
\pgfmath@def{cosfrac}{24}{0.97238}		\pgfmath@def{cosfrac}{25}{0.97014}
\pgfmath@def{cosfrac}{26}{0.96782}		\pgfmath@def{cosfrac}{27}{0.96542}
\pgfmath@def{cosfrac}{28}{0.96296}		\pgfmath@def{cosfrac}{29}{0.96042}
\pgfmath@def{cosfrac}{30}{0.95782}		\pgfmath@def{cosfrac}{31}{0.95515}
\pgfmath@def{cosfrac}{32}{0.95242}		\pgfmath@def{cosfrac}{33}{0.94962}
\pgfmath@def{cosfrac}{34}{0.94677}		\pgfmath@def{cosfrac}{35}{0.94385}
\pgfmath@def{cosfrac}{36}{0.94088}		\pgfmath@def{cosfrac}{37}{0.93786}
\pgfmath@def{cosfrac}{38}{0.93478}		\pgfmath@def{cosfrac}{39}{0.93165}
\pgfmath@def{cosfrac}{40}{0.92847}		\pgfmath@def{cosfrac}{41}{0.92525}
\pgfmath@def{cosfrac}{42}{0.92198}		\pgfmath@def{cosfrac}{43}{0.91866}
\pgfmath@def{cosfrac}{44}{0.91531}		\pgfmath@def{cosfrac}{45}{0.91192}
\pgfmath@def{cosfrac}{46}{0.90849}		\pgfmath@def{cosfrac}{47}{0.90502}
\pgfmath@def{cosfrac}{48}{0.90152}		\pgfmath@def{cosfrac}{49}{0.89799}
\pgfmath@def{cosfrac}{50}{0.89442}		\pgfmath@def{cosfrac}{51}{0.89083}
\pgfmath@def{cosfrac}{52}{0.88721}		\pgfmath@def{cosfrac}{53}{0.88357}
\pgfmath@def{cosfrac}{54}{0.87990}		\pgfmath@def{cosfrac}{55}{0.87621}
\pgfmath@def{cosfrac}{56}{0.87250}		\pgfmath@def{cosfrac}{57}{0.86877}
\pgfmath@def{cosfrac}{58}{0.86503}		\pgfmath@def{cosfrac}{59}{0.86126}
\pgfmath@def{cosfrac}{60}{0.85749}		\pgfmath@def{cosfrac}{61}{0.85370}
\pgfmath@def{cosfrac}{62}{0.84990}		\pgfmath@def{cosfrac}{63}{0.84609}
\pgfmath@def{cosfrac}{64}{0.84227}		\pgfmath@def{cosfrac}{65}{0.83844}
\pgfmath@def{cosfrac}{66}{0.83460}		\pgfmath@def{cosfrac}{67}{0.83077}
\pgfmath@def{cosfrac}{68}{0.82692}		\pgfmath@def{cosfrac}{69}{0.82308}
\pgfmath@def{cosfrac}{70}{0.81923}		\pgfmath@def{cosfrac}{71}{0.81538}
\pgfmath@def{cosfrac}{72}{0.81153}		\pgfmath@def{cosfrac}{73}{0.80768}
\pgfmath@def{cosfrac}{74}{0.80384}		\pgfmath@def{cosfrac}{75}{0.80000}
\pgfmath@def{cosfrac}{76}{0.79616}		\pgfmath@def{cosfrac}{77}{0.79232}
\pgfmath@def{cosfrac}{78}{0.78850}		\pgfmath@def{cosfrac}{79}{0.78468}
\pgfmath@def{cosfrac}{80}{0.78086}		\pgfmath@def{cosfrac}{81}{0.77706}
\pgfmath@def{cosfrac}{82}{0.77326}		\pgfmath@def{cosfrac}{83}{0.76948}
\pgfmath@def{cosfrac}{84}{0.76570}		\pgfmath@def{cosfrac}{85}{0.76193}
\pgfmath@def{cosfrac}{86}{0.75818}		\pgfmath@def{cosfrac}{87}{0.75444}
\pgfmath@def{cosfrac}{88}{0.75071}		\pgfmath@def{cosfrac}{89}{0.74699}
\pgfmath@def{cosfrac}{90}{0.74329}		\pgfmath@def{cosfrac}{91}{0.73960}
\pgfmath@def{cosfrac}{92}{0.73593}		\pgfmath@def{cosfrac}{93}{0.73227}
\pgfmath@def{cosfrac}{94}{0.72862}		\pgfmath@def{cosfrac}{95}{0.72499}
\pgfmath@def{cosfrac}{96}{0.72138}		\pgfmath@def{cosfrac}{97}{0.71779}
\pgfmath@def{cosfrac}{98}{0.71421}		\pgfmath@def{cosfrac}{99}{0.71065}
\pgfmath@def{cosfrac}{100}{0.70710}	

% Oh No! Some really loooooong tables.
\pgfmath@def{atan}{0}{0.00000}		\pgfmath@def{atan}{1}{0.05729}
\pgfmath@def{atan}{2}{0.11459}		\pgfmath@def{atan}{3}{0.17188}
\pgfmath@def{atan}{4}{0.22918}		\pgfmath@def{atan}{5}{0.28647}
\pgfmath@def{atan}{6}{0.34377}		\pgfmath@def{atan}{7}{0.40106}
\pgfmath@def{atan}{8}{0.45835}		\pgfmath@def{atan}{9}{0.51564}
\pgfmath@def{atan}{10}{0.57293}		\pgfmath@def{atan}{11}{0.63022}
\pgfmath@def{atan}{12}{0.68751}		\pgfmath@def{atan}{13}{0.74480}
\pgfmath@def{atan}{14}{0.80208}		\pgfmath@def{atan}{15}{0.85937}
\pgfmath@def{atan}{16}{0.91665}		\pgfmath@def{atan}{17}{0.97393}
\pgfmath@def{atan}{18}{1.03121}		\pgfmath@def{atan}{19}{1.08848}
\pgfmath@def{atan}{20}{1.14576}		\pgfmath@def{atan}{21}{1.20303}
\pgfmath@def{atan}{22}{1.26030}		\pgfmath@def{atan}{23}{1.31757}
\pgfmath@def{atan}{24}{1.37483}		\pgfmath@def{atan}{25}{1.43209}
\pgfmath@def{atan}{26}{1.48935}		\pgfmath@def{atan}{27}{1.54661}
\pgfmath@def{atan}{28}{1.60386}		\pgfmath@def{atan}{29}{1.66111}
\pgfmath@def{atan}{30}{1.71835}		\pgfmath@def{atan}{31}{1.77560}
\pgfmath@def{atan}{32}{1.83284}		\pgfmath@def{atan}{33}{1.89007}
\pgfmath@def{atan}{34}{1.94730}		\pgfmath@def{atan}{35}{2.00453}
\pgfmath@def{atan}{36}{2.06175}		\pgfmath@def{atan}{37}{2.11897}
\pgfmath@def{atan}{38}{2.17619}		\pgfmath@def{atan}{39}{2.23340}
\pgfmath@def{atan}{40}{2.29061}		\pgfmath@def{atan}{41}{2.34781}
\pgfmath@def{atan}{42}{2.40500}		\pgfmath@def{atan}{43}{2.46220}
\pgfmath@def{atan}{44}{2.51938}		\pgfmath@def{atan}{45}{2.57657}
\pgfmath@def{atan}{46}{2.63374}		\pgfmath@def{atan}{47}{2.69092}
\pgfmath@def{atan}{48}{2.74808}		\pgfmath@def{atan}{49}{2.80524}
\pgfmath@def{atan}{50}{2.86240}		\pgfmath@def{atan}{51}{2.91955}
\pgfmath@def{atan}{52}{2.97669}		\pgfmath@def{atan}{53}{3.03383}
\pgfmath@def{atan}{54}{3.09097}		\pgfmath@def{atan}{55}{3.14809}
\pgfmath@def{atan}{56}{3.20521}		\pgfmath@def{atan}{57}{3.26232}
\pgfmath@def{atan}{58}{3.31943}		\pgfmath@def{atan}{59}{3.37653}
\pgfmath@def{atan}{60}{3.43363}		\pgfmath@def{atan}{61}{3.49071}
\pgfmath@def{atan}{62}{3.54779}		\pgfmath@def{atan}{63}{3.60487}
\pgfmath@def{atan}{64}{3.66193}		\pgfmath@def{atan}{65}{3.71899}
\pgfmath@def{atan}{66}{3.77604}		\pgfmath@def{atan}{67}{3.83308}
\pgfmath@def{atan}{68}{3.89012}		\pgfmath@def{atan}{69}{3.94715}
\pgfmath@def{atan}{70}{4.00417}		\pgfmath@def{atan}{71}{4.06118}
\pgfmath@def{atan}{72}{4.11819}		\pgfmath@def{atan}{73}{4.17518}
\pgfmath@def{atan}{74}{4.23217}		\pgfmath@def{atan}{75}{4.28915}
\pgfmath@def{atan}{76}{4.34612}		\pgfmath@def{atan}{77}{4.40308}
\pgfmath@def{atan}{78}{4.46004}		\pgfmath@def{atan}{79}{4.51698}
\pgfmath@def{atan}{80}{4.57392}		\pgfmath@def{atan}{81}{4.63084}
\pgfmath@def{atan}{82}{4.68776}		\pgfmath@def{atan}{83}{4.74467}
\pgfmath@def{atan}{84}{4.80157}		\pgfmath@def{atan}{85}{4.85846}
\pgfmath@def{atan}{86}{4.91534}		\pgfmath@def{atan}{87}{4.97221}
\pgfmath@def{atan}{88}{5.02907}		\pgfmath@def{atan}{89}{5.08592}
\pgfmath@def{atan}{90}{5.14276}		\pgfmath@def{atan}{91}{5.19959}
\pgfmath@def{atan}{92}{5.25641}		\pgfmath@def{atan}{93}{5.31322}
\pgfmath@def{atan}{94}{5.37002}		\pgfmath@def{atan}{95}{5.42681}
\pgfmath@def{atan}{96}{5.48359}		\pgfmath@def{atan}{97}{5.54035}
\pgfmath@def{atan}{98}{5.59711}		\pgfmath@def{atan}{99}{5.65385}
\pgfmath@def{atan}{100}{5.71059}		\pgfmath@def{atan}{101}{5.76731}
\pgfmath@def{atan}{102}{5.82402}		\pgfmath@def{atan}{103}{5.88072}
\pgfmath@def{atan}{104}{5.93741}		\pgfmath@def{atan}{105}{5.99409}
\pgfmath@def{atan}{106}{6.05075}		\pgfmath@def{atan}{107}{6.10741}
\pgfmath@def{atan}{108}{6.16405}		\pgfmath@def{atan}{109}{6.22068}
\pgfmath@def{atan}{110}{6.27729}		\pgfmath@def{atan}{111}{6.33390}
\pgfmath@def{atan}{112}{6.39049}		\pgfmath@def{atan}{113}{6.44707}
\pgfmath@def{atan}{114}{6.50364}		\pgfmath@def{atan}{115}{6.56019}
\pgfmath@def{atan}{116}{6.61673}		\pgfmath@def{atan}{117}{6.67326}
\pgfmath@def{atan}{118}{6.72978}		\pgfmath@def{atan}{119}{6.78628}
\pgfmath@def{atan}{120}{6.84277}		\pgfmath@def{atan}{121}{6.89924}
\pgfmath@def{atan}{122}{6.95571}		\pgfmath@def{atan}{123}{7.01216}
\pgfmath@def{atan}{124}{7.06859}		\pgfmath@def{atan}{125}{7.12501}
\pgfmath@def{atan}{126}{7.18142}		\pgfmath@def{atan}{127}{7.23781}
\pgfmath@def{atan}{128}{7.29419}		\pgfmath@def{atan}{129}{7.35056}
\pgfmath@def{atan}{130}{7.40691}		\pgfmath@def{atan}{131}{7.46324}
\pgfmath@def{atan}{132}{7.51957}		\pgfmath@def{atan}{133}{7.57587}
\pgfmath@def{atan}{134}{7.63217}		\pgfmath@def{atan}{135}{7.68844}
\pgfmath@def{atan}{136}{7.74471}		\pgfmath@def{atan}{137}{7.80095}
\pgfmath@def{atan}{138}{7.85719}		\pgfmath@def{atan}{139}{7.91340}
\pgfmath@def{atan}{140}{7.96961}		\pgfmath@def{atan}{141}{8.02579}
\pgfmath@def{atan}{142}{8.08196}		\pgfmath@def{atan}{143}{8.13812}
\pgfmath@def{atan}{144}{8.19426}		\pgfmath@def{atan}{145}{8.25038}
\pgfmath@def{atan}{146}{8.30649}		\pgfmath@def{atan}{147}{8.36258}
\pgfmath@def{atan}{148}{8.41866}		\pgfmath@def{atan}{149}{8.47472}
\pgfmath@def{atan}{150}{8.53076}		\pgfmath@def{atan}{151}{8.58679}
\pgfmath@def{atan}{152}{8.64280}		\pgfmath@def{atan}{153}{8.69879}
\pgfmath@def{atan}{154}{8.75477}		\pgfmath@def{atan}{155}{8.81073}
\pgfmath@def{atan}{156}{8.86667}		\pgfmath@def{atan}{157}{8.92260}
\pgfmath@def{atan}{158}{8.97851}		\pgfmath@def{atan}{159}{9.03440}
\pgfmath@def{atan}{160}{9.09027}		\pgfmath@def{atan}{161}{9.14613}
\pgfmath@def{atan}{162}{9.20197}		\pgfmath@def{atan}{163}{9.25779}
\pgfmath@def{atan}{164}{9.31359}		\pgfmath@def{atan}{165}{9.36938}
\pgfmath@def{atan}{166}{9.42515}		\pgfmath@def{atan}{167}{9.48090}
\pgfmath@def{atan}{168}{9.53663}		\pgfmath@def{atan}{169}{9.59235}
\pgfmath@def{atan}{170}{9.64804}		\pgfmath@def{atan}{171}{9.70372}
\pgfmath@def{atan}{172}{9.75938}		\pgfmath@def{atan}{173}{9.81502}
\pgfmath@def{atan}{174}{9.87064}		\pgfmath@def{atan}{175}{9.92624}
\pgfmath@def{atan}{176}{9.98182}		\pgfmath@def{atan}{177}{10.03739}
\pgfmath@def{atan}{178}{10.09294}		\pgfmath@def{atan}{179}{10.14846}
\pgfmath@def{atan}{180}{10.20397}		\pgfmath@def{atan}{181}{10.25946}
\pgfmath@def{atan}{182}{10.31493}		\pgfmath@def{atan}{183}{10.37037}
\pgfmath@def{atan}{184}{10.42580}		\pgfmath@def{atan}{185}{10.48121}
\pgfmath@def{atan}{186}{10.53660}		\pgfmath@def{atan}{187}{10.59197}
\pgfmath@def{atan}{188}{10.64732}		\pgfmath@def{atan}{189}{10.70265}
\pgfmath@def{atan}{190}{10.75796}		\pgfmath@def{atan}{191}{10.81325}
\pgfmath@def{atan}{192}{10.86852}		\pgfmath@def{atan}{193}{10.92377}
\pgfmath@def{atan}{194}{10.97900}		\pgfmath@def{atan}{195}{11.03420}
\pgfmath@def{atan}{196}{11.08939}		\pgfmath@def{atan}{197}{11.14456}
\pgfmath@def{atan}{198}{11.19970}		\pgfmath@def{atan}{199}{11.25483}
\pgfmath@def{atan}{200}{11.30993}		\pgfmath@def{atan}{201}{11.36501}
\pgfmath@def{atan}{202}{11.42007}		\pgfmath@def{atan}{203}{11.47511}
\pgfmath@def{atan}{204}{11.53013}		\pgfmath@def{atan}{205}{11.58512}
\pgfmath@def{atan}{206}{11.64010}		\pgfmath@def{atan}{207}{11.69505}
\pgfmath@def{atan}{208}{11.74998}		\pgfmath@def{atan}{209}{11.80489}
\pgfmath@def{atan}{210}{11.85977}		\pgfmath@def{atan}{211}{11.91464}
\pgfmath@def{atan}{212}{11.96948}		\pgfmath@def{atan}{213}{12.02430}
\pgfmath@def{atan}{214}{12.07910}		\pgfmath@def{atan}{215}{12.13388}
\pgfmath@def{atan}{216}{12.18863}		\pgfmath@def{atan}{217}{12.24336}
\pgfmath@def{atan}{218}{12.29807}		\pgfmath@def{atan}{219}{12.35275}
\pgfmath@def{atan}{220}{12.40741}		\pgfmath@def{atan}{221}{12.46205}
\pgfmath@def{atan}{222}{12.51667}		\pgfmath@def{atan}{223}{12.57126}
\pgfmath@def{atan}{224}{12.62583}		\pgfmath@def{atan}{225}{12.68038}
\pgfmath@def{atan}{226}{12.73490}		\pgfmath@def{atan}{227}{12.78940}
\pgfmath@def{atan}{228}{12.84388}		\pgfmath@def{atan}{229}{12.89833}
\pgfmath@def{atan}{230}{12.95276}		\pgfmath@def{atan}{231}{13.00717}
\pgfmath@def{atan}{232}{13.06155}		\pgfmath@def{atan}{233}{13.11590}
\pgfmath@def{atan}{234}{13.17024}		\pgfmath@def{atan}{235}{13.22455}
\pgfmath@def{atan}{236}{13.27883}		\pgfmath@def{atan}{237}{13.33309}
\pgfmath@def{atan}{238}{13.38733}		\pgfmath@def{atan}{239}{13.44154}
\pgfmath@def{atan}{240}{13.49573}		\pgfmath@def{atan}{241}{13.54989}
\pgfmath@def{atan}{242}{13.60403}		\pgfmath@def{atan}{243}{13.65814}
\pgfmath@def{atan}{244}{13.71223}		\pgfmath@def{atan}{245}{13.76630}
\pgfmath@def{atan}{246}{13.82034}		\pgfmath@def{atan}{247}{13.87435}
\pgfmath@def{atan}{248}{13.92834}		\pgfmath@def{atan}{249}{13.98230}
\pgfmath@def{atan}{250}{14.03624}		\pgfmath@def{atan}{251}{14.09015}
\pgfmath@def{atan}{252}{14.14404}		\pgfmath@def{atan}{253}{14.19790}
\pgfmath@def{atan}{254}{14.25174}		\pgfmath@def{atan}{255}{14.30555}
\pgfmath@def{atan}{256}{14.35933}		\pgfmath@def{atan}{257}{14.41309}
\pgfmath@def{atan}{258}{14.46682}		\pgfmath@def{atan}{259}{14.52053}
\pgfmath@def{atan}{260}{14.57421}		\pgfmath@def{atan}{261}{14.62787}
\pgfmath@def{atan}{262}{14.68149}		\pgfmath@def{atan}{263}{14.73510}
\pgfmath@def{atan}{264}{14.78867}		\pgfmath@def{atan}{265}{14.84222}
\pgfmath@def{atan}{266}{14.89575}		\pgfmath@def{atan}{267}{14.94924}
\pgfmath@def{atan}{268}{15.00271}		\pgfmath@def{atan}{269}{15.05615}
\pgfmath@def{atan}{270}{15.10957}		\pgfmath@def{atan}{271}{15.16296}
\pgfmath@def{atan}{272}{15.21632}		\pgfmath@def{atan}{273}{15.26966}
\pgfmath@def{atan}{274}{15.32297}		\pgfmath@def{atan}{275}{15.37625}
\pgfmath@def{atan}{276}{15.42950}		\pgfmath@def{atan}{277}{15.48273}
\pgfmath@def{atan}{278}{15.53593}		\pgfmath@def{atan}{279}{15.58910}
\pgfmath@def{atan}{280}{15.64224}		\pgfmath@def{atan}{281}{15.69536}
\pgfmath@def{atan}{282}{15.74845}		\pgfmath@def{atan}{283}{15.80151}
\pgfmath@def{atan}{284}{15.85454}		\pgfmath@def{atan}{285}{15.90755}
\pgfmath@def{atan}{286}{15.96052}		\pgfmath@def{atan}{287}{16.01347}
\pgfmath@def{atan}{288}{16.06640}		\pgfmath@def{atan}{289}{16.11929}
\pgfmath@def{atan}{290}{16.17215}		\pgfmath@def{atan}{291}{16.22499}
\pgfmath@def{atan}{292}{16.27780}		\pgfmath@def{atan}{293}{16.33058}
\pgfmath@def{atan}{294}{16.38333}		\pgfmath@def{atan}{295}{16.43605}
\pgfmath@def{atan}{296}{16.48875}		\pgfmath@def{atan}{297}{16.54142}
\pgfmath@def{atan}{298}{16.59405}		\pgfmath@def{atan}{299}{16.64666}
\pgfmath@def{atan}{300}{16.69924}		\pgfmath@def{atan}{301}{16.75179}
\pgfmath@def{atan}{302}{16.80431}		\pgfmath@def{atan}{303}{16.85680}
\pgfmath@def{atan}{304}{16.90927}		\pgfmath@def{atan}{305}{16.96170}
\pgfmath@def{atan}{306}{17.01411}		\pgfmath@def{atan}{307}{17.06648}
\pgfmath@def{atan}{308}{17.11883}		\pgfmath@def{atan}{309}{17.17114}
\pgfmath@def{atan}{310}{17.22343}		\pgfmath@def{atan}{311}{17.27569}
\pgfmath@def{atan}{312}{17.32792}		\pgfmath@def{atan}{313}{17.38012}
\pgfmath@def{atan}{314}{17.43228}		\pgfmath@def{atan}{315}{17.48442}
\pgfmath@def{atan}{316}{17.53653}		\pgfmath@def{atan}{317}{17.58861}
\pgfmath@def{atan}{318}{17.64066}		\pgfmath@def{atan}{319}{17.69268}
\pgfmath@def{atan}{320}{17.74467}		\pgfmath@def{atan}{321}{17.79663}
\pgfmath@def{atan}{322}{17.84855}		\pgfmath@def{atan}{323}{17.90045}
\pgfmath@def{atan}{324}{17.95232}		\pgfmath@def{atan}{325}{18.00416}
\pgfmath@def{atan}{326}{18.05596}		\pgfmath@def{atan}{327}{18.10774}
\pgfmath@def{atan}{328}{18.15949}		\pgfmath@def{atan}{329}{18.21120}
\pgfmath@def{atan}{330}{18.26289}		\pgfmath@def{atan}{331}{18.31454}
\pgfmath@def{atan}{332}{18.36616}		\pgfmath@def{atan}{333}{18.41775}
\pgfmath@def{atan}{334}{18.46931}		\pgfmath@def{atan}{335}{18.52084}
\pgfmath@def{atan}{336}{18.57234}		\pgfmath@def{atan}{337}{18.62381}
\pgfmath@def{atan}{338}{18.67525}		\pgfmath@def{atan}{339}{18.72665}
\pgfmath@def{atan}{340}{18.77803}		\pgfmath@def{atan}{341}{18.82937}
\pgfmath@def{atan}{342}{18.88068}		\pgfmath@def{atan}{343}{18.93196}
\pgfmath@def{atan}{344}{18.98321}		\pgfmath@def{atan}{345}{19.03443}
\pgfmath@def{atan}{346}{19.08562}		\pgfmath@def{atan}{347}{19.13677}
\pgfmath@def{atan}{348}{19.18789}		\pgfmath@def{atan}{349}{19.23898}
\pgfmath@def{atan}{350}{19.29004}		\pgfmath@def{atan}{351}{19.34107}
\pgfmath@def{atan}{352}{19.39206}		\pgfmath@def{atan}{353}{19.44303}
\pgfmath@def{atan}{354}{19.49396}		\pgfmath@def{atan}{355}{19.54486}
\pgfmath@def{atan}{356}{19.59572}		\pgfmath@def{atan}{357}{19.64656}
\pgfmath@def{atan}{358}{19.69736}		\pgfmath@def{atan}{359}{19.74813}
\pgfmath@def{atan}{360}{19.79887}		\pgfmath@def{atan}{361}{19.84958}
\pgfmath@def{atan}{362}{19.90025}		\pgfmath@def{atan}{363}{19.95089}
\pgfmath@def{atan}{364}{20.00150}		\pgfmath@def{atan}{365}{20.05208}
\pgfmath@def{atan}{366}{20.10262}		\pgfmath@def{atan}{367}{20.15313}
\pgfmath@def{atan}{368}{20.20361}		\pgfmath@def{atan}{369}{20.25406}
\pgfmath@def{atan}{370}{20.30447}		\pgfmath@def{atan}{371}{20.35485}
\pgfmath@def{atan}{372}{20.40520}		\pgfmath@def{atan}{373}{20.45551}
\pgfmath@def{atan}{374}{20.50579}		\pgfmath@def{atan}{375}{20.55604}
\pgfmath@def{atan}{376}{20.60626}		\pgfmath@def{atan}{377}{20.65644}
\pgfmath@def{atan}{378}{20.70659}		\pgfmath@def{atan}{379}{20.75670}
\pgfmath@def{atan}{380}{20.80679}		\pgfmath@def{atan}{381}{20.85684}
\pgfmath@def{atan}{382}{20.90685}		\pgfmath@def{atan}{383}{20.95684}
\pgfmath@def{atan}{384}{21.00678}		\pgfmath@def{atan}{385}{21.05670}
\pgfmath@def{atan}{386}{21.10658}		\pgfmath@def{atan}{387}{21.15643}
\pgfmath@def{atan}{388}{21.20625}		\pgfmath@def{atan}{389}{21.25603}
\pgfmath@def{atan}{390}{21.30578}		\pgfmath@def{atan}{391}{21.35549}
\pgfmath@def{atan}{392}{21.40517}		\pgfmath@def{atan}{393}{21.45482}
\pgfmath@def{atan}{394}{21.50444}		\pgfmath@def{atan}{395}{21.55402}
\pgfmath@def{atan}{396}{21.60356}		\pgfmath@def{atan}{397}{21.65307}
\pgfmath@def{atan}{398}{21.70255}		\pgfmath@def{atan}{399}{21.75200}
\pgfmath@def{atan}{400}{21.80140}		\pgfmath@def{atan}{401}{21.85078}
\pgfmath@def{atan}{402}{21.90012}		\pgfmath@def{atan}{403}{21.94943}
\pgfmath@def{atan}{404}{21.99870}		\pgfmath@def{atan}{405}{22.04794}
\pgfmath@def{atan}{406}{22.09715}		\pgfmath@def{atan}{407}{22.14632}
\pgfmath@def{atan}{408}{22.19546}		\pgfmath@def{atan}{409}{22.24456}
\pgfmath@def{atan}{410}{22.29362}		\pgfmath@def{atan}{411}{22.34266}
\pgfmath@def{atan}{412}{22.39166}		\pgfmath@def{atan}{413}{22.44062}
\pgfmath@def{atan}{414}{22.48955}		\pgfmath@def{atan}{415}{22.53845}
\pgfmath@def{atan}{416}{22.58731}		\pgfmath@def{atan}{417}{22.63613}
\pgfmath@def{atan}{418}{22.68492}		\pgfmath@def{atan}{419}{22.73368}
\pgfmath@def{atan}{420}{22.78240}		\pgfmath@def{atan}{421}{22.83109}
\pgfmath@def{atan}{422}{22.87974}		\pgfmath@def{atan}{423}{22.92836}
\pgfmath@def{atan}{424}{22.97694}		\pgfmath@def{atan}{425}{23.02549}
\pgfmath@def{atan}{426}{23.07400}		\pgfmath@def{atan}{427}{23.12248}
\pgfmath@def{atan}{428}{23.17092}		\pgfmath@def{atan}{429}{23.21933}
\pgfmath@def{atan}{430}{23.26770}		\pgfmath@def{atan}{431}{23.31604}
\pgfmath@def{atan}{432}{23.36434}		\pgfmath@def{atan}{433}{23.41261}
\pgfmath@def{atan}{434}{23.46084}		\pgfmath@def{atan}{435}{23.50904}
\pgfmath@def{atan}{436}{23.55720}		\pgfmath@def{atan}{437}{23.60532}
\pgfmath@def{atan}{438}{23.65341}		\pgfmath@def{atan}{439}{23.70147}
\pgfmath@def{atan}{440}{23.74949}		\pgfmath@def{atan}{441}{23.79747}
\pgfmath@def{atan}{442}{23.84542}		\pgfmath@def{atan}{443}{23.89334}
\pgfmath@def{atan}{444}{23.94122}		\pgfmath@def{atan}{445}{23.98906}
\pgfmath@def{atan}{446}{24.03687}		\pgfmath@def{atan}{447}{24.08464}
\pgfmath@def{atan}{448}{24.13238}		\pgfmath@def{atan}{449}{24.18008}
\pgfmath@def{atan}{450}{24.22774}		\pgfmath@def{atan}{451}{24.27537}
\pgfmath@def{atan}{452}{24.32296}		\pgfmath@def{atan}{453}{24.37052}
\pgfmath@def{atan}{454}{24.41804}		\pgfmath@def{atan}{455}{24.46553}
\pgfmath@def{atan}{456}{24.51298}		\pgfmath@def{atan}{457}{24.56040}
\pgfmath@def{atan}{458}{24.60778}		\pgfmath@def{atan}{459}{24.65512}
\pgfmath@def{atan}{460}{24.70243}		\pgfmath@def{atan}{461}{24.74970}
\pgfmath@def{atan}{462}{24.79693}		\pgfmath@def{atan}{463}{24.84413}
\pgfmath@def{atan}{464}{24.89130}		\pgfmath@def{atan}{465}{24.93842}
\pgfmath@def{atan}{466}{24.98551}		\pgfmath@def{atan}{467}{25.03257}
\pgfmath@def{atan}{468}{25.07959}		\pgfmath@def{atan}{469}{25.12657}
\pgfmath@def{atan}{470}{25.17352}		\pgfmath@def{atan}{471}{25.22043}
\pgfmath@def{atan}{472}{25.26731}		\pgfmath@def{atan}{473}{25.31414}
\pgfmath@def{atan}{474}{25.36095}		\pgfmath@def{atan}{475}{25.40771}
\pgfmath@def{atan}{476}{25.45444}		\pgfmath@def{atan}{477}{25.50114}
\pgfmath@def{atan}{478}{25.54780}		\pgfmath@def{atan}{479}{25.59442}
\pgfmath@def{atan}{480}{25.64100}		\pgfmath@def{atan}{481}{25.68755}
\pgfmath@def{atan}{482}{25.73406}		\pgfmath@def{atan}{483}{25.78054}
\pgfmath@def{atan}{484}{25.82698}		\pgfmath@def{atan}{485}{25.87338}
\pgfmath@def{atan}{486}{25.91975}		\pgfmath@def{atan}{487}{25.96608}
\pgfmath@def{atan}{488}{26.01237}		\pgfmath@def{atan}{489}{26.05863}
\pgfmath@def{atan}{490}{26.10485}		\pgfmath@def{atan}{491}{26.15103}
\pgfmath@def{atan}{492}{26.19718}		\pgfmath@def{atan}{493}{26.24329}
\pgfmath@def{atan}{494}{26.28937}		\pgfmath@def{atan}{495}{26.33541}
\pgfmath@def{atan}{496}{26.38141}		\pgfmath@def{atan}{497}{26.42737}
\pgfmath@def{atan}{498}{26.47330}		\pgfmath@def{atan}{499}{26.51919}
\pgfmath@def{atan}{500}{26.56505}		\pgfmath@def{atan}{501}{26.61086}
\pgfmath@def{atan}{502}{26.65665}		\pgfmath@def{atan}{503}{26.70239}
\pgfmath@def{atan}{504}{26.74810}		\pgfmath@def{atan}{505}{26.79377}
\pgfmath@def{atan}{506}{26.83941}		\pgfmath@def{atan}{507}{26.88500}
\pgfmath@def{atan}{508}{26.93057}		\pgfmath@def{atan}{509}{26.97609}
\pgfmath@def{atan}{510}{27.02158}		\pgfmath@def{atan}{511}{27.06703}
\pgfmath@def{atan}{512}{27.11244}		\pgfmath@def{atan}{513}{27.15782}
\pgfmath@def{atan}{514}{27.20316}		\pgfmath@def{atan}{515}{27.24846}
\pgfmath@def{atan}{516}{27.29373}		\pgfmath@def{atan}{517}{27.33896}
\pgfmath@def{atan}{518}{27.38415}		\pgfmath@def{atan}{519}{27.42931}
\pgfmath@def{atan}{520}{27.47443}		\pgfmath@def{atan}{521}{27.51951}
\pgfmath@def{atan}{522}{27.56455}		\pgfmath@def{atan}{523}{27.60956}
\pgfmath@def{atan}{524}{27.65453}		\pgfmath@def{atan}{525}{27.69947}
\pgfmath@def{atan}{526}{27.74437}		\pgfmath@def{atan}{527}{27.78923}
\pgfmath@def{atan}{528}{27.83405}		\pgfmath@def{atan}{529}{27.87884}
\pgfmath@def{atan}{530}{27.92359}		\pgfmath@def{atan}{531}{27.96830}
\pgfmath@def{atan}{532}{28.01297}		\pgfmath@def{atan}{533}{28.05761}
\pgfmath@def{atan}{534}{28.10221}		\pgfmath@def{atan}{535}{28.14678}
\pgfmath@def{atan}{536}{28.19130}		\pgfmath@def{atan}{537}{28.23579}
\pgfmath@def{atan}{538}{28.28025}		\pgfmath@def{atan}{539}{28.32466}
\pgfmath@def{atan}{540}{28.36904}		\pgfmath@def{atan}{541}{28.41338}
\pgfmath@def{atan}{542}{28.45769}		\pgfmath@def{atan}{543}{28.50196}
\pgfmath@def{atan}{544}{28.54619}		\pgfmath@def{atan}{545}{28.59038}
\pgfmath@def{atan}{546}{28.63454}		\pgfmath@def{atan}{547}{28.67865}
\pgfmath@def{atan}{548}{28.72274}		\pgfmath@def{atan}{549}{28.76678}
\pgfmath@def{atan}{550}{28.81079}		\pgfmath@def{atan}{551}{28.85476}
\pgfmath@def{atan}{552}{28.89869}		\pgfmath@def{atan}{553}{28.94259}
\pgfmath@def{atan}{554}{28.98645}		\pgfmath@def{atan}{555}{29.03027}
\pgfmath@def{atan}{556}{29.07405}		\pgfmath@def{atan}{557}{29.11780}
\pgfmath@def{atan}{558}{29.16151}		\pgfmath@def{atan}{559}{29.20519}
\pgfmath@def{atan}{560}{29.24882}		\pgfmath@def{atan}{561}{29.29242}
\pgfmath@def{atan}{562}{29.33598}		\pgfmath@def{atan}{563}{29.37951}
\pgfmath@def{atan}{564}{29.42299}		\pgfmath@def{atan}{565}{29.46644}
\pgfmath@def{atan}{566}{29.50986}		\pgfmath@def{atan}{567}{29.55323}
\pgfmath@def{atan}{568}{29.59657}		\pgfmath@def{atan}{569}{29.63987}
\pgfmath@def{atan}{570}{29.68314}		\pgfmath@def{atan}{571}{29.72636}
\pgfmath@def{atan}{572}{29.76955}		\pgfmath@def{atan}{573}{29.81270}
\pgfmath@def{atan}{574}{29.85582}		\pgfmath@def{atan}{575}{29.89890}
\pgfmath@def{atan}{576}{29.94194}		\pgfmath@def{atan}{577}{29.98494}
\pgfmath@def{atan}{578}{30.02791}		\pgfmath@def{atan}{579}{30.07084}
\pgfmath@def{atan}{580}{30.11373}		\pgfmath@def{atan}{581}{30.15658}
\pgfmath@def{atan}{582}{30.19940}		\pgfmath@def{atan}{583}{30.24218}
\pgfmath@def{atan}{584}{30.28492}		\pgfmath@def{atan}{585}{30.32763}
\pgfmath@def{atan}{586}{30.37030}		\pgfmath@def{atan}{587}{30.41293}
\pgfmath@def{atan}{588}{30.45552}		\pgfmath@def{atan}{589}{30.49808}
\pgfmath@def{atan}{590}{30.54060}		\pgfmath@def{atan}{591}{30.58308}
\pgfmath@def{atan}{592}{30.62553}		\pgfmath@def{atan}{593}{30.66794}
\pgfmath@def{atan}{594}{30.71031}		\pgfmath@def{atan}{595}{30.75264}
\pgfmath@def{atan}{596}{30.79494}		\pgfmath@def{atan}{597}{30.83720}
\pgfmath@def{atan}{598}{30.87942}		\pgfmath@def{atan}{599}{30.92160}
\pgfmath@def{atan}{600}{30.96375}		\pgfmath@def{atan}{601}{31.00586}
\pgfmath@def{atan}{602}{31.04794}		\pgfmath@def{atan}{603}{31.08997}
\pgfmath@def{atan}{604}{31.13197}		\pgfmath@def{atan}{605}{31.17393}
\pgfmath@def{atan}{606}{31.21586}		\pgfmath@def{atan}{607}{31.25775}
\pgfmath@def{atan}{608}{31.29960}		\pgfmath@def{atan}{609}{31.34141}
\pgfmath@def{atan}{610}{31.38319}		\pgfmath@def{atan}{611}{31.42493}
\pgfmath@def{atan}{612}{31.46663}		\pgfmath@def{atan}{613}{31.50829}
\pgfmath@def{atan}{614}{31.54992}		\pgfmath@def{atan}{615}{31.59151}
\pgfmath@def{atan}{616}{31.63306}		\pgfmath@def{atan}{617}{31.67458}
\pgfmath@def{atan}{618}{31.71606}		\pgfmath@def{atan}{619}{31.75750}
\pgfmath@def{atan}{620}{31.79891}		\pgfmath@def{atan}{621}{31.84028}
\pgfmath@def{atan}{622}{31.88161}		\pgfmath@def{atan}{623}{31.92290}
\pgfmath@def{atan}{624}{31.96416}		\pgfmath@def{atan}{625}{32.00538}
\pgfmath@def{atan}{626}{32.04656}		\pgfmath@def{atan}{627}{32.08771}
\pgfmath@def{atan}{628}{32.12882}		\pgfmath@def{atan}{629}{32.16989}
\pgfmath@def{atan}{630}{32.21092}		\pgfmath@def{atan}{631}{32.25192}
\pgfmath@def{atan}{632}{32.29288}		\pgfmath@def{atan}{633}{32.33381}
\pgfmath@def{atan}{634}{32.37469}		\pgfmath@def{atan}{635}{32.41554}
\pgfmath@def{atan}{636}{32.45636}		\pgfmath@def{atan}{637}{32.49713}
\pgfmath@def{atan}{638}{32.53787}		\pgfmath@def{atan}{639}{32.57857}
\pgfmath@def{atan}{640}{32.61924}		\pgfmath@def{atan}{641}{32.65987}
\pgfmath@def{atan}{642}{32.70046}		\pgfmath@def{atan}{643}{32.74101}
\pgfmath@def{atan}{644}{32.78153}		\pgfmath@def{atan}{645}{32.82201}
\pgfmath@def{atan}{646}{32.86246}		\pgfmath@def{atan}{647}{32.90286}
\pgfmath@def{atan}{648}{32.94323}		\pgfmath@def{atan}{649}{32.98357}
\pgfmath@def{atan}{650}{33.02386}		\pgfmath@def{atan}{651}{33.06412}
\pgfmath@def{atan}{652}{33.10435}		\pgfmath@def{atan}{653}{33.14453}
\pgfmath@def{atan}{654}{33.18468}		\pgfmath@def{atan}{655}{33.22479}
\pgfmath@def{atan}{656}{33.26487}		\pgfmath@def{atan}{657}{33.30491}
\pgfmath@def{atan}{658}{33.34491}		\pgfmath@def{atan}{659}{33.38488}
\pgfmath@def{atan}{660}{33.42481}		\pgfmath@def{atan}{661}{33.46470}
\pgfmath@def{atan}{662}{33.50455}		\pgfmath@def{atan}{663}{33.54437}
\pgfmath@def{atan}{664}{33.58416}		\pgfmath@def{atan}{665}{33.62390}
\pgfmath@def{atan}{666}{33.66361}		\pgfmath@def{atan}{667}{33.70328}
\pgfmath@def{atan}{668}{33.74292}		\pgfmath@def{atan}{669}{33.78252}
\pgfmath@def{atan}{670}{33.82208}		\pgfmath@def{atan}{671}{33.86161}
\pgfmath@def{atan}{672}{33.90110}		\pgfmath@def{atan}{673}{33.94055}
\pgfmath@def{atan}{674}{33.97997}		\pgfmath@def{atan}{675}{34.01935}
\pgfmath@def{atan}{676}{34.05869}		\pgfmath@def{atan}{677}{34.09800}
\pgfmath@def{atan}{678}{34.13727}		\pgfmath@def{atan}{679}{34.17650}
\pgfmath@def{atan}{680}{34.21570}		\pgfmath@def{atan}{681}{34.25486}
\pgfmath@def{atan}{682}{34.29398}		\pgfmath@def{atan}{683}{34.33307}
\pgfmath@def{atan}{684}{34.37212}		\pgfmath@def{atan}{685}{34.41114}
\pgfmath@def{atan}{686}{34.45012}		\pgfmath@def{atan}{687}{34.48906}
\pgfmath@def{atan}{688}{34.52797}		\pgfmath@def{atan}{689}{34.56684}
\pgfmath@def{atan}{690}{34.60567}		\pgfmath@def{atan}{691}{34.64447}
\pgfmath@def{atan}{692}{34.68323}		\pgfmath@def{atan}{693}{34.72195}
\pgfmath@def{atan}{694}{34.76064}		\pgfmath@def{atan}{695}{34.79930}
\pgfmath@def{atan}{696}{34.83791}		\pgfmath@def{atan}{697}{34.87649}
\pgfmath@def{atan}{698}{34.91504}		\pgfmath@def{atan}{699}{34.95354}
\pgfmath@def{atan}{700}{34.99202}		\pgfmath@def{atan}{701}{35.03045}
\pgfmath@def{atan}{702}{35.06885}		\pgfmath@def{atan}{703}{35.10721}
\pgfmath@def{atan}{704}{35.14554}		\pgfmath@def{atan}{705}{35.18383}
\pgfmath@def{atan}{706}{35.22209}		\pgfmath@def{atan}{707}{35.26031}
\pgfmath@def{atan}{708}{35.29849}		\pgfmath@def{atan}{709}{35.33664}
\pgfmath@def{atan}{710}{35.37475}		\pgfmath@def{atan}{711}{35.41282}
\pgfmath@def{atan}{712}{35.45086}		\pgfmath@def{atan}{713}{35.48886}
\pgfmath@def{atan}{714}{35.52683}		\pgfmath@def{atan}{715}{35.56476}
\pgfmath@def{atan}{716}{35.60266}		\pgfmath@def{atan}{717}{35.64052}
\pgfmath@def{atan}{718}{35.67834}		\pgfmath@def{atan}{719}{35.71613}
\pgfmath@def{atan}{720}{35.75388}		\pgfmath@def{atan}{721}{35.79160}
\pgfmath@def{atan}{722}{35.82928}		\pgfmath@def{atan}{723}{35.86692}
\pgfmath@def{atan}{724}{35.90453}		\pgfmath@def{atan}{725}{35.94211}
\pgfmath@def{atan}{726}{35.97965}		\pgfmath@def{atan}{727}{36.01715}
\pgfmath@def{atan}{728}{36.05461}		\pgfmath@def{atan}{729}{36.09204}
\pgfmath@def{atan}{730}{36.12944}		\pgfmath@def{atan}{731}{36.16680}
\pgfmath@def{atan}{732}{36.20412}		\pgfmath@def{atan}{733}{36.24141}
\pgfmath@def{atan}{734}{36.27866}		\pgfmath@def{atan}{735}{36.31588}
\pgfmath@def{atan}{736}{36.35306}		\pgfmath@def{atan}{737}{36.39021}
\pgfmath@def{atan}{738}{36.42732}		\pgfmath@def{atan}{739}{36.46440}
\pgfmath@def{atan}{740}{36.50144}		\pgfmath@def{atan}{741}{36.53844}
\pgfmath@def{atan}{742}{36.57541}		\pgfmath@def{atan}{743}{36.61234}
\pgfmath@def{atan}{744}{36.64924}		\pgfmath@def{atan}{745}{36.68611}
\pgfmath@def{atan}{746}{36.72293}		\pgfmath@def{atan}{747}{36.75973}
\pgfmath@def{atan}{748}{36.79648}		\pgfmath@def{atan}{749}{36.83321}
\pgfmath@def{atan}{750}{36.86989}		\pgfmath@def{atan}{751}{36.90654}
\pgfmath@def{atan}{752}{36.94316}		\pgfmath@def{atan}{753}{36.97974}
\pgfmath@def{atan}{754}{37.01629}		\pgfmath@def{atan}{755}{37.05280}
\pgfmath@def{atan}{756}{37.08928}		\pgfmath@def{atan}{757}{37.12572}
\pgfmath@def{atan}{758}{37.16212}		\pgfmath@def{atan}{759}{37.19849}
\pgfmath@def{atan}{760}{37.23483}		\pgfmath@def{atan}{761}{37.27113}
\pgfmath@def{atan}{762}{37.30740}		\pgfmath@def{atan}{763}{37.34363}
\pgfmath@def{atan}{764}{37.37982}		\pgfmath@def{atan}{765}{37.41598}
\pgfmath@def{atan}{766}{37.45211}		\pgfmath@def{atan}{767}{37.48820}
\pgfmath@def{atan}{768}{37.52426}		\pgfmath@def{atan}{769}{37.56028}
\pgfmath@def{atan}{770}{37.59627}		\pgfmath@def{atan}{771}{37.63222}
\pgfmath@def{atan}{772}{37.66814}		\pgfmath@def{atan}{773}{37.70402}
\pgfmath@def{atan}{774}{37.73987}		\pgfmath@def{atan}{775}{37.77568}
\pgfmath@def{atan}{776}{37.81146}		\pgfmath@def{atan}{777}{37.84720}
\pgfmath@def{atan}{778}{37.88291}		\pgfmath@def{atan}{779}{37.91859}
\pgfmath@def{atan}{780}{37.95423}		\pgfmath@def{atan}{781}{37.98983}
\pgfmath@def{atan}{782}{38.02540}		\pgfmath@def{atan}{783}{38.06094}
\pgfmath@def{atan}{784}{38.09644}		\pgfmath@def{atan}{785}{38.13191}
\pgfmath@def{atan}{786}{38.16734}		\pgfmath@def{atan}{787}{38.20274}
\pgfmath@def{atan}{788}{38.23811}		\pgfmath@def{atan}{789}{38.27344}
\pgfmath@def{atan}{790}{38.30873}		\pgfmath@def{atan}{791}{38.34399}
\pgfmath@def{atan}{792}{38.37922}		\pgfmath@def{atan}{793}{38.41441}
\pgfmath@def{atan}{794}{38.44957}		\pgfmath@def{atan}{795}{38.48469}
\pgfmath@def{atan}{796}{38.51979}		\pgfmath@def{atan}{797}{38.55484}
\pgfmath@def{atan}{798}{38.58986}		\pgfmath@def{atan}{799}{38.62485}
\pgfmath@def{atan}{800}{38.65980}		\pgfmath@def{atan}{801}{38.69472}
\pgfmath@def{atan}{802}{38.72961}		\pgfmath@def{atan}{803}{38.76446}
\pgfmath@def{atan}{804}{38.79928}		\pgfmath@def{atan}{805}{38.83406}
\pgfmath@def{atan}{806}{38.86881}		\pgfmath@def{atan}{807}{38.90353}
\pgfmath@def{atan}{808}{38.93821}		\pgfmath@def{atan}{809}{38.97285}
\pgfmath@def{atan}{810}{39.00747}		\pgfmath@def{atan}{811}{39.04205}
\pgfmath@def{atan}{812}{39.07659}		\pgfmath@def{atan}{813}{39.11111}
\pgfmath@def{atan}{814}{39.14558}		\pgfmath@def{atan}{815}{39.18003}
\pgfmath@def{atan}{816}{39.21444}		\pgfmath@def{atan}{817}{39.24882}
\pgfmath@def{atan}{818}{39.28316}		\pgfmath@def{atan}{819}{39.31747}
\pgfmath@def{atan}{820}{39.35175}		\pgfmath@def{atan}{821}{39.38599}
\pgfmath@def{atan}{822}{39.42020}		\pgfmath@def{atan}{823}{39.45438}
\pgfmath@def{atan}{824}{39.48852}		\pgfmath@def{atan}{825}{39.52263}
\pgfmath@def{atan}{826}{39.55670}		\pgfmath@def{atan}{827}{39.59074}
\pgfmath@def{atan}{828}{39.62475}		\pgfmath@def{atan}{829}{39.65873}
\pgfmath@def{atan}{830}{39.69267}		\pgfmath@def{atan}{831}{39.72658}
\pgfmath@def{atan}{832}{39.76045}		\pgfmath@def{atan}{833}{39.79429}
\pgfmath@def{atan}{834}{39.82810}		\pgfmath@def{atan}{835}{39.86188}
\pgfmath@def{atan}{836}{39.89562}		\pgfmath@def{atan}{837}{39.92933}
\pgfmath@def{atan}{838}{39.96300}		\pgfmath@def{atan}{839}{39.99665}
\pgfmath@def{atan}{840}{40.03025}		\pgfmath@def{atan}{841}{40.06383}
\pgfmath@def{atan}{842}{40.09737}		\pgfmath@def{atan}{843}{40.13088}
\pgfmath@def{atan}{844}{40.16436}		\pgfmath@def{atan}{845}{40.19781}
\pgfmath@def{atan}{846}{40.23122}		\pgfmath@def{atan}{847}{40.26459}
\pgfmath@def{atan}{848}{40.29794}		\pgfmath@def{atan}{849}{40.33125}
\pgfmath@def{atan}{850}{40.36453}		\pgfmath@def{atan}{851}{40.39778}
\pgfmath@def{atan}{852}{40.43099}		\pgfmath@def{atan}{853}{40.46417}
\pgfmath@def{atan}{854}{40.49732}		\pgfmath@def{atan}{855}{40.53044}
\pgfmath@def{atan}{856}{40.56352}		\pgfmath@def{atan}{857}{40.59657}
\pgfmath@def{atan}{858}{40.62959}		\pgfmath@def{atan}{859}{40.66257}
\pgfmath@def{atan}{860}{40.69553}		\pgfmath@def{atan}{861}{40.72845}
\pgfmath@def{atan}{862}{40.76133}		\pgfmath@def{atan}{863}{40.79419}
\pgfmath@def{atan}{864}{40.82701}		\pgfmath@def{atan}{865}{40.85980}
\pgfmath@def{atan}{866}{40.89256}		\pgfmath@def{atan}{867}{40.92528}
\pgfmath@def{atan}{868}{40.95798}		\pgfmath@def{atan}{869}{40.99064}
\pgfmath@def{atan}{870}{41.02326}		\pgfmath@def{atan}{871}{41.05586}
\pgfmath@def{atan}{872}{41.08842}		\pgfmath@def{atan}{873}{41.12095}
\pgfmath@def{atan}{874}{41.15345}		\pgfmath@def{atan}{875}{41.18592}
\pgfmath@def{atan}{876}{41.21836}		\pgfmath@def{atan}{877}{41.25076}
\pgfmath@def{atan}{878}{41.28313}		\pgfmath@def{atan}{879}{41.31547}
\pgfmath@def{atan}{880}{41.34777}		\pgfmath@def{atan}{881}{41.38005}
\pgfmath@def{atan}{882}{41.41229}		\pgfmath@def{atan}{883}{41.44450}
\pgfmath@def{atan}{884}{41.47668}		\pgfmath@def{atan}{885}{41.50882}
\pgfmath@def{atan}{886}{41.54094}		\pgfmath@def{atan}{887}{41.57302}
\pgfmath@def{atan}{888}{41.60507}		\pgfmath@def{atan}{889}{41.63709}
\pgfmath@def{atan}{890}{41.66908}		\pgfmath@def{atan}{891}{41.70103}
\pgfmath@def{atan}{892}{41.73296}		\pgfmath@def{atan}{893}{41.76485}
\pgfmath@def{atan}{894}{41.79671}		\pgfmath@def{atan}{895}{41.82854}
\pgfmath@def{atan}{896}{41.86034}		\pgfmath@def{atan}{897}{41.89210}
\pgfmath@def{atan}{898}{41.92383}		\pgfmath@def{atan}{899}{41.95554}
\pgfmath@def{atan}{900}{41.98721}		\pgfmath@def{atan}{901}{42.01885}
\pgfmath@def{atan}{902}{42.05046}		\pgfmath@def{atan}{903}{42.08203}
\pgfmath@def{atan}{904}{42.11358}		\pgfmath@def{atan}{905}{42.14509}
\pgfmath@def{atan}{906}{42.17657}		\pgfmath@def{atan}{907}{42.20802}
\pgfmath@def{atan}{908}{42.23944}		\pgfmath@def{atan}{909}{42.27083}
\pgfmath@def{atan}{910}{42.30219}		\pgfmath@def{atan}{911}{42.33352}
\pgfmath@def{atan}{912}{42.36481}		\pgfmath@def{atan}{913}{42.39607}
\pgfmath@def{atan}{914}{42.42731}		\pgfmath@def{atan}{915}{42.45851}
\pgfmath@def{atan}{916}{42.48968}		\pgfmath@def{atan}{917}{42.52082}
\pgfmath@def{atan}{918}{42.55193}		\pgfmath@def{atan}{919}{42.58300}
\pgfmath@def{atan}{920}{42.61405}		\pgfmath@def{atan}{921}{42.64507}
\pgfmath@def{atan}{922}{42.67605}		\pgfmath@def{atan}{923}{42.70701}
\pgfmath@def{atan}{924}{42.73793}		\pgfmath@def{atan}{925}{42.76882}
\pgfmath@def{atan}{926}{42.79968}		\pgfmath@def{atan}{927}{42.83051}
\pgfmath@def{atan}{928}{42.86131}		\pgfmath@def{atan}{929}{42.89208}
\pgfmath@def{atan}{930}{42.92282}		\pgfmath@def{atan}{931}{42.95353}
\pgfmath@def{atan}{932}{42.98421}		\pgfmath@def{atan}{933}{43.01485}
\pgfmath@def{atan}{934}{43.04547}		\pgfmath@def{atan}{935}{43.07605}
\pgfmath@def{atan}{936}{43.10661}		\pgfmath@def{atan}{937}{43.13713}
\pgfmath@def{atan}{938}{43.16763}		\pgfmath@def{atan}{939}{43.19809}
\pgfmath@def{atan}{940}{43.22853}		\pgfmath@def{atan}{941}{43.25893}
\pgfmath@def{atan}{942}{43.28930}		\pgfmath@def{atan}{943}{43.31964}
\pgfmath@def{atan}{944}{43.34996}		\pgfmath@def{atan}{945}{43.38024}
\pgfmath@def{atan}{946}{43.41049}		\pgfmath@def{atan}{947}{43.44071}
\pgfmath@def{atan}{948}{43.47090}		\pgfmath@def{atan}{949}{43.50106}
\pgfmath@def{atan}{950}{43.53119}		\pgfmath@def{atan}{951}{43.56130}
\pgfmath@def{atan}{952}{43.59137}		\pgfmath@def{atan}{953}{43.62141}
\pgfmath@def{atan}{954}{43.65142}		\pgfmath@def{atan}{955}{43.68140}
\pgfmath@def{atan}{956}{43.71135}		\pgfmath@def{atan}{957}{43.74127}
\pgfmath@def{atan}{958}{43.77116}		\pgfmath@def{atan}{959}{43.80102}
\pgfmath@def{atan}{960}{43.83086}		\pgfmath@def{atan}{961}{43.86066}
\pgfmath@def{atan}{962}{43.89043}		\pgfmath@def{atan}{963}{43.92017}
\pgfmath@def{atan}{964}{43.94988}		\pgfmath@def{atan}{965}{43.97957}
\pgfmath@def{atan}{966}{44.00922}		\pgfmath@def{atan}{967}{44.03884}
\pgfmath@def{atan}{968}{44.06844}		\pgfmath@def{atan}{969}{44.09800}
\pgfmath@def{atan}{970}{44.12754}		\pgfmath@def{atan}{971}{44.15704}
\pgfmath@def{atan}{972}{44.18652}		\pgfmath@def{atan}{973}{44.21597}
\pgfmath@def{atan}{974}{44.24538}		\pgfmath@def{atan}{975}{44.27477}
\pgfmath@def{atan}{976}{44.30413}		\pgfmath@def{atan}{977}{44.33346}
\pgfmath@def{atan}{978}{44.36276}		\pgfmath@def{atan}{979}{44.39203}
\pgfmath@def{atan}{980}{44.42127}		\pgfmath@def{atan}{981}{44.45048}
\pgfmath@def{atan}{982}{44.47966}		\pgfmath@def{atan}{983}{44.50882}
\pgfmath@def{atan}{984}{44.53794}		\pgfmath@def{atan}{985}{44.56704}
\pgfmath@def{atan}{986}{44.59610}		\pgfmath@def{atan}{987}{44.62514}
\pgfmath@def{atan}{988}{44.65415}		\pgfmath@def{atan}{989}{44.68313}
\pgfmath@def{atan}{990}{44.71208}		\pgfmath@def{atan}{991}{44.74100}
\pgfmath@def{atan}{992}{44.76989}		\pgfmath@def{atan}{993}{44.79876}
\pgfmath@def{atan}{994}{44.82759}		\pgfmath@def{atan}{995}{44.85640}
\pgfmath@def{atan}{996}{44.88517}		\pgfmath@def{atan}{997}{44.91392}
\pgfmath@def{atan}{998}{44.94264}		\pgfmath@def{atan}{999}{44.97133}
\pgfmath@def{atan}{1000}{45.00000}		

\pgfmath@def{asin}{0}{0.00000}		\pgfmath@def{asin}{1}{0.05729}
\pgfmath@def{asin}{2}{0.11459}		\pgfmath@def{asin}{3}{0.17188}
\pgfmath@def{asin}{4}{0.22918}		\pgfmath@def{asin}{5}{0.28648}
\pgfmath@def{asin}{6}{0.34377}		\pgfmath@def{asin}{7}{0.40107}
\pgfmath@def{asin}{8}{0.45837}		\pgfmath@def{asin}{9}{0.51566}
\pgfmath@def{asin}{10}{0.57296}		\pgfmath@def{asin}{11}{0.63026}
\pgfmath@def{asin}{12}{0.68756}		\pgfmath@def{asin}{13}{0.74486}
\pgfmath@def{asin}{14}{0.80216}		\pgfmath@def{asin}{15}{0.85946}
\pgfmath@def{asin}{16}{0.91677}		\pgfmath@def{asin}{17}{0.97407}
\pgfmath@def{asin}{18}{1.03138}		\pgfmath@def{asin}{19}{1.08868}
\pgfmath@def{asin}{20}{1.14599}		\pgfmath@def{asin}{21}{1.20330}
\pgfmath@def{asin}{22}{1.26060}		\pgfmath@def{asin}{23}{1.31791}
\pgfmath@def{asin}{24}{1.37523}		\pgfmath@def{asin}{25}{1.43254}
\pgfmath@def{asin}{26}{1.48985}		\pgfmath@def{asin}{27}{1.54717}
\pgfmath@def{asin}{28}{1.60449}		\pgfmath@def{asin}{29}{1.66181}
\pgfmath@def{asin}{30}{1.71913}		\pgfmath@def{asin}{31}{1.77645}
\pgfmath@def{asin}{32}{1.83377}		\pgfmath@def{asin}{33}{1.89110}
\pgfmath@def{asin}{34}{1.94843}		\pgfmath@def{asin}{35}{2.00576}
\pgfmath@def{asin}{36}{2.06309}		\pgfmath@def{asin}{37}{2.12042}
\pgfmath@def{asin}{38}{2.17776}		\pgfmath@def{asin}{39}{2.23510}
\pgfmath@def{asin}{40}{2.29244}		\pgfmath@def{asin}{41}{2.34978}
\pgfmath@def{asin}{42}{2.40713}		\pgfmath@def{asin}{43}{2.46447}
\pgfmath@def{asin}{44}{2.52182}		\pgfmath@def{asin}{45}{2.57918}
\pgfmath@def{asin}{46}{2.63653}		\pgfmath@def{asin}{47}{2.69389}
\pgfmath@def{asin}{48}{2.75125}		\pgfmath@def{asin}{49}{2.80861}
\pgfmath@def{asin}{50}{2.86598}		\pgfmath@def{asin}{51}{2.92335}
\pgfmath@def{asin}{52}{2.98072}		\pgfmath@def{asin}{53}{3.03810}
\pgfmath@def{asin}{54}{3.09547}		\pgfmath@def{asin}{55}{3.15285}
\pgfmath@def{asin}{56}{3.21024}		\pgfmath@def{asin}{57}{3.26763}
\pgfmath@def{asin}{58}{3.32502}		\pgfmath@def{asin}{59}{3.38241}
\pgfmath@def{asin}{60}{3.43981}		\pgfmath@def{asin}{61}{3.49721}
\pgfmath@def{asin}{62}{3.55461}		\pgfmath@def{asin}{63}{3.61202}
\pgfmath@def{asin}{64}{3.66943}		\pgfmath@def{asin}{65}{3.72685}
\pgfmath@def{asin}{66}{3.78427}		\pgfmath@def{asin}{67}{3.84169}
\pgfmath@def{asin}{68}{3.89912}		\pgfmath@def{asin}{69}{3.95655}
\pgfmath@def{asin}{70}{4.01398}		\pgfmath@def{asin}{71}{4.07142}
\pgfmath@def{asin}{72}{4.12886}		\pgfmath@def{asin}{73}{4.18631}
\pgfmath@def{asin}{74}{4.24376}		\pgfmath@def{asin}{75}{4.30122}
\pgfmath@def{asin}{76}{4.35868}		\pgfmath@def{asin}{77}{4.41614}
\pgfmath@def{asin}{78}{4.47361}		\pgfmath@def{asin}{79}{4.53108}
\pgfmath@def{asin}{80}{4.58856}		\pgfmath@def{asin}{81}{4.64604}
\pgfmath@def{asin}{82}{4.70353}		\pgfmath@def{asin}{83}{4.76102}
\pgfmath@def{asin}{84}{4.81852}		\pgfmath@def{asin}{85}{4.87602}
\pgfmath@def{asin}{86}{4.93353}		\pgfmath@def{asin}{87}{4.99104}
\pgfmath@def{asin}{88}{5.04855}		\pgfmath@def{asin}{89}{5.10608}
\pgfmath@def{asin}{90}{5.16360}		\pgfmath@def{asin}{91}{5.22113}
\pgfmath@def{asin}{92}{5.27867}		\pgfmath@def{asin}{93}{5.33621}
\pgfmath@def{asin}{94}{5.39376}		\pgfmath@def{asin}{95}{5.45132}
\pgfmath@def{asin}{96}{5.50887}		\pgfmath@def{asin}{97}{5.56644}
\pgfmath@def{asin}{98}{5.62401}		\pgfmath@def{asin}{99}{5.68158}
\pgfmath@def{asin}{100}{5.73917}		\pgfmath@def{asin}{101}{5.79675}
\pgfmath@def{asin}{102}{5.85435}		\pgfmath@def{asin}{103}{5.91195}
\pgfmath@def{asin}{104}{5.96955}		\pgfmath@def{asin}{105}{6.02716}
\pgfmath@def{asin}{106}{6.08478}		\pgfmath@def{asin}{107}{6.14240}
\pgfmath@def{asin}{108}{6.20003}		\pgfmath@def{asin}{109}{6.25767}
\pgfmath@def{asin}{110}{6.31531}		\pgfmath@def{asin}{111}{6.37296}
\pgfmath@def{asin}{112}{6.43062}		\pgfmath@def{asin}{113}{6.48828}
\pgfmath@def{asin}{114}{6.54595}		\pgfmath@def{asin}{115}{6.60362}
\pgfmath@def{asin}{116}{6.66130}		\pgfmath@def{asin}{117}{6.71899}
\pgfmath@def{asin}{118}{6.77669}		\pgfmath@def{asin}{119}{6.83439}
\pgfmath@def{asin}{120}{6.89210}		\pgfmath@def{asin}{121}{6.94981}
\pgfmath@def{asin}{122}{7.00754}		\pgfmath@def{asin}{123}{7.06527}
\pgfmath@def{asin}{124}{7.12301}		\pgfmath@def{asin}{125}{7.18075}
\pgfmath@def{asin}{126}{7.23850}		\pgfmath@def{asin}{127}{7.29626}
\pgfmath@def{asin}{128}{7.35403}		\pgfmath@def{asin}{129}{7.41181}
\pgfmath@def{asin}{130}{7.46959}		\pgfmath@def{asin}{131}{7.52738}
\pgfmath@def{asin}{132}{7.58518}		\pgfmath@def{asin}{133}{7.64298}
\pgfmath@def{asin}{134}{7.70079}		\pgfmath@def{asin}{135}{7.75862}
\pgfmath@def{asin}{136}{7.81644}		\pgfmath@def{asin}{137}{7.87428}
\pgfmath@def{asin}{138}{7.93213}		\pgfmath@def{asin}{139}{7.98998}
\pgfmath@def{asin}{140}{8.04784}		\pgfmath@def{asin}{141}{8.10571}
\pgfmath@def{asin}{142}{8.16359}		\pgfmath@def{asin}{143}{8.22148}
\pgfmath@def{asin}{144}{8.27937}		\pgfmath@def{asin}{145}{8.33727}
\pgfmath@def{asin}{146}{8.39519}		\pgfmath@def{asin}{147}{8.45311}
\pgfmath@def{asin}{148}{8.51104}		\pgfmath@def{asin}{149}{8.56898}
\pgfmath@def{asin}{150}{8.62692}		\pgfmath@def{asin}{151}{8.68488}
\pgfmath@def{asin}{152}{8.74284}		\pgfmath@def{asin}{153}{8.80082}
\pgfmath@def{asin}{154}{8.85880}		\pgfmath@def{asin}{155}{8.91679}
\pgfmath@def{asin}{156}{8.97479}		\pgfmath@def{asin}{157}{9.03280}
\pgfmath@def{asin}{158}{9.09082}		\pgfmath@def{asin}{159}{9.14885}
\pgfmath@def{asin}{160}{9.20689}		\pgfmath@def{asin}{161}{9.26494}
\pgfmath@def{asin}{162}{9.32300}		\pgfmath@def{asin}{163}{9.38107}
\pgfmath@def{asin}{164}{9.43914}		\pgfmath@def{asin}{165}{9.49723}
\pgfmath@def{asin}{166}{9.55533}		\pgfmath@def{asin}{167}{9.61343}
\pgfmath@def{asin}{168}{9.67155}		\pgfmath@def{asin}{169}{9.72968}
\pgfmath@def{asin}{170}{9.78781}		\pgfmath@def{asin}{171}{9.84596}
\pgfmath@def{asin}{172}{9.90412}		\pgfmath@def{asin}{173}{9.96229}
\pgfmath@def{asin}{174}{10.02047}		\pgfmath@def{asin}{175}{10.07865}
\pgfmath@def{asin}{176}{10.13685}		\pgfmath@def{asin}{177}{10.19506}
\pgfmath@def{asin}{178}{10.25328}		\pgfmath@def{asin}{179}{10.31151}
\pgfmath@def{asin}{180}{10.36976}		\pgfmath@def{asin}{181}{10.42801}
\pgfmath@def{asin}{182}{10.48627}		\pgfmath@def{asin}{183}{10.54455}
\pgfmath@def{asin}{184}{10.60283}		\pgfmath@def{asin}{185}{10.66113}
\pgfmath@def{asin}{186}{10.71944}		\pgfmath@def{asin}{187}{10.77775}
\pgfmath@def{asin}{188}{10.83608}		\pgfmath@def{asin}{189}{10.89443}
\pgfmath@def{asin}{190}{10.95278}		\pgfmath@def{asin}{191}{11.01114}
\pgfmath@def{asin}{192}{11.06952}		\pgfmath@def{asin}{193}{11.12791}
\pgfmath@def{asin}{194}{11.18631}		\pgfmath@def{asin}{195}{11.24472}
\pgfmath@def{asin}{196}{11.30314}		\pgfmath@def{asin}{197}{11.36158}
\pgfmath@def{asin}{198}{11.42002}		\pgfmath@def{asin}{199}{11.47848}
\pgfmath@def{asin}{200}{11.53695}		\pgfmath@def{asin}{201}{11.59544}
\pgfmath@def{asin}{202}{11.65393}		\pgfmath@def{asin}{203}{11.71244}
\pgfmath@def{asin}{204}{11.77096}		\pgfmath@def{asin}{205}{11.82949}
\pgfmath@def{asin}{206}{11.88804}		\pgfmath@def{asin}{207}{11.94660}
\pgfmath@def{asin}{208}{12.00517}		\pgfmath@def{asin}{209}{12.06375}
\pgfmath@def{asin}{210}{12.12235}		\pgfmath@def{asin}{211}{12.18096}
\pgfmath@def{asin}{212}{12.23958}		\pgfmath@def{asin}{213}{12.29821}
\pgfmath@def{asin}{214}{12.35686}		\pgfmath@def{asin}{215}{12.41552}
\pgfmath@def{asin}{216}{12.47420}		\pgfmath@def{asin}{217}{12.53288}
\pgfmath@def{asin}{218}{12.59159}		\pgfmath@def{asin}{219}{12.65030}
\pgfmath@def{asin}{220}{12.70903}		\pgfmath@def{asin}{221}{12.76777}
\pgfmath@def{asin}{222}{12.82653}		\pgfmath@def{asin}{223}{12.88529}
\pgfmath@def{asin}{224}{12.94408}		\pgfmath@def{asin}{225}{13.00287}
\pgfmath@def{asin}{226}{13.06168}		\pgfmath@def{asin}{227}{13.12051}
\pgfmath@def{asin}{228}{13.17935}		\pgfmath@def{asin}{229}{13.23820}
\pgfmath@def{asin}{230}{13.29707}		\pgfmath@def{asin}{231}{13.35595}
\pgfmath@def{asin}{232}{13.41484}		\pgfmath@def{asin}{233}{13.47375}
\pgfmath@def{asin}{234}{13.53268}		\pgfmath@def{asin}{235}{13.59162}
\pgfmath@def{asin}{236}{13.65057}		\pgfmath@def{asin}{237}{13.70954}
\pgfmath@def{asin}{238}{13.76852}		\pgfmath@def{asin}{239}{13.82752}
\pgfmath@def{asin}{240}{13.88654}		\pgfmath@def{asin}{241}{13.94556}
\pgfmath@def{asin}{242}{14.00461}		\pgfmath@def{asin}{243}{14.06367}
\pgfmath@def{asin}{244}{14.12274}		\pgfmath@def{asin}{245}{14.18183}
\pgfmath@def{asin}{246}{14.24093}		\pgfmath@def{asin}{247}{14.30005}
\pgfmath@def{asin}{248}{14.35919}		\pgfmath@def{asin}{249}{14.41834}
\pgfmath@def{asin}{250}{14.47751}		\pgfmath@def{asin}{251}{14.53669}
\pgfmath@def{asin}{252}{14.59589}		\pgfmath@def{asin}{253}{14.65510}
\pgfmath@def{asin}{254}{14.71433}		\pgfmath@def{asin}{255}{14.77358}
\pgfmath@def{asin}{256}{14.83284}		\pgfmath@def{asin}{257}{14.89212}
\pgfmath@def{asin}{258}{14.95142}		\pgfmath@def{asin}{259}{15.01073}
\pgfmath@def{asin}{260}{15.07006}		\pgfmath@def{asin}{261}{15.12940}
\pgfmath@def{asin}{262}{15.18876}		\pgfmath@def{asin}{263}{15.24814}
\pgfmath@def{asin}{264}{15.30754}		\pgfmath@def{asin}{265}{15.36695}
\pgfmath@def{asin}{266}{15.42638}		\pgfmath@def{asin}{267}{15.48582}
\pgfmath@def{asin}{268}{15.54529}		\pgfmath@def{asin}{269}{15.60477}
\pgfmath@def{asin}{270}{15.66426}		\pgfmath@def{asin}{271}{15.72378}
\pgfmath@def{asin}{272}{15.78331}		\pgfmath@def{asin}{273}{15.84286}
\pgfmath@def{asin}{274}{15.90243}		\pgfmath@def{asin}{275}{15.96201}
\pgfmath@def{asin}{276}{16.02161}		\pgfmath@def{asin}{277}{16.08123}
\pgfmath@def{asin}{278}{16.14087}		\pgfmath@def{asin}{279}{16.20053}
\pgfmath@def{asin}{280}{16.26020}		\pgfmath@def{asin}{281}{16.31989}
\pgfmath@def{asin}{282}{16.37960}		\pgfmath@def{asin}{283}{16.43933}
\pgfmath@def{asin}{284}{16.49908}		\pgfmath@def{asin}{285}{16.55884}
\pgfmath@def{asin}{286}{16.61863}		\pgfmath@def{asin}{287}{16.67843}
\pgfmath@def{asin}{288}{16.73825}		\pgfmath@def{asin}{289}{16.79809}
\pgfmath@def{asin}{290}{16.85795}		\pgfmath@def{asin}{291}{16.91783}
\pgfmath@def{asin}{292}{16.97773}		\pgfmath@def{asin}{293}{17.03764}
\pgfmath@def{asin}{294}{17.09758}		\pgfmath@def{asin}{295}{17.15753}
\pgfmath@def{asin}{296}{17.21751}		\pgfmath@def{asin}{297}{17.27750}
\pgfmath@def{asin}{298}{17.33751}		\pgfmath@def{asin}{299}{17.39755}
\pgfmath@def{asin}{300}{17.45760}		\pgfmath@def{asin}{301}{17.51767}
\pgfmath@def{asin}{302}{17.57776}		\pgfmath@def{asin}{303}{17.63788}
\pgfmath@def{asin}{304}{17.69801}		\pgfmath@def{asin}{305}{17.75816}
\pgfmath@def{asin}{306}{17.81833}		\pgfmath@def{asin}{307}{17.87852}
\pgfmath@def{asin}{308}{17.93874}		\pgfmath@def{asin}{309}{17.99897}
\pgfmath@def{asin}{310}{18.05923}		\pgfmath@def{asin}{311}{18.11950}
\pgfmath@def{asin}{312}{18.17980}		\pgfmath@def{asin}{313}{18.24011}
\pgfmath@def{asin}{314}{18.30045}		\pgfmath@def{asin}{315}{18.36081}
\pgfmath@def{asin}{316}{18.42119}		\pgfmath@def{asin}{317}{18.48159}
\pgfmath@def{asin}{318}{18.54201}		\pgfmath@def{asin}{319}{18.60246}
\pgfmath@def{asin}{320}{18.66292}		\pgfmath@def{asin}{321}{18.72341}
\pgfmath@def{asin}{322}{18.78392}		\pgfmath@def{asin}{323}{18.84445}
\pgfmath@def{asin}{324}{18.90500}		\pgfmath@def{asin}{325}{18.96557}
\pgfmath@def{asin}{326}{19.02617}		\pgfmath@def{asin}{327}{19.08678}
\pgfmath@def{asin}{328}{19.14742}		\pgfmath@def{asin}{329}{19.20809}
\pgfmath@def{asin}{330}{19.26877}		\pgfmath@def{asin}{331}{19.32948}
\pgfmath@def{asin}{332}{19.39021}		\pgfmath@def{asin}{333}{19.45096}
\pgfmath@def{asin}{334}{19.51174}		\pgfmath@def{asin}{335}{19.57253}
\pgfmath@def{asin}{336}{19.63335}		\pgfmath@def{asin}{337}{19.69420}
\pgfmath@def{asin}{338}{19.75507}		\pgfmath@def{asin}{339}{19.81596}
\pgfmath@def{asin}{340}{19.87687}		\pgfmath@def{asin}{341}{19.93781}
\pgfmath@def{asin}{342}{19.99877}		\pgfmath@def{asin}{343}{20.05975}
\pgfmath@def{asin}{344}{20.12076}		\pgfmath@def{asin}{345}{20.18179}
\pgfmath@def{asin}{346}{20.24285}		\pgfmath@def{asin}{347}{20.30393}
\pgfmath@def{asin}{348}{20.36503}		\pgfmath@def{asin}{349}{20.42616}
\pgfmath@def{asin}{350}{20.48731}		\pgfmath@def{asin}{351}{20.54849}
\pgfmath@def{asin}{352}{20.60969}		\pgfmath@def{asin}{353}{20.67091}
\pgfmath@def{asin}{354}{20.73216}		\pgfmath@def{asin}{355}{20.79344}
\pgfmath@def{asin}{356}{20.85474}		\pgfmath@def{asin}{357}{20.91607}
\pgfmath@def{asin}{358}{20.97742}		\pgfmath@def{asin}{359}{21.03879}
\pgfmath@def{asin}{360}{21.10019}		\pgfmath@def{asin}{361}{21.16162}
\pgfmath@def{asin}{362}{21.22307}		\pgfmath@def{asin}{363}{21.28455}
\pgfmath@def{asin}{364}{21.34605}		\pgfmath@def{asin}{365}{21.40758}
\pgfmath@def{asin}{366}{21.46913}		\pgfmath@def{asin}{367}{21.53071}
\pgfmath@def{asin}{368}{21.59232}		\pgfmath@def{asin}{369}{21.65395}
\pgfmath@def{asin}{370}{21.71561}		\pgfmath@def{asin}{371}{21.77730}
\pgfmath@def{asin}{372}{21.83901}		\pgfmath@def{asin}{373}{21.90075}
\pgfmath@def{asin}{374}{21.96252}		\pgfmath@def{asin}{375}{22.02431}
\pgfmath@def{asin}{376}{22.08613}		\pgfmath@def{asin}{377}{22.14797}
\pgfmath@def{asin}{378}{22.20985}		\pgfmath@def{asin}{379}{22.27175}
\pgfmath@def{asin}{380}{22.33368}		\pgfmath@def{asin}{381}{22.39563}
\pgfmath@def{asin}{382}{22.45762}		\pgfmath@def{asin}{383}{22.51963}
\pgfmath@def{asin}{384}{22.58167}		\pgfmath@def{asin}{385}{22.64374}
\pgfmath@def{asin}{386}{22.70583}		\pgfmath@def{asin}{387}{22.76795}
\pgfmath@def{asin}{388}{22.83011}		\pgfmath@def{asin}{389}{22.89229}
\pgfmath@def{asin}{390}{22.95449}		\pgfmath@def{asin}{391}{23.01673}
\pgfmath@def{asin}{392}{23.07900}		\pgfmath@def{asin}{393}{23.14129}
\pgfmath@def{asin}{394}{23.20362}		\pgfmath@def{asin}{395}{23.26597}
\pgfmath@def{asin}{396}{23.32835}		\pgfmath@def{asin}{397}{23.39076}
\pgfmath@def{asin}{398}{23.45320}		\pgfmath@def{asin}{399}{23.51567}
\pgfmath@def{asin}{400}{23.57817}		\pgfmath@def{asin}{401}{23.64070}
\pgfmath@def{asin}{402}{23.70326}		\pgfmath@def{asin}{403}{23.76585}
\pgfmath@def{asin}{404}{23.82847}		\pgfmath@def{asin}{405}{23.89112}
\pgfmath@def{asin}{406}{23.95380}		\pgfmath@def{asin}{407}{24.01651}
\pgfmath@def{asin}{408}{24.07926}		\pgfmath@def{asin}{409}{24.14203}
\pgfmath@def{asin}{410}{24.20483}		\pgfmath@def{asin}{411}{24.26766}
\pgfmath@def{asin}{412}{24.33053}		\pgfmath@def{asin}{413}{24.39343}
\pgfmath@def{asin}{414}{24.45635}		\pgfmath@def{asin}{415}{24.51931}
\pgfmath@def{asin}{416}{24.58230}		\pgfmath@def{asin}{417}{24.64532}
\pgfmath@def{asin}{418}{24.70838}		\pgfmath@def{asin}{419}{24.77146}
\pgfmath@def{asin}{420}{24.83458}		\pgfmath@def{asin}{421}{24.89773}
\pgfmath@def{asin}{422}{24.96092}		\pgfmath@def{asin}{423}{25.02413}
\pgfmath@def{asin}{424}{25.08738}		\pgfmath@def{asin}{425}{25.15066}
\pgfmath@def{asin}{426}{25.21397}		\pgfmath@def{asin}{427}{25.27732}
\pgfmath@def{asin}{428}{25.34070}		\pgfmath@def{asin}{429}{25.40411}
\pgfmath@def{asin}{430}{25.46756}		\pgfmath@def{asin}{431}{25.53103}
\pgfmath@def{asin}{432}{25.59455}		\pgfmath@def{asin}{433}{25.65809}
\pgfmath@def{asin}{434}{25.72167}		\pgfmath@def{asin}{435}{25.78529}
\pgfmath@def{asin}{436}{25.84894}		\pgfmath@def{asin}{437}{25.91262}
\pgfmath@def{asin}{438}{25.97634}		\pgfmath@def{asin}{439}{26.04009}
\pgfmath@def{asin}{440}{26.10388}		\pgfmath@def{asin}{441}{26.16770}
\pgfmath@def{asin}{442}{26.23155}		\pgfmath@def{asin}{443}{26.29545}
\pgfmath@def{asin}{444}{26.35937}		\pgfmath@def{asin}{445}{26.42333}
\pgfmath@def{asin}{446}{26.48733}		\pgfmath@def{asin}{447}{26.55136}
\pgfmath@def{asin}{448}{26.61543}		\pgfmath@def{asin}{449}{26.67954}
\pgfmath@def{asin}{450}{26.74368}		\pgfmath@def{asin}{451}{26.80786}
\pgfmath@def{asin}{452}{26.87207}		\pgfmath@def{asin}{453}{26.93632}
\pgfmath@def{asin}{454}{27.00061}		\pgfmath@def{asin}{455}{27.06493}
\pgfmath@def{asin}{456}{27.12929}		\pgfmath@def{asin}{457}{27.19369}
\pgfmath@def{asin}{458}{27.25812}		\pgfmath@def{asin}{459}{27.32259}
\pgfmath@def{asin}{460}{27.38710}		\pgfmath@def{asin}{461}{27.45165}
\pgfmath@def{asin}{462}{27.51623}		\pgfmath@def{asin}{463}{27.58086}
\pgfmath@def{asin}{464}{27.64552}		\pgfmath@def{asin}{465}{27.71022}
\pgfmath@def{asin}{466}{27.77496}		\pgfmath@def{asin}{467}{27.83973}
\pgfmath@def{asin}{468}{27.90455}		\pgfmath@def{asin}{469}{27.96940}
\pgfmath@def{asin}{470}{28.03429}		\pgfmath@def{asin}{471}{28.09922}
\pgfmath@def{asin}{472}{28.16419}		\pgfmath@def{asin}{473}{28.22921}
\pgfmath@def{asin}{474}{28.29426}		\pgfmath@def{asin}{475}{28.35935}
\pgfmath@def{asin}{476}{28.42448}		\pgfmath@def{asin}{477}{28.48965}
\pgfmath@def{asin}{478}{28.55486}		\pgfmath@def{asin}{479}{28.62011}
\pgfmath@def{asin}{480}{28.68540}		\pgfmath@def{asin}{481}{28.75073}
\pgfmath@def{asin}{482}{28.81610}		\pgfmath@def{asin}{483}{28.88152}
\pgfmath@def{asin}{484}{28.94697}		\pgfmath@def{asin}{485}{29.01247}
\pgfmath@def{asin}{486}{29.07801}		\pgfmath@def{asin}{487}{29.14359}
\pgfmath@def{asin}{488}{29.20921}		\pgfmath@def{asin}{489}{29.27487}
\pgfmath@def{asin}{490}{29.34058}		\pgfmath@def{asin}{491}{29.40633}
\pgfmath@def{asin}{492}{29.47212}		\pgfmath@def{asin}{493}{29.53795}
\pgfmath@def{asin}{494}{29.60383}		\pgfmath@def{asin}{495}{29.66975}
\pgfmath@def{asin}{496}{29.73571}		\pgfmath@def{asin}{497}{29.80171}
\pgfmath@def{asin}{498}{29.86776}		\pgfmath@def{asin}{499}{29.93386}
\pgfmath@def{asin}{500}{30.00000}		\pgfmath@def{asin}{501}{30.06618}
\pgfmath@def{asin}{502}{30.13240}		\pgfmath@def{asin}{503}{30.19867}
\pgfmath@def{asin}{504}{30.26499}		\pgfmath@def{asin}{505}{30.33135}
\pgfmath@def{asin}{506}{30.39775}		\pgfmath@def{asin}{507}{30.46420}
\pgfmath@def{asin}{508}{30.53070}		\pgfmath@def{asin}{509}{30.59724}
\pgfmath@def{asin}{510}{30.66383}		\pgfmath@def{asin}{511}{30.73046}
\pgfmath@def{asin}{512}{30.79714}		\pgfmath@def{asin}{513}{30.86386}
\pgfmath@def{asin}{514}{30.93063}		\pgfmath@def{asin}{515}{30.99745}
\pgfmath@def{asin}{516}{31.06432}		\pgfmath@def{asin}{517}{31.13123}
\pgfmath@def{asin}{518}{31.19819}		\pgfmath@def{asin}{519}{31.26519}
\pgfmath@def{asin}{520}{31.33225}		\pgfmath@def{asin}{521}{31.39935}
\pgfmath@def{asin}{522}{31.46650}		\pgfmath@def{asin}{523}{31.53370}
\pgfmath@def{asin}{524}{31.60094}		\pgfmath@def{asin}{525}{31.66824}
\pgfmath@def{asin}{526}{31.73558}		\pgfmath@def{asin}{527}{31.80298}
\pgfmath@def{asin}{528}{31.87042}		\pgfmath@def{asin}{529}{31.93791}
\pgfmath@def{asin}{530}{32.00545}		\pgfmath@def{asin}{531}{32.07304}
\pgfmath@def{asin}{532}{32.14068}		\pgfmath@def{asin}{533}{32.20837}
\pgfmath@def{asin}{534}{32.27611}		\pgfmath@def{asin}{535}{32.34391}
\pgfmath@def{asin}{536}{32.41175}		\pgfmath@def{asin}{537}{32.47964}
\pgfmath@def{asin}{538}{32.54759}		\pgfmath@def{asin}{539}{32.61559}
\pgfmath@def{asin}{540}{32.68363}		\pgfmath@def{asin}{541}{32.75173}
\pgfmath@def{asin}{542}{32.81989}		\pgfmath@def{asin}{543}{32.88809}
\pgfmath@def{asin}{544}{32.95635}		\pgfmath@def{asin}{545}{33.02466}
\pgfmath@def{asin}{546}{33.09302}		\pgfmath@def{asin}{547}{33.16144}
\pgfmath@def{asin}{548}{33.22991}		\pgfmath@def{asin}{549}{33.29843}
\pgfmath@def{asin}{550}{33.36701}		\pgfmath@def{asin}{551}{33.43564}
\pgfmath@def{asin}{552}{33.50433}		\pgfmath@def{asin}{553}{33.57307}
\pgfmath@def{asin}{554}{33.64186}		\pgfmath@def{asin}{555}{33.71071}
\pgfmath@def{asin}{556}{33.77962}		\pgfmath@def{asin}{557}{33.84858}
\pgfmath@def{asin}{558}{33.91759}		\pgfmath@def{asin}{559}{33.98666}
\pgfmath@def{asin}{560}{34.05579}		\pgfmath@def{asin}{561}{34.12498}
\pgfmath@def{asin}{562}{34.19422}		\pgfmath@def{asin}{563}{34.26352}
\pgfmath@def{asin}{564}{34.33287}		\pgfmath@def{asin}{565}{34.40229}
\pgfmath@def{asin}{566}{34.47176}		\pgfmath@def{asin}{567}{34.54129}
\pgfmath@def{asin}{568}{34.61087}		\pgfmath@def{asin}{569}{34.68052}
\pgfmath@def{asin}{570}{34.75022}		\pgfmath@def{asin}{571}{34.81998}
\pgfmath@def{asin}{572}{34.88981}		\pgfmath@def{asin}{573}{34.95969}
\pgfmath@def{asin}{574}{35.02963}		\pgfmath@def{asin}{575}{35.09963}
\pgfmath@def{asin}{576}{35.16969}		\pgfmath@def{asin}{577}{35.23981}
\pgfmath@def{asin}{578}{35.30999}		\pgfmath@def{asin}{579}{35.38023}
\pgfmath@def{asin}{580}{35.45054}		\pgfmath@def{asin}{581}{35.52090}
\pgfmath@def{asin}{582}{35.59133}		\pgfmath@def{asin}{583}{35.66182}
\pgfmath@def{asin}{584}{35.73237}		\pgfmath@def{asin}{585}{35.80299}
\pgfmath@def{asin}{586}{35.87366}		\pgfmath@def{asin}{587}{35.94440}
\pgfmath@def{asin}{588}{36.01521}		\pgfmath@def{asin}{589}{36.08607}
\pgfmath@def{asin}{590}{36.15700}		\pgfmath@def{asin}{591}{36.22800}
\pgfmath@def{asin}{592}{36.29906}		\pgfmath@def{asin}{593}{36.37018}
\pgfmath@def{asin}{594}{36.44137}		\pgfmath@def{asin}{595}{36.51263}
\pgfmath@def{asin}{596}{36.58395}		\pgfmath@def{asin}{597}{36.65533}
\pgfmath@def{asin}{598}{36.72679}		\pgfmath@def{asin}{599}{36.79831}
\pgfmath@def{asin}{600}{36.86989}		\pgfmath@def{asin}{601}{36.94155}
\pgfmath@def{asin}{602}{37.01327}		\pgfmath@def{asin}{603}{37.08506}
\pgfmath@def{asin}{604}{37.15691}		\pgfmath@def{asin}{605}{37.22884}
\pgfmath@def{asin}{606}{37.30083}		\pgfmath@def{asin}{607}{37.37289}
\pgfmath@def{asin}{608}{37.44503}		\pgfmath@def{asin}{609}{37.51723}
\pgfmath@def{asin}{610}{37.58950}		\pgfmath@def{asin}{611}{37.66184}
\pgfmath@def{asin}{612}{37.73425}		\pgfmath@def{asin}{613}{37.80674}
\pgfmath@def{asin}{614}{37.87929}		\pgfmath@def{asin}{615}{37.95192}
\pgfmath@def{asin}{616}{38.02461}		\pgfmath@def{asin}{617}{38.09738}
\pgfmath@def{asin}{618}{38.17023}		\pgfmath@def{asin}{619}{38.24314}
\pgfmath@def{asin}{620}{38.31613}		\pgfmath@def{asin}{621}{38.38919}
\pgfmath@def{asin}{622}{38.46233}		\pgfmath@def{asin}{623}{38.53554}
\pgfmath@def{asin}{624}{38.60882}		\pgfmath@def{asin}{625}{38.68218}
\pgfmath@def{asin}{626}{38.75562}		\pgfmath@def{asin}{627}{38.82913}
\pgfmath@def{asin}{628}{38.90272}		\pgfmath@def{asin}{629}{38.97638}
\pgfmath@def{asin}{630}{39.05012}		\pgfmath@def{asin}{631}{39.12393}
\pgfmath@def{asin}{632}{39.19783}		\pgfmath@def{asin}{633}{39.27180}
\pgfmath@def{asin}{634}{39.34585}		\pgfmath@def{asin}{635}{39.41998}
\pgfmath@def{asin}{636}{39.49419}		\pgfmath@def{asin}{637}{39.56847}
\pgfmath@def{asin}{638}{39.64284}		\pgfmath@def{asin}{639}{39.71729}
\pgfmath@def{asin}{640}{39.79181}		\pgfmath@def{asin}{641}{39.86642}
\pgfmath@def{asin}{642}{39.94111}		\pgfmath@def{asin}{643}{40.01588}
\pgfmath@def{asin}{644}{40.09074}		\pgfmath@def{asin}{645}{40.16567}
\pgfmath@def{asin}{646}{40.24069}		\pgfmath@def{asin}{647}{40.31579}
\pgfmath@def{asin}{648}{40.39098}		\pgfmath@def{asin}{649}{40.46624}
\pgfmath@def{asin}{650}{40.54160}		\pgfmath@def{asin}{651}{40.61704}
\pgfmath@def{asin}{652}{40.69256}		\pgfmath@def{asin}{653}{40.76817}
\pgfmath@def{asin}{654}{40.84386}		\pgfmath@def{asin}{655}{40.91965}
\pgfmath@def{asin}{656}{40.99551}		\pgfmath@def{asin}{657}{41.07147}
\pgfmath@def{asin}{658}{41.14751}		\pgfmath@def{asin}{659}{41.22365}
\pgfmath@def{asin}{660}{41.29987}		\pgfmath@def{asin}{661}{41.37618}
\pgfmath@def{asin}{662}{41.45258}		\pgfmath@def{asin}{663}{41.52907}
\pgfmath@def{asin}{664}{41.60565}		\pgfmath@def{asin}{665}{41.68232}
\pgfmath@def{asin}{666}{41.75908}		\pgfmath@def{asin}{667}{41.83594}
\pgfmath@def{asin}{668}{41.91289}		\pgfmath@def{asin}{669}{41.98993}
\pgfmath@def{asin}{670}{42.06706}		\pgfmath@def{asin}{671}{42.14429}
\pgfmath@def{asin}{672}{42.22161}		\pgfmath@def{asin}{673}{42.29903}
\pgfmath@def{asin}{674}{42.37654}		\pgfmath@def{asin}{675}{42.45415}
\pgfmath@def{asin}{676}{42.53185}		\pgfmath@def{asin}{677}{42.60965}
\pgfmath@def{asin}{678}{42.68755}		\pgfmath@def{asin}{679}{42.76554}
\pgfmath@def{asin}{680}{42.84364}		\pgfmath@def{asin}{681}{42.92183}
\pgfmath@def{asin}{682}{43.00012}		\pgfmath@def{asin}{683}{43.07852}
\pgfmath@def{asin}{684}{43.15701}		\pgfmath@def{asin}{685}{43.23560}
\pgfmath@def{asin}{686}{43.31430}		\pgfmath@def{asin}{687}{43.39310}
\pgfmath@def{asin}{688}{43.47199}		\pgfmath@def{asin}{689}{43.55100}
\pgfmath@def{asin}{690}{43.63010}		\pgfmath@def{asin}{691}{43.70932}
\pgfmath@def{asin}{692}{43.78863}		\pgfmath@def{asin}{693}{43.86805}
\pgfmath@def{asin}{694}{43.94758}		\pgfmath@def{asin}{695}{44.02721}
\pgfmath@def{asin}{696}{44.10695}		\pgfmath@def{asin}{697}{44.18680}
\pgfmath@def{asin}{698}{44.26676}		\pgfmath@def{asin}{699}{44.34682}
\pgfmath@def{asin}{700}{44.42700}		\pgfmath@def{asin}{701}{44.50728}
\pgfmath@def{asin}{702}{44.58768}		\pgfmath@def{asin}{703}{44.66819}
\pgfmath@def{asin}{704}{44.74881}		\pgfmath@def{asin}{705}{44.82954}
\pgfmath@def{asin}{706}{44.91038}		\pgfmath@def{asin}{707}{44.99134}
\pgfmath@def{asin}{708}{45.07242}		\pgfmath@def{asin}{709}{45.15361}
\pgfmath@def{asin}{710}{45.23491}		\pgfmath@def{asin}{711}{45.31633}
\pgfmath@def{asin}{712}{45.39787}		\pgfmath@def{asin}{713}{45.47953}
\pgfmath@def{asin}{714}{45.56130}		\pgfmath@def{asin}{715}{45.64319}
\pgfmath@def{asin}{716}{45.72521}		\pgfmath@def{asin}{717}{45.80734}
\pgfmath@def{asin}{718}{45.88960}		\pgfmath@def{asin}{719}{45.97198}
\pgfmath@def{asin}{720}{46.05448}		\pgfmath@def{asin}{721}{46.13710}
\pgfmath@def{asin}{722}{46.21985}		\pgfmath@def{asin}{723}{46.30272}
\pgfmath@def{asin}{724}{46.38572}		\pgfmath@def{asin}{725}{46.46884}
\pgfmath@def{asin}{726}{46.55210}		\pgfmath@def{asin}{727}{46.63548}
\pgfmath@def{asin}{728}{46.71898}		\pgfmath@def{asin}{729}{46.80262}
\pgfmath@def{asin}{730}{46.88639}		\pgfmath@def{asin}{731}{46.97029}
\pgfmath@def{asin}{732}{47.05432}		\pgfmath@def{asin}{733}{47.13848}
\pgfmath@def{asin}{734}{47.22278}		\pgfmath@def{asin}{735}{47.30721}
\pgfmath@def{asin}{736}{47.39178}		\pgfmath@def{asin}{737}{47.47648}
\pgfmath@def{asin}{738}{47.56132}		\pgfmath@def{asin}{739}{47.64630}
\pgfmath@def{asin}{740}{47.73141}		\pgfmath@def{asin}{741}{47.81667}
\pgfmath@def{asin}{742}{47.90206}		\pgfmath@def{asin}{743}{47.98760}
\pgfmath@def{asin}{744}{48.07327}		\pgfmath@def{asin}{745}{48.15909}
\pgfmath@def{asin}{746}{48.24506}		\pgfmath@def{asin}{747}{48.33117}
\pgfmath@def{asin}{748}{48.41742}		\pgfmath@def{asin}{749}{48.50382}
\pgfmath@def{asin}{750}{48.59037}		\pgfmath@def{asin}{751}{48.67707}
\pgfmath@def{asin}{752}{48.76392}		\pgfmath@def{asin}{753}{48.85092}
\pgfmath@def{asin}{754}{48.93806}		\pgfmath@def{asin}{755}{49.02537}
\pgfmath@def{asin}{756}{49.11282}		\pgfmath@def{asin}{757}{49.20043}
\pgfmath@def{asin}{758}{49.28819}		\pgfmath@def{asin}{759}{49.37611}
\pgfmath@def{asin}{760}{49.46419}		\pgfmath@def{asin}{761}{49.55243}
\pgfmath@def{asin}{762}{49.64083}		\pgfmath@def{asin}{763}{49.72939}
\pgfmath@def{asin}{764}{49.81810}		\pgfmath@def{asin}{765}{49.90699}
\pgfmath@def{asin}{766}{49.99603}		\pgfmath@def{asin}{767}{50.08525}
\pgfmath@def{asin}{768}{50.17462}		\pgfmath@def{asin}{769}{50.26417}
\pgfmath@def{asin}{770}{50.35388}		\pgfmath@def{asin}{771}{50.44377}
\pgfmath@def{asin}{772}{50.53382}		\pgfmath@def{asin}{773}{50.62405}
\pgfmath@def{asin}{774}{50.71445}		\pgfmath@def{asin}{775}{50.80503}
\pgfmath@def{asin}{776}{50.89578}		\pgfmath@def{asin}{777}{50.98671}
\pgfmath@def{asin}{778}{51.07782}		\pgfmath@def{asin}{779}{51.16910}
\pgfmath@def{asin}{780}{51.26057}		\pgfmath@def{asin}{781}{51.35222}
\pgfmath@def{asin}{782}{51.44406}		\pgfmath@def{asin}{783}{51.53607}
\pgfmath@def{asin}{784}{51.62828}		\pgfmath@def{asin}{785}{51.72067}
\pgfmath@def{asin}{786}{51.81326}		\pgfmath@def{asin}{787}{51.90603}
\pgfmath@def{asin}{788}{51.99899}		\pgfmath@def{asin}{789}{52.09215}
\pgfmath@def{asin}{790}{52.18551}		\pgfmath@def{asin}{791}{52.27906}
\pgfmath@def{asin}{792}{52.37280}		\pgfmath@def{asin}{793}{52.46675}
\pgfmath@def{asin}{794}{52.56090}		\pgfmath@def{asin}{795}{52.65525}
\pgfmath@def{asin}{796}{52.74981}		\pgfmath@def{asin}{797}{52.84457}
\pgfmath@def{asin}{798}{52.93953}		\pgfmath@def{asin}{799}{53.03471}
\pgfmath@def{asin}{800}{53.13010}		\pgfmath@def{asin}{801}{53.22570}
\pgfmath@def{asin}{802}{53.32151}		\pgfmath@def{asin}{803}{53.41754}
\pgfmath@def{asin}{804}{53.51379}		\pgfmath@def{asin}{805}{53.61025}
\pgfmath@def{asin}{806}{53.70694}		\pgfmath@def{asin}{807}{53.80385}
\pgfmath@def{asin}{808}{53.90098}		\pgfmath@def{asin}{809}{53.99834}
\pgfmath@def{asin}{810}{54.09593}		\pgfmath@def{asin}{811}{54.19374}
\pgfmath@def{asin}{812}{54.29180}		\pgfmath@def{asin}{813}{54.39008}
\pgfmath@def{asin}{814}{54.48860}		\pgfmath@def{asin}{815}{54.58736}
\pgfmath@def{asin}{816}{54.68636}		\pgfmath@def{asin}{817}{54.78560}
\pgfmath@def{asin}{818}{54.88508}		\pgfmath@def{asin}{819}{54.98481}
\pgfmath@def{asin}{820}{55.08479}		\pgfmath@def{asin}{821}{55.18502}
\pgfmath@def{asin}{822}{55.28550}		\pgfmath@def{asin}{823}{55.38624}
\pgfmath@def{asin}{824}{55.48723}		\pgfmath@def{asin}{825}{55.58849}
\pgfmath@def{asin}{826}{55.69000}		\pgfmath@def{asin}{827}{55.79178}
\pgfmath@def{asin}{828}{55.89383}		\pgfmath@def{asin}{829}{55.99615}
\pgfmath@def{asin}{830}{56.09873}		\pgfmath@def{asin}{831}{56.20160}
\pgfmath@def{asin}{832}{56.30473}		\pgfmath@def{asin}{833}{56.40815}
\pgfmath@def{asin}{834}{56.51185}		\pgfmath@def{asin}{835}{56.61583}
\pgfmath@def{asin}{836}{56.72010}		\pgfmath@def{asin}{837}{56.82467}
\pgfmath@def{asin}{838}{56.92952}		\pgfmath@def{asin}{839}{57.03467}
\pgfmath@def{asin}{840}{57.14012}		\pgfmath@def{asin}{841}{57.24586}
\pgfmath@def{asin}{842}{57.35192}		\pgfmath@def{asin}{843}{57.45828}
\pgfmath@def{asin}{844}{57.56495}		\pgfmath@def{asin}{845}{57.67193}
\pgfmath@def{asin}{846}{57.77923}		\pgfmath@def{asin}{847}{57.88685}
\pgfmath@def{asin}{848}{57.99480}		\pgfmath@def{asin}{849}{58.10307}
\pgfmath@def{asin}{850}{58.21166}		\pgfmath@def{asin}{851}{58.32060}
\pgfmath@def{asin}{852}{58.42987}		\pgfmath@def{asin}{853}{58.53948}
\pgfmath@def{asin}{854}{58.64943}		\pgfmath@def{asin}{855}{58.75973}
\pgfmath@def{asin}{856}{58.87038}		\pgfmath@def{asin}{857}{58.98139}
\pgfmath@def{asin}{858}{59.09275}		\pgfmath@def{asin}{859}{59.20448}
\pgfmath@def{asin}{860}{59.31658}		\pgfmath@def{asin}{861}{59.42904}
\pgfmath@def{asin}{862}{59.54189}		\pgfmath@def{asin}{863}{59.65511}
\pgfmath@def{asin}{864}{59.76871}		\pgfmath@def{asin}{865}{59.88270}
\pgfmath@def{asin}{866}{59.99708}		\pgfmath@def{asin}{867}{60.11187}
\pgfmath@def{asin}{868}{60.22705}		\pgfmath@def{asin}{869}{60.34264}
\pgfmath@def{asin}{870}{60.45863}		\pgfmath@def{asin}{871}{60.57505}
\pgfmath@def{asin}{872}{60.69189}		\pgfmath@def{asin}{873}{60.80915}
\pgfmath@def{asin}{874}{60.92684}		\pgfmath@def{asin}{875}{61.04497}
\pgfmath@def{asin}{876}{61.16354}		\pgfmath@def{asin}{877}{61.28256}
\pgfmath@def{asin}{878}{61.40203}		\pgfmath@def{asin}{879}{61.52196}
\pgfmath@def{asin}{880}{61.64236}		\pgfmath@def{asin}{881}{61.76322}
\pgfmath@def{asin}{882}{61.88457}		\pgfmath@def{asin}{883}{62.00639}
\pgfmath@def{asin}{884}{62.12871}		\pgfmath@def{asin}{885}{62.25152}
\pgfmath@def{asin}{886}{62.37483}		\pgfmath@def{asin}{887}{62.49865}
\pgfmath@def{asin}{888}{62.62299}		\pgfmath@def{asin}{889}{62.74785}
\pgfmath@def{asin}{890}{62.87324}		\pgfmath@def{asin}{891}{62.99917}
\pgfmath@def{asin}{892}{63.12565}		\pgfmath@def{asin}{893}{63.25268}
\pgfmath@def{asin}{894}{63.38027}		\pgfmath@def{asin}{895}{63.50843}
\pgfmath@def{asin}{896}{63.63716}		\pgfmath@def{asin}{897}{63.76649}
\pgfmath@def{asin}{898}{63.89640}		\pgfmath@def{asin}{899}{64.02693}
\pgfmath@def{asin}{900}{64.15806}		\pgfmath@def{asin}{901}{64.28982}
\pgfmath@def{asin}{902}{64.42221}		\pgfmath@def{asin}{903}{64.55524}
\pgfmath@def{asin}{904}{64.68893}		\pgfmath@def{asin}{905}{64.82328}
\pgfmath@def{asin}{906}{64.95830}		\pgfmath@def{asin}{907}{65.09401}
\pgfmath@def{asin}{908}{65.23041}		\pgfmath@def{asin}{909}{65.36752}
\pgfmath@def{asin}{910}{65.50535}		\pgfmath@def{asin}{911}{65.64391}
\pgfmath@def{asin}{912}{65.78321}		\pgfmath@def{asin}{913}{65.92327}
\pgfmath@def{asin}{914}{66.06411}		\pgfmath@def{asin}{915}{66.20572}
\pgfmath@def{asin}{916}{66.34814}		\pgfmath@def{asin}{917}{66.49136}
\pgfmath@def{asin}{918}{66.63542}		\pgfmath@def{asin}{919}{66.78032}
\pgfmath@def{asin}{920}{66.92608}		\pgfmath@def{asin}{921}{67.07271}
\pgfmath@def{asin}{922}{67.22024}		\pgfmath@def{asin}{923}{67.36867}
\pgfmath@def{asin}{924}{67.51804}		\pgfmath@def{asin}{925}{67.66835}
\pgfmath@def{asin}{926}{67.81963}		\pgfmath@def{asin}{927}{67.97189}
\pgfmath@def{asin}{928}{68.12516}		\pgfmath@def{asin}{929}{68.27946}
\pgfmath@def{asin}{930}{68.43481}		\pgfmath@def{asin}{931}{68.59123}
\pgfmath@def{asin}{932}{68.74875}		\pgfmath@def{asin}{933}{68.90739}
\pgfmath@def{asin}{934}{69.06718}		\pgfmath@def{asin}{935}{69.22814}
\pgfmath@def{asin}{936}{69.39030}		\pgfmath@def{asin}{937}{69.55369}
\pgfmath@def{asin}{938}{69.71835}		\pgfmath@def{asin}{939}{69.88429}
\pgfmath@def{asin}{940}{70.05155}		\pgfmath@def{asin}{941}{70.22017}
\pgfmath@def{asin}{942}{70.39018}		\pgfmath@def{asin}{943}{70.56162}
\pgfmath@def{asin}{944}{70.73453}		\pgfmath@def{asin}{945}{70.90894}
\pgfmath@def{asin}{946}{71.08490}		\pgfmath@def{asin}{947}{71.26245}
\pgfmath@def{asin}{948}{71.44164}		\pgfmath@def{asin}{949}{71.62251}
\pgfmath@def{asin}{950}{71.80512}		\pgfmath@def{asin}{951}{71.98952}
\pgfmath@def{asin}{952}{72.17576}		\pgfmath@def{asin}{953}{72.36390}
\pgfmath@def{asin}{954}{72.55401}		\pgfmath@def{asin}{955}{72.74614}
\pgfmath@def{asin}{956}{72.94037}		\pgfmath@def{asin}{957}{73.13678}
\pgfmath@def{asin}{958}{73.33542}		\pgfmath@def{asin}{959}{73.53640}
\pgfmath@def{asin}{960}{73.73979}		\pgfmath@def{asin}{961}{73.94569}
\pgfmath@def{asin}{962}{74.15419}		\pgfmath@def{asin}{963}{74.36540}
\pgfmath@def{asin}{964}{74.57942}		\pgfmath@def{asin}{965}{74.79639}
\pgfmath@def{asin}{966}{75.01642}		\pgfmath@def{asin}{967}{75.23966}
\pgfmath@def{asin}{968}{75.46625}		\pgfmath@def{asin}{969}{75.69635}
\pgfmath@def{asin}{970}{75.93013}		\pgfmath@def{asin}{971}{76.16778}
\pgfmath@def{asin}{972}{76.40950}		\pgfmath@def{asin}{973}{76.65552}
\pgfmath@def{asin}{974}{76.90607}		\pgfmath@def{asin}{975}{77.16143}
\pgfmath@def{asin}{976}{77.42188}		\pgfmath@def{asin}{977}{77.68774}
\pgfmath@def{asin}{978}{77.95939}		\pgfmath@def{asin}{979}{78.23721}
\pgfmath@def{asin}{980}{78.52165}		\pgfmath@def{asin}{981}{78.81323}
\pgfmath@def{asin}{982}{79.11251}		\pgfmath@def{asin}{983}{79.42015}
\pgfmath@def{asin}{984}{79.73690}		\pgfmath@def{asin}{985}{80.06363}
\pgfmath@def{asin}{986}{80.40136}		\pgfmath@def{asin}{987}{80.75129}
\pgfmath@def{asin}{988}{81.11487}		\pgfmath@def{asin}{989}{81.49385}
\pgfmath@def{asin}{990}{81.89038}		\pgfmath@def{asin}{991}{82.30718}
\pgfmath@def{asin}{992}{82.74775}		\pgfmath@def{asin}{993}{83.21671}
\pgfmath@def{asin}{994}{83.72041}		\pgfmath@def{asin}{995}{84.26803}
\pgfmath@def{asin}{996}{84.87360}		\pgfmath@def{asin}{997}{85.56077}
\pgfmath@def{asin}{998}{86.37569}		\pgfmath@def{asin}{999}{87.43744}
\pgfmath@def{asin}{1000}{90.00000}	
	
\pgfmath@def{acos}{0}{90.00000}		\pgfmath@def{acos}{1}{89.94270}
\pgfmath@def{acos}{2}{89.88540}		\pgfmath@def{acos}{3}{89.82811}
\pgfmath@def{acos}{4}{89.77081}		\pgfmath@def{acos}{5}{89.71352}
\pgfmath@def{acos}{6}{89.65622}		\pgfmath@def{acos}{7}{89.59892}
\pgfmath@def{acos}{8}{89.54162}		\pgfmath@def{acos}{9}{89.48433}
\pgfmath@def{acos}{10}{89.42703}		\pgfmath@def{acos}{11}{89.36973}
\pgfmath@def{acos}{12}{89.31243}		\pgfmath@def{acos}{13}{89.25513}
\pgfmath@def{acos}{14}{89.19783}		\pgfmath@def{acos}{15}{89.14053}
\pgfmath@def{acos}{16}{89.08322}		\pgfmath@def{acos}{17}{89.02592}
\pgfmath@def{acos}{18}{88.96862}		\pgfmath@def{acos}{19}{88.91131}
\pgfmath@def{acos}{20}{88.85400}		\pgfmath@def{acos}{21}{88.79670}
\pgfmath@def{acos}{22}{88.73939}		\pgfmath@def{acos}{23}{88.68208}
\pgfmath@def{acos}{24}{88.62476}		\pgfmath@def{acos}{25}{88.56745}
\pgfmath@def{acos}{26}{88.51014}		\pgfmath@def{acos}{27}{88.45282}
\pgfmath@def{acos}{28}{88.39550}		\pgfmath@def{acos}{29}{88.33818}
\pgfmath@def{acos}{30}{88.28086}		\pgfmath@def{acos}{31}{88.22354}
\pgfmath@def{acos}{32}{88.16622}		\pgfmath@def{acos}{33}{88.10889}
\pgfmath@def{acos}{34}{88.05156}		\pgfmath@def{acos}{35}{87.99423}
\pgfmath@def{acos}{36}{87.93690}		\pgfmath@def{acos}{37}{87.87957}
\pgfmath@def{acos}{38}{87.82223}		\pgfmath@def{acos}{39}{87.76489}
\pgfmath@def{acos}{40}{87.70755}		\pgfmath@def{acos}{41}{87.65021}
\pgfmath@def{acos}{42}{87.59286}		\pgfmath@def{acos}{43}{87.53552}
\pgfmath@def{acos}{44}{87.47817}		\pgfmath@def{acos}{45}{87.42081}
\pgfmath@def{acos}{46}{87.36346}		\pgfmath@def{acos}{47}{87.30610}
\pgfmath@def{acos}{48}{87.24874}		\pgfmath@def{acos}{49}{87.19138}
\pgfmath@def{acos}{50}{87.13401}		\pgfmath@def{acos}{51}{87.07664}
\pgfmath@def{acos}{52}{87.01927}		\pgfmath@def{acos}{53}{86.96190}
\pgfmath@def{acos}{54}{86.90452}		\pgfmath@def{acos}{55}{86.84714}
\pgfmath@def{acos}{56}{86.78975}		\pgfmath@def{acos}{57}{86.73237}
\pgfmath@def{acos}{58}{86.67497}		\pgfmath@def{acos}{59}{86.61758}
\pgfmath@def{acos}{60}{86.56018}		\pgfmath@def{acos}{61}{86.50278}
\pgfmath@def{acos}{62}{86.44538}		\pgfmath@def{acos}{63}{86.38797}
\pgfmath@def{acos}{64}{86.33056}		\pgfmath@def{acos}{65}{86.27314}
\pgfmath@def{acos}{66}{86.21572}		\pgfmath@def{acos}{67}{86.15830}
\pgfmath@def{acos}{68}{86.10087}		\pgfmath@def{acos}{69}{86.04344}
\pgfmath@def{acos}{70}{85.98601}		\pgfmath@def{acos}{71}{85.92857}
\pgfmath@def{acos}{72}{85.87113}		\pgfmath@def{acos}{73}{85.81368}
\pgfmath@def{acos}{74}{85.75623}		\pgfmath@def{acos}{75}{85.69877}
\pgfmath@def{acos}{76}{85.64131}		\pgfmath@def{acos}{77}{85.58385}
\pgfmath@def{acos}{78}{85.52638}		\pgfmath@def{acos}{79}{85.46891}
\pgfmath@def{acos}{80}{85.41143}		\pgfmath@def{acos}{81}{85.35395}
\pgfmath@def{acos}{82}{85.29646}		\pgfmath@def{acos}{83}{85.23897}
\pgfmath@def{acos}{84}{85.18147}		\pgfmath@def{acos}{85}{85.12397}
\pgfmath@def{acos}{86}{85.06646}		\pgfmath@def{acos}{87}{85.00895}
\pgfmath@def{acos}{88}{84.95144}		\pgfmath@def{acos}{89}{84.89392}
\pgfmath@def{acos}{90}{84.83639}		\pgfmath@def{acos}{91}{84.77886}
\pgfmath@def{acos}{92}{84.72132}		\pgfmath@def{acos}{93}{84.66378}
\pgfmath@def{acos}{94}{84.60623}		\pgfmath@def{acos}{95}{84.54868}
\pgfmath@def{acos}{96}{84.49112}		\pgfmath@def{acos}{97}{84.43355}
\pgfmath@def{acos}{98}{84.37598}		\pgfmath@def{acos}{99}{84.31841}
\pgfmath@def{acos}{100}{84.26083}		\pgfmath@def{acos}{101}{84.20324}
\pgfmath@def{acos}{102}{84.14564}		\pgfmath@def{acos}{103}{84.08805}
\pgfmath@def{acos}{104}{84.03044}		\pgfmath@def{acos}{105}{83.97283}
\pgfmath@def{acos}{106}{83.91521}		\pgfmath@def{acos}{107}{83.85759}
\pgfmath@def{acos}{108}{83.79996}		\pgfmath@def{acos}{109}{83.74232}
\pgfmath@def{acos}{110}{83.68468}		\pgfmath@def{acos}{111}{83.62703}
\pgfmath@def{acos}{112}{83.56938}		\pgfmath@def{acos}{113}{83.51171}
\pgfmath@def{acos}{114}{83.45405}		\pgfmath@def{acos}{115}{83.39637}
\pgfmath@def{acos}{116}{83.33869}		\pgfmath@def{acos}{117}{83.28100}
\pgfmath@def{acos}{118}{83.22330}		\pgfmath@def{acos}{119}{83.16560}
\pgfmath@def{acos}{120}{83.10789}		\pgfmath@def{acos}{121}{83.05018}
\pgfmath@def{acos}{122}{82.99245}		\pgfmath@def{acos}{123}{82.93472}
\pgfmath@def{acos}{124}{82.87698}		\pgfmath@def{acos}{125}{82.81924}
\pgfmath@def{acos}{126}{82.76149}		\pgfmath@def{acos}{127}{82.70373}
\pgfmath@def{acos}{128}{82.64596}		\pgfmath@def{acos}{129}{82.58819}
\pgfmath@def{acos}{130}{82.53040}		\pgfmath@def{acos}{131}{82.47261}
\pgfmath@def{acos}{132}{82.41482}		\pgfmath@def{acos}{133}{82.35701}
\pgfmath@def{acos}{134}{82.29920}		\pgfmath@def{acos}{135}{82.24138}
\pgfmath@def{acos}{136}{82.18355}		\pgfmath@def{acos}{137}{82.12571}
\pgfmath@def{acos}{138}{82.06786}		\pgfmath@def{acos}{139}{82.01001}
\pgfmath@def{acos}{140}{81.95215}		\pgfmath@def{acos}{141}{81.89428}
\pgfmath@def{acos}{142}{81.83640}		\pgfmath@def{acos}{143}{81.77851}
\pgfmath@def{acos}{144}{81.72062}		\pgfmath@def{acos}{145}{81.66272}
\pgfmath@def{acos}{146}{81.60480}		\pgfmath@def{acos}{147}{81.54688}
\pgfmath@def{acos}{148}{81.48895}		\pgfmath@def{acos}{149}{81.43102}
\pgfmath@def{acos}{150}{81.37307}		\pgfmath@def{acos}{151}{81.31511}
\pgfmath@def{acos}{152}{81.25715}		\pgfmath@def{acos}{153}{81.19917}
\pgfmath@def{acos}{154}{81.14119}		\pgfmath@def{acos}{155}{81.08320}
\pgfmath@def{acos}{156}{81.02520}		\pgfmath@def{acos}{157}{80.96719}
\pgfmath@def{acos}{158}{80.90917}		\pgfmath@def{acos}{159}{80.85114}
\pgfmath@def{acos}{160}{80.79310}		\pgfmath@def{acos}{161}{80.73505}
\pgfmath@def{acos}{162}{80.67699}		\pgfmath@def{acos}{163}{80.61893}
\pgfmath@def{acos}{164}{80.56085}		\pgfmath@def{acos}{165}{80.50276}
\pgfmath@def{acos}{166}{80.44466}		\pgfmath@def{acos}{167}{80.38656}
\pgfmath@def{acos}{168}{80.32844}		\pgfmath@def{acos}{169}{80.27031}
\pgfmath@def{acos}{170}{80.21218}		\pgfmath@def{acos}{171}{80.15403}
\pgfmath@def{acos}{172}{80.09587}		\pgfmath@def{acos}{173}{80.03770}
\pgfmath@def{acos}{174}{79.97953}		\pgfmath@def{acos}{175}{79.92134}
\pgfmath@def{acos}{176}{79.86314}		\pgfmath@def{acos}{177}{79.80493}
\pgfmath@def{acos}{178}{79.74671}		\pgfmath@def{acos}{179}{79.68848}
\pgfmath@def{acos}{180}{79.63024}		\pgfmath@def{acos}{181}{79.57198}
\pgfmath@def{acos}{182}{79.51372}		\pgfmath@def{acos}{183}{79.45545}
\pgfmath@def{acos}{184}{79.39716}		\pgfmath@def{acos}{185}{79.33886}
\pgfmath@def{acos}{186}{79.28056}		\pgfmath@def{acos}{187}{79.22224}
\pgfmath@def{acos}{188}{79.16391}		\pgfmath@def{acos}{189}{79.10556}
\pgfmath@def{acos}{190}{79.04721}		\pgfmath@def{acos}{191}{78.98885}
\pgfmath@def{acos}{192}{78.93047}		\pgfmath@def{acos}{193}{78.87208}
\pgfmath@def{acos}{194}{78.81368}		\pgfmath@def{acos}{195}{78.75527}
\pgfmath@def{acos}{196}{78.69685}		\pgfmath@def{acos}{197}{78.63841}
\pgfmath@def{acos}{198}{78.57997}		\pgfmath@def{acos}{199}{78.52151}
\pgfmath@def{acos}{200}{78.46304}		\pgfmath@def{acos}{201}{78.40455}
\pgfmath@def{acos}{202}{78.34606}		\pgfmath@def{acos}{203}{78.28755}
\pgfmath@def{acos}{204}{78.22903}		\pgfmath@def{acos}{205}{78.17050}
\pgfmath@def{acos}{206}{78.11195}		\pgfmath@def{acos}{207}{78.05339}
\pgfmath@def{acos}{208}{77.99482}		\pgfmath@def{acos}{209}{77.93624}
\pgfmath@def{acos}{210}{77.87764}		\pgfmath@def{acos}{211}{77.81903}
\pgfmath@def{acos}{212}{77.76041}		\pgfmath@def{acos}{213}{77.70178}
\pgfmath@def{acos}{214}{77.64313}		\pgfmath@def{acos}{215}{77.58447}
\pgfmath@def{acos}{216}{77.52579}		\pgfmath@def{acos}{217}{77.46711}
\pgfmath@def{acos}{218}{77.40841}		\pgfmath@def{acos}{219}{77.34969}
\pgfmath@def{acos}{220}{77.29096}		\pgfmath@def{acos}{221}{77.23222}
\pgfmath@def{acos}{222}{77.17347}		\pgfmath@def{acos}{223}{77.11470}
\pgfmath@def{acos}{224}{77.05591}		\pgfmath@def{acos}{225}{76.99712}
\pgfmath@def{acos}{226}{76.93831}		\pgfmath@def{acos}{227}{76.87948}
\pgfmath@def{acos}{228}{76.82064}		\pgfmath@def{acos}{229}{76.76179}
\pgfmath@def{acos}{230}{76.70292}		\pgfmath@def{acos}{231}{76.64404}
\pgfmath@def{acos}{232}{76.58515}		\pgfmath@def{acos}{233}{76.52624}
\pgfmath@def{acos}{234}{76.46731}		\pgfmath@def{acos}{235}{76.40837}
\pgfmath@def{acos}{236}{76.34942}		\pgfmath@def{acos}{237}{76.29045}
\pgfmath@def{acos}{238}{76.23147}		\pgfmath@def{acos}{239}{76.17247}
\pgfmath@def{acos}{240}{76.11346}		\pgfmath@def{acos}{241}{76.05443}
\pgfmath@def{acos}{242}{75.99538}		\pgfmath@def{acos}{243}{75.93632}
\pgfmath@def{acos}{244}{75.87725}		\pgfmath@def{acos}{245}{75.81816}
\pgfmath@def{acos}{246}{75.75906}		\pgfmath@def{acos}{247}{75.69994}
\pgfmath@def{acos}{248}{75.64080}		\pgfmath@def{acos}{249}{75.58165}
\pgfmath@def{acos}{250}{75.52248}		\pgfmath@def{acos}{251}{75.46330}
\pgfmath@def{acos}{252}{75.40410}		\pgfmath@def{acos}{253}{75.34489}
\pgfmath@def{acos}{254}{75.28566}		\pgfmath@def{acos}{255}{75.22641}
\pgfmath@def{acos}{256}{75.16715}		\pgfmath@def{acos}{257}{75.10787}
\pgfmath@def{acos}{258}{75.04857}		\pgfmath@def{acos}{259}{74.98926}
\pgfmath@def{acos}{260}{74.92993}		\pgfmath@def{acos}{261}{74.87059}
\pgfmath@def{acos}{262}{74.81123}		\pgfmath@def{acos}{263}{74.75185}
\pgfmath@def{acos}{264}{74.69245}		\pgfmath@def{acos}{265}{74.63304}
\pgfmath@def{acos}{266}{74.57361}		\pgfmath@def{acos}{267}{74.51417}
\pgfmath@def{acos}{268}{74.45471}		\pgfmath@def{acos}{269}{74.39523}
\pgfmath@def{acos}{270}{74.33573}		\pgfmath@def{acos}{271}{74.27621}
\pgfmath@def{acos}{272}{74.21668}		\pgfmath@def{acos}{273}{74.15713}
\pgfmath@def{acos}{274}{74.09757}		\pgfmath@def{acos}{275}{74.03798}
\pgfmath@def{acos}{276}{73.97838}		\pgfmath@def{acos}{277}{73.91876}
\pgfmath@def{acos}{278}{73.85912}		\pgfmath@def{acos}{279}{73.79946}
\pgfmath@def{acos}{280}{73.73979}		\pgfmath@def{acos}{281}{73.68010}
\pgfmath@def{acos}{282}{73.62039}		\pgfmath@def{acos}{283}{73.56066}
\pgfmath@def{acos}{284}{73.50091}		\pgfmath@def{acos}{285}{73.44115}
\pgfmath@def{acos}{286}{73.38136}		\pgfmath@def{acos}{287}{73.32156}
\pgfmath@def{acos}{288}{73.26174}		\pgfmath@def{acos}{289}{73.20190}
\pgfmath@def{acos}{290}{73.14204}		\pgfmath@def{acos}{291}{73.08216}
\pgfmath@def{acos}{292}{73.02226}		\pgfmath@def{acos}{293}{72.96235}
\pgfmath@def{acos}{294}{72.90241}		\pgfmath@def{acos}{295}{72.84246}
\pgfmath@def{acos}{296}{72.78248}		\pgfmath@def{acos}{297}{72.72249}
\pgfmath@def{acos}{298}{72.66248}		\pgfmath@def{acos}{299}{72.60244}
\pgfmath@def{acos}{300}{72.54239}		\pgfmath@def{acos}{301}{72.48232}
\pgfmath@def{acos}{302}{72.42223}		\pgfmath@def{acos}{303}{72.36212}
\pgfmath@def{acos}{304}{72.30198}		\pgfmath@def{acos}{305}{72.24183}
\pgfmath@def{acos}{306}{72.18166}		\pgfmath@def{acos}{307}{72.12147}
\pgfmath@def{acos}{308}{72.06125}		\pgfmath@def{acos}{309}{72.00102}
\pgfmath@def{acos}{310}{71.94077}		\pgfmath@def{acos}{311}{71.88049}
\pgfmath@def{acos}{312}{71.82019}		\pgfmath@def{acos}{313}{71.75988}
\pgfmath@def{acos}{314}{71.69954}		\pgfmath@def{acos}{315}{71.63918}
\pgfmath@def{acos}{316}{71.57880}		\pgfmath@def{acos}{317}{71.51840}
\pgfmath@def{acos}{318}{71.45798}		\pgfmath@def{acos}{319}{71.39754}
\pgfmath@def{acos}{320}{71.33707}		\pgfmath@def{acos}{321}{71.27658}
\pgfmath@def{acos}{322}{71.21608}		\pgfmath@def{acos}{323}{71.15555}
\pgfmath@def{acos}{324}{71.09499}		\pgfmath@def{acos}{325}{71.03442}
\pgfmath@def{acos}{326}{70.97382}		\pgfmath@def{acos}{327}{70.91321}
\pgfmath@def{acos}{328}{70.85257}		\pgfmath@def{acos}{329}{70.79190}
\pgfmath@def{acos}{330}{70.73122}		\pgfmath@def{acos}{331}{70.67051}
\pgfmath@def{acos}{332}{70.60978}		\pgfmath@def{acos}{333}{70.54903}
\pgfmath@def{acos}{334}{70.48826}		\pgfmath@def{acos}{335}{70.42746}
\pgfmath@def{acos}{336}{70.36664}		\pgfmath@def{acos}{337}{70.30579}
\pgfmath@def{acos}{338}{70.24493}		\pgfmath@def{acos}{339}{70.18404}
\pgfmath@def{acos}{340}{70.12312}		\pgfmath@def{acos}{341}{70.06218}
\pgfmath@def{acos}{342}{70.00122}		\pgfmath@def{acos}{343}{69.94024}
\pgfmath@def{acos}{344}{69.87923}		\pgfmath@def{acos}{345}{69.81820}
\pgfmath@def{acos}{346}{69.75714}		\pgfmath@def{acos}{347}{69.69606}
\pgfmath@def{acos}{348}{69.63496}		\pgfmath@def{acos}{349}{69.57383}
\pgfmath@def{acos}{350}{69.51268}		\pgfmath@def{acos}{351}{69.45150}
\pgfmath@def{acos}{352}{69.39030}		\pgfmath@def{acos}{353}{69.32908}
\pgfmath@def{acos}{354}{69.26783}		\pgfmath@def{acos}{355}{69.20655}
\pgfmath@def{acos}{356}{69.14525}		\pgfmath@def{acos}{357}{69.08393}
\pgfmath@def{acos}{358}{69.02258}		\pgfmath@def{acos}{359}{68.96120}
\pgfmath@def{acos}{360}{68.89980}		\pgfmath@def{acos}{361}{68.83837}
\pgfmath@def{acos}{362}{68.77692}		\pgfmath@def{acos}{363}{68.71544}
\pgfmath@def{acos}{364}{68.65394}		\pgfmath@def{acos}{365}{68.59241}
\pgfmath@def{acos}{366}{68.53086}		\pgfmath@def{acos}{367}{68.46928}
\pgfmath@def{acos}{368}{68.40767}		\pgfmath@def{acos}{369}{68.34604}
\pgfmath@def{acos}{370}{68.28438}		\pgfmath@def{acos}{371}{68.22269}
\pgfmath@def{acos}{372}{68.16098}		\pgfmath@def{acos}{373}{68.09924}
\pgfmath@def{acos}{374}{68.03748}		\pgfmath@def{acos}{375}{67.97568}
\pgfmath@def{acos}{376}{67.91386}		\pgfmath@def{acos}{377}{67.85202}
\pgfmath@def{acos}{378}{67.79014}		\pgfmath@def{acos}{379}{67.72824}
\pgfmath@def{acos}{380}{67.66631}		\pgfmath@def{acos}{381}{67.60436}
\pgfmath@def{acos}{382}{67.54237}		\pgfmath@def{acos}{383}{67.48036}
\pgfmath@def{acos}{384}{67.41832}		\pgfmath@def{acos}{385}{67.35626}
\pgfmath@def{acos}{386}{67.29416}		\pgfmath@def{acos}{387}{67.23204}
\pgfmath@def{acos}{388}{67.16988}		\pgfmath@def{acos}{389}{67.10770}
\pgfmath@def{acos}{390}{67.04550}		\pgfmath@def{acos}{391}{66.98326}
\pgfmath@def{acos}{392}{66.92099}		\pgfmath@def{acos}{393}{66.85870}
\pgfmath@def{acos}{394}{66.79637}		\pgfmath@def{acos}{395}{66.73402}
\pgfmath@def{acos}{396}{66.67164}		\pgfmath@def{acos}{397}{66.60923}
\pgfmath@def{acos}{398}{66.54679}		\pgfmath@def{acos}{399}{66.48432}
\pgfmath@def{acos}{400}{66.42182}		\pgfmath@def{acos}{401}{66.35929}
\pgfmath@def{acos}{402}{66.29673}		\pgfmath@def{acos}{403}{66.23414}
\pgfmath@def{acos}{404}{66.17152}		\pgfmath@def{acos}{405}{66.10887}
\pgfmath@def{acos}{406}{66.04619}		\pgfmath@def{acos}{407}{65.98348}
\pgfmath@def{acos}{408}{65.92074}		\pgfmath@def{acos}{409}{65.85796}
\pgfmath@def{acos}{410}{65.79516}		\pgfmath@def{acos}{411}{65.73233}
\pgfmath@def{acos}{412}{65.66946}		\pgfmath@def{acos}{413}{65.60657}
\pgfmath@def{acos}{414}{65.54364}		\pgfmath@def{acos}{415}{65.48068}
\pgfmath@def{acos}{416}{65.41769}		\pgfmath@def{acos}{417}{65.35467}
\pgfmath@def{acos}{418}{65.29161}		\pgfmath@def{acos}{419}{65.22853}
\pgfmath@def{acos}{420}{65.16541}		\pgfmath@def{acos}{421}{65.10226}
\pgfmath@def{acos}{422}{65.03908}		\pgfmath@def{acos}{423}{64.97586}
\pgfmath@def{acos}{424}{64.91261}		\pgfmath@def{acos}{425}{64.84933}
\pgfmath@def{acos}{426}{64.78602}		\pgfmath@def{acos}{427}{64.72267}
\pgfmath@def{acos}{428}{64.65929}		\pgfmath@def{acos}{429}{64.59588}
\pgfmath@def{acos}{430}{64.53244}		\pgfmath@def{acos}{431}{64.46896}
\pgfmath@def{acos}{432}{64.40544}		\pgfmath@def{acos}{433}{64.34190}
\pgfmath@def{acos}{434}{64.27832}		\pgfmath@def{acos}{435}{64.21470}
\pgfmath@def{acos}{436}{64.15105}		\pgfmath@def{acos}{437}{64.08737}
\pgfmath@def{acos}{438}{64.02365}		\pgfmath@def{acos}{439}{63.95990}
\pgfmath@def{acos}{440}{63.89611}		\pgfmath@def{acos}{441}{63.83229}
\pgfmath@def{acos}{442}{63.76844}		\pgfmath@def{acos}{443}{63.70455}
\pgfmath@def{acos}{444}{63.64062}		\pgfmath@def{acos}{445}{63.57666}
\pgfmath@def{acos}{446}{63.51266}		\pgfmath@def{acos}{447}{63.44863}
\pgfmath@def{acos}{448}{63.38456}		\pgfmath@def{acos}{449}{63.32045}
\pgfmath@def{acos}{450}{63.25631}		\pgfmath@def{acos}{451}{63.19213}
\pgfmath@def{acos}{452}{63.12792}		\pgfmath@def{acos}{453}{63.06367}
\pgfmath@def{acos}{454}{62.99938}		\pgfmath@def{acos}{455}{62.93506}
\pgfmath@def{acos}{456}{62.87070}		\pgfmath@def{acos}{457}{62.80630}
\pgfmath@def{acos}{458}{62.74187}		\pgfmath@def{acos}{459}{62.67740}
\pgfmath@def{acos}{460}{62.61289}		\pgfmath@def{acos}{461}{62.54834}
\pgfmath@def{acos}{462}{62.48376}		\pgfmath@def{acos}{463}{62.41913}
\pgfmath@def{acos}{464}{62.35447}		\pgfmath@def{acos}{465}{62.28977}
\pgfmath@def{acos}{466}{62.22504}		\pgfmath@def{acos}{467}{62.16026}
\pgfmath@def{acos}{468}{62.09545}		\pgfmath@def{acos}{469}{62.03059}
\pgfmath@def{acos}{470}{61.96570}		\pgfmath@def{acos}{471}{61.90077}
\pgfmath@def{acos}{472}{61.83580}		\pgfmath@def{acos}{473}{61.77079}
\pgfmath@def{acos}{474}{61.70574}		\pgfmath@def{acos}{475}{61.64065}
\pgfmath@def{acos}{476}{61.57552}		\pgfmath@def{acos}{477}{61.51035}
\pgfmath@def{acos}{478}{61.44514}		\pgfmath@def{acos}{479}{61.37988}
\pgfmath@def{acos}{480}{61.31459}		\pgfmath@def{acos}{481}{61.24926}
\pgfmath@def{acos}{482}{61.18389}		\pgfmath@def{acos}{483}{61.11847}
\pgfmath@def{acos}{484}{61.05302}		\pgfmath@def{acos}{485}{60.98752}
\pgfmath@def{acos}{486}{60.92199}		\pgfmath@def{acos}{487}{60.85641}
\pgfmath@def{acos}{488}{60.79078}		\pgfmath@def{acos}{489}{60.72512}
\pgfmath@def{acos}{490}{60.65941}		\pgfmath@def{acos}{491}{60.59367}
\pgfmath@def{acos}{492}{60.52787}		\pgfmath@def{acos}{493}{60.46204}
\pgfmath@def{acos}{494}{60.39616}		\pgfmath@def{acos}{495}{60.33025}
\pgfmath@def{acos}{496}{60.26428}		\pgfmath@def{acos}{497}{60.19828}
\pgfmath@def{acos}{498}{60.13223}		\pgfmath@def{acos}{499}{60.06613}
\pgfmath@def{acos}{500}{60.00000}		\pgfmath@def{acos}{501}{59.93381}
\pgfmath@def{acos}{502}{59.86759}		\pgfmath@def{acos}{503}{59.80132}
\pgfmath@def{acos}{504}{59.73500}		\pgfmath@def{acos}{505}{59.66864}
\pgfmath@def{acos}{506}{59.60224}		\pgfmath@def{acos}{507}{59.53579}
\pgfmath@def{acos}{508}{59.46929}		\pgfmath@def{acos}{509}{59.40275}
\pgfmath@def{acos}{510}{59.33617}		\pgfmath@def{acos}{511}{59.26953}
\pgfmath@def{acos}{512}{59.20285}		\pgfmath@def{acos}{513}{59.13613}
\pgfmath@def{acos}{514}{59.06936}		\pgfmath@def{acos}{515}{59.00254}
\pgfmath@def{acos}{516}{58.93568}		\pgfmath@def{acos}{517}{58.86876}
\pgfmath@def{acos}{518}{58.80180}		\pgfmath@def{acos}{519}{58.73480}
\pgfmath@def{acos}{520}{58.66774}		\pgfmath@def{acos}{521}{58.60064}
\pgfmath@def{acos}{522}{58.53349}		\pgfmath@def{acos}{523}{58.46629}
\pgfmath@def{acos}{524}{58.39905}		\pgfmath@def{acos}{525}{58.33175}
\pgfmath@def{acos}{526}{58.26441}		\pgfmath@def{acos}{527}{58.19702}
\pgfmath@def{acos}{528}{58.12957}		\pgfmath@def{acos}{529}{58.06208}
\pgfmath@def{acos}{530}{57.99454}		\pgfmath@def{acos}{531}{57.92695}
\pgfmath@def{acos}{532}{57.85931}		\pgfmath@def{acos}{533}{57.79162}
\pgfmath@def{acos}{534}{57.72388}		\pgfmath@def{acos}{535}{57.65608}
\pgfmath@def{acos}{536}{57.58824}		\pgfmath@def{acos}{537}{57.52035}
\pgfmath@def{acos}{538}{57.45240}		\pgfmath@def{acos}{539}{57.38441}
\pgfmath@def{acos}{540}{57.31636}		\pgfmath@def{acos}{541}{57.24826}
\pgfmath@def{acos}{542}{57.18010}		\pgfmath@def{acos}{543}{57.11190}
\pgfmath@def{acos}{544}{57.04364}		\pgfmath@def{acos}{545}{56.97533}
\pgfmath@def{acos}{546}{56.90697}		\pgfmath@def{acos}{547}{56.83855}
\pgfmath@def{acos}{548}{56.77008}		\pgfmath@def{acos}{549}{56.70156}
\pgfmath@def{acos}{550}{56.63298}		\pgfmath@def{acos}{551}{56.56435}
\pgfmath@def{acos}{552}{56.49567}		\pgfmath@def{acos}{553}{56.42693}
\pgfmath@def{acos}{554}{56.35813}		\pgfmath@def{acos}{555}{56.28928}
\pgfmath@def{acos}{556}{56.22038}		\pgfmath@def{acos}{557}{56.15141}
\pgfmath@def{acos}{558}{56.08240}		\pgfmath@def{acos}{559}{56.01333}
\pgfmath@def{acos}{560}{55.94420}		\pgfmath@def{acos}{561}{55.87501}
\pgfmath@def{acos}{562}{55.80577}		\pgfmath@def{acos}{563}{55.73647}
\pgfmath@def{acos}{564}{55.66712}		\pgfmath@def{acos}{565}{55.59770}
\pgfmath@def{acos}{566}{55.52823}		\pgfmath@def{acos}{567}{55.45871}
\pgfmath@def{acos}{568}{55.38912}		\pgfmath@def{acos}{569}{55.31947}
\pgfmath@def{acos}{570}{55.24977}		\pgfmath@def{acos}{571}{55.18001}
\pgfmath@def{acos}{572}{55.11019}		\pgfmath@def{acos}{573}{55.04030}
\pgfmath@def{acos}{574}{54.97036}		\pgfmath@def{acos}{575}{54.90036}
\pgfmath@def{acos}{576}{54.83030}		\pgfmath@def{acos}{577}{54.76018}
\pgfmath@def{acos}{578}{54.69000}		\pgfmath@def{acos}{579}{54.61976}
\pgfmath@def{acos}{580}{54.54945}		\pgfmath@def{acos}{581}{54.47909}
\pgfmath@def{acos}{582}{54.40866}		\pgfmath@def{acos}{583}{54.33817}
\pgfmath@def{acos}{584}{54.26762}		\pgfmath@def{acos}{585}{54.19701}
\pgfmath@def{acos}{586}{54.12633}		\pgfmath@def{acos}{587}{54.05559}
\pgfmath@def{acos}{588}{53.98479}		\pgfmath@def{acos}{589}{53.91392}
\pgfmath@def{acos}{590}{53.84299}		\pgfmath@def{acos}{591}{53.77199}
\pgfmath@def{acos}{592}{53.70093}		\pgfmath@def{acos}{593}{53.62981}
\pgfmath@def{acos}{594}{53.55862}		\pgfmath@def{acos}{595}{53.48736}
\pgfmath@def{acos}{596}{53.41604}		\pgfmath@def{acos}{597}{53.34466}
\pgfmath@def{acos}{598}{53.27320}		\pgfmath@def{acos}{599}{53.20168}
\pgfmath@def{acos}{600}{53.13010}		\pgfmath@def{acos}{601}{53.05844}
\pgfmath@def{acos}{602}{52.98672}		\pgfmath@def{acos}{603}{52.91494}
\pgfmath@def{acos}{604}{52.84308}		\pgfmath@def{acos}{605}{52.77115}
\pgfmath@def{acos}{606}{52.69916}		\pgfmath@def{acos}{607}{52.62710}
\pgfmath@def{acos}{608}{52.55497}		\pgfmath@def{acos}{609}{52.48276}
\pgfmath@def{acos}{610}{52.41049}		\pgfmath@def{acos}{611}{52.33815}
\pgfmath@def{acos}{612}{52.26574}		\pgfmath@def{acos}{613}{52.19326}
\pgfmath@def{acos}{614}{52.12070}		\pgfmath@def{acos}{615}{52.04808}
\pgfmath@def{acos}{616}{51.97538}		\pgfmath@def{acos}{617}{51.90261}
\pgfmath@def{acos}{618}{51.82976}		\pgfmath@def{acos}{619}{51.75685}
\pgfmath@def{acos}{620}{51.68386}		\pgfmath@def{acos}{621}{51.61080}
\pgfmath@def{acos}{622}{51.53766}		\pgfmath@def{acos}{623}{51.46445}
\pgfmath@def{acos}{624}{51.39117}		\pgfmath@def{acos}{625}{51.31781}
\pgfmath@def{acos}{626}{51.24437}		\pgfmath@def{acos}{627}{51.17086}
\pgfmath@def{acos}{628}{51.09728}		\pgfmath@def{acos}{629}{51.02361}
\pgfmath@def{acos}{630}{50.94987}		\pgfmath@def{acos}{631}{50.87606}
\pgfmath@def{acos}{632}{50.80216}		\pgfmath@def{acos}{633}{50.72819}
\pgfmath@def{acos}{634}{50.65414}		\pgfmath@def{acos}{635}{50.58001}
\pgfmath@def{acos}{636}{50.50580}		\pgfmath@def{acos}{637}{50.43152}
\pgfmath@def{acos}{638}{50.35715}		\pgfmath@def{acos}{639}{50.28270}
\pgfmath@def{acos}{640}{50.20818}		\pgfmath@def{acos}{641}{50.13357}
\pgfmath@def{acos}{642}{50.05888}		\pgfmath@def{acos}{643}{49.98411}
\pgfmath@def{acos}{644}{49.90926}		\pgfmath@def{acos}{645}{49.83432}
\pgfmath@def{acos}{646}{49.75930}		\pgfmath@def{acos}{647}{49.68420}
\pgfmath@def{acos}{648}{49.60902}		\pgfmath@def{acos}{649}{49.53375}
\pgfmath@def{acos}{650}{49.45839}		\pgfmath@def{acos}{651}{49.38296}
\pgfmath@def{acos}{652}{49.30743}		\pgfmath@def{acos}{653}{49.23182}
\pgfmath@def{acos}{654}{49.15613}		\pgfmath@def{acos}{655}{49.08035}
\pgfmath@def{acos}{656}{49.00448}		\pgfmath@def{acos}{657}{48.92852}
\pgfmath@def{acos}{658}{48.85248}		\pgfmath@def{acos}{659}{48.77634}
\pgfmath@def{acos}{660}{48.70012}		\pgfmath@def{acos}{661}{48.62381}
\pgfmath@def{acos}{662}{48.54741}		\pgfmath@def{acos}{663}{48.47092}
\pgfmath@def{acos}{664}{48.39434}		\pgfmath@def{acos}{665}{48.31767}
\pgfmath@def{acos}{666}{48.24091}		\pgfmath@def{acos}{667}{48.16405}
\pgfmath@def{acos}{668}{48.08710}		\pgfmath@def{acos}{669}{48.01006}
\pgfmath@def{acos}{670}{47.93293}		\pgfmath@def{acos}{671}{47.85570}
\pgfmath@def{acos}{672}{47.77838}		\pgfmath@def{acos}{673}{47.70096}
\pgfmath@def{acos}{674}{47.62345}		\pgfmath@def{acos}{675}{47.54585}
\pgfmath@def{acos}{676}{47.46814}		\pgfmath@def{acos}{677}{47.39034}
\pgfmath@def{acos}{678}{47.31244}		\pgfmath@def{acos}{679}{47.23445}
\pgfmath@def{acos}{680}{47.15635}		\pgfmath@def{acos}{681}{47.07816}
\pgfmath@def{acos}{682}{46.99987}		\pgfmath@def{acos}{683}{46.92147}
\pgfmath@def{acos}{684}{46.84298}		\pgfmath@def{acos}{685}{46.76439}
\pgfmath@def{acos}{686}{46.68569}		\pgfmath@def{acos}{687}{46.60690}
\pgfmath@def{acos}{688}{46.52800}		\pgfmath@def{acos}{689}{46.44899}
\pgfmath@def{acos}{690}{46.36989}		\pgfmath@def{acos}{691}{46.29068}
\pgfmath@def{acos}{692}{46.21136}		\pgfmath@def{acos}{693}{46.13194}
\pgfmath@def{acos}{694}{46.05241}		\pgfmath@def{acos}{695}{45.97278}
\pgfmath@def{acos}{696}{45.89304}		\pgfmath@def{acos}{697}{45.81319}
\pgfmath@def{acos}{698}{45.73323}		\pgfmath@def{acos}{699}{45.65317}
\pgfmath@def{acos}{700}{45.57299}		\pgfmath@def{acos}{701}{45.49271}
\pgfmath@def{acos}{702}{45.41231}		\pgfmath@def{acos}{703}{45.33180}
\pgfmath@def{acos}{704}{45.25118}		\pgfmath@def{acos}{705}{45.17045}
\pgfmath@def{acos}{706}{45.08961}		\pgfmath@def{acos}{707}{45.00865}
\pgfmath@def{acos}{708}{44.92757}		\pgfmath@def{acos}{709}{44.84638}
\pgfmath@def{acos}{710}{44.76508}		\pgfmath@def{acos}{711}{44.68366}
\pgfmath@def{acos}{712}{44.60212}		\pgfmath@def{acos}{713}{44.52046}
\pgfmath@def{acos}{714}{44.43869}		\pgfmath@def{acos}{715}{44.35680}
\pgfmath@def{acos}{716}{44.27478}		\pgfmath@def{acos}{717}{44.19265}
\pgfmath@def{acos}{718}{44.11039}		\pgfmath@def{acos}{719}{44.02802}
\pgfmath@def{acos}{720}{43.94552}		\pgfmath@def{acos}{721}{43.86289}
\pgfmath@def{acos}{722}{43.78014}		\pgfmath@def{acos}{723}{43.69727}
\pgfmath@def{acos}{724}{43.61427}		\pgfmath@def{acos}{725}{43.53115}
\pgfmath@def{acos}{726}{43.44790}		\pgfmath@def{acos}{727}{43.36452}
\pgfmath@def{acos}{728}{43.28101}		\pgfmath@def{acos}{729}{43.19737}
\pgfmath@def{acos}{730}{43.11360}		\pgfmath@def{acos}{731}{43.02970}
\pgfmath@def{acos}{732}{42.94567}		\pgfmath@def{acos}{733}{42.86151}
\pgfmath@def{acos}{734}{42.77721}		\pgfmath@def{acos}{735}{42.69278}
\pgfmath@def{acos}{736}{42.60821}		\pgfmath@def{acos}{737}{42.52351}
\pgfmath@def{acos}{738}{42.43867}		\pgfmath@def{acos}{739}{42.35370}
\pgfmath@def{acos}{740}{42.26858}		\pgfmath@def{acos}{741}{42.18333}
\pgfmath@def{acos}{742}{42.09793}		\pgfmath@def{acos}{743}{42.01240}
\pgfmath@def{acos}{744}{41.92672}		\pgfmath@def{acos}{745}{41.84090}
\pgfmath@def{acos}{746}{41.75493}		\pgfmath@def{acos}{747}{41.66882}
\pgfmath@def{acos}{748}{41.58257}		\pgfmath@def{acos}{749}{41.49617}
\pgfmath@def{acos}{750}{41.40962}		\pgfmath@def{acos}{751}{41.32292}
\pgfmath@def{acos}{752}{41.23607}		\pgfmath@def{acos}{753}{41.14908}
\pgfmath@def{acos}{754}{41.06193}		\pgfmath@def{acos}{755}{40.97463}
\pgfmath@def{acos}{756}{40.88717}		\pgfmath@def{acos}{757}{40.79956}
\pgfmath@def{acos}{758}{40.71180}		\pgfmath@def{acos}{759}{40.62388}
\pgfmath@def{acos}{760}{40.53580}		\pgfmath@def{acos}{761}{40.44756}
\pgfmath@def{acos}{762}{40.35916}		\pgfmath@def{acos}{763}{40.27061}
\pgfmath@def{acos}{764}{40.18189}		\pgfmath@def{acos}{765}{40.09300}
\pgfmath@def{acos}{766}{40.00396}		\pgfmath@def{acos}{767}{39.91474}
\pgfmath@def{acos}{768}{39.82537}		\pgfmath@def{acos}{769}{39.73582}
\pgfmath@def{acos}{770}{39.64611}		\pgfmath@def{acos}{771}{39.55622}
\pgfmath@def{acos}{772}{39.46617}		\pgfmath@def{acos}{773}{39.37594}
\pgfmath@def{acos}{774}{39.28554}		\pgfmath@def{acos}{775}{39.19496}
\pgfmath@def{acos}{776}{39.10421}		\pgfmath@def{acos}{777}{39.01328}
\pgfmath@def{acos}{778}{38.92218}		\pgfmath@def{acos}{779}{38.83089}
\pgfmath@def{acos}{780}{38.73942}		\pgfmath@def{acos}{781}{38.64777}
\pgfmath@def{acos}{782}{38.55594}		\pgfmath@def{acos}{783}{38.46392}
\pgfmath@def{acos}{784}{38.37171}		\pgfmath@def{acos}{785}{38.27932}
\pgfmath@def{acos}{786}{38.18673}		\pgfmath@def{acos}{787}{38.09396}
\pgfmath@def{acos}{788}{38.00100}		\pgfmath@def{acos}{789}{37.90784}
\pgfmath@def{acos}{790}{37.81448}		\pgfmath@def{acos}{791}{37.72093}
\pgfmath@def{acos}{792}{37.62719}		\pgfmath@def{acos}{793}{37.53324}
\pgfmath@def{acos}{794}{37.43909}		\pgfmath@def{acos}{795}{37.34474}
\pgfmath@def{acos}{796}{37.25019}		\pgfmath@def{acos}{797}{37.15542}
\pgfmath@def{acos}{798}{37.06046}		\pgfmath@def{acos}{799}{36.96528}
\pgfmath@def{acos}{800}{36.86989}		\pgfmath@def{acos}{801}{36.77429}
\pgfmath@def{acos}{802}{36.67848}		\pgfmath@def{acos}{803}{36.58245}
\pgfmath@def{acos}{804}{36.48621}		\pgfmath@def{acos}{805}{36.38974}
\pgfmath@def{acos}{806}{36.29305}		\pgfmath@def{acos}{807}{36.19615}
\pgfmath@def{acos}{808}{36.09901}		\pgfmath@def{acos}{809}{36.00165}
\pgfmath@def{acos}{810}{35.90406}		\pgfmath@def{acos}{811}{35.80625}
\pgfmath@def{acos}{812}{35.70820}		\pgfmath@def{acos}{813}{35.60991}
\pgfmath@def{acos}{814}{35.51139}		\pgfmath@def{acos}{815}{35.41263}
\pgfmath@def{acos}{816}{35.31364}		\pgfmath@def{acos}{817}{35.21440}
\pgfmath@def{acos}{818}{35.11491}		\pgfmath@def{acos}{819}{35.01518}
\pgfmath@def{acos}{820}{34.91520}		\pgfmath@def{acos}{821}{34.81497}
\pgfmath@def{acos}{822}{34.71449}		\pgfmath@def{acos}{823}{34.61375}
\pgfmath@def{acos}{824}{34.51276}		\pgfmath@def{acos}{825}{34.41150}
\pgfmath@def{acos}{826}{34.30999}		\pgfmath@def{acos}{827}{34.20821}
\pgfmath@def{acos}{828}{34.10616}		\pgfmath@def{acos}{829}{34.00385}
\pgfmath@def{acos}{830}{33.90126}		\pgfmath@def{acos}{831}{33.79840}
\pgfmath@def{acos}{832}{33.69526}		\pgfmath@def{acos}{833}{33.59184}
\pgfmath@def{acos}{834}{33.48814}		\pgfmath@def{acos}{835}{33.38416}
\pgfmath@def{acos}{836}{33.27989}		\pgfmath@def{acos}{837}{33.17533}
\pgfmath@def{acos}{838}{33.07047}		\pgfmath@def{acos}{839}{32.96532}
\pgfmath@def{acos}{840}{32.85988}		\pgfmath@def{acos}{841}{32.75413}
\pgfmath@def{acos}{842}{32.64807}		\pgfmath@def{acos}{843}{32.54171}
\pgfmath@def{acos}{844}{32.43504}		\pgfmath@def{acos}{845}{32.32806}
\pgfmath@def{acos}{846}{32.22076}		\pgfmath@def{acos}{847}{32.11314}
\pgfmath@def{acos}{848}{32.00520}		\pgfmath@def{acos}{849}{31.89693}
\pgfmath@def{acos}{850}{31.78833}		\pgfmath@def{acos}{851}{31.67939}
\pgfmath@def{acos}{852}{31.57012}		\pgfmath@def{acos}{853}{31.46051}
\pgfmath@def{acos}{854}{31.35056}		\pgfmath@def{acos}{855}{31.24026}
\pgfmath@def{acos}{856}{31.12961}		\pgfmath@def{acos}{857}{31.01860}
\pgfmath@def{acos}{858}{30.90724}		\pgfmath@def{acos}{859}{30.79551}
\pgfmath@def{acos}{860}{30.68341}		\pgfmath@def{acos}{861}{30.57095}
\pgfmath@def{acos}{862}{30.45811}		\pgfmath@def{acos}{863}{30.34489}
\pgfmath@def{acos}{864}{30.23128}		\pgfmath@def{acos}{865}{30.11729}
\pgfmath@def{acos}{866}{30.00291}		\pgfmath@def{acos}{867}{29.88813}
\pgfmath@def{acos}{868}{29.77294}		\pgfmath@def{acos}{869}{29.65736}
\pgfmath@def{acos}{870}{29.54136}		\pgfmath@def{acos}{871}{29.42494}
\pgfmath@def{acos}{872}{29.30810}		\pgfmath@def{acos}{873}{29.19084}
\pgfmath@def{acos}{874}{29.07315}		\pgfmath@def{acos}{875}{28.95502}
\pgfmath@def{acos}{876}{28.83645}		\pgfmath@def{acos}{877}{28.71743}
\pgfmath@def{acos}{878}{28.59796}		\pgfmath@def{acos}{879}{28.47803}
\pgfmath@def{acos}{880}{28.35763}		\pgfmath@def{acos}{881}{28.23677}
\pgfmath@def{acos}{882}{28.11542}		\pgfmath@def{acos}{883}{27.99360}
\pgfmath@def{acos}{884}{27.87128}		\pgfmath@def{acos}{885}{27.74847}
\pgfmath@def{acos}{886}{27.62516}		\pgfmath@def{acos}{887}{27.50134}
\pgfmath@def{acos}{888}{27.37700}		\pgfmath@def{acos}{889}{27.25214}
\pgfmath@def{acos}{890}{27.12675}		\pgfmath@def{acos}{891}{27.00082}
\pgfmath@def{acos}{892}{26.87434}		\pgfmath@def{acos}{893}{26.74731}
\pgfmath@def{acos}{894}{26.61973}		\pgfmath@def{acos}{895}{26.49157}
\pgfmath@def{acos}{896}{26.36283}		\pgfmath@def{acos}{897}{26.23351}
\pgfmath@def{acos}{898}{26.10359}		\pgfmath@def{acos}{899}{25.97306}
\pgfmath@def{acos}{900}{25.84193}		\pgfmath@def{acos}{901}{25.71017}
\pgfmath@def{acos}{902}{25.57778}		\pgfmath@def{acos}{903}{25.44475}
\pgfmath@def{acos}{904}{25.31106}		\pgfmath@def{acos}{905}{25.17671}
\pgfmath@def{acos}{906}{25.04169}		\pgfmath@def{acos}{907}{24.90598}
\pgfmath@def{acos}{908}{24.76958}		\pgfmath@def{acos}{909}{24.63247}
\pgfmath@def{acos}{910}{24.49464}		\pgfmath@def{acos}{911}{24.35608}
\pgfmath@def{acos}{912}{24.21678}		\pgfmath@def{acos}{913}{24.07672}
\pgfmath@def{acos}{914}{23.93588}		\pgfmath@def{acos}{915}{23.79427}
\pgfmath@def{acos}{916}{23.65185}		\pgfmath@def{acos}{917}{23.50863}
\pgfmath@def{acos}{918}{23.36457}		\pgfmath@def{acos}{919}{23.21967}
\pgfmath@def{acos}{920}{23.07391}		\pgfmath@def{acos}{921}{22.92728}
\pgfmath@def{acos}{922}{22.77975}		\pgfmath@def{acos}{923}{22.63132}
\pgfmath@def{acos}{924}{22.48195}		\pgfmath@def{acos}{925}{22.33164}
\pgfmath@def{acos}{926}{22.18036}		\pgfmath@def{acos}{927}{22.02810}
\pgfmath@def{acos}{928}{21.87483}		\pgfmath@def{acos}{929}{21.72053}
\pgfmath@def{acos}{930}{21.56518}		\pgfmath@def{acos}{931}{21.40876}
\pgfmath@def{acos}{932}{21.25124}		\pgfmath@def{acos}{933}{21.09260}
\pgfmath@def{acos}{934}{20.93281}		\pgfmath@def{acos}{935}{20.77185}
\pgfmath@def{acos}{936}{20.60969}		\pgfmath@def{acos}{937}{20.44630}
\pgfmath@def{acos}{938}{20.28165}		\pgfmath@def{acos}{939}{20.11570}
\pgfmath@def{acos}{940}{19.94844}		\pgfmath@def{acos}{941}{19.77982}
\pgfmath@def{acos}{942}{19.60981}		\pgfmath@def{acos}{943}{19.43837}
\pgfmath@def{acos}{944}{19.26546}		\pgfmath@def{acos}{945}{19.09105}
\pgfmath@def{acos}{946}{18.91509}		\pgfmath@def{acos}{947}{18.73754}
\pgfmath@def{acos}{948}{18.55835}		\pgfmath@def{acos}{949}{18.37748}
\pgfmath@def{acos}{950}{18.19487}		\pgfmath@def{acos}{951}{18.01047}
\pgfmath@def{acos}{952}{17.82423}		\pgfmath@def{acos}{953}{17.63609}
\pgfmath@def{acos}{954}{17.44598}		\pgfmath@def{acos}{955}{17.25385}
\pgfmath@def{acos}{956}{17.05962}		\pgfmath@def{acos}{957}{16.86322}
\pgfmath@def{acos}{958}{16.66457}		\pgfmath@def{acos}{959}{16.46359}
\pgfmath@def{acos}{960}{16.26020}		\pgfmath@def{acos}{961}{16.05430}
\pgfmath@def{acos}{962}{15.84580}		\pgfmath@def{acos}{963}{15.63459}
\pgfmath@def{acos}{964}{15.42057}		\pgfmath@def{acos}{965}{15.20360}
\pgfmath@def{acos}{966}{14.98357}		\pgfmath@def{acos}{967}{14.76033}
\pgfmath@def{acos}{968}{14.53374}		\pgfmath@def{acos}{969}{14.30364}
\pgfmath@def{acos}{970}{14.06986}		\pgfmath@def{acos}{971}{13.83221}
\pgfmath@def{acos}{972}{13.59049}		\pgfmath@def{acos}{973}{13.34447}
\pgfmath@def{acos}{974}{13.09392}		\pgfmath@def{acos}{975}{12.83856}
\pgfmath@def{acos}{976}{12.57811}		\pgfmath@def{acos}{977}{12.31225}
\pgfmath@def{acos}{978}{12.04060}		\pgfmath@def{acos}{979}{11.76278}
\pgfmath@def{acos}{980}{11.47834}		\pgfmath@def{acos}{981}{11.18676}
\pgfmath@def{acos}{982}{10.88748}		\pgfmath@def{acos}{983}{10.57984}
\pgfmath@def{acos}{984}{10.26309}		\pgfmath@def{acos}{985}{9.93636}
\pgfmath@def{acos}{986}{9.59863}		\pgfmath@def{acos}{987}{9.24870}
\pgfmath@def{acos}{988}{8.88512}		\pgfmath@def{acos}{989}{8.50614}
\pgfmath@def{acos}{990}{8.10961}		\pgfmath@def{acos}{991}{7.69281}
\pgfmath@def{acos}{992}{7.25224}		\pgfmath@def{acos}{993}{6.78328}
\pgfmath@def{acos}{994}{6.27958}		\pgfmath@def{acos}{995}{5.73196}
\pgfmath@def{acos}{996}{5.12640}		\pgfmath@def{acos}{997}{4.43922}
\pgfmath@def{acos}{998}{3.62430}		\pgfmath@def{acos}{999}{2.56255}
\pgfmath@def{acos}{1000}{0.00000}% Load the trig. stuff.
\input{pgfmathrnd.code.tex}%  Load the random stuff.

% \pgfmathadd
%
% Add #1 and #2.
%
\def\pgfmathadd#1#2{%
	\pgfmathparse{#1}\let\pgfmath@adda=\pgfmathresult%
	\pgfmathparse{#2}\let\pgfmath@addb=\pgfmathresult%
	\pgfmathadd@{\pgfmath@adda}{\pgfmath@addb}}
\def\pgfmathadd@#1#2{%
	\begingroup%
		\pgfmath@x=#1pt\relax%
		\pgfmath@y=#2pt\relax%
		\advance\pgfmath@x by\pgfmath@y%
		\pgfmath@returnone\pgfmath@x%
	\endgroup%
}

% \pgfmathsubtract
%
% Subtract #2 from #1.
%
\def\pgfmathsubtract#1#2{%
	\pgfmathparse{#1}\let\pgfmath@subtracta=\pgfmathresult%
	\pgfmathparse{#2}\let\pgfmath@subtractb=\pgfmathresult%
	\pgfmathsubtract@{\pgfmath@subtracta}{\pgfmath@subtractb}}

\def\pgfmathsubtract@#1#2{%
	\begingroup%
		\pgfmath@x=#1pt\relax%
		\pgfmath@y=#2pt\relax%
		\advance\pgfmath@x by-\pgfmath@y%
		\pgfmath@returnone\pgfmath@x%
	\endgroup%
}

% \pgfmathmultiply
%
% Multiply #1 by #2.
%
\def\pgfmathmultiply#1#2{%
	\pgfmathparse{#1}\let\pgfmath@multiplya=\pgfmathresult%
	\pgfmathparse{#2}\let\pgfmath@multiplyb=\pgfmathresult%
	\pgfmathmultiply@{\pgfmath@multiplya}{\pgfmath@multiplyb}}
\def\pgfmathmultiply@#1#2{%
	\begingroup%
		\pgfmath@x=#1pt\relax%
		\pgfmath@x=#2\pgfmath@x%
		\pgfmath@returnone\pgfmath@x%
	\endgroup%
}

% \pgfmathdivide
%
% Divide #1 by #2.
%
\def\pgfmathdivide#1#2{%
	\pgfmathparse{#1}\let\pgfmath@dividea=\pgfmathresult%
	\pgfmathparse{#2}\let\pgfmath@divideb=\pgfmathresult%
	\pgfmathdivide@{\pgfmath@dividea}{\pgfmath@divideb}}
\def\pgfmathdivide@#1#2{%
  \begingroup%
    \pgfmath@x#1pt\relax%
    \pgfmath@ifinteger{#2}{%
      \afterassignment\pgfmath@gobbletilpgfmath@%
      \c@pgfmath@counta#2\relax\pgfmath@%
      \divide\pgfmath@x\c@pgfmath@counta}{%
      \pgfmathreciprocal@{#2}%
      \pgfmath@x=\pgfmathresult\pgfmath@x%
    }%
    \pgfmath@returnone\pgfmath@x%
  \endgroup%
}

% \pgfmathgreaterthan
%
% 1.0 if #1 > #2. Otherwise 0.0
%
\def\pgfmathgreaterthan#1#2{%
	\pgfmathparse{#1}\let\pgfmath@greaterthana=\pgfmathresult%
	\pgfmathparse{#2}\let\pgfmath@greaterthanb=\pgfmathresult%
	\pgfmathgreaterthan@{\pgfmath@greaterthana}{\pgfmath@greaterthanb}}
\def\pgfmathgreaterthan@#1#2{%
	\begingroup%
		\pgfmath@x#1pt\relax%
		\advance\pgfmath@x-#2pt\relax%
		\ifdim\pgfmath@x>0pt\relax%
			\pgfmath@x1pt\relax%
		\else%
			\pgfmath@x0pt\relax%
		\fi%
		\pgfmath@returnone\pgfmath@x%
	\endgroup%
}

% \pgfmathlessthan
%
% 1.0 if #1< #2. Otherwise 0.0
%
\def\pgfmathlessthan#1#2{%
	\pgfmathparse{#1}\let\pgfmath@lessthana=\pgfmathresult%
	\pgfmathparse{#2}\let\pgfmath@lessthanb=\pgfmathresult%
	\pgfmathlessthan@{\pgfmath@lessthana}{\pgfmath@lessthanb}}
\def\pgfmathlessthan@#1#2{%
	\begingroup%
		\pgfmath@x#1pt\relax%
		\advance\pgfmath@x-#2pt\relax%
		\ifdim\pgfmath@x<0pt\relax%
			\pgfmath@x1pt\relax%
		\else%
			\pgfmath@x0pt\relax%
		\fi%
		\pgfmath@returnone\pgfmath@x%
	\endgroup%
}

% \pgfmathequalto
%
% 1.0 if #1 = #2. Otherwise 0.0
%
\def\pgfmathequalto#1#2{%
	\pgfmathparse{#1}\let\pgfmath@equaltoa=\pgfmathresult%
	\pgfmathparse{#2}\let\pgfmath@equaltob=\pgfmathresult%
	\pgfmathadd@{\pgfmath@equaltoa}{\pgfmath@equaltob}}
\def\pgfmathequalto@#1#2{%
	\begingroup%
		\pgfmath@x#1pt\relax%
		\advance\pgfmath@x-#2pt\relax%
		\ifdim\pgfmath@x=0pt\relax%
			\pgfmath@x1pt\relax%
		\else%
			\pgfmath@x0pt\relax%
		\fi%
		\pgfmath@returnone\pgfmath@x%
	\endgroup%
}

% \pgfmathreciprocal
%
% 1 / #1
%
\def\pgfmathreciprocal#1{%
	\pgfmathparse{#1}%
	\pgfmathreciprocal@{\pgfmathresult}}
\def\pgfmathreciprocal@#1{%
	\begingroup%
		\pgfmath@x=#1pt\relax%
		\ifdim\pgfmath@x=0pt\relax%
			\pgfmath@error{Division by zero}{}%
		\fi%
		\edef\pgfmath@temp{\pgfmath@tonumber{\pgfmath@x}}%
		% If #1 is an integer, no fancy approximation is needed.
		\pgfmath@ifinteger{\pgfmath@temp}{%
			\afterassignment\pgfmath@gobbletilpgfmath@%
			\c@pgfmath@counta=\pgfmath@temp\relax\pgfmath@%
			\pgfmath@x=1pt%
			\divide\pgfmath@x by\c@pgfmath@counta}{%
				% Oh well, here goes...
				%
				% ...approximate root...
				\c@pgfmath@counta=\pgfmath@x\relax%
				\divide\c@pgfmath@counta by65535\relax%
				\ifnum\c@pgfmath@counta=0\relax%
					\c@pgfmath@counta=1\relax%
				\fi%
				\pgfmath@x=1pt\relax%
				\divide\pgfmath@x by\c@pgfmath@counta\relax%
				% ...and now iterate..
				%
				\pgfmath@xb=\pgfmath@x%
				\pgfmathloop%
					\pgfmath@xa=-\pgfmath@temp\pgfmath@x\relax%
					\advance\pgfmath@xa by2pt\relax%
					\pgfmath@xa=\pgfmath@tonumber{\pgfmath@x}\pgfmath@xa%
					\pgfmath@x=\pgfmath@xa%
					% If the the root is the same as the last root...
					\ifdim\pgfmath@x=\pgfmath@xb% Stop!
					\else%
						\pgfmath@xb=\pgfmath@x%
					\repeatpgfmathloop}%
		\pgfmath@returnone\pgfmath@x%
	\endgroup%
}%

% \pgfmathabs
%
% Calculate |#1|
%
\def\pgfmathabs#1{%
	\pgfmathparse{#1}%
	\pgfmathabsolute@{\pgfmathresult}}
\def\pgfmathabs@#1{%
	\begingroup%
		\pgfmath@x=#1pt\relax%
		\ifdim\pgfmath@x<0pt\relax%
			\pgfmath@x=-\pgfmath@x%
		\fi%
	\pgfmath@returnone\pgfmath@x%
	\endgroup%
}

% \pgfmathmod
%
% Calculate #1 mod #2.
%
\def\pgfmathmod#1#2{%
	\pgfmathparse{#1}\edef\pgfmath@moda{\pgfmathresult}%
	\pgfmathparse{#2}\edef\pgfmath@modb{\pgfmathresult}%
	\pgfmathmod@{\pgfmath@mod@a}{\pgfmath@modb}%
}
\def\pgfmathmod@#1#2{%
	\begingroup%
		\pgfmath@x#1pt\relax%
		\pgfmath@xa\pgfmath@x%
		\pgfmath@xb#2pt\relax%
		\c@pgfmath@counta=\pgfmath@xa%
		\c@pgfmath@countb=\pgfmath@xb%
		\divide\c@pgfmath@counta\c@pgfmath@countb%
		\multiply\pgfmath@xb\c@pgfmath@counta%
		\advance\pgfmath@x-\pgfmath@xb%
		\pgfmath@returnone\pgfmath@x%
	\endgroup%
}

% \pgfmathsqrt
%
% Square-root of #1.
%
%
\def\pgfmathsqrt#1{%
	\pgfmathparse{#1}%
	\pgfmathsqrt@{\pgfmathresult}}
\def\pgfmathsqrt@#1{%
	\begingroup%
		\pgfmath@x=#1pt\relax%
		\pgfmath@x=.01\pgfmath@x%
		\pgfmath@xa=\pgfmath@x%
		\pgfmath@xb=\pgfmath@x%
		\pgfmathloop
			\pgfmath@xc=\pgfmath@x%
			% If pgfmath@x >= 128pt, we get an Arithmetic overflow, so...
			% If x^2 >= 16384 then 16384/x < x
			\pgfmath@y=16383.99999pt\relax%
			\c@pgfmath@counta=\pgfmath@x%
			\divide\c@pgfmath@counta by655360\relax% Can't remember why we need the extra zero.
			\ifnum\c@pgfmath@counta=0\relax%
				\c@pgfmath@counta=1\relax%
			\fi%
			\divide\pgfmath@y by\c@pgfmath@counta%
			\ifdim\pgfmath@y<\pgfmath@x%
			\else%
				\pgfmath@x=\pgfmath@tonumber{\pgfmath@x}\pgfmath@x%
				\advance\pgfmath@x by-\pgfmath@xa\relax%
				\pgfmath@ya=\pgfmath@x%
				\pgfmathreciprocal@{\pgfmath@tonumber{\pgfmath@xc}}%
				\pgfmath@x=\pgfmathresult\pgfmath@ya%
				\pgfmath@x=-.5\pgfmath@x%
				\advance\pgfmath@x by\pgfmath@xb%
			\fi%
			% If the new root equals the old root, stop.
			\ifdim\pgfmath@x=\pgfmath@xb%
			\else%
				\pgfmath@xb=\pgfmath@x%
		\repeatpgfmathloop%
		\pgfmath@x=10.0\pgfmath@x%
		\pgfmath@returnone\pgfmath@x%
	\endgroup%
}

% \pgfmathpower
%
% Calculates #1 ^ #2
%
% #2 is expected to be an integer.
%
\def\pgfmathpower#1#2{%
	\pgfmathparse{#1}\let\pgfmath@powera=\pgfmathresult%
	\pgfmathparse{#2}\let\pgfmath@powerb=\pgfmathresult%
	\pgfmathpower@{\pgfmath@powera}{\pgfmath@powerb}}
\def\pgfmathpower@#1#2{%
	\begingroup%
		\pgfmath@xa#1pt\relax%
		\pgfmath@xb#2pt\relax%
		% If #2 is negative, take the reciprocal of #1
		% and the absolute value of #2, and carry on.
		%
		\ifdim\pgfmath@xb<0pt\relax%
			\pgfmath@xb-\pgfmath@xb%
			\pgfmathreciprocal@{\pgfmath@tonumber{\pgfmath@xa}}%
			\pgfmath@xa\pgfmathresult pt\relax%
		\fi%
		\afterassignment\pgfmath@gobbletilpgfmath@%
		\expandafter\c@pgfmath@counta\expandafter=\the\pgfmath@xb\relax\pgfmath@% Gobble decimal place.
		\pgfmath@x=1pt\relax%
		\pgfmathloop%
			\ifnum\c@pgfmath@counta>0\relax%
				\ifodd\c@pgfmath@counta%
					\pgfmath@x=\pgfmath@tonumber{\pgfmath@x}\pgfmath@xa%
				\fi
				% If \pgfmath@xa*\pgfmath@xa >=16384pt we get an error, so...		
				\pgfmath@xc=16383.99999pt\relax%
				\pgfmath@y=\pgfmath@xa%
				\ifdim\pgfmath@y<0pt\relax%
					\pgfmath@y=-\pgfmath@y%
				\fi%
				\c@pgfmath@countb=\pgfmath@y%
				\divide\c@pgfmath@countb by 65536\relax%
				\ifnum\c@pgfmath@countb=0\relax%
					\c@pgfmath@countb=1\relax%
				\fi%
				\divide\pgfmath@xc by\c@pgfmath@countb%
				\ifdim\pgfmath@xc<\pgfmath@y%
					\c@pgfmath@counta=0\relax%
				\else%
					\pgfmath@xa=\pgfmath@tonumber{\pgfmath@xa}\pgfmath@xa%
					\divide\c@pgfmath@counta by 2\relax%
				\fi%
		\repeatpgfmathloop%
		\pgfmath@returnone\pgfmath@x%
	\endgroup%
}	


% \pgfmathround
% 
% Half-up rounding.
%
\def\pgfmathround#1{%
	\pgfmathparse{#1}%
	\pgfmathround@{\pgfmathresult}}
\def\pgfmathround@#1{%
	\begingroup%
		\pgfmath@x#1pt\relax%
		\afterassignment\pgfmath@gobbletilpgfmath@%
		\expandafter\c@pgfmath@counta\the\pgfmath@x\relax\pgfmath@%
		\pgfmath@y\pgfmath@x%
		\advance\pgfmath@y-\c@pgfmath@counta pt\relax%
		\pgfmath@x\c@pgfmath@counta pt\relax%
		\ifdim\pgfmath@x<0pt\relax%
			\advance\pgfmath@x-1pt\relax%
		\fi%
		\ifdim\pgfmath@y<0.5pt\relax%
		\else%
			\advance\pgfmath@x1pt\relax%
		\fi%
		\pgfmath@returnone\pgfmath@x%
	\endgroup%
}%

% \pgfmathfloor
% 
% Floor function.
%
\def\pgfmathfloor#1{%
	\pgfmathparse{#1}%
	\expandafter\pgfmathfloor@\expandafter{\pgfmathresult}}
\def\pgfmathfloor@#1{%
	\begingroup%
		\pgfmath@x#1pt\relax%
		\afterassignment\pgfmath@gobbletilpgfmath@%
		\expandafter\c@pgfmath@counta\the\pgfmath@x\relax\pgfmath@%
		\pgfmath@x\c@pgfmath@counta pt\relax%
		\pgfmath@returnone\pgfmath@x%
	\endgroup
}%

% \pgfmathceil
% 
% Ceiling function.
%
\def\pgfmathceil#1{%
	\pgfmathparse{#1}%
	\expandafter\pgfmathceil@\expandafter{\pgfmathresult}}
\def\pgfmathceil@#1{%
	\begingroup%
		\pgfmath@x#1pt\relax%
		\afterassignment\pgfmath@gobbletilpgfmath@%
		\expandafter\c@pgfmath@counta\the\pgfmath@x\relax\pgfmath@%
		\pgfmath@y\pgfmath@x%
		\advance\pgfmath@y-\c@pgfmath@counta pt\relax%
		\pgfmath@x\c@pgfmath@counta pt\relax%
		\ifdim\pgfmath@y>0pt\relax%
			\advance\pgfmath@x1pt\relax%
		\fi%
	\pgfmath@returnone\pgfmath@x%
	\endgroup%
}%

% \pgfmathexp
%
% A Maclaurens expansion for e^#1.
% 0 <= #1 < ln(16384).
%
\def\pgfmathexp#1{%
	\pgfmathparse{#1}%
	\expandafter\pgfmathexp@\expandafter{\pgfmathresult}}
\def\pgfmathexp@#1{%
	\begingroup%
		\pgfmath@x1pt\relax%
		\pgfmath@xa1pt\relax%
		\pgfmath@xb\pgfmath@x%
		\pgfmathloop%
			\pgfmath@xc\pgfmathcounter pt\relax%
			\c@pgfmath@counta\pgfmath@xc%
			\divide\c@pgfmath@counta65536\relax%
			\pgfmath@xc1pt\relax%
			\divide\pgfmath@xc\c@pgfmath@counta%
			\pgfmath@xa\pgfmath@tonumber{\pgfmath@xc}\pgfmath@xa%
			\pgfmath@xa#1\pgfmath@xa%
			\advance\pgfmath@x\pgfmath@xa%
			\ifdim\pgfmath@x=\pgfmath@xb%
			\else%
				\pgfmath@xb\pgfmath@x%
		\repeatpgfmathloop%
	\pgfmath@returnone\pgfmath@x%
	\endgroup%
}



% \pgfmathvectorlength
%
% Calcluate the Eulidean length of a 2D vector.
%
% This based on polynomial approximation co-efficents
% contributed by Rouben Rostamian.
%
% #1 - the x component of the vector.
% #2 - the y component of the vector.
%
% P(x) = c0 + x^2 * (c1 + x^2 * (c2 + x^2 * ( c3 + c4 * x^2)))
\def\pgfmath@cE{-0.01019}
\def\pgfmath@cD{0.04453}
\def\pgfmath@cC{-0.11951}
\def\pgfmath@cB{0.49936}
\def\pgfmath@cA{1.00001}

\def\pgfmathveclen#1#2{%
	\pgfmathparse{#1}\let\pgfmath@vecx=\pgfmathresult%
	\pgfmathparse{#2}\let\pgfmath@vecy=\pgfmathresult%
	\pgfmathveclen@{\pgfmath@vecx}{\pgfmath@vecy}%
}
\def\pgfmathveclen@#1#2{%
	\begingroup%
		\pgfmath@x#1pt\relax%
		\pgfmath@y#2pt\relax%
		\ifdim\pgfmath@x>\pgfmath@y%
			\pgfmath@xa\pgfmath@x%
			\pgfmath@x\pgfmath@y%
			\pgfmath@y\pgfmath@xa%
		\fi%
		% We use a scaling factor to reduce errors.
		\ifdim\pgfmath@y>10000pt\relax%
			\c@pgfmath@counta1500\relax%
		\else%
			\ifdim\pgfmath@y>1000pt\relax%
				\c@pgfmath@counta150\relax%
			\else%
				\ifdim\pgfmath@y>100pt\relax%
					\c@pgfmath@counta50\relax%
				\else%
					\c@pgfmath@counta1\relax%
				\fi%
			\fi%
		\fi
		\divide\pgfmath@x by\c@pgfmath@counta\relax%
		\divide\pgfmath@y by\c@pgfmath@counta\relax%
		\pgfmathreciprocal{\pgfmath@tonumber{\pgfmath@y}}%
		\pgfmath@x=\pgfmathresult\pgfmath@x%
		\pgfmath@xa=\pgfmath@tonumber{\pgfmath@x}\pgfmath@x%
		\edef\pgfmath@xsq{\pgfmath@tonumber{\pgfmath@xa}}%
		\pgfmath@x=\pgfmath@cE\pgfmath@xa%
		\advance\pgfmath@x by\pgfmath@cD pt\relax%
		\pgfmath@x=\pgfmath@xsq\pgfmath@x%
		\advance\pgfmath@x by\pgfmath@cC pt\relax%
		\pgfmath@x=\pgfmath@xsq\pgfmath@x%
		\advance\pgfmath@x by\pgfmath@cB pt\relax%
		\pgfmath@x=\pgfmath@xsq\pgfmath@x%
		\advance\pgfmath@x by\pgfmath@cA pt\relax%
		\ifdim\pgfmath@y<0pt\relax%
			\pgfmath@y=-\pgfmath@y%
		\fi%
		\pgfmath@x=\pgfmath@tonumber{\pgfmath@y}\pgfmath@x%
		% Invert the scaling factor.
		\multiply\pgfmath@x by\c@pgfmath@counta\relax%
		\pgfmath@returnone\pgfmath@x%
	\endgroup%
}
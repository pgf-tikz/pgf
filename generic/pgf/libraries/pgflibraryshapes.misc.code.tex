% Copyright 2006 by Till Tantau
%
% This file may be distributed and/or modified
%
% 1. under the LaTeX Project Public License and/or
% 2. under the GNU Public License.
%
% See the file doc/generic/pgf/licenses/LICENSE for more details.

\ProvidesFileRCS[v\pgfversion] $Header$


\pgfdeclareshape{cross out}
{
  \inheritsavedanchors[from=rectangle] % this is nearly a rectangle
  \inheritanchorborder[from=rectangle]
  \inheritanchor[from=rectangle]{north}
  \inheritanchor[from=rectangle]{north west}
  \inheritanchor[from=rectangle]{north east}
  \inheritanchor[from=rectangle]{center}
  \inheritanchor[from=rectangle]{west}
  \inheritanchor[from=rectangle]{east}
  \inheritanchor[from=rectangle]{mid}
  \inheritanchor[from=rectangle]{mid west}
  \inheritanchor[from=rectangle]{mid east}
  \inheritanchor[from=rectangle]{base}
  \inheritanchor[from=rectangle]{base west}
  \inheritanchor[from=rectangle]{base east}
  \inheritanchor[from=rectangle]{south}
  \inheritanchor[from=rectangle]{south west}
  \inheritanchor[from=rectangle]{south east}
  \foregroundpath{
    % store lower right in xa/ya and upper right in xb/yb
    \southwest \pgf@xa=\pgf@x \pgf@ya=\pgf@y
    \northeast \pgf@xb=\pgf@x \pgf@yb=\pgf@y
    \pgfpathmoveto{\pgfqpoint{\pgf@xa}{\pgf@ya}}
    \pgfpathlineto{\pgfqpoint{\pgf@xb}{\pgf@yb}}
    \pgfpathmoveto{\pgfqpoint{\pgf@xa}{\pgf@yb}}
    \pgfpathlineto{\pgfqpoint{\pgf@xb}{\pgf@ya}}
 }
}


\pgfdeclareshape{strike out}
{
  \inheritsavedanchors[from=rectangle] % this is nearly a rectangle
  \inheritanchorborder[from=rectangle]
  \inheritanchor[from=rectangle]{north}
  \inheritanchor[from=rectangle]{north west}
  \inheritanchor[from=rectangle]{north east}
  \inheritanchor[from=rectangle]{center}
  \inheritanchor[from=rectangle]{west}
  \inheritanchor[from=rectangle]{east}
  \inheritanchor[from=rectangle]{mid}
  \inheritanchor[from=rectangle]{mid west}
  \inheritanchor[from=rectangle]{mid east}
  \inheritanchor[from=rectangle]{base}
  \inheritanchor[from=rectangle]{base west}
  \inheritanchor[from=rectangle]{base east}
  \inheritanchor[from=rectangle]{south}
  \inheritanchor[from=rectangle]{south west}
  \inheritanchor[from=rectangle]{south east}
  \foregroundpath{
    \pgfpathmoveto{\southwest}
    \pgfpathlineto{\northeast}
 }
}




% Keys for starburst shape
%
% /pgf/starburst point height : The maximum height of the outer points.
% /pgf/starburst points       : The number of points.
% /pgf/random starburst       : The seed for the random number generator.
%
\pgfkeys{/pgf/random starburst/%
	.code={%
		\ifx\pgfkeysnovalue#1%
			\pgfmathgeneratepseudorandomnumber%
		\else%
			\pgfmathtruncatemacro\pgfmathresult{#1}%
		\fi%
		\pgfkeyslet{/pgf/random starburst}{\pgfmathresult}%
	}%
}
\pgfkeys{/pgf/random starburst=100}

\pgfkeys{/pgf/starburst point height/.value required}
\pgfkeys{/pgf/starburst point height/.code={%
		\pgfmathparse{#1}%
		\edef\pgfmathresult{\pgfmathresult pt}%
		\pgfkeyslet{/pgf/starburst point height}{\pgfmathresult}%
	}%
}%
\pgfkeys{/pgf/starburst point height=.5cm}

\pgfkeys{/pgf/starburst points/.value required}
\pgfkeys{/pgf/starburst points/.code={%
		\pgfmathtruncatemacro\pgfmathresult{#1}%
		\pgfkeyslet{/pgf/starburst points}{\pgfmathresult}%
	}%	
}%
\pgfkeys{/pgf/starburst points=17}

\pgfdeclareshape{starburst}{%
	\savedmacro\anglestep{%
		\pgfmathdivide@{180}{\pgfkeysvalueof{/pgf/starburst points}}%
		\let\anglestep\pgfmathresult%
	}
	\savedmacro\calculatestarburstpoints{%
		%
		% Get the angle step.
		%
		\pgfmathdivide@{180}{\pgfkeysvalueof{/pgf/starburst points}}%
		\let\anglestep\pgfmathresult%
		%
		% Get the total number of points.
		%
		\pgfmathsetcounter{pgf@counta}{\pgfkeysvalueof{/pgf/starburst points}}%
		\multiply\c@pgf@counta2\relax%
		\edef\totalpoints{\the\c@pgf@counta}%
		\addtosavedmacro{\totalpoints}%
		%
		% Calculate the centerpoint.
		%
		\pgfsavepgf@process\centerpoint{%
			\pgfmathsetlength\pgf@x{+.5\wd\pgfnodeparttextbox}%
			\pgfmathsetlength\pgf@y{+.5\ht\pgfnodeparttextbox}%
			\pgfmathaddtolength\pgf@y{-.5\dp\pgfnodeparttextbox}%
		}%
		%
		% Get the larger of the outer sep.
		%
		\pgfmathsetlength\pgf@x{+\pgfkeysvalueof{/pgf/outer xsep}}%
		\pgfmathsetlength\pgf@y{+\pgfkeysvalueof{/pgf/outer ysep}}%
		\ifdim\pgf@x<\pgf@y%
			\pgf@x\pgf@y%
		\fi%
		\edef\outersep{\the\pgf@x}%
		% 
		% Get the node dimensions.
		% 
		\pgfmathsetlength\pgf@x{+\pgfkeysvalueof{/pgf/inner xsep}}%
		\pgfmathaddtolength\pgf@x{+.5\wd\pgfnodeparttextbox}%		
		\pgfmathsetlength\pgf@y{+\pgfkeysvalueof{/pgf/inner ysep}}%
		\pgfmathaddtolength\pgf@y{+.5\ht\pgfnodeparttextbox}%
		\pgfmathaddtolength\pgf@y{+.5\dp\pgfnodeparttextbox}%
		%
		%  Calculate the inner radii.
		%
		\ifpgfshapeborderusesincircle%
			\pgfkeysgetvalue{/pgf/shape border rotate}{\rotate}%
			%
			% Use the incircle...
			%
			\ifdim\pgf@y>\pgf@x%
				\pgf@x\pgf@y%
			\fi%
			\pgf@x1.41421\pgf@x%
			\pgf@y\pgf@x%
		\else%
			%
			% Get the rotation (with rounding).
			%
			\pgfkeysgetvalue{/pgf/shape border rotate}{\rotate}%
			\pgfmathmod@{\rotate}{360}%
			\afterassignment\pgfmath@gobbletilpgfmath@%
			\expandafter\c@pgf@counta\pgfmathresult\relax\pgfmath@%
			\advance\c@pgf@counta45\relax%
			\divide\c@pgf@counta90\relax%
			\multiply\c@pgf@counta90\relax%
			\ifnum\c@pgf@counta<0\relax%
				\advance\c@pgf@counta360\relax%
			\fi%
			%
			% Calculate the width and height of the node
			% contents, according to any border rotation.
			%
			\ifnum\c@pgf@counta=90\relax%
				\pgf@xc\pgf@x%
				\pgf@x\pgf@y%
				\pgf@y\pgf@xc%
			\else%
				\ifnum\c@pgf@counta=270\relax%
					\pgf@xc\pgf@x%
					\pgf@x\pgf@y%
					\pgf@y\pgf@xc%
				\fi%
			\fi%
			\edef\rotate{\the\c@pgf@counta}%
			%
			% ...or not.
			%		
			\pgf@x=1.41421\pgf@x%
			\pgf@y=1.41421\pgf@y%
		\fi%
		\addtosavedmacro{\rotate}%
		% 
		% Adjust innerradius for minimum width and height.
		%
		\pgf@xa\pgf@x% 
		\pgfmathsetlength\pgf@xb{+\pgfkeysvalueof{/pgf/starburst point height}}%
		\advance\pgf@xa\pgf@xb%
		\pgfmathsetlength\pgf@xc{+\pgfkeysvalueof{/pgf/minimum width}}%
		\ifdim\pgf@xa<.5\pgf@xc%
			\pgf@x.5\pgf@xc%
			\advance\pgf@x-\pgf@xb%
		\fi%
		\pgf@ya\pgf@y% 
		\pgfmathsetlength\pgf@yb{+\pgfkeysvalueof{/pgf/starburst point height}}%
		\advance\pgf@ya\pgf@yb%
		\pgfmathsetlength\pgf@yc{+\pgfkeysvalueof{/pgf/minimum height}}%
		\ifdim\pgf@ya<.5\pgf@yc%
			\pgf@y.5\pgf@yc%
			\advance\pgf@y-\pgf@yb%
		\fi%	
		\edef\xinnerradius{\the\pgf@x}%
		\edef\yinnerradius{\the\pgf@y}%
		\addtosavedmacro{\xinnerradius}%
		\addtosavedmacro{\yinnerradius}%
		%
		% Calculate a radius outside the starburst.
		%
		\ifdim\pgf@y>\pgf@x%
			\pgf@x\pgf@y%
		\fi%
		\pgfmathaddtolength\pgf@x{+\pgfkeysvalueof{/pgf/starburst point height}}%
		\edef\externalradius{\the\pgf@x}%
		\addtosavedmacro{\externalradius}%
		%
		% Set the seed for the random number generator.
		%
		\pgfmathsetseed{\pgfkeysvalueof{/pgf/random starburst}}%	
		%
		% Now create the points on the shape and also 
		% the miter length and angle for each point.
		%
		\def\angle{90}% Start at the top.
		%
		% At point a, the miter length and angle are calculated for point b = a - 1.
		%
		\c@pgf@counta1\relax%
		\c@pgf@countb0\relax%
		%
		% As 3 consecutive points are required to be defined for miter
		% calculations, it is necessary to go over the first two points
		% again.
		% 
		\c@pgf@countc\totalpoints\relax%
		\advance\c@pgf@countc2\relax%
		\edef\looppoints{\the\c@pgf@countc}%
		\let\secondpoint\pgfutil@empty%
		\let\thirdpoint\pgfutil@empty%
		\pgfmathloop%
			%
			% Cycle the point definitions.
			%
			\let\firstpoint\secondpoint%	
			\let\secondpoint\thirdpoint%	
			\ifnum\pgfmathcounter>\looppoints%
			\else%
				\ifnum\pgfmathcounter>\totalpoints%
					\expandafter\let\expandafter\thirdpoint\csname point@\the\c@pgf@counta @\endcsname%
				\else%
					\ifodd\pgfmathcounter%
						%
						% An outer point.
						%
						\ifnum\pgfkeysvalueof{/pgf/random starburst}=0\relax%
							\pgf@xa\pgfkeysvalueof{/pgf/starburst point height}\relax%
						\else%
							\pgf@x\pgfkeysvalueof{/pgf/starburst point height}\relax%
							\pgf@xa.75\pgf@x%
							\pgf@xb.25\pgf@x%
							\pgfmathrnd%
							\pgf@xa\pgfmathresult\pgf@xa%
							\advance\pgf@xa\pgf@xb%
						\fi%
						\pgf@x\xinnerradius\relax%
						\advance\pgf@x\pgf@xa%
						\pgf@y\yinnerradius\relax%
						\advance\pgf@y\pgf@xa%
						\expandafter\pgfsavepgf@process\csname point@\the\c@pgf@counta @\endcsname{%
							\pgfpointpolar{\angle}{\the\pgf@x and \the\pgf@y}%
							\pgf@xa\pgf@x%
							\pgf@ya\pgf@y%
							\centerpoint%
							\advance\pgf@x\pgf@xa%
							\advance\pgf@y\pgf@ya%
						}%
					\else%
						%
						% An inner point.
						%
						\expandafter\pgfsavepgf@process\csname point@\the\c@pgf@counta @\endcsname{%
							\pgfpointpolar{\angle}{\xinnerradius and \yinnerradius}%
							\pgf@xa\pgf@x%
							\pgf@ya\pgf@y%
							\centerpoint%
							\advance\pgf@x\pgf@xa%
							\advance\pgf@y\pgf@ya%
						}%						
					\fi%
					%
					% Add the points to the saved macro.
					%
					\expandafter\let\expandafter\thirdpoint\csname point@\the\c@pgf@counta @\endcsname%
					\expandafter\addtosavedmacro\expandafter{\csname point@\the\c@pgf@counta @\endcsname}%
				\fi%				
				%
				% It is only possible to do the miter calculations if three points are defined.
				%
				\ifx\firstpoint\pgfutil@empty%
				\else%
					%
					% Calculate the miter length...
					%
					\pgfmathanglebetweenlines{\secondpoint}{\thirdpoint}{\secondpoint}{\firstpoint}%
					\pgfmathdivide@{\pgfmathresult}{2}%
					\let\defaultmiterangle\pgfmathresult%
					\pgfmathcosec@{\pgfmathresult}%
					\pgf@x\outersep\relax%
					\pgf@x\pgfmathresult\pgf@x%
					\edef\miterlength{\the\pgf@x}%
					%
					% ...the miter angle...
					%
					\pgfmathanglebetweenlines{\firstpoint}{\secondpoint}{\firstpoint}{\thirdpoint}%
					\pgfmathadd@{\pgfmathresult}{\defaultmiterangle}%
					\pgfmathsubtract@{180}{\pgfmathresult}%
					\let\angletemp\pgfmathresult%
					\pgfmathanglebetweenpoints{\firstpoint}{\thirdpoint}%
					\pgfmathsubtract@{180}{\pgfmathresult}%
					\pgfmathsubtract@{\angletemp}{\pgfmathresult}%
					\edef\miterangle{\pgfmathresult}%
					%
					% ...and thus the border point.
					%
					\pgfsavepgf@process\borderpoint{%
						\secondpoint%
						\pgf@xa\pgf@x
						\pgf@ya\pgf@y%
						\pgfpointpolar{\miterangle}{\miterlength}%
						\advance\pgf@x\pgf@xa%
						\advance\pgf@y\pgf@ya%
					}%
					%
					% Get the angle from the centerpoint to the *unrotated* border points.
					%
					\pgfmathanglebetweenpoints{\centerpoint}{\borderpoint}%
					\expandafter\edef\csname angletoborderpoint@\the\c@pgf@countb @\endcsname{\pgfmathresult}%
					\expandafter\addtosavedmacro\expandafter{\csname angletoborderpoint@\the\c@pgf@countb @\endcsname}%
					%
					% Rotatee the border points and save.
					%
					\expandafter\pgfsavepgf@process\csname borderpoint@\the\c@pgf@countb @\endcsname{%
						\pgfmathrotatepointaround{\borderpoint}{\centerpoint}{\rotate}%
					}%
					\expandafter\addtosavedmacro\expandafter{\csname borderpoint@\the\c@pgf@countb @\endcsname}%				
					%
					% Now create the anchors.
					%
					\c@pgf@countc\c@pgf@countb%
					\advance\c@pgf@countc1\relax%
					\divide\c@pgf@countc2\relax%
					\ifodd\c@pgf@countb\relax%
						\pgfutil@ifundefined{pgf@anchor@starburst@outer point\space\the\c@pgf@countc}{%
							\expandafter\xdef\csname pgf@anchor@starburst@outer point\space\the\c@pgf@countc\endcsname{%
								\noexpand\calculatestarburstpoints%
								\noexpand\csname borderpoint@\the\c@pgf@countb @\noexpand\endcsname%
							}%
						}{}%
					\else%
						\pgfutil@ifundefined{pgf@anchor@starburst@inner point\space\the\c@pgf@countc}{%
							\expandafter\xdef\csname pgf@anchor@starburst@inner point\space\the\c@pgf@countc\endcsname{%
								\noexpand\calculatestarburstpoints%
								\noexpand\csname borderpoint@\the\c@pgf@countb @\noexpand\endcsname%
							}%
						}{}%
					\fi%
				\fi%
				\pgfmathadd@{\angle}{\anglestep}%
				\pgfmathmod@{\pgfmathresult}{360}%	
				\let\angle\pgfmathresult%
				\advance\c@pgf@counta1\relax%
				\ifnum\c@pgf@counta>\totalpoints%
					\c@pgf@counta1\relax%
				\fi%
				\advance\c@pgf@countb1\relax%
				\ifnum\c@pgf@countb>\totalpoints%
					\c@pgf@countb1\relax%
				\fi%
		\repeatpgfmathloop%
	}
	\savedanchor\centerpoint{%
		\pgfmathsetlength\pgf@x{+.5\wd\pgfnodeparttextbox}%
		\pgfmathsetlength\pgf@y{+.5\ht\pgfnodeparttextbox}%
		\pgfmathaddtolength\pgf@y{-.5\dp\pgfnodeparttextbox}%
	}%
	\savedanchor\midpoint{%
		\pgfmathsetlength\pgf@x{+.5\wd\pgfnodeparttextbox}%
		\pgfmathsetlength\pgf@y{+.5ex}%
	}%
	\savedanchor\basepoint{%
		\pgfmathsetlength\pgf@x{+.5\wd\pgfnodeparttextbox}%
		\pgf@y0pt\relax%
	}%
	\anchor{center}{\centerpoint}
	\anchor{base}{\basepoint}
	\anchor{mid}{\midpoint}
	\anchor{north}{%
		\calculatestarburstpoints%
		\csname pgf@anchor@starburst@border\endcsname{\pgfqpoint{0pt}{\externalradius}}%
	}
	\anchor{south}{%
		\calculatestarburstpoints%
		\csname pgf@anchor@starburst@border\endcsname{\pgfqpoint{0pt}{-\externalradius}}%
	}
	\anchor{east}{%
		\calculatestarburstpoints%
		\csname pgf@anchor@starburst@border\endcsname{\pgfqpoint{\externalradius}{0pt}}%
	}
	\anchor{west}{%
		\calculatestarburstpoints%
		\csname pgf@anchor@starburst@border\endcsname{\pgfqpoint{-\externalradius}{0pt}}%
	}
	\anchor{north west}{%
		\calculatestarburstpoints%
		\csname pgf@anchor@starburst@border\endcsname{\pgfqpoint{-\externalradius}{\externalradius}}%
	}
	\anchor{south west}{%
		\calculatestarburstpoints%
		\csname pgf@anchor@starburst@border\endcsname{\pgfqpoint{-\externalradius}{-\externalradius}}%
	}
	\anchor{north east}{%
		\calculatestarburstpoints%
		\csname pgf@anchor@starburst@border\endcsname{\pgfqpoint{\externalradius}{\externalradius}}%
	}
	\anchor{south east}{%
		\calculatestarburstpoints%
		\csname pgf@anchor@starburst@border\endcsname{\pgfqpoint{\externalradius}{-\externalradius}}%
	}
	\backgroundpath{%	
		\calculatestarburstpoints%
		\pgfmathloop%
			\ifnum\pgfmathcounter>\totalpoints%
			\else%
				\ifnum\pgfmathcounter=1\relax%
					\let\starburstaction\pgfpathmoveto%
				\else%
					\let\starburstaction\pgfpathlineto%
				\fi%
				\starburstaction{%
					\pgfmathrotatepointaround{\csname point@\pgfmathcounter @\endcsname}{\centerpoint}{\rotate}}%
			\repeatpgfmathloop%
		\pgfpathclose%		
	}
	\anchorborder{%
		%
		% Save x and y.
		%
		\edef\externalx{\the\pgf@x}%
		\edef\externaly{\the\pgf@y}%
		%
		% Adjust the location of the external 
		% point relative to \centerpoint.
		%
		\centerpoint%
		\pgf@xa\externalx\relax%
		\pgf@ya\externaly\relax%
		\advance\pgf@xa\pgf@x%
		\advance\pgf@ya\pgf@y%
		\edef\externalx{\the\pgf@xa}%
		\edef\externaly{\the\pgf@ya}%
		%
		% Get the starburst points.
		%
		\calculatestarburstpoints%
		%
		% Get the angle of the external point to the \centerpoint.
		%
		\pgfmathanglebetweenpoints{\centerpoint}{\pgfqpoint{\externalx}{\externaly}}%
		\pgfmathsubtract@{\pgfmathresult}{\rotate}%
		\ifdim\pgfmathresult pt<0pt\relax%
			\pgfmathadd@{\pgfmathresult}{360}%
		\fi%
		\let\externalangle\pgfmathresult%
		%
		% Locate the appropriate sides on the starburst border...
		%
		\ifdim\externalangle pt<90pt\relax%
			\c@pgf@counta0\relax%
			\c@pgf@countb\totalpoints\relax%
			\pgfmathloop%
			\ifnum\c@pgf@counta>0\relax%
			\else%
				\ifdim\csname angletoborderpoint@\the\c@pgf@countb @\endcsname pt>90pt\relax%
					\c@pgf@counta\c@pgf@countb%
				\else%
					\ifdim\externalangle pt>\csname angletoborderpoint@\the\c@pgf@countb @\endcsname pt\relax%
						\c@pgf@counta\c@pgf@countb%
					\fi%
				\fi%
				\advance\c@pgf@countb-1\relax%
			\repeatpgfmathloop%
			\edef\first{\the\c@pgf@counta}%
			\advance\c@pgf@counta1\relax%
			\ifnum\c@pgf@counta>\totalpoints\relax%
				\c@pgf@counta1\relax%
			\fi%
			\edef\second{\the\c@pgf@counta}%
		\else%
			\c@pgf@counta0\relax%
			\pgfmathloop%
			\ifnum\c@pgf@counta>0\relax%
			\else%
				\ifdim\csname angletoborderpoint@\pgfmathcounter @\endcsname pt<90pt\relax%
					\c@pgf@counta\pgfmathcounter%
				\else%
					\ifdim\externalangle pt<\csname angletoborderpoint@\pgfmathcounter @\endcsname pt\relax%
						\c@pgf@counta\pgfmathcounter%	
					\fi%
				\fi%					
			\repeatpgfmathloop%
			\edef\first{\the\c@pgf@counta}%
			\advance\c@pgf@counta-1\relax%
			\ifnum\c@pgf@counta=0\relax%
				\c@pgf@counta\totalpoints\relax%
			\fi%
			\edef\second{\the\c@pgf@counta}%
		\fi%
		%
		% ...and thus, the point on the star border.
		%
		\pgfpointintersectionoflines{\centerpoint}{\pgfqpoint{\externalx}{\externaly}}%
				{\csname borderpoint@\first @\endcsname}{\csname borderpoint@\second @\endcsname}%
	}%
}

\endinput

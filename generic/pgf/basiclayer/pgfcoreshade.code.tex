\ProvidesFileRCS $Header$

% Copyright 2005 by Till Tantau <tantau@cs.tu-berlin.de>.
%
% This program can be redistributed and/or modified under the terms
% of the GNU Public License, version 2.


%
% Parsing functions
%
\newdimen\pgf@max
\newcount\pgf@sys@shading@range@num
\def\pgf@parsefunc#1{%
  \edef\temp{{#1}}%
  \expandafter\pgf@convertrgbstring\temp%
  \edef\temp{{\pgf@rgbconv}}%
  \expandafter\pgf@@parsefunc\temp}
\def\pgf@@parsefunc#1{%
  \let\pgf@bounds=\pgf@empty%
  \let\pgf@funcs=\pgf@empty%
  \let\pgf@psfuncs=\pgf@empty%
  \let\pgf@encode=\pgf@empty%
  \let\pgf@sys@shading@ranges=\pgf@empty%
  \pgf@sys@shading@range@num=0\relax%
  \pgf@parsefirst[#1; ]%
  \pgf@parselastdom[#1; ]%
  \pgf@parsemid[#1; ]%
  \ifx\pgf@bounds\pgf@empty%
    \edef\pgf@pdfparseddomain{0 1}%
    \edef\pgf@pdfparsedfunction{\pgf@singlefunc\space}%
  \else%
    \edef\pgf@pdfparseddomain{\pgf@doma\space\pgf@domb}%
    \edef\pgf@pdfparsedfunction{%
      << /FunctionType 3 /Domain [\pgf@doma\space\pgf@domb] /Functions
      [\pgf@funcs\space] /Bounds [\pgf@bounds] /Encode [0 1 \pgf@encode]
      >> }%
  \fi%
  \xdef\pgf@psfuncs{\pgf@psfuncs}%
  }
\def\pgf@parsefirst[rgb(#1)=(#2,#3,#4)#5]{%
  \setlength\pgf@x{#1}%
  \edef\pgf@sys@shading@start@pos{\the\pgf@x}%
  \pgf@sys@bp@correct\pgf@x%
  \edef\pgf@doma{\pgf@sys@tonumber{\pgf@x}}%
  \edef\pgf@prevx{\pgf@sys@tonumber{\pgf@x}}%
  \pgf@getrgbtuplewithmixin{#2}{#3}{#4}%
  \edef\pgf@sys@shading@start@rgb{\pgf@sys@rgb}%
  \let\pgf@sys@prevcolor=\pgf@sys@shading@start@rgb%
  \let\pgf@sys@prevpos=\pgf@sys@shading@start@pos%
  \edef\pgf@prevcolor{\pgf@rgb}%
  \edef\pgf@firstcolor{\pgf@rgb}}
\def\pgf@parselastdom[rgb(#1)=(#2,#3,#4); {%
  \pgf@ifnextchar]{%
    \setlength\pgf@x{#1}%
    \edef\pgf@sys@shading@end@pos{\the\pgf@x}%
    \pgf@max=\pgf@x\relax%
    \pgf@sys@bp@correct\pgf@x%
    \edef\pgf@domb{\pgf@sys@tonumber{\pgf@x}}%
    \pgf@getrgbtuplewithmixin{#2}{#3}{#4}%
    \edef\pgf@sys@shading@end@rgb{\pgf@sys@rgb}%
    \pgf@gobble}{\pgf@parselastdom[}}
\def\pgf@parsemid[rgb(#1)=(#2,#3,#4); {\pgf@parserest[}
\def\pgf@parserest[rgb(#1)=(#2,#3,#4); {%
  \advance\pgf@sys@shading@range@num by1\relax%
  \pgf@ifnextchar]{%
    \pgf@getrgbtuplewithmixin{#2}{#3}{#4}%
    \edef\pgf@singlefunc{\space%
      << /FunctionType 2 /Domain [0 1] /C0
      [\pgf@prevcolor] /C1 [\pgf@rgb] /N 1 >> }%
    \edef\pgf@funcs{\pgf@funcs\space%
      << /FunctionType 2 /Domain [\pgf@doma\space\pgf@domb] /C0
      [\pgf@prevcolor] /C1 [\pgf@rgb] /N 1 >> }%
    \edef\pgf@psfuncs{\pgf@prevx\space \pgf@rgb\space \pgf@prevcolor\space pgfshade \pgf@psfuncs}%
    \setlength\pgf@x{#1}%
    \edef\pgf@sys@shading@ranges{\pgf@sys@shading@ranges{{\pgf@sys@prevpos}{\the\pgf@x}{\pgf@sys@prevcolor}{\pgf@sys@rgb}}}%
    \edef\pgf@sys@prevpos{\the\pgf@x}%
    \let\pgf@sys@prevcolor=\pgf@sys@rgb%
    \pgf@gobble}{%
    \setlength\pgf@x{#1}%
    \pgf@getrgbtuplewithmixin{#2}{#3}{#4}%
    \edef\pgf@sys@shading@ranges{\pgf@sys@shading@ranges{{\pgf@sys@prevpos}{\the\pgf@x}{\pgf@sys@prevcolor}{\pgf@sys@rgb}}}%
    \edef\pgf@sys@prevpos{\the\pgf@x}%
    \let\pgf@sys@prevcolor=\pgf@sys@rgb%
    \edef\pgf@psfuncs{\pgf@prevx\space \pgf@rgb\space \pgf@prevcolor\space pgfshade \pgf@psfuncs}%
    \pgf@sys@bp@correct\pgf@x%
    \edef\pgf@prevx{\pgf@sys@tonumber{\pgf@x}}%
    \edef\pgf@bounds{\pgf@bounds\space\pgf@sys@tonumber{\pgf@x}}%
    \edef\pgf@encode{\pgf@encode\space0 1}%
    \edef\pgf@singlefunc{\space%
      << /FunctionType 2 /Domain [0 1] /C0
      [\pgf@prevcolor] /C1 [\pgf@rgb] /N 1 >> }% 
    \edef\pgf@funcs{\pgf@funcs\space%
      << /FunctionType 2 /Domain [\pgf@doma\space\pgf@domb] /C0
      [\pgf@prevcolor] /C1 [\pgf@rgb] /N 1 >> }% 
    \edef\pgf@prevcolor{\pgf@rgb}%
    \pgf@parserest[}}

\def\pgf@getrgbtuplewithmixin#1#2#3{%
  \definecolor{pgfshadetemp}{rgb}{#1,#2,#3}%
  \pgf@ifundefined{applycolormixins}{}{\applycolormixins{pgfshadetemp}}%
  \extractcolorspec{pgfshadetemp}{\pgf@tempcolor}%
  \expandafter\convertcolorspec\pgf@tempcolor{rgb}{\pgf@rgbcolor}%
  \expandafter\pgf@getrgb@@\pgf@rgbcolor!}
\def\pgf@getrgb@@#1,#2,#3!{%
  \def\pgf@rgb{#1 #2 #3}%
  \def\pgf@sys@rgb{{#1}{#2}{#3}}%
}


\def\pgf@convertrgbstring#1{%
  \def\pgf@rgbconv{}%
  \pgf@converttorgb#1]%
  }
\def\pgf@converttorgb{%
  \pgf@ifnextchar]{\pgf@gobble}%done!
  {%
    \pgf@ifnextchar;{\pgf@grabsemicolor}%
    {%
      \pgf@ifnextchar r{\pgf@grabrgb}%
      {%
        \pgf@ifnextchar g{\pgf@grabgray}%
        {%
          \pgf@ifnextchar c{\pgf@grabcolor}%
          {\PackageError{pgfshade}{Illformed shading
              specification}{}\pgf@converttorgb}%
        }%
      }%
    }%
  }%
}
\def\pgf@grabsemicolor;{%
  \edef\pgf@rgbconv{\pgf@rgbconv; }\pgf@converttorgb}
\def\pgf@grabrgb rgb(#1)=(#2,#3,#4){%
  \edef\pgf@rgbconv{\pgf@rgbconv rgb(#1)=(#2,#3,#4)}\pgf@converttorgb}
\def\pgf@grabgray gray(#1)=(#2){%
  \edef\pgf@rgbconv{\pgf@rgbconv rgb(#1)=(#2,#2,#2)}\pgf@converttorgb}
\def\pgf@grabcolor color(#1)=(#2){%
  \colorlet{pgf@tempcol}{#2}%
  \extractcolorspec{pgf@tempcol}{\pgf@tempcolor}%
  \expandafter\convertcolorspec\pgf@tempcolor{rgb}{\pgf@rgbcolor}%
  \expandafter\pgf@convgetrgb@\expandafter{\pgf@rgbcolor}{#1}%
}
\def\pgf@convgetrgb@#1#2{%
  \edef\pgf@rgbconv{\pgf@rgbconv rgb(#2)=(#1)}\pgf@converttorgb}




% Declares a horizontal shading for later use. The shading is a
% horizontal bar that changes its color.
%
% #1 = name of the shading for later use
% #2 = height of the shading
% #3 = color specification in the following format: A list of colors
%      that the bar should have at certain points. If the bar should
%      be red at 1cm, this is specified as
%      "rgb(1cm)=(1,0,0)". Multiple specifications are separated by a
%      semicolon and a space. At least two specifications must be
%      given. The specified positions must be given in increasing
%      order. 
%
% Example:
%
% \pgfdeclarehorizontalshading{redtogreentoblue}{1cm}{%
%   rgb(0cm)=(1,0,0); % red
%   rgb(1cm)=(0,1,0); % green
%   rgb(2cm)=(0,0,1)}
%
% \begin{document}
%   The following bar is 2cm long: \pgfuseshading{redtogreentoblue}.
% \end{document}

\def\pgfdeclarehorizontalshading{\pgf@ifnextchar[\pgf@declarehorizontalshading{\pgf@declarehorizontalshading[]}}%
\def\pgf@declarehorizontalshading[#1]#2#3#4{%
  \expandafter\def\csname pgf@deps@pgfshading#2!\endcsname{#1}%
  \expandafter\ifx\csname pgf@deps@pgfshading#2!\endcsname\pgf@empty%
    \pgfsys@horishading{#2}{#3}{#4}%
  \else%
    \expandafter\def\csname pgf@func@pgfshading#2!\endcsname{\pgfsys@horishading}%
    \expandafter\def\csname pgf@args@pgfshading#2!\endcsname{{#3}{#4}}%
    \expandafter\let\csname @pgfshading#2!\endcsname=\pgf@empty%
  \fi}


% Declares a vertical shading for later use. 
%
% #1 = name of the shading for later use
% #2 = height of the shading
% #3 = color specification 
%
% Example:
%
% \pgfdeclareverticalshading{redtogreentoblue}{1cm}{%
%   rgb(0cm)=(1,0,0); % red
%   rgb(1cm)=(0,1,0); % green
%   rgb(2cm)=(0,0,1)}
%
% \begin{document}
%   The following bar is 2cm high: \pgfuseshading{redtogreentoblue}.
% \end{document}

\def\pgfdeclareverticalshading{\pgf@ifnextchar[\pgf@declareverticalshading{\pgf@declareverticalshading[]}}%
\def\pgf@declareverticalshading[#1]#2#3#4{%
  \expandafter\def\csname pgf@deps@pgfshading#2!\endcsname{#1}%
  \expandafter\ifx\csname pgf@deps@pgfshading#2!\endcsname\pgf@empty%
    \pgfsys@vertshading{#2}{#3}{#4}%
  \else%
    \expandafter\def\csname pgf@func@pgfshading#2!\endcsname{\pgfsys@vertshading}%
    \expandafter\def\csname pgf@args@pgfshading#2!\endcsname{{#3}{#4}}%
    \expandafter\let\csname @pgfshading#2!\endcsname=\pgf@empty%
  \fi}


% Declares a radial shading for later use. 
%
% #1 = name of the shading for later use
% #2 = center of inner circle
% #3 = color specification 
%
% Description:
%
% A radial shading creates a smooth color transition between two
% circles. The center of the inner circle is at the give position. Its
% radius is the start of the color specification. The
% center of the outer circle is at the center of the whole shading,
% whose radius is the end of the color specification. For example,
% suppose the color specification is "rgb(1cm)=(1,1,1); rgb(2cm)=(0,0,0)".  
% Then the shading would be 4cm times 4cm large. The inner circle would
% have diameter 1cm and the outer circle would have diameter 2cm. The
% outer circle would be centered in the middle of the shading, whereas
% the outer circle would be centered at the given position.
%
% Example:
%
% \pgfdeclareradialshading{redtogreentoblue}{\pgfpoint{2cm}{2cm}}{%
%   rgb(10pt)=(1,0,0); % red
%   rgb(2cm)=(0,1,0); % green
%   rgb(3cm)=(0,0,1)}
%
% \begin{document}
%   The following ball has diameter 3cm: \pgfuseshading{redtogreentoblue}.
% \end{document}

\def\pgfdeclareradialshading{\pgf@ifnextchar[\pgf@declareradialshading{\pgf@declareradialshading[]}}%
\def\pgf@declareradialshading[#1]#2#3#4{%
  \expandafter\def\csname pgf@deps@pgfshading#2!\endcsname{#1}%
  \expandafter\ifx\csname pgf@deps@pgfshading#2!\endcsname\pgf@empty%
    \pgfsys@radialshading{#2}{#3}{#4}%
  \else%
    \expandafter\def\csname pgf@func@pgfshading#2!\endcsname{\pgfsys@radialshading}%
    \expandafter\def\csname pgf@args@pgfshading#2!\endcsname{{#3}{#4}}%
    \expandafter\let\csname @pgfshading#2!\endcsname=\pgf@empty%
  \fi}
  

% Inserts a box into the text that contains a previously defined
% shading. 
%
% #1 = Name of a shading
%
% Example:
%
% \pgfuseshading{redtogreentoblue}

\def\pgfuseshading#1{%
  \edef\pgf@shadingname{@pgfshading#1}%
  \pgf@tryextensions{\pgf@shadingname}{\pgfalternateextension}%
  \expandafter\pgf@ifundefined\expandafter{\pgf@shadingname}%
  {\PackageError{pgfshade}{Undefined shading "#1"}{}}%
  {%
    {%
      \globalcolorsfalse%
      \def\pgf@shade@adds{}%
      \pgf@ifundefined{pgf@deps\pgf@shadingname}%
      {}%
      {%
        \edef\@list{\csname pgf@deps\pgf@shadingname\endcsname}%
        \@for\@temp:=\@list\do{%
          {%
            \pgf@ifundefined{applycolormixins}{}{\applycolormixins{\@temp}}%
            \extractcolorspec{\@temp}{\pgf@tempcolor}%
            \expandafter\convertcolorspec\pgf@tempcolor{rgb}{\pgf@rgbcolor}%
            \xdef\pgf@shade@adds{\pgf@shade@adds,\pgf@rgbcolor}%
          }%
        }%
      }%
      \expandafter\pgf@strip@shadename\pgf@shadingname!!%
      \pgf@ifundefined{@pgfshading\pgf@basename\pgf@shade@adds!}%
      {%
        {%
          \expandafter\def\expandafter\@temp\expandafter{\csname pgf@func\pgf@shadingname\endcsname}%
          \edef\@args{{\pgf@basename\pgf@shade@adds}}%
          \expandafter\expandafter\expandafter\def%
          \expandafter\expandafter\expandafter\@@args%
          \expandafter\expandafter\expandafter{\csname pgf@args\pgf@shadingname\endcsname}%
          \expandafter\expandafter\expandafter\@temp\expandafter\@args\@@args%
          %
        }%
      }%
      {}%
      \pgf@invokeshading{\csname @pgfshading\pgf@basename\pgf@shade@adds!\endcsname}%
    }%
  }%
}

\def\pgf@strip@shadename @pgfshading#1!!!{\def\pgf@basename{#1}}

\def\pgf@invokeshading#1{%
  \ifpgfpicture%
    \pgfsys@shadinginsidepgfpicture{#1}%
  \else%
    \pgfsys@shadingoutsidepgfpicture{#1}%  
  \fi%
}


% Create an alias name for a shading
%
% #1 = name of the alias
% #2 = name of the original
%
% Example:
%
% \pgfaliasshading{shading!25}{shadingshaded}

\def\pgfaliasshading#1#2{%
  \expandafter\let\expandafter\pgf@temp\expandafter=\csname @pgfshading#2!\endcsname%
  \expandafter\let\csname @pgfshading#1!\endcsname=\pgf@temp%
  \expandafter\let\expandafter\pgf@temp\expandafter=\csname pgf@deps@pgfshading#2!\endcsname%
  \expandafter\let\csname pgf@deps@pgfshading#1!\endcsname=\pgf@temp%
  \expandafter\let\expandafter\pgf@temp\expandafter=\csname pgf@func@pgfshading#2!\endcsname%
  \expandafter\let\csname pgf@func@pgfshading#1!\endcsname=\pgf@temp%
  \expandafter\let\expandafter\pgf@temp\expandafter=\csname pgf@args@pgfshading#2!\endcsname%
  \expandafter\let\csname pgf@args@pgfshading#1!\endcsname=\pgf@temp%
}




% Shades the current path, but does not discard it.
%
% #1 - a shading (see below)
% #2 - an angle
%
% Description:
%
% \pgfshadepath  ``tries'' to fill the
% current path with a shading. The shading's original size should
% completely cover the area between (0,0) and (100bp,100bp). The
% shading will be rotated by #2 and then rescaled so that it
% completely covers the path. Then the path will be used (locally) for
% clipping and the shading is drawn.
%
% After all this, the path can still be used for the normal
% stroking/clipping operations.
%
% The shading is rotated around its middle. If no rotation occurs, the
% lower left corner of the path will lie on (25bp, 25bp), the upper
% right corner on (75bp, 75bp).
%
% Example:
%
% \pgfdeclareverticalshading{myshading}{100bp}{color(0pt)=(red); color(100bp)=(green)}
%
% \pgfpathmoveto{\pgforigin}
% \pgfpathlineto{\pgfxy(1,0)}
% \pgfpathlineto{\pgfxy(1,1)}
% \pgfshadepath{myshading}{0}
% \pgfusepath{stroke}

\def\pgfshadepath#1#2{%
  \ifdim\pgf@pathminx=16000pt%
    \PackageWarning{pgfshade}{No path specified that can be filled}%
  \else%
    \begingroup%  
      \pgfsys@beginscope%
        % Calculate center:
        \pgf@xb=.5\pgf@pathmaxx%
        \advance\pgf@xb by.5\pgf@pathminx%
        \pgf@yb=.5\pgf@pathmaxy%
        \advance\pgf@yb by.5\pgf@pathminy%
        % Calculate scaling:
        \pgf@xc=\pgf@pathmaxx%
        \advance\pgf@xc by-\pgf@pathminx%
        \pgf@yc=\pgf@pathmaxy%
        \advance\pgf@yc by-\pgf@pathminy%
        \pgf@xc=.02\pgf@xc%
        \pgf@yc=.02\pgf@yc%
        \pgfsyssoftpath@invokecurrentpath%
        \pgfsys@clipnext%
        \pgfsys@discardpath%
        % Compute new transformation matrix:
        \pgf@process{\pgfsincos{#2}}%
        \pgf@xa=-\pgf@x%
        \pgfsys@transformcm{1}{0}{0}{1}{\pgf@xb}{\pgf@yb}%
        \pgfsys@transformcm%
        {\pgf@sys@tonumber{\pgf@xc}}{0}%
        {0}{\pgf@sys@tonumber{\pgf@yc}}{0pt}{0pt}%
        \pgfsys@transformcm%
        {\pgf@sys@tonumber{\pgf@y}}{\pgf@sys@tonumber{\pgf@x}}%
        {\pgf@sys@tonumber{\pgf@xa}}{\pgf@sys@tonumber{\pgf@y}}{0pt}{0pt}%
        \pgfuseshading{#1}%
      \pgfsys@endscope%
    \endgroup%
  \fi%
}

\endinput

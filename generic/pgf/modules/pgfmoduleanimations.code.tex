% Copyright 2015 by Till Tantau
%
% This file may be distributed and/or modified
%
% 1. under the LaTeX Project Public License and/or
% 2. under the GNU Public License.
%
% See the file doc/generic/pgf/licenses/LICENSE for more details.

\ProvidesFileRCS $Header$



% Animate an attribute over time
%
% #1 = an attribute
% #2 = configuration keys
%
% This command adds animation commands for the attribute given in #1
% to the current graphic scope. For instance, when #1 is 
% "fill opacity" then the fill opacity will change over time within
% the current scope. 
%
% The keys in #2 specify how quickly the changes occur. It turns out
% that specifying timings is a bit of an art and there are a great
% number of keys for this. Basically, the keys and the logic follows
% the SVG specification since, at the time of writing, this is the
% only output format that supports animation; nevertheless pgf
% provides an complete abstraction and renames a few keys that have
% been named very poorly in SVG (such as keyPoints, which are neither
% keys nor points). The semantics is, however, the SVG semantics.
%
% The keys in #2 are given in key-value syntax and can be given in any
% order. However, not all keys apply to all attributes in #1, but only
% certain types. In detail, these are (more may be added in the
% future): 
%
% Attribute                 Value Type
% ---------------------------------------
% fill opacity              scalar
% stroke opacity            scalar
% line width                dimension
% fill                      color
% stroke                    color
% motion                    motion
% translate                 translation
% scale                     scaling
% rotate                    rotation
% xskew                     scalar
% yskew                     scalar
% path                      path
%
% The keys for #2 either concern the value types listed above or
% general. The general keys are the same for all attributes. When a
% key is for the wrong type, it is ignored. This is useful when you
% wish to specify several animations that share the same timing
% information. 
%
% One case is special: The path attribute. Unlike all other
% attributes, this attribute does not concern the current scope, but
% the next \pgfusepath: It animates the path used by this usepath
% command. 
% 
%
% Example:
%
% \pgfanimateattribute{fill opacity}{begin=0s,dur=3s,from=0,to=1}

\def\pgfanimateattribute#1#2{%
  {%
    \pgfkeys{/pgf/@animation attributes/.cd,#1}%
    \pgfkeys{/pgf/animation/.cd,#2}%
    \pgf@do@animate%
  }%
}


% The animation attributes

\pgfkeys{/pgf/@animation attributes/.cd,
  fill opacity/.code=\pgf@animate@attr{fillopacity}{scalar},
  draw opacity/.code=\pgf@animate@attr{strokeopacity}{scalar},
  stroke opacity/.style=draw opacity,
  line width/.code=\pgf@animate@attr{linewidth}{dimension},
  fill/.code=\pgf@animate@attr{fillcolor}{color},
  draw/.code=\pgf@animate@attr{strokecolor}{color},
  stroke/.code=draw,
  motion/.code=\pgf@animate@attr{motion}{motion},
  translate/.code=\pgf@animate@attr{translate}{translation},
  scale/.code=\pgf@animate@attr{scale}{scaling},
  rotate/.code=\pgf@animate@attr{rotate}{rotation},
  xskew/.code=\pgf@animate@attr{skewx}{scalar},
  yskew/.code=\pgf@animate@attr{skewy}{scalar},
  skew x/.style=xskew,
  skew y/.style=yskew,
  path/.code=\pgf@animate@attr{path}{path}
}

\def\pgf@animate@attr#1#2{%
  \expandafter\let\expandafter\pgf@do@animate\csname pgfsys@animate#1\endcsname%
  \expandafter\let\expandafter\pgfanim@type@parser\csname pgfanim@parse@type@#2\endcsname%
}



% The time parser
% 
% #1 = a time
% 
% This macro parses the time in #1, but adds some support for times:
% 
% 1) The postfix operator "s" is added, which has no effect.
% 2) The postfix operator "ms" is added, which devides a number by
%    1000, so "2ms" equals "0.002".
% 3) The postfix operator "min" is added, which multiplies a number by
%    60.
% 4) The postfix operator "h" is added, which multiplies a number by
%    3600.
% 5) The infix operator ":" is redefined, so that it multiplies its
%    first argument by 60 and adds the second. This implies that
%    "1:20" equals "80" and "01:00:00" equals "3600".
% 6) The parsing of octal numbers is switched off to allow things like
%    "01:08" for 68s.

\def\pgfparsetime#1{%
  \begingroup%
    \pgfmathdeclareoperator{s}{@seconds}{1}{postfix}{600}
    \pgfmathdeclareoperator{m}{m@encountered}{1}{postfix}{600}
    \pgfmathdeclareoperator{i}{i@encountered}{1}{postfix}{600}
    \pgfmathdeclareoperator{n}{@minutes}{1}{postfix}{600}
    \pgfmathdeclareoperator{h}{@hours}{1}{postfix}{600}
    \pgfmathdeclareoperator{:}{time@colon}{2}{infix}{50}
    \pgfmath@octalparsingfalse%
    \pgfmathparse{#1}%
  \expandafter\endgroup%
  \expandafter\def\expandafter\pgftimeresult\expandafter{\pgfmathresult}  
}

\newif\ifpgfanim@m@encountered
\newif\ifpgfanim@i@encountered
\pgfmathdeclarefunction{@seconds}{1}{%  
  \begingroup%
    \expandafter\pgfmath@x#1pt\relax%
    \ifpgfanim@m@encountered%
      \divide\pgfmath@x by1000\relax%
    \fi%
    \global\pgfanim@m@encounteredfalse%
    \pgfmath@returnone\pgfmath@x%
  \endgroup%
}%
\pgfmathdeclarefunction{m@encountered}{1}{%
  \global\pgfanim@m@encounteredtrue%
  \begingroup%
    \expandafter\pgfmath@x#1pt\relax%
    \pgfmath@returnone\pgfmath@x%
  \endgroup%
}%
\pgfmathdeclarefunction{i@encountered}{1}{%
  \ifpgfanim@m@encountered%
    \global\pgfanim@m@encounteredfalse%
    \global\pgfanim@i@encounteredtrue%
  \else%
    \pgfmath@error{Unexpected i}{}%
  \fi%
  \begingroup%
    \expandafter\pgfmath@x#1pt\relax%
    \pgfmath@returnone\pgfmath@x%
  \endgroup%
}%
\pgfmathdeclarefunction{@minutes}{1}{%  
  \ifpgfanim@i@encountered%
    \global\pgfanim@i@encounteredfalse%
  \else%
    \pgfmath@error{Unexpected n}{}%
  \fi%
  \begingroup%
    \expandafter\pgfmath@x#1pt\relax%
    \pgfmath@x60\pgfmath@x\relax%
    \pgfmath@returnone\pgfmath@x%
  \endgroup%
}%
\pgfmathdeclarefunction{@hours}{1}{%  
  \begingroup%
    \expandafter\pgfmath@x#1pt\relax%
    \pgfmath@x3600\pgfmath@x\relax%
    \pgfmath@returnone\pgfmath@x%
  \endgroup%
}%
\pgfmathdeclarefunction{time@colon}{2}{%
  \begingroup%
    \pgfmath@x=#1pt\relax%
    \pgfmath@x=60\pgfmath@x\relax%
    \pgfmath@y=#2pt\relax%
    \advance\pgfmath@x by\pgfmath@y%
    \pgfmath@returnone\pgfmath@x%
  \endgroup%
}



% Sets general animation attributes
% 
% #1 = key-value pairs that are executed for the path /pgf/animation/

\def\pgfanimationset{\pgfqkeys{/pgf/animation}}


% Sets the target animation scope that should be animated.
% 
% #1 = A name previously passed to \pgfanimationscope
%
% Example:
% 
% \begin{pgfanimationscope}{some scope}
%   \draw (0,0) -- (1,1);
%   \draw (1,0) -- (2,1);
% \end{pgfanimationscope}
%
% \pgfanimateattribute{fill opacity}{whom=some scope, duration=6s, to=0}

\pgfanimationset{
  whom/.code={%
    \expandafter\let\expandafter\pgf@anim@temp\csname pgf@anim@scope@name@#1\endcsname%
    \ifx\pgf@anim@temp\relax%
      \pgferror{No animation scope named '#1' known}
    \else%
      \pgfsys@animation@whom{\pgf@anim@temp},
    \fi%
  }
}



% Assigns a name to this animation so that it can be used as an event
% id in another animation
% 
% #1 = A name
% 
% The name "previous" is special and always refers to the most recent
% animation before the animation currently being constructed. 
%
% Example:
% 
% \pgfanimateattribute{fill opacity}{name=anim 1, duration=1s, from=1, to=0}
% \pgfanimateattribute{fill opacity}{begin on={end,of=anim 1}, duration=1s, from=0, to=1}

\pgfanimationset{
  name/.code={%
    {%
      \expandafter\xdef\csname pgf@anim@anim@name@#1\endcsname{\pgfsys@next@animation@id}%
    }%
  }
}

\def\pgf@anim@anim@name@previous{\pgfsys@last@animation@id}




% Sets the duration of an animation
% 
% #1 = The duration as a time. 
%
% Has the semantics of SVG's
% 
%   dur="#1"
%
% Example:
% 
% \pgfanimateattribute{fill opacity}{duration=6s, to=0}
% \pgfanimateattribute{fill opacity}{to=0, duration=5s}

\pgfanimationset{
  duration/.code=\pgfparsetime{#1}\pgfsys@animation@dur{\pgftimeresult},
  duration/.value required
}


% Sets the minimum / maximum  duration of an animation
% 
% #1 = The duration as a time. 
%
% Has the semantics of SVG's
% 
%   min="#1"
%
% and
%
%   max="#1"
%
% Example:
% 
% \pgfanimateattribute{fill opacity}{duration=6s, repeats, max duration=20s, to=0}

\pgfanimationset{
  min duration/.code=\pgfparsetime{#1}\pgfsys@animation@min{\pgftimeresult},
  max duration/.code=\pgfparsetime{#1}\pgfsys@animation@max{\pgftimeresult}
}



% Configures what happens when an animation ends.
% 
% When an animation ends, its "effect" can either persist or it can be
% removed. Setting "freeze at end" to true will cause it to persist,
% otherwise it will be removed.
%
% Example:
% 
% \pgfanimateattribute{fill opacity}{to=0, duration=5s, freeze at end}

\pgfanimationset{
  freeze at end/.is choice,
  freeze at end/.default=true,
  freeze at end/true/.code=\pgfsys@animation@freezeatend,
  freeze at end/false/.code=\pgfsys@animation@removeatend,
}


% Configures whether an animation can be restarted
% 
% Some animations can restart when certain events take place. This key
% configures this. Setting it to 
% 
% "true" allows a restart at any time,
% "false" does not allow a restart,
% "when not active" allows a restart only, when the element is not
% active. 
%
% Example:
% 
% \pgfanimateattribute{fill opacity}{to=0, duration=5s, restart=false}

\pgfanimationset{
  restart/.is choice,
  restart/.default=true,
  restart/true/.code=\pgfsys@animation@restart@always,
  restart/false/.code=\pgfsys@animation@restart@never,
  restart/when not active/.code=\pgfsys@animation@restart@whennotactive
}


% Sets the number times an animation should repeat
% 
% #1 = A string of the following form:
% 
% [<empty> | <number> | for <time>] ["accumulating"]
%      
% When empty (the default value) the animation repeats forever.
%
% When a <number> is provided, the animation will repeat <number>
% times, which need not be an integer. 
% 
% When a <time> is given, the repeating will stop after this much time. 
% 
% When the optional "accumulating" is specified, the repeat
% accumulates, otherwise each repeat begins with the original start
% value. 
%
% Example:
% 
% \pgfanimateattribute{fill opacity}{duration=6s, repeats, to=0}
% \pgfanimateattribute{fill opacity}{duration=6s, repeats=2, to=0}
% \pgfanimateattribute{fill opacity}{duration=6s, repeats=2 accumulating, to=0}
% \pgfanimateattribute{fill opacity}{duration=6s, repeats=for 15s, to=0}

\pgfanimationset{
  repeats/.code={%
    \pgfutil@in@{accumulating\pgf@stop}{#1\pgf@stop}%
    \ifpgfutil@in@%
      \pgfsys@animation@accumulate%
      \pgfanim@parse@acc#1\pgf@stop%
    \else%
      \pgfsys@animation@noaccumulate%
      \pgfanim@parse@noacc{#1}%
    \fi%
  },
  repeats/.default=
}
\def\pgfanim@parse@acc#1accumulating\pgf@stop{%
  \pgfanim@parse@noacc{#1}%
}
\def\pgfanim@parse@noacc#1{%
  \pgfutil@in@{\pgf@stop for}{\pgf@stop#1}%    
  \ifpgfutil@in@%
    \pgfanim@parse@for#1\pgf@stop%
  \else%
    \def\pgf@temp{#1}%
    \ifx\pgf@temp\pgfutil@empty%
      \pgfsys@animation@repeat@indefinite%
    \else% 
      \pgfmathparse{#1}%
      \pgfsys@animation@repeat{\pgfmathresult}%
    \fi%
  \fi%
}
\def\pgfanim@parse@for for#1\pgf@stop{\pgfparsetime{#1}\pgfsys@animation@repeat@dur{\pgftimeresult}}



% Specifies whether the animation values are added to the existing
% values or whether the animation values replace them.
% 
% #1 = true or false (with true being the default value, false being
% the initial value)
% 
% Example:
% 
% \pgfanimateattribute{fill opacity}{duration=1s, by=0.1, add}

\pgfanimationset{
  add/.is choice,
  add/.default=true,
  add/true/.code=\pgfsys@animation@sum,
  add/false/.code=\pgfsys@animation@replace
}



% Specifies that an animation should begin or end at a certain time
% (relative to the current context). The "begin..." options may be
% used multiple times and their effects accumulate, likewise for the
% "end..." options.
% 
% #1 = A time
% 
% Example:
% 
% \pgfanimateattribute{fill opacity}{begin=2s, end=2.5s, duration=1s, to=0}

\pgfanimationset{
  begin/.code=\pgfparsetime{#1}\pgfsys@animation@offset{\pgftimeresult}{begin},
  end/.code=\pgfparsetime{#1}\pgfsys@animation@offset{\pgftimeresult}{end},
}



% Specifies that an animation should begin (or end) when a certain
% event takes place. 
% 
% #1 = A list of key-value pairs.
% 
% The following keys are allowed:
% 
% event=some event : The begin / end is triggered by that event. The
%                    list of allowed events is defined in the SVG
%                    standard. 
% of=some name     : The event does not refer to the current group,
%                    but to the object named "some name", which must
%                    previously have been named using \pgfname
% key=some key     : The trigger is some key being pressed
% repeat=number    : The trigger is that an animation has been
%                    repeated number times.
% delay=time       : A delay, may be negative.
% 
% These keys are executed with the path prefix /pgf/animation/events.
% 
% Some of them are predefined:
% 
% "click" is a shorthand for "event=click"
% "focus in" is a shorthand for "event=focusin"
% "focus out" is a shorthand for "event=focusout"
% "activate" is a shorthand for "event=actiate"
% "mouse down" is a shorthand for "event=mousedown"
% "mouse up" is a shorthand for "event=mouseup"
% "mouse over" is a shorthand for "event=mouseover"
% "mouse move" is a shorthand for "event=mousemove"
% "mouse out" is a shorthand for "event=mouseout"
% "begin" is a shorthand for "event=begin"
% "end" is a shorthand for "event=end"
%
% Example:
%
% % Begin after 5s or when clicked:
% \pgfanimateattribute{fill opacity}{begin=5s, begin on=click, duration=1s, to=0}
% 
% % Begin, when an object named "button" is clicked
% \pgfanimateattribute{fill opacity}{begin on={click, of=button}, duration=1s, to=0, name=another animation}
%
% % Begin 1s after some other animation ends
% \pgfanimateattribute{fill opacity}{begin on={end, of=another animation, delay=1s}, duration=1s, to=0}

\pgfanimationset{
  begin on/.code={\pgfanim@make@event{#1}{begin}},%
  end on/.code={\pgfanim@make@event{#1}{end}}
}

\def\pgfanim@make@event#1#2{%
  \let\pgfanim@event@event\pgfutil@empty%
  \let\pgfanim@event@id\pgfutil@empty%
  \let\pgfanim@event@key\pgfutil@empty%
  \let\pgfanim@event@repeat\pgfutil@empty%
  \let\pgfanim@event@delay\pgfutil@empty%
  \pgfkeys{/pgf/animation/events/.cd,#1}%
  \ifx\pgfanim@event@key\pgfutil@empty%
    \ifx\pgfanim@event@repeat\pgfutil@empty%
      \ifx\pgfanim@event@event\pgfutil@empty%
        \pgferror{No event specified}%
      \else%
        \pgfsys@animation@event{\pgfanim@event@id}{\pgfanim@event@event}{\pgfanim@event@delay}{#2}%
      \fi%
    \else%
      \pgfsys@animation@repeat@event{\pgfanim@event@id}{\pgfanim@event@repeat}{\pgfanim@event@delay}{#2}%
    \fi%
  \else%
    \pgfsys@animation@accesskey{\pgfanim@event@key}{\pgfanim@event@delay}{#2}%
  \fi%
}

\pgfkeys{/pgf/animation/events/.cd,
  event/.store in=\pgfanim@event@event,
  of/.code={%
    \expandafter\let\expandafter\pgf@anim@temp\csname pgf@anim@anim@name@#1\endcsname%
    \ifx\pgf@anim@temp\relax%
      \expandafter\let\expandafter\pgf@anim@temp\csname pgf@anim@scope@name@#1\endcsname%
      \ifx\pgf@anim@temp\relax%
        \pgferror{Unknown animation name '#1'.}%
        \let\pgf@anim@temp\pgfutil@empty%
      \fi%
    \fi%
    \edef\pgfanim@event@id{\pgf@anim@temp}%
  },
  key/.store in=\pgfanim@event@key,
  repeat/.code=\pgfmathparse{#1}\let\pgfanim@event@repeat\pgfmathresult,
  delay/.code=\pgfparsetime{#1}\let\pgfanim@event@delay\pgftimeresult,
  click/.style={event=click},
  focus in/.style={event=focusin},
  focus out/.style={event=focusout},
  activate/.style={event=actiate},
  mouse down/.style={event=mousedown},
  mouse up/.style={event=mouseup},
  mouse over/.style={event=mouseover},
  mouse move/.style={event=mousemove},
  mouse out/.style={event=mouseout},
  begin/.style={event=begin},
  end/.style={event=end}
}  


% Specifies the interpolation model.
% 
% #1 = "discrete" or "linear" or "paced" or "spline"
% 
% Example:
% 
% \pgfanimateattribute{fill opacity}{interpolation=descrete, duration=3s, value=0, value=0.25, value=1}

\pgfanimationset{
  interpolation/.is choice,
  interpolation/discrete/.code=\pgfsys@animation@discrete,
  interpolation/linear/.code=\pgfsys@animation@linear,
  interpolation/paced/.code=\pgfsys@animation@paced,
  interpolation/spline/.code=\pgfsys@animation@spline
}



% Specifies a key times, key splines, and key progress.
% 
% #1 = a scalar (between 0 and 1) as a fraction of the total time
%
% Repeated uses of this key accumulate. 
%
% Example:
% 
% \pgfanimateattribute{fill opacity}{key time=0, key time=0.5, key time=1, duration=3s, value=0, value=0.25, value=1}

\pgfanimationset{
  key time/.code=\pgfmathparse{#1}\pgfsys@animation@keytime{\pgfmathresult},
  key progress/.code=\pgfmathparse{#1}\pgfsys@animation@keyprogress{\pgfmathresult},
  key spline control/.code n args={4}{%
    \pgfmathsetmacro\pgfanim@cont@a{#1}%
    \pgfmathsetmacro\pgfanim@cont@a{#2}%
    \pgfmathsetmacro\pgfanim@cont@a{#3}%
    \pgfmathsetmacro\pgfanim@cont@a{#4}%
    \pgfsys@animation@keytime{\pgfanim@cont@a}{\pgfanim@cont@b}{\pgfanim@cont@c}{\pgfanim@cont@d}%
  }
}



% Specifies a motion path.
% 
% #1 = pgf commands for constructing a path
%
% Use this key to specify a motion path for the motion attribute.
%
% Example:
% 
% \pgfanimateattribute{motion}{duration=1s, along=\pgfpathmoveto{\pgfpointorigin} \pgfpathlineto{\pgfpoint{1cm}{1cm}}}

\pgfanimationset{
  along/.code=%
  {%
    \pgfsyssoftpath@getcurrentpath\pgfanim@save@path%
    \pgfsyssoftpath@setcurrentpath\pgfutil@empty%
    #1%
    \pgfsyssoftpath@getcurrentpath\pgfanim@the@path%
    \global\let\pgfanim@the@path\pgfanim@the@path%
    \pgfsyssoftpath@setcurrentpath\pgfanim@save@path%
  }%
  \pgfsys@animation@movealong{\pgfanim@the@path}%
}


% Specifies a motion soft path.
% 
% #1 = a pgf soft path (result of calling \pgfsyssoftpath@getcurrentpath)
%
% Use this key to specify a motion path for the motion attribute when
% a softpath has already been constructed.
%
% Example:
% 
% \pgfanimateattribute{motion}{duration=1s, 
%    along softpath=\pgfsyssoftpath@movetotoken {0pt}{0pt}\pgfsyssoftpath@linetotoken {10pt}{10pt}}

\pgfanimationset{
  along softpath/.code=\pgfsys@animation@movealong{#1}%
}



% Configures whether the motion is rotated along the path.
% 
% Example:
% 
% \pgfanimateattribute{motion}{rotate along, duration=1s, along=\pgfpathmoveto{\pgfpointorigin} \pgfpathlineto{\pgfpoint{1cm}{1cm}}}

\pgfanimationset{
  rotate along/.is choice,
  rotate along/.default=true,
  rotate along/true/.code=\pgfsys@animation@rotatealong,
  rotate along/false/.code=\pgfsys@animation@norotatealong,
}



% Specifies a value in a sequence of values.
% 
% #1 = A value. The format of this value depends on the current type
% of the animation attribute. (So for an animation of the "fill
% opacity", #1 must be a scalar value, for an animation of the "fill"
% color, #1 must be a color, and so on.)
%
% Repeated uses of this key accumulate.
%
% Example:
% 
% \pgfanimateattribute{fill opacity}{duration=3s, value=0, value=0.25, value=1}

\pgfanimationset{value/.code=\pgfanim@type@parser{#1}{values}}
\pgfanimationset{values/.code=\foreach\pgf@anim@val in{#1}{\pgfanim@type@parser{\pgf@anim@val}{values}}}


% Specifies a value for the from, to, or by keys of an animation.
% 
% #1 = A value whose format depends on the current type of the
% animation attribute. 
%
% Repeated uses of these keys do not accumulate, the last value gets replaced.
%
% Example:
% 
% \pgfanimateattribute{fill opacity}{duration=3s, from=0, to=1}

\pgfanimationset{
  from/.code=\pgfanim@type@parser{#1}{from},
  to/.code=\pgfanim@type@parser{#1}{to},
  by/.code=\pgfanim@type@parser{#1}{by},
}


% The parsers

% Parse a scalar in "value=#1"
% 
% #1 is a number parsed using \pgfmathparse
% 
% Example:
% 
% value = 5+6
% to = 9.5

\def\pgfanim@parse@type@scalar#1#2{%
  \pgfmathparse{#1}%
  \pgfsys@animation@scalar{\pgfmathresult}{#2}%
}


% Parse a dimension in "value=#1"
% 
% #1 is a number parsed using \pgfmathparse
% 
% Example:
% 
% value = 5cm + 2pt

\def\pgfanim@parse@type@dimension#1#2{%
  \pgfmathparse{#1}%
  \pgfsys@animation@scalar{\pgfmathresult}{#2}%
}


% Parse a color in "value=#1"
% 
% #1 is a color
% 
% from = red
% value = blue!20

\def\pgfanim@parse@type@color#1#2{%
  \pgfutil@colorlet{pgf@anim@temp}{#1}%
  \pgfutil@ifundefined{applycolormixins}{}{\applycolormixins{pgf@anim@temp}}%
  \expandafter\let\expandafter\pgf@temp\csname\string\color@pgf@anim@temp\endcsname
  % for arrow tips:
  \expandafter\pgfanim@parse@type@color@\pgf@temp{#2}%
}
\def\pgfanim@parse@type@color@#1#2#3#4#5#6{%
  \expandafter\ifx\csname pgfsys@animation@color@#4\endcsname\relax%
    \pgferror{Unsupported color model `#4'. Sorry}%
  \else%
    \edef\pgf@colmarshal{\expandafter\noexpand\csname pgfsys@animation@color@#4\endcsname}%
    \pgf@uncomma#5,,%
    \pgf@colmarshal{#6}%
  \fi%
}


% Parse a path in "value=#1"
% 
% #1 is a sequence of pgf path construction commands
% 
% value = \pgfpathmoveto{\pgfpointorigin} \pgfpathlineto{\pgfpoint{1cm}{1cm}}

\def\pgfanim@parse@type@path#1#2{%
  {%
    \pgfsyssoftpath@getcurrentpath\pgfanim@save@path%
    \pgfsyssoftpath@setcurrentpath\pgfutil@empty%
    #1%
    \pgfsyssoftpath@getcurrentpath\pgfanim@the@path%
    \global\let\pgfanim@the@path\pgfanim@the@path%
    \pgfsyssoftpath@setcurrentpath\pgfanim@save@path%
  }%
  \pgfsys@animation@path{\pgfanim@the@path}{#2}%
}


% Parse a translation in "value=#1"
% 
% #1 is a point
% 
% value = \pgfpoint{1cm}{2cm}

\def\pgfanim@parse@type@translation#1#2{%
  \pgf@process{#1}%
  \pgfsys@animation@translate{\pgf@x}{\pgf@y}{#2}%
}


% Parse a scaling in "value=#1"
% 
% #1 = if #1 contains a comma, the two parts are passed to
% pgfmathparse, otherwise the whole thing is passed to pgfmathparse. 
% 
% value = 2
% value = {2,1}

\def\pgfanim@parse@type@scaling#1#2{%
  \pgfutil@in@{,}{#1}%
  \ifpgfutil@in@%
    \pgfanim@unpack@comma#1\pgf@stop{#2}%
  \else%
    \pgfmathparse{#1}%
    \pgfsys@animation@scale{\pgfmathresult}{}{#2}%
  \fi%
}
\def\pgfanim@unpack@comma#1,#2\pgf@stop#3{%
  \pgfmathsetmacro{\pgf@anim@temp@x}{#1}%
  \pgfmathsetmacro{\pgf@anim@temp@y}{#2}%
  \pgfsys@animation@scale{\pgf@anim@temp@x}{\pgf@anim@temp@y}{#3}%
}



% Parse a rotation in "value=#1"
% 
% #1 = if #1 contains a comma, the first part is a number, the second
% must be a point. Otherwise, #1 is just a number.
% 
% value = 90
% value = {45, \pgfpoint{1cm}{1cm}}

\def\pgfanim@parse@type@rotation#1#2{%
  \pgfutil@in@{,}{#1}%
  \ifpgfutil@in@%
    \pgfanim@unpack@comma@rot#1\pgf@stop{#2}%
  \else%
    \pgfmathparse{#1}%
    \pgfsys@animation@rotate{\pgfmathresult}{0pt}{0pt}{#2}%
  \fi%
}
\def\pgfanim@unpack@comma@rot#1,#2\pgf@stop#3{%
  \pgf@process{#2}%
  \pgfmathparse{#1}%
  \pgfsys@animation@rotate{\pgfmathresult}{\pgf@x}{\pgf@y}{#3}%
}




\endinput

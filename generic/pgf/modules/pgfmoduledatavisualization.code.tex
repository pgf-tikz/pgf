% Copyright 2006 by Till Tantau
%
% This file may be distributed and/or modified
%
% 1. under the LaTeX Project Public License and/or
% 2. under the GNU Public License.
%
% See the file doc/generic/pgf/licenses/LICENSE for more details.

\ProvidesFileRCS $Header$

\usepgfmodule{oo}


% This module defines the basic framework for data visualization. 

% In order to visualize data, you first need data. The format for this
% data is not specified, indeed, different formats are possible. A
% data point is created each time the command \pgfdatapoint is
% used. The "parameters" of the data point are just the current values
% of the keys or macros in the current scope. 
%
% A set of data points is framed by calls to \pgfdataset and
% \pgfenddataset, which just signal the events "begin data set" and
% "end data set". Inside a data set, you can use \pgfdatacluster
% and \endpgfdatacluster to further cluster data. These events will
% cause the "begin data cluster" and "end data cluster" events to be
% issued. Every data set implicitly opens a cluster.
%
% When a data point is created, a numer of events are signaled, see
% the description of \pgfdatapoint. To actually visualize something,
% objects should be created that listen to these events and that
% handle them. 
%
% The command \pgfnewdatavisualization inits these events. 


% Initialize a data visualization
%
% Description:
%
% Use this command to start a data visualization. It will clear
% (locally for the current group) all event listeners.

\def\pgfnewdatavisualization{%
  \pgfooinitevent{prepare data point}%
  \pgfooinitevent{locate data point}%
  \pgfooinitevent{transform data point}%
  \pgfooinitevent{visualize data point}%
  \pgfooinitevent{finish data point}%
  \pgfooinitevent{begin data set}%
  \pgfooinitevent{end data set}%
  \pgfooinitevent{begin data cluster}%
  \pgfooinitevent{end data cluster}%
}      




% Create and handle a data point
%
% Description:
%
% This command may only be called after \pgfnewdatavisualization has
% been called. It will then signal a number of events, to which
% (hopefully) some smart objects are listening.
%
% The data that is represented by the data point is not given as a
% parameter. Rather, it is stored in macros and keys, that is, the
% data point is conceptually given by the settgins of all the keys and
% macros in the local scope.
%
% In detail, a prepare event is signaled first, given all objects a
% chance to "prepare" for the data point. Note that it is permissible
% for an object to manipulate the data point in this phase (and also
% in later phases).
%
% Next, the command \pgfcanvaspositionofdatapoint is called. Mainly, the effect
% of this command is to setup the keys /data point/canvas x and /data
% point/canvas y, see the description of this command for more
% details. 
%
% The next phase consists of signaling the "visualize data point"
% event. Objects listening to this event will cause some form of
% visualization of the data point to occur.
%
% Finally, finish data point allows objects to do any final processing
% of the data point.

\def\pgfdatapoint{%
  \pgfooevent{prepare data point}%
  \pgfcanvaspositionofdatapoint%
  \pgfooevent{visualize data point}%
  \pgfooevent{finish data point}%
}



% Compute a position of a data point
%
% Description:
%
% This command uses a special event to compute the position where a
% data point should be visualized on the canvas. In detail, the
% following happens:
%
% The next phase is special: A local scope is created and the
% transformation matrix is reset. Then, two events are signled: First,
% "locate data points" and then "transform data point". The first
% event lets listening objects "locate" the object by setting up
% attributes of the data point. The second event asks objects
% listening to this event to transform  the current transformation
% matrix. After the events, we compute where 
% the origin lies inside this transformed coordinate system. Then the
% two keys /data point/canvas x and /data point/canvas y are set to
% the values of this position. The local scope ends (but the settings
% of the keys persist by a bit of magic), thus restoring the
% transformation matrix to its original value.

\def\pgfcanvaspositionofdatapoint{%
  \pgf@canvaspositionofdatapoint{\pgfdatapointvirtualfalse}%
}
\def\pgf@canvaspositionofdatapoint#1{%
  {
    \pgftransformreset
    #1%
    \pgfooevent{locate data point}
    \pgfooevent{transform data point}%
    \pgfpointtransformed{\pgfpointorigin}
    % Smuggle outside group
    \expandafter
  }%
  \edef\pgf@marshal{%
    \noexpand\pgfkeyssetvalue{/data point/canvas x}{\the\pgf@x}
    \noexpand\pgfkeyssetvalue{/data point/canvas y}{\the\pgf@y}
  }%
  \pgf@marshal%
}

% Help functions for locating a canvas data point
%
% Description:
%
% The first function returns the data point computed by a
% canvasposition... call. The second function stores this position in
% a macro

\def\pgfpointcanvasposition{%
  \pgfqpoint{\pgfkeysvalueof{/data point/canvas x}}{\pgfkeysvalueof{/data point/canvas y}}%
}

\def\pgfsettocanvasposition#1{%
  \edef#1{\noexpand\pgfqpoint{\pgfkeysvalueof{/data point/canvas x}}{\pgfkeysvalueof{/data point/canvas y}}}%
}



% Compute a position of a virtual data point
%
% Description:
%
% This macors works like the previous macor, but the data point is
% considered to be "virtual". This happens when you wish to compute
% the position where a data point should be visualized even when no
% "real" data point is there. For virtual data points, bounders and
% coordinate trackers should not update internal data
% structures. They can detect whether a data point is virtual by
% testing \ifpgfdatapointvirtual

\newif\ifpgfdatapointvirtual
\def\pgfcanvaspositionofvirtualdatapoint{%
  \pgf@canvaspositionofdatapoint{\pgfdatapointvirtualtrue}%
}

\pgfkeys{% do not even think of changing the values of the following:
  /data point/always true/.initial=true,
  /data point/always false/.initial=false,
  /data point/always 0/.initial=0,
  /data point/always 1/.initial=1,
  /data point/always empty/.initial=%
}


% Start/stop a data set
%
% Description:
%
% These command is used to signal the start/end of a data set. They
% cause the "begin data set" or "end data set" event to be
% signaled. Objects listening for these event can use it to setup
% their internal settings. 

\def\pgfdataset{\pgfooevent{begin data set}\pgfooevent{begin data cluster}}
\def\endpgfdataset{\pgfooevent{end data cluster}\pgfooevent{end data set}}

\let\startpgfdataset=\pgfdataset
\let\stoppgfdataset=\endpgfdataset



% Start/stop a data cluster
%
% Description:
%
% Inside a data set the data can be further clustered using these
% commands. Like the start/stop of data sets, they signal events.

\def\pgfdatacluster{\pgfooevent{begin data cluster}}
\def\endpgfdatacluster{\pgfooevent{end data cluster}}

\let\startpgfdatacluster=\pgfdatacluster
\let\stoppgfdatacluster=\endpgfdatacluster
















%
%
% Standard objects for data visualization
% 
%


%
% Transformers
%

\pgfooclass{line transformer}
{
  \method line transformer(#1,#2) {
    \pgfoothis.listen to(transform data point)
    \pgfooset{attribute}{#1}
    \pgfooset{origin}{\pgfpointorigin}%
    \pgfooset{unit}{#2}%
  }

  \method set origin(#1) {
    \pgfooset{origin}{#1}
  }

  \method transform data point() {
    \pgfkeysgetvalue{/data point/\pgfoovalue{attribute}}\pgf@dv@val%
    \ifx\pgf@dv@val\pgfutil@empty%
    \else%
      \ifx\pgf@dv@val\relax%
      \else%
        \pgftransformshift{\pgfoovalue{origin}}%
        \pgftransformshift{\pgfpointscale{\pgf@dv@val}{\pgfoovalue{unit}}}%
      \fi%
    \fi%
  }
}




%
% Locators
%

\pgfooclass{attribute locator}
{
  \method attribute locator(#1,#2,#3,#4,#5,#6,#7) {
    \pgfoothis.listen to(locate data point)
    {%
      % Let's start with the output, it's easier...
      \pgfooset{out}{#4}
      \pgfmathparse{#5}
      \pgfooeset{out min}{\pgfmathresult}
      \pgfmathsubtract{#6}{\pgfoovalue{out min}}%
      \let\pgf@dv@out@diff=\pgfmathresult%
      % Now comes the transformation code:
      \def\pgf@temp{#7}%
      \ifx\pgf@temp\pgfutil@empty%
      \else%
        \pgfooset{trans}{%
          \let\pgfvalue\pgfmathresult%
          \pgfmathparse{#7}%
        }%
      \fi%
      % Ok, now the transformed input:
      \pgfooset{in}{#1}
      \pgfmathparse{#2}
      \pgfoovalue{trans}%
      \let\pgf@dv@min=\pgfmathresult
      \pgfooeset{trans in min}{\pgfmathresult}
      \pgfmathparse{#3}
      \pgfoovalue{trans}%
      \pgfmathsubtract{\pgfmathresult}{\pgfoovalue{trans in min}}%
      % 
      % Precompute the scaling
      %
      \pgfmathdivide{\pgf@dv@out@diff}{\pgfmathresult}%
      \pgfooeset{scale}{\pgfmathresult}%
    }%
  } 

  \method locate data point() {
    \pgfkeysgetvalue{/data point/\pgfoovalue{in}}\pgfmathresult%
    \ifx\pgfmathresult\pgfutil@empty%
    \else%
      \ifx\pgfmathresult\relax%
      \else%
        \pgfoovalue{trans}%
        \pgfmathsubtract{\pgfmathresult}{\pgfoovalue{trans in min}}%
        \pgfmathmultiply{\pgfmathresult}{\pgfoovalue{scale}}%
        \pgfmathadd{\pgfmathresult}{\pgfoovalue{out min}}%
        \pgfkeyslet{/data point/\pgfoovalue{out}}\pgfmathresult%
      \fi%
    \fi%
  }
}


\pgfooclass{polar locator}
{
  \method polar locator(#1,#2,#3,#4) {
    \pgfoothis.listen to(locate data point)
    \pgfooset{angle}{#1}
    \pgfooset{radius}{#2}
    \pgfooset{x}{#3}
    \pgfooset{y}{#4}
  } 

  \method locate data point() {
    \pgfkeysgetvalue{/data point/\pgfoovalue{angle}}\pgf@dv@angle%
    \ifx\pgf@dv@angle\pgfutil@empty%
    \else%
      \ifx\pgf@dv@angle\relax%
      \else%
        \pgfkeysgetvalue{/data point/\pgfoovalue{radius}}\pgf@dv@radius%
        \ifx\pgf@dv@radius\pgfutil@empty%
        \else%
          \ifx\pgf@dv@radius\relax%
          \else%
            \pgfmathsincos{\pgf@dv@angle}%
            \pgfmathmultiply{\pgfmathresultx}{\pgf@dv@radius}%
            \pgfkeyslet{/data point/\pgfoovalue{x}}\pgfmathresult%
            \pgfmathmultiply{\pgfmathresulty}{\pgf@dv@radius}%
            \pgfkeyslet{/data point/\pgfoovalue{y}}\pgfmathresult%
          \fi%
        \fi%
      \fi%
    \fi%
  }

  \method get angle and radius attributes() {
    \pgfooget{angle}\pgfdvangleattribute
    \pgfooget{radius}\pgfdvradiusattribute
  }
}



%
% Bounders
%

\pgfooclass{attribute bounder}
{
  
  \method attribute bounder(#1,#2) {
    \pgfoothis.listen to(locate data point)
    \pgfooset{attribute}{#1}
    \pgfoolet{interval obj}#2
    \pgfooset{drop min}{}
    \pgfooset{drop max}{}
    \pgfooset{lift min}{}
    \pgfooset{lift max}{}
    \pgfooset{ignore min}{}
    \pgfooset{ignore max}{}
  }

  \method drop values below(#1) {
    \pgfooset{drop min}{#1}
  }

  \method drop values above(#1) {
    \pgfooset{drop max}{#1}
  }

  \method ignore values below(#1) {
    \pgfooset{ignore min}{#1}
  }

  \method ignore values above(#1) {
    \pgfooset{ignore max}{#1}
  }

  \method raise values lower than(#1) {
    \pgfooset{lift min}{#1}
  }

  \method lower values higher than(#1) {
    \pgfooset{lift max}{#1}
  }

  \method goto min() {
    \pgfooget{interval obj}\pgf@dv@interval
    \pgf@dv@interval.get interval()
    \pgfkeyslet{/data point/\pgfoovalue{attribute}}\pgfdvmin
  }

  \method goto max() {
    \pgfooget{interval obj}\pgf@dv@interval
    \pgf@dv@interval.get interval()
    \pgfkeyslet{/data point/\pgfoovalue{attribute}}\pgfdvmax
  }

  \method goto adjusted min() {
    \pgfooget{interval obj}\pgf@dv@interval
    \pgf@dv@interval.get adjusted interval()
    \pgfkeyslet{/data point/\pgfoovalue{attribute}}\pgfdvmin
  }

  \method goto adjusted max() {
    \pgfooget{interval obj}\pgf@dv@interval
    \pgf@dv@interval.get adjusted interval()
    \pgfkeyslet{/data point/\pgfoovalue{attribute}}\pgfdvmax
  }
  
  \method locate data point() {
    \ifpgfdatapointvirtual%
    \else%
      \pgfkeysgetvalue{/data point/\pgfoovalue{attribute}}\pgf@dv@val%
      \ifx\pgf@dv@val\pgfutil@empty%
      \else%
        \ifx\pgf@dv@val\relax%
        \else%
          \pgfmathparse{\pgf@dv@val}%
          % Ignore?
          \pgfooget{ignore min}\pgf@dv@ignore%
          \ifx\pgf@dv@ignore\pgfutil@empty%
          \else%
            \ifdim\pgfmathresult pt<\pgf@dv@ignore pt%
              \pgf@dvignoretrue%
            \fi%
          \fi%
          \pgfooget{ignore max}\pgf@dv@ignore%
          \ifx\pgf@dv@ignore\pgfutil@empty%
          \else%
            \ifdim\pgfmathresult pt>\pgf@dv@ignore pt%
              \pgf@dvignoretrue%
            \fi%
          \fi%
          \ifpgf@dvignore%
          \else%
            % Drop?
            \pgfooget{drop min}\pgf@dv@drop%
            \ifx\pgf@dv@drop\pgfutil@empty%
            \else%
              \ifdim\pgfmathresult pt<\pgf@dv@drop pt%
                \pgf@dvdroptrue%
              \fi%
            \fi%
            \pgfooget{drop max}\pgf@dv@drop%
            \ifx\pgf@dv@drop\pgfutil@empty%
            \else%
              \ifdim\pgfmathresult pt>\pgf@dv@drop pt%
                \pgf@dvdroptrue%
              \fi%
            \fi%
            \ifpgf@dvdrop%
              \pgfkeyslet{/data point/\pgfoovalue{attribute}}\pgfutil@empty%
            \else%
              % Lift?
              \pgfooget{lift min}\pgf@dv@lift%
              \ifx\pgf@dv@lift\pgfutil@empty%
              \else%
                \ifdim\pgfmathresult pt<\pgf@dv@lift pt%
                  \let\pgfmathresult=\pgf@dv@lift%
                  \pgfkeyslet{/data point/\pgfoovalue{attribute}}\pgfmathresult            
                \fi%
              \fi%
              % Lift?
              \pgfooget{lift max}\pgf@dv@lift%
              \ifx\pgf@dv@lift\pgfutil@empty%
              \else%
                \ifdim\pgfmathresult pt>\pgf@dv@lift pt%
                  \let\pgfmathresult=\pgf@dv@lift%
                  \pgfkeyslet{/data point/\pgfoovalue{attribute}}\pgfmathresult            
                \fi%
              \fi%
              % Now, protocol value
              \pgfooget{interval obj}\pgf@dv@interval
              \pgf@dv@interval.get interval()%
              \ifx\pgfdvmin\pgfutil@empty%
                \pgf@dv@interval.set min(\pgfmathresult)%
              \else%
                \ifdim\pgfmathresult pt<\pgfdvmin pt%
                  \pgf@dv@interval.set min(\pgfmathresult)%
                \fi%
              \fi%
              \ifx\pgfdvmax\pgfutil@empty%
                \pgf@dv@interval.set max(\pgfmathresult)%
              \else%
                \ifdim\pgfmathresult pt>\pgfdvmax pt%
                  \pgf@dv@interval.set max(\pgfmathresult)%
                \fi%
              \fi%
            \fi%
          \fi%
        \fi%
      \fi%
    \fi%
  }

  \newif\ifpgf@dvdrop
  \newif\ifpgf@dvignore
}



%
% Help class: Interval
%

\pgfooclass{interval}
{
  \method interval(#1,#2) {
    \pgfooeset{min}{#1}
    \pgfooeset{max}{#2}
    \pgfooset{min adjust absolute}{}
    \pgfooset{min adjust relative}{}
    \pgfooset{max adjust absolute}{}
    \pgfooset{max adjust relative}{}
  }
  \method set min(#1) {
    \pgfooeset{min}{#1}
  }
  \method set max(#1) {
    \pgfooeset{max}{#1}
  }
  \method set min adjust absolute(#1) {
    \pgfooset{min adjust absolute}{#1}    
  }
  \method set min adjust relative(#1) {
    \pgfooset{min adjust relative}{#1}    
  }
  \method set max adjust absolute(#1) {
    \pgfooset{max adjust absolute}{#1}    
  }
  \method set max adjust relative(#1) {
    \pgfooset{max adjust relative}{#1}    
  }

  \method get interval() {
    \pgfooget{min}\pgfdvmin
    \pgfooget{max}\pgfdvmax
  }

  \method get adjusted interval() {
    \pgfooget{min}\pgfdvmin
    \pgfooget{max}\pgfdvmax
    \ifx\pgfdvmin\pgfutil@empty%
    \else%
      \ifx\pgfdvmax\pgfutil@empty%
      \else%
        \pgfooget{min adjust absolute}\pgf@dv@minpa
        \pgfooget{min adjust relative}\pgf@dv@minpr
        \pgfooget{max adjust absolute}\pgf@dv@maxpa
        \pgfooget{max adjust relative}\pgf@dv@maxpr
        \pgfmathsubtract{\pgfdvmax}{\pgfdvmin}%
        \let\pgf@dv@diff\pgfmathresult%
        % Start with minimum
        \let\pgfmathresult\pgfdvmin%
        \ifx\pgf@dv@minpa\pgfutil@empty%
        \else%
          \pgfmathadd{\pgfmathresult}{\pgf@dv@minpa}%
        \fi%
        \ifx\pgf@dv@minpr\pgfutil@empty%
        \else%
          \pgfmathadd{\pgfmathresult}{\pgf@dv@minpr*\pgf@dv@diff}%
        \fi%
        \let\pgfdvmin\pgfmathresult%  
        % Maximum next
        \let\pgfmathresult\pgfdvmax%
        \ifx\pgf@dv@maxpa\pgfutil@empty%
        \else%
          \pgfmathadd{\pgfmathresult}{\pgf@dv@maxpa}%
        \fi%
        \ifx\pgf@dv@maxpr\pgfutil@empty%
        \else%
          \pgfmathadd{\pgfmathresult}{\pgf@dv@maxpr*\pgf@dv@diff}%
        \fi%
        \let\pgfdvmax\pgfmathresult%  
      \fi%
    \fi%
  }
}


\pgfooclass{count up}
{
  \method count up(#1) {
    \pgfoothis.listen to(prepare data point)
    \pgfooset{attribute}{#1}
    \pgfooset{val}{0}
  }

  \method set value(#1) {
    \pgfooset{val}{#1}
    \pgfkeyssetvalue{/data point/\pgfoovalue{attribute}}{#1}%
  }

  \method prepare data point() {
    {%
      \c@pgf@counta=\pgfoovalue{val}\relax%
      \advance\c@pgf@counta by1\relax%
      \pgfooeset{val}{\the\c@pgf@counta}%
    }%
    \pgfooget{val}\pgf@dv@temp%
    \pgfkeyslet{/data point/\pgfoovalue{attribute}}\pgf@dv@temp%
  }  
}


\pgfooclass{count down}
{
  \method count down(#1,#2) {
    \pgfoothis.listen to(prepare data point)
    \pgfooset{attribute}{#1}
    \pgfooset{max}{#2}
    \pgfooset{val}{#2}
  }

  \method set value(#1) {
    \pgfooset{val}{#1}
    \pgfkeyssetvalue{/data point/\pgfoovalue{attribute}}{#1}%
  }

  \method prepare data point() {
    {%
      \c@pgf@counta=\pgfoovalue{val}\relax%
      \advance\c@pgf@counta by-1\relax%
      \ifnum\c@pgf@counta<0\relax%
        \advance\c@pgf@counta by\pgfoovalue{max}\relax%
      \fi%
      \pgfooeset{val}{\the\c@pgf@counta}%
    }%
    \pgfooget{val}\pgf@dv@temp%
    \pgfkeyslet{/data point/\pgfoovalue{attribute}}\pgf@dv@temp%
  }  
}



\pgfooclass{accumulator}
{
  \method accumulator(#1,#2) {
    \pgfoothis.listen to(prepare data point)
    \pgfooeset{what}{#1}
    \pgfooeset{where}{#2}
    \pgfkeysifdefined{/data point/#2}{}{\pgfkeyssetvalue{/data point/#2}{0}}
  }

  \method prepare data point() {
    \pgfmathadd{\pgfkeysvalueof{/data point/\pgfoovalue{where}}}{\pgfkeysvalueof{/data point/\pgfoovalue{what}}}
    \pgfkeyslet{/data point/\pgfoovalue{where}}\pgfmathresult
  }
}


%
% Axes
%

\pgfooclass{straight axis}
{
  \method straight axis(#1,#2) {
    \pgfooset{attribute}{#1}
    \pgfoolet{interval obj}#2
    \pgfooset{use path}{\pgfusepath{stroke}}
  }

  \method set use path(#1) {
    \pgfooset{use path}{#1}
  }
  
  \method visualize axis() {
    {%
      % we assume that the current data point settings determine where
      % the axis should go
      %
      % We now compute the start and the end points
      %
      \pgfooget{interval obj}\pgf@dv@interval
      \pgf@dv@interval.get adjusted interval()%
      % Ok, got the values
      % Save them.
      \let\pgf@dv@sa@min=\pgfdvmin
      \let\pgf@dv@sa@max=\pgfdvmax
      \pgfkeyssetvalue{/data point/\pgfoovalue{attribute}}{\pgf@dv@sa@min}%
      \pgfcanvaspositionofvirtualdatapoint%
      % Ok, that's where we start
      \pgfpathmoveto{\pgfpointcanvasposition}%
      % now the end:
      \pgfkeyssetvalue{/data point/\pgfoovalue{attribute}}{\pgf@dv@sa@max}%
      \pgfcanvaspositionofvirtualdatapoint%
      \pgfpathlineto{\pgfpointcanvasposition}%
      % Finally, use the path:
      \pgfoovalue{use path}%
    }%
  }
}



\pgfooclass{polar axis}
{
  \method polar axis(#1,#2) {
    \pgfoolet{locator obj}#1
    \pgfoolet{interval obj}#2
    \pgfooset{use path}{\pgfusepath{stroke}}
  }

  \method set use path(#1) {
    \pgfooset{use path}{#1}
  }
  
  \method visualize axis() {
    {%
      %
      \pgfooget{locator obj}\pgf@dv@trans%
      \pgf@dv@trans.get angle and radius attributes()%
      % we assume that the current data point settings determine where
      % the axis should go
      %
      % we also assume that the axis will, indeed, be visualized as an
      % arc. 
      %
      % Compute the origin:
      {%
        \pgfkeyssetvalue{/data point/\pgfdvradiusattribute}{0}%
        \pgfcanvaspositionofvirtualdatapoint%
        \pgfsettocanvasposition\pgf@temp
        \global\let\pgf@dv@temp@orig=\pgf@temp%
      }%
      % Compute the first axis:
      \pgfkeyssetvalue{/data point/\pgfdvangleattribute}{0}%
      \pgfcanvaspositionofvirtualdatapoint%
      \pgfsettocanvasposition\pgf@dv@veczero%
      % Compute the second axis:
      \pgfkeyssetvalue{/data point/\pgfdvangleattribute}{90}%
      \pgfcanvaspositionofvirtualdatapoint%
      \pgfsettocanvasposition\pgf@dv@vecninety%
      %
      % Start and end angle
      %
      \pgfooget{interval obj}\pgf@dv@interval
      \pgf@dv@interval.get adjusted interval()%
      % Ok, got the values
      % Compute the first axis:
      \pgfkeyssetvalue{/data point/\pgfdvangleattribute}{\pgfdvmin}%
      \pgfcanvaspositionofvirtualdatapoint%
      \pgfpathmoveto{\pgfpointcanvasposition}%
      % Do arc:
      \pgftransformtriangle{\pgf@dv@temp@orig}{\pgf@dv@veczero}{\pgf@dv@vecninety}%
      \pgfpatharc{\pgfdvmin}{\pgfdvmax}{1pt}%
      % Finally, use the path:
      \pgfoovalue{use path}%
    }%
  }
}





%
% Visualizers
%

\usepgflibrary{plothandlers}


\pgfooclass{plot mark visualizer}
{
  \method plot mark visualizer(#1) {
    \pgfoothis.listen to(visualize data point)
    \pgfooeset{mark}{#1}
    \pgfooeset{zero trigger}{always 0}
    \pgfooeset{size}{\pgfplotmarksize}
  }

  \method set size(#1) {
    \pgfooset{size}{#1}
  }

  \method set zero trigger(#1) {
    \pgfooset{zero trigger}{#1}
  }
  
  \method visualize data point() {
    {%
      \pgfkeysgetvalue{/data point/\pgfoovalue{zero trigger}}\pgf@dv@trig%
      \ifx\pgf@dv@trig\pgf@dv@zero@text%
        \pgfplotmarksize\pgfoovalue{size}
        \pgftransformshift{\pgfpointcanvasposition}
        \pgfuseplotmark{\pgfoovalue{mark}}
      \fi%
    }%
  }

  \def\pgf@dv@zero@text{0}
}



\pgfooclass{plot handler visualizer}
{
  \method plot handler visualizer(#1) {
    \pgfoothis.listen to(visualize data point)
    \pgfoothis.listen to(end data cluster)
    \pgfooset{handler}{#1}
    \pgfooset{use path}{\pgfusepath{stroke}}
    \pgfooset{positions}{}
  }

  \method set use path(#1) {
    \pgfooset{use path}{#1}
  }
  
  \method visualize data point() {
    % Just append it to positions
    \pgfooget{positions}\pgf@dv@temp
    \edef\pgf@dv@add{\noexpand\pgf@dv@ph{\pgfkeysvalueof{/data point/canvas x}}{\pgfkeysvalueof{/data point/canvas y}}}%
    \expandafter\expandafter\expandafter\def%
    \expandafter\expandafter\expandafter\pgf@dv@temp%
    \expandafter\expandafter\expandafter{\expandafter\pgf@dv@temp\pgf@dv@add}%
    \pgfoolet{positions}\pgf@dv@temp
  }

  \method end data cluster() {
    \pgfoovalue{handler}
    \pgfplotstreamstart
    \pgfoovalue{positions}
    \pgfplotstreamend
    \pgfoovalue{use path}
    \pgfoolet{positions}\pgfutil@empty
  }

  \def\pgf@dv@ph#1#2{\pgfplotstreampoint{\pgfqpoint{#1}{#2}}}
}




\endinput

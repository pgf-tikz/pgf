% Copyright 2007 by Till Tantau
%
% This file may be distributed and/or modified
%
% 1. under the LaTeX Project Public License and/or
% 2. under the GNU Public License.
%
% See the file doc/generic/pgf/licenses/LICENSE for more details.


% The purpose of this file is to provide a general settings engine that 
% works with all TeX formats and has no save-stack impact


% This is useful:

\def\pgfutil@ifcsname#1\endcsname#2\else#3\fi{\expandafter\ifx\csname#1\endcsname\relax#3\else#2\fi}%
\ifx\eTeXrevision\undefined%
\else%
  \expandafter\let\expandafter\pgfutil@ifcsname\csname ifcsname\endcsname%
\fi



% Set a key to a value
%
% #1 = key
% #2 = a parameter pattern
% #3 = a value
%
% Description:
%
% This command sets the key to the given value. The value can be
% parametrized (that is, it may contain parameters like #1 or
% #2). This allows you to modify the value of the key when it is
% retrieved. The list of parameters (more generally, the parameter
% pattern) is given in the second argument.
%
% Keys are organized hierarchically using something similar to Unix
% paths. Thus, a typically key might be called "/tikz/length" or
% "/tikz/length dimension/.code". Some keys starting with a dot are
% special, so they should not be used as normal key names (they are
% similar to Unix files starting with a dot -- you can use them, but
% be careful).
%
% Keys are always local to the current TeX group.
%
% Example:
%
% \pgfsetkey{/tikz/length/.value}{}{2cm-3cm}
% \pgfsetkey{/algo/swap}{#1#2}{{#2}{#1}}

\def\pgfsetkey#1#2#3{%
  \expandafter\def\csname pgfk@#1\endcsname#2{#3}%
}




% Makes a key equal a given macro
%
% #1 = key
% #2 = a macro name
%
% Description:
%
% This command executes a \let command so that when the key is
% executed, then the parameter #2 is executed "instead".
%
% Keys are always local to the current TeX group.
%
% Example:
%
% \pgfletkey{/algo/swap}{\myswap}

\def\pgfletkey#1#2{%
  \expandafter\let\csname pgfk@#1\endcsname#2%
}



% Retrieve the value stored in a key
%
% #1 = key
%
% Description:
%
% This command will expand to the value stored in the key. If the key
% is parametrized, the parameters should follow. The key should
% previously have been set using \pgfsetkey. 
%
% Example:
%
% The length is \pgfdokey{/tikz/length}.
% This \pgfdokey{/algo/swap}{correct}{is }.

\def\pgfdokey#1{\csname pgfk@#1\endcsname}



% Retrieve the code stored in a key into a macro
%
% #1 = key
% #2 = macro
%
% Description:
%
% This command will make #2 have the same effect as \pgfdokey{#1}.
%
% Example:
%
% \pgfretrievekeycode{/tikz/swap}{\myswap}

\def\pgfretrievekeycode#1#2{\expandafter\let\expandafter#2\csname pgfk@#1\endcsname}



% Execute settings
%
% #1 = list of settings
%
% Description:
%
% The list of settings should contain comma-separated settings. Each
% setting has the following form:
%
% /path/key=value
%
% The parts "/path/" and "=value" are optional. When the path is not
% specified, the value of the token register "\pgfkeypath" is used. If
% "=value" is missing, the value of the setting "/path/key/.default" is used
% instead. If this key is set to "\pgfvaluerequired", the key
% "/errors/value required/.code" is executed. The macro \pgfcurrentkey
% will store the name of the current key.
%
% Any spaces at the beginning and at the end and around the
% equals-sign are removed. The key with the complete path is set to
% the macro \pgfcurrentkey.
%
% The setting is then processed according to the following rules:
%
% 1) If the key /path/key/.code" is present, its code is executed
%    with the value computed above, followed by \pgfeov (end of
%    value). So, to handle
%
%    "/stuff/height=  1.5 ,"
%
%    /stuff/height/.code should be set to some code, that can
%    handle the parameter
%
%    "1.5\pgfeov"
%
%    For instance, saying
%
%    \pgfsetkey{/stuff/height/.code}{#1\pgfeov}{\def\myheight{#1}}
%
%    will do nicely.
%
% 2) Otherwise, if the key /path/key/.value is present, this key is
%    set to the value computed above.
%
% 3) Otherwise, if the key /handlers/key/.code is present, it is executed
%    with the same parameters as in 1). Additionally, the
%    token register \pgfcurrentkeypath will be set to "/path/" and the
%    macor \pgfcurrentkeywithoutpath to "key". So, in the above
%    example if neither "/stuff/height/.code" nor
%    "/stuff/height/.value" is present, but "/handlers/height" is,
%    then "/handlers/height" is executed with the parameters:
%
%    "1.5\pgfeov"
%
%    and \pgfcurrentkey is set to "/stuff/height" and \pgfcurrentkeypath
%    is set to "/stuff/" and \pgfcurrentkeywithoutpath to "height".
%
% 4) Otherwise, if the key "/path/.unknown/.code" is present, its code is
%    executed with the same parameters as in 3).
%
% 5) Otherwise, the key "/handlers/.unknown/.code" is executed with the same
%    parameters as in 1).
%
% After all settings have been processed, the value of the token
% register \pgfdefaultkeypath is set to its original value. Thus, any local
% change of this token register has no effect outside the call.
%
% Example:
%
% \setoptions{/handlers/.style/.set code[1]={\setoptionsinner{#1}}}
% \setoptions{/tikz/.style={/path=/tikz}}}
% \setoptions{tikz,length/.macro=\tikz@length}
% \setoptions{tikz,hidden/.if=\iftikz@hidden}
% \setoptions{tikz,length=1cm,hidden}
% \setoptions{/tikz/length=2cm}
% \setoptions{tikz,length=1cm, height=2cm, }

\newtoks\pgfdefaultkeypath
\newtoks\pgfcurrentkeypath
\pgfdefaultkeypath{/}


\def\pgfset{% Save path
  \expandafter\pgfset@withpath\expandafter{\the\pgfdefaultkeypath}%
}

\def\pgfset@withpath#1#2{%
  \pgfset@parse#2,\relax,%
  \pgfdefaultkeypath{#1}% restore path
}

\def\pgfset@parse#1,{%
  \ifx\relax#1\pgfutil@empty%
  \else%
    \pgfset@unpack#1=\pgf@novalue=\pgfset@stop%
    \expandafter\pgfset@parse%
  \fi%
}

\def\pgf@novalue{\pgf@novalue} % equals only itself
\def\pgfvaluerequired{\pgfvaluerequired} % equals only itself

\def\pgfset@unpack#1=#2=#3\pgfset@stop{%
  \pgfset@spdef\pgfcurrentkey{#1}%
  \ifx\pgfcurrentkey\pgfutil@empty%
    % Skip
  \else%
    \pgfset@add@path@as@needed%
    \pgfset@spdef\pgfset@val{#2}%
    \ifx\pgfset@val\pgf@novalue% Hmm... no value
      \pgfutil@ifcsname pgfk@\pgfcurrentkey/.default\endcsname%
        % Aha. Use this!
        \pgfretrievekeycode{\pgfcurrentkey/.default}{\pgfset@val}%
      \else%
        \let\pgfset@val\pgfutil@empty% no default, so use empty string
      \fi%
    \fi%
    \ifx\pgfset@val\pgfvaluerequired%
      \pgfdokey{/errors/value required/.code}%
    \else%
      \pgfset@case@one%
    \fi%
  \fi}

\def\pgfset@case@one{%
  \pgfutil@ifcsname pgfk@\pgfcurrentkey/.code\endcsname%
    \expandafter\let\expandafter\pgfset@code\csname pgfk@\pgfcurrentkey/.code\endcsname%
    \expandafter\pgfset@code\pgfset@val\pgfeov%
    % done
  \else%
    \pgfset@case@two%
  \fi%
}

\def\pgfset@case@two{%
  \pgfutil@ifcsname pgfk@\pgfcurrentkey/.value\endcsname%
    \expandafter\def\pgfset@name{\pgfcurrentkey/.value}%
    \expandafter\pgfsetkey\expandafter{\expandafter\pgfset@name\expandafter}\expandafter{\pgfset@val}%
    % done
  \else%
    \pgfset@case@three%
  \fi%
}

\def\pgfset@case@three{%
  \pgfset@split@path%
  \pgfutil@ifcsname pgfk@/handlers/\pgfcurrentkeywithoutpath/.code\endcsname%
    \expandafter\let\expandafter\pgfset@code\csname pgfk@/handlers/\pgfcurrentkeywithoutpath/.code\endcsname%
    \expandafter\pgfset@code\pgfset@val\pgfeov%
    % done
  \else%
    \pgfset@case@four%
  \fi%
}

\def\pgfset@case@four{%
  \pgfutil@ifcsname pgfk@\the\pgfcurrentkeypath.unknown/.code\endcsname%
    \expandafter\let\expandafter\pgfset@code\csname pgfk@\the\pgfcurrentkeypath.unknown/.code\endcsname%
    \expandafter\pgfset@code\pgfset@val\pgfeov%
    % done
  \else%
    \pgfset@case@five%
  \fi%
}

\def\pgfset@case@five{%
  \expandafter\let\expandafter\pgfset@code\csname pgfk@/handlers/.unknown/.code\endcsname%
  \expandafter\pgfset@code\pgfset@val\pgfeov%
}


\def\pgfkey@argumentisspace#1{%
  \def\pgfset@spdef##1##2{%
    \futurelet\pgfset@possiblespace\pgfset@sp@a##2\pgfset@stop\pgfset@stop#1\pgfset@stop\relax##1}%
  \def\pgfset@sp@a{%
    \ifx\pgfset@possiblespace\pgfutil@sptoken%
      \expandafter\pgfset@sp@b%
    \else%
      \expandafter\pgfset@sp@b\expandafter#1%
    \fi}%
  \def\pgfset@sp@b#1##1 \pgfset@stop{\pgfset@sp@c##1}%
}
\pgfkey@argumentisspace{ }
\def\pgfset@sp@c#1\pgfset@stop#2\relax#3{\def#3{#1}}


\def\pgfset@add@path@as@needed{% Should add the path if the
  % \pgfcurrentkey does not start with /
  \expandafter\futurelet\expandafter\pgfset@possibleslash\expandafter\pgfset@check@slash\pgfcurrentkey\relax%
}
\def\pgfset@check@slash{%
  \ifx\pgfset@possibleslash/%
    \expandafter\pgfset@nevermind%
  \else%
    \expandafter\pgfset@addpath%
  \fi%
}

\def\pgfset@nevermind#1\relax{}
\def\pgfset@addpath#1\relax{\def\pgfcurrentkey{\the\pgfdefaultkeypath#1}}

\def\pgfset@split@path{% Should assign the two macros
                       % \pgfcurrentkeywithoutpath and \pgfcurrentlkeypath
  \pgfcurrentkeypath{}%
  \expandafter\pgfset@splitter\pgfcurrentkey//%
}
\def\pgfset@splitter#1/#2/{%
  \def\pgfset@temp{#2}%
  \ifx\pgfset@temp\pgfutil@empty%
    % Ah. done
    \def\pgfcurrentkeywithoutpath{#1}%
    \expandafter\pgfset@gobbletoslash%
  \else%
    \expandafter\pgfcurrentkeypath\expandafter{\the\pgfcurrentkeypath#1/}%
  \fi%
  \pgfset@splitter#2/%
}
\def\pgfset@gobbletoslash\pgfset@splitter/{}%




% Sets keys while setting keys
%
% #1 = key-value pairs
%
% Desscription:
%
% This macro may only be called inside the code that is executed for a
% key. The #1 should be a list of settings pairs. They will be executed
% as if they had been given as the argument to the \pgfset command.
%
% Example:
%
% \pgfset{tikz,myother length.set code=\def\myotherlength{#1}\pgfsetinner{length=#1}}

\def\pgfsetinner#1{\pgfset@parse#1,\relax,}





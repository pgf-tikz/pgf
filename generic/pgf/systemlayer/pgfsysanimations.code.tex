% Copyright 2006 by Till Tantau
%
% This file may be distributed and/or modified
%
% 1. under the LaTeX Project Public License and/or
% 2. under the GNU Public License.
%
% See the file doc/generic/pgf/licenses/LICENSE for more details.



% Guard against reading twice
\ifx\pgfsysanimationsloaded\pgfutil@undefined
  \let\pgfsysanimationsloaded=\relax
\else
  \expandafter\endinput
\fi


% We need some support:
\ifx\pgfmathloaded\pgfutil@undefined
  \input pgfmath.code.tex
\fi


% Animation abstraction layer
%
% This layer provides an abstraction of the
% \pgfsys@anim... commands. The idea is to provide an interface that
% can map animation commands either to "real" animations (for instance
% in the sense of SVG) or to "snapshots" of animations, which are
% especially useful for printing. For these snapshots, instead of
% adding an animation property to a scope, the property is actually
% set to a certain value computed by \TeX.
%
%
% For each \pgfsys@animation@... command there is a corresponding
% \pgfsysanim command.


% Set the current time to a snapshot time
%
% #1 = a global snapshot time (a dimensionless number, measured in
%      seconds). 
% #2 = must be + or -, which means that the time meant is #1+eps or
%      #1-eps (this is relevant when multiple values are specified for
%      the same time; in this case, #1-eps refers to the first of
%      them, #1+eps to the last of them)
% #3 = a sequence of tuples of the form {object}{type}{event}{event options}{time}
%      that specifies when the specified events have occurred.
%
% Description:
%
% When this command is used in a scope, instead of creating an
% animation, pgf will insert appropriate commands that show the state
% of the animation at the given time.
%
% The value #1 can be thought of as the "global time", which is the
% time that elapsed since the "onload" event. The sequence in #3 can
% be used to specify which events have occurred: Some animations only
% start when certain events have occurred; in this case the sequence
% in #3 tells us when these happend relative to the global time. The
% "event options" are relevant only for the accesskey event and the
% repeats event, where they specify the key or the number of repeats.
%
% The following commands and effect for animations are ignored / not
% implemented for snapshots (meaning that even if the "real" animation
% would look in a certain way because of one of the following commands
% or effects, the snapshot will not reflect this):
%
% - All events whose target is "end". Thus, snapshots ignore the end
%   of animation (however, this may change in the future, so it is
%   best not to use snapshots in conjunction with "end" targets.
% - All restart settings. Snapshots currently always assume the
%   "never" is the restart setting. Again, this may change in the
%   future and should not be taken for granted.
% - The "current value" as starting value (pgf has no real chance of
%   determining the correct value of this). Using this with a snapshot
%   will raise an errer.
% - The "paced" speed in "exotic" situations such as colors or
%   scalars. Using this with a snapshot will raise an error.
% - The "sum" versus "replace" properties. In a snapshot, if there are
%   several animations for an attribute, they are treated as if they
%   were separate animations of the attribute on nested scopes. This
%   means that for attributes like a color the "replace" is used in
%   effect, while for transformations the "sum" is used.
%
% Example:
%
% {
%   \pgfsyssnapshot{2}{+}{}%
%   \pgfsysanimkeytime{0}{1}{1}{0}{0}
%   \pgfsysanimvalscalar{1}
%   \pgfsysanimkeytime{5}{1}{1}{0}{0}
%   \pgfsysanimvalscalar{0}
%   \pgfsysanimkeywhom{\someid}{}
%   \pgfsysanimate{fillopacity}%
%   % Will set the opacity of \someid to 0.6
% }

\def\pgfsyssnapshot#1#2#3{%
  \def\pgfsysanim@snaptime{#1}%
  \def\pgfsysanim@snapevents{#3}%
  \ifx#2\pgfsysanim@plus@text%
    \pgfsysanim@snap@firstfalse%
  \else%
    \pgfsysanim@snap@firsttrue%
  \fi
  \pgfsysanim@is@snaptrue%
}
\newif\ifpgfsysanim@is@snap
\newif\ifpgfsysanim@snap@first
\def\pgfsysanim@plus@text{+}



% Companion for \pgfsys@animate
%
% #1 = attribute
%
% Description:
%
% If not snapshot is set, \pgfsys@animate is simply called. Otherwise,
% appropriate code is generated that sets the specified attribute of
% the whom object to the value it would have during the animation at
% the moment of the snapshot.

\def\pgfsysanimate#1{%
  \ifpgfsysanim@is@snap%
    \expandafter\ifx\csname pgfsysanim@prep@attr@#1\endcsname\pgfutil@undefined%
    \else%
      \csname pgfsysanim@prep@attr@#1\endcsname%
      \expandafter\pgfsys@attach@to@id\expandafter{\expandafter\pgfsysanim@whom@id\expandafter}%
      \expandafter{\expandafter\pgfsysanim@whom@type\expandafter}\expandafter{\pgfsys@anim@code}{}{}%
    \fi%
  \else%
    \pgfsys@animate{#1}%
  \fi%
}


% Companions for \pgfsys@animation@restart@...
%
% Description:
%
% These commands just call \pgfsys@animation@restart@... and they are
% *ignored* when it comes to snapshots.

\def\pgfsysanimkeyrestartalways{\pgfsys@animation@restart@always}
\def\pgfsysanimkeyrestartnever{\pgfsys@animation@restart@never}
\def\pgfsysanimkeyrestartwhennotactive{\pgfsys@animation@restart@whennotactive}



% Companions for \pgfsys@animation@repeat...
%
% Description:
%
% These commands call \pgfsys@animation@repeat... and, if there is a
% snapshot installed, the passed values are taken into consideration.

\def\pgfsysanimkeyrepeat#1{%
  \pgfsys@animation@repeat{#1}%
  % Snapshot
  \let\pgfsysanim@snap@repeat@kind\pgfsysanim@snap@repeat@normal%
  \def\pgfsysanim@snap@repeat@arg{#1}%
}
\def\pgfsysanimkeyrepeatindefinite{%
  \pgfsys@animation@repeat@indefinite%
  % Snapshot
  \let\pgfsysanim@snap@repeat@kind\pgfsysanim@snap@repeat@indefinite%
  \let\pgfsysanim@snap@repeat@arg\pgfutil@empty%
}
\def\pgfsysanimkeyrepeatduration#1{%
  \pgfsys@animation@repeat@dur{#1}%
  % Snapshot
  \let\pgfsysanim@snap@repeat@kind\pgfsysanim@snap@repeat@dur%
  \def\pgfsysanim@snap@repeat@arg{#1}%
}
\def\pgfsysanim@snap@repeat@none{n}%
\def\pgfsysanim@snap@repeat@normal{m}%
\def\pgfsysanim@snap@repeat@indefinite{i}%
\def\pgfsysanim@snap@repeat@dur{d}%
\let\pgfsysanim@snap@repeat@kind\pgfsysanim@snap@repeat@none
\let\pgfsysanim@snap@repeat@arg\pgfutil@empty



% Companions for \pgfsys@animation@freezeatend and
% \pgfsys@animation@removeatend 
%
% Description:
%
% These commands call the two system commands and, if there is a
% snapshot installed, the settings are taken into consideration.

\def\pgfsysanimkeyfreezeatend{\pgfsys@animation@freezeatend\pgfsysanim@freezeatendtrue}
\def\pgfsysanimkeyremoveatend{\pgfsys@animation@removeatend\pgfsysanim@freezeatendfalse}
\newif\ifpgfsysanim@freezeatend



% Companion for \pgfsys@animation@time
%
% #1 to #5 = as in \pgfsys@animation@time
%
% Description:
%
% Calls \pgfsys@animation@time. Furthermore, the specified times are
% recorded so that they can later be analysed for the computation
% needed for a snapshot.

\def\pgfsysanimkeytime#1#2#3#4#5{%
  % Setup animation
  \pgfsys@animation@time{#1}{#2}{#3}{#4}{#5}%
  % and rember for snapshots
  \def\pgfsysanim@snap@time{{#1}{#2}{#3}{#4}{#5}}%
}



% Companion for \pgfsys@animation@offset
%
% #1, #2 = as in \pgfsys@animation@offset
%
% Description:
%
% Calls \pgfsys@animation@offset. Furthermore, if #2 is "begin", the
% specified offset is recorded so that it can later be analysed for
% the computation needed for a snapshot.

\def\pgfsysanimkeyoffset{\pgfsys@animation@offset}
\def\pgfsysanimkeysyncbegin{\pgfsys@animation@syncbegin}
\def\pgfsysanimkeysyncend{\pgfsys@animation@syncend}
\def\pgfsysanimkeyevent{\pgfsys@animation@event}
\def\pgfsysanimkeyrepeatevent{\pgfsys@animation@repeat@event}
\def\pgfsysanimkeyaccesskey{\pgfsys@animation@accesskey}
% not yet implemented for snapshots



% Companions for \pgfsys@animation@sum and \pgfsys@animation@replace
%
% Description:
%
% Calls \pgfsys@animation@sum or \pgfsys@animation@replace. Note that
% these commands are *ignored* for snapshots: In a snapshot, all
% attribute animations have the effect of being added to a new scope.

\def\pgfsysanimkeysum{\pgfsys@animation@sum}
\def\pgfsysanimkeyreplace{\pgfsys@animation@replace}


% Companions for \pgfsys@animation@(no)accumulate
%
% Description:
%
% Calls \pgfsys@animaion@(no)accumulate and records the setting for
% snapshots. 

\def\pgfsysanimkeyaccumulate{\pgfsys@animation@accumulate\pgfsysanim@accumulatetrue}
\def\pgfsysanimkeynoaccumulate{\pgfsys@animation@noaccumulate\pgfsysanim@accumulatefalse}
\newif\ifpgfsysanim@accumulate




% Companion for \pgfsys@animation@whom
%
% #1 and #2 = as for \pgfsys@animation@whom
%
% Description:
%
% Calls \pgfsys@animation@whom and records the setting for snapshots. 

\def\pgfsysanimkeywhom#1#2{%
  \pgfsys@animation@whom{#1}{#2}%
  % Snapshots
  \def\pgfsysanim@whom@id{#1}%
  \def\pgfsysanim@whom@type{#2}%
}




% Companion for \pgfsys@animation@paced
%
% #1 and #2 = as for \pgfsys@animation@paced
%
% Description:
%
% Calls the system layer command and records the setting for
% snapshots. Note that a paced setting is not possible for all
% attributes. 

\def\pgfsysanimkeypaced#1#2{%
  \pgfsys@animation@paced{#1}{#2}%
  % Snapshot
  \pgfsysanim@pacedtrue%
  \def\pgfsysanim@snap@paced{{#1}{#2}}%
}


% Companion for \pgfsys@animation@spline
%
% Description:
%
% Calls the system layer command and records the setting for
% snapshots. 

\def\pgfsysanimkeyspline{\pgfsys@animation@spline\pgfsysanim@pacedfalse}
\newif\ifpgfsysanim@paced



% Companions for \pgfsys@animation@rotatealong and
% \pgfsys@animation@norotatealong 
%
% Description:
%
% Calls the system layer command and records the setting for
% snapshots. 

\def\pgfsysanimkeyrotatealong{\pgfsys@animation@rotatealong}
\def\pgfsysanimkeynorotatealong{\pgfsys@animation@norotatealong}



% Companion for \pgfsys@animation@movealong
%
% #1 = the path
%
% Description:
%
% Calls the system layer command and records the setting for
% snapshots.

\def\pgfsysanimkeymovealong#1{%
  \pgfsys@animation@movealong{#1}%
  % Snapshot
  \def\pgfsysanim@snap@movealong{#1}%
}


% Companion for \pgfsys@animation@canvas@transform
%
% Description:
%
% Calls the system layer command and records the setting for
% snapshots.

\def\pgfsysanimkeycanvastransform#1#2{%
  \pgfsys@animation@canvas@transform{#1}{#2}%
  % Snapshot
  \def\pgfsysanim@snap@canvas@transform{{#1}{#2}}%
}





% Companion for \pgfsys@animation@val@current
%
% Description:
%
% Calls the system layer command. If a snapshot is currently active,
% an error results.

\def\pgfsysanimvalcurrent{%
  \pgfsys@animation@val@current%
  \ifpgfsysanim@is@snap%
    \pgferror{You may not use "current value" with an animation snapshot}%
  \fi%
}



% Companion for \pgfsys@animation@val@text
%
% #1 = as for \pgfsys@animation@val@text
%
% Description:
%
% Calls the system layer command. If a snapshot is currently active,
% the value is recorded in the timeline.

\def\pgfsysanimvaltext#1{%
  \pgfsys@animation@val@text{#1}%
  \ifpgfsysanim@is@snap%
    \pgfsysanim@snap@record{#1}%
  \fi%
}




% Companion for \pgfsys@animation@val@scalar
%
% #1 = as for \pgfsys@animation@val@scalar
%
% Description:
%
% Calls the system layer command. If a snapshot is currently active,
% the value is recorded in the timeline.

\def\pgfsysanimvalscalar#1{%
  \pgfsys@animation@val@scalar{#1}%
  \ifpgfsysanim@is@snap%
    \pgfsysanim@snap@record{#1}%
  \fi%
}



% Companion for \pgfsys@animation@val@dimension
%
% #1 = as for \pgfsys@animation@val@dimension
%
% Description:
%
% Calls the system layer command. If a snapshot is currently active,
% the value is recorded in the timeline.

\def\pgfsysanimvaldimension#1{%
  \pgfsys@animation@val@dimension{#1}%
  \ifpgfsysanim@is@snap%
    \pgfsysanim@snap@record{#1}%
  \fi%
}



% Dispatcher for different color values
%
% #1 = a color value (like "red" or "black!20")
%
% Description:
%
% This macro transforms #1 into its correct color model and, then,
% calls the correct \pgfsysanimcolorXXX macro.

\def\pgfsysanimvalcolor#1{%
  \pgfutil@colorlet{pgf@anim@temp}{#1}%
  \pgfutil@ifundefined{applycolormixins}{}{\applycolormixins{pgf@anim@temp}}%
  \expandafter\let\expandafter\pgf@sys@temp\csname\string\color@pgf@anim@temp\endcsname
  \expandafter\pgfanim@parse@type@color@\pgf@sys@temp%
}
\def\pgfanim@parse@type@color@#1#2#3#4#5{%
  \expandafter\ifx\csname pgfsysanimvalcolor#4\endcsname\relax%
    \pgferror{Unsupported color model `#4'}%
  \else%
    \edef\pgf@sys@colmarshal{\expandafter\noexpand\csname pgfsysanimvalcolor#4\endcsname}%
    \pgf@sys@uncomma#5,,%
    \pgf@sys@colmarshal%
  \fi%
}

\def\pgf@sys@uncomma#1,{%
  \def\pgf@sys@coltest{#1}%
  \ifx\pgf@sys@coltest\pgfutil@empty%
  \else%
    \expandafter\def\expandafter\pgf@sys@colmarshal\expandafter{\pgf@sys@colmarshal{#1}}%
    \expandafter\pgf@sys@uncomma%
  \fi%
}




% Companion for \pgfsys@animation@val@color@rgb
%
% #1, #2, #3 = as for \pgfsys@animation@val@color@rgb
%
% Description:
%
% Calls the system layer command. If a snapshot is currently active,
% the value is recorded in the timeline.

\def\pgfsysanimvalcolorrgb#1#2#3{%
  \pgfsys@animation@val@color@rgb{#1}{#2}{#3}%
  \ifpgfsysanim@is@snap%
    \pgfsysanim@snap@record{{#1}{#2}{#3}}%
  \fi%  
}



% Companion for \pgfsys@animation@val@color@cmyk
%
% #1, #2, #3, #4 = as for \pgfsys@animation@val@color@cmyk
%
% Description:
%
% Calls the system layer command. If a snapshot is currently active,
% the value is recorded in the timeline.

\def\pgfsysanimvalcolorcmyk#1#2#3#4{%
  \pgfsys@animation@val@color@cmyk{#1}{#2}{#3}{#4}%
  \ifpgfsysanim@is@snap%
    \pgferror{not yet implemented}% must call
                                % \pgfsysanim@snap@record{{coverted rgb values}}
  \fi%  
}


% Companion for \pgfsys@animation@val@color@cmy
%
% #1, #2, #3 = as for \pgfsys@animation@val@color@cmy
%
% Description:
%
% Calls the system layer command. If a snapshot is currently active,
% the value is recorded in the timeline.

\def\pgfsysanimvalcolorcmy#1#2#3{%
  \pgfsys@animation@val@color@cmy{#1}{#2}{#3}%
  \ifpgfsysanim@is@snap%
    \pgferror{not yet implemented}% must call
                                % \pgfsysanim@snap@record{{coverted rgb values}}
  \fi%  
}



% Companion for \pgfsys@animation@val@color@gray
%
% #1 = as for \pgfsys@animation@val@color@gray
%
% Description:
%
% Calls the system layer command. If a snapshot is currently active,
% the value is recorded in the timeline.

\def\pgfsysanimvalcolorgray#1{%
  \pgfsys@animation@val@color@gray{#1}%
  \ifpgfsysanim@is@snap%
    \pgfsysanim@snap@record{{#1}{#1}{#1}}%
  \fi%  
}



% Companion for \pgfsys@animation@val@path
%
% #1 = as for \pgfsys@animation@val@path
%
% Description:
%
% Calls the system layer command. If a snapshot is currently active,
% the value is recorded in the timeline.

\def\pgfsysanimvalpath#1{
  \pgfsys@animation@val@path{#1}%
  \ifpgfsysanim@is@snap%
    \pgfsysanim@snap@record{#1}%
  \fi%  
}


% Companion for \pgfsys@animation@val@translate
%
% #1, #2 = as for \pgfsys@animation@val@translate
%
% Description:
%
% Calls the system layer command. If a snapshot is currently active,
% the value is recorded in the timeline.

\def\pgfsysanimvaltranslate#1#2{
  \pgfsys@animation@val@translate{#1}{#2}%
  \ifpgfsysanim@is@snap%
    \pgfsysanim@snap@record{{#1}{#2}}%
  \fi%  
}



% Companion for \pgfsys@animation@val@scale
%
% #1, #2 = as for \pgfsys@animation@val@scale
%
% Description:
%
% Calls the system layer command. If a snapshot is currently active,
% the value is recorded in the timeline.

\def\pgfsysanimvalscale#1#2{
  \pgfsys@animation@val@scale{#1}{#2}%
  \ifpgfsysanim@is@snap%
    \pgfsysanim@snap@record{{#1}{#2}}%
  \fi%  
}



% Companion for \pgfsys@animation@val@viewbox
%
% #1, #2, #3, #4 = as for \pgfsys@animation@val@viewbox
%
% Description:
%
% Calls the system layer command. If a snapshot is currently active,
% the value is recorded in the timeline.

\def\pgfsysanimvalviewbox#1#2#3#4{
  \pgfsys@animation@val@viewbox{#1}{#2}{#3}{#4}%
  \ifpgfsysanim@is@snap%
    \pgfsysanim@snap@record{{#1}{#2}{#3}{#4}}%
  \fi%  
}




% Companion for \pgfsys@animation@val@dash
%
% #1, #2 = as for \pgfsys@animation@val@dash
%
% Description:
%
% Calls the system layer command. If a snapshot is currently active,
% the value is recorded in the timeline.

\def\pgfsysanimvaldash#1#2{
  \pgfsys@animation@val@dash{#1}{#2}%
  \ifpgfsysanim@is@snap%
    \pgfsysanim@snap@record{{#1}{#2}}%
  \fi%  
}


% To be removed!

\def\pgfsysanimvaldashphase{\pgfsys@animation@val@dashphase}
\def\pgfsysanimvaldashpattern{\pgfsys@animation@val@dashpattern}







% To be implemented later...

\def\pgfsysanim@test@time#1{%
  \ifdim\pgfsysanim@time pt<\pgfsysanim@snaptime pt%
    % snap time not yet reached. Update previous:
    \def\pgfsysanim@prev@val{#1}%
    \let\pgfsysanim@prev@time\pgfsysanim@time%
    \let\pgfsysanim@next@val\relax%
    \let\pgfsysanim@next@time\relax%
  \else%
    \ifdim\pgfsysanim@time pt=\pgfsysanim@snaptime pt%
      \ifpgfsysanim@snap@first%
        \ifx\pgfsysanim@next@val\relax%
          % first? Save!
          \def\pgfsysanim@prev@val{#1}%
          \let\pgfsysanim@prev@time\pgfsysanim@time%
          \let\pgfsysanim@next@val\pgfsysanim@prev@val%
          \let\pgfsysanim@next@time\pgfsysanim@prev@time%
        % else, do nothing
        \fi
      \else%
        % always overwrite:  
        \def\pgfsysanim@prev@val{#1}%
        \let\pgfsysanim@prev@time\pgfsysanim@time%
        \let\pgfsysanim@next@val\pgfsysanim@prev@val%
        \let\pgfsysanim@next@time\pgfsysanim@prev@time%
      \fi%
    \else%
      % we passed the time!
      \ifx\pgfsysanim@next@val\relax%
        % first? Then save!
        \def\pgfsysanim@next@val{#1}%
        \let\pgfsysanim@next@time\pgfsysanim@time%
      % else: ignore later times!
      \fi%
    \fi%
  \fi%  
}

\let\pgfsysanim@prev@val\relax%
\let\pgfsysanim@prev@time\relax%
\let\pgfsysanim@next@val\relax%
\let\pgfsysanim@next@time\relax%


% The code for the different attributes

\def\pgfsysanim@prep@attr@opacity{%
  \pgfsysanim@comp@scalar%
  \edef\pgfsys@anim@code{\noexpand\pgfsys@opacity{\pgfsysanim@x@val}}%
}

\def\pgfsysanim@prep@attr@fillopacity{%
  \pgfsysanim@comp@scalar%
  \edef\pgfsys@anim@code{\noexpand\pgfsys@fill@opacity{\pgfsysanim@x@val}}%
}

\def\pgfsysanim@prep@attr@strokeopacity{%
  \pgfsysanim@comp@scalar%
  \edef\pgfsys@anim@code{\noexpand\pgfsys@stroke@opacity{\pgfsysanim@x@val}}%
}

\def\pgfsysanim@prep@attr@visibility{%
  \pgfsysanim@comp@text%
  \pgferror{snapshot of visibility not yet implemented}%
}

\def\pgfsysanim@prep@attr@staged{%
  \pgfsysanim@comp@text%
  \pgferror{snapshot of visibility not yet implemented}%
}

\def\pgfsysanim@prep@attr@linewidth{%
  \pgfsysanim@comp@dimension%
  \edef\pgfsys@anim@code{\noexpand\pgfsys@setlinewidth{\pgfsysanim@x@val pt}}%
}

\def\pgfsysanim@prep@attr@dashphase{%
  \pgfsysanim@comp@dimension%
  \pgferror{snapshot of dash not yet implemented}%
}

\def\pgfsysanim@prep@attr@dashpattern{%
  \pgfsysanim@comp@dasharray%
  \pgferror{snapshot of dash not yet implemented}%
}

\def\pgfsysanim@prep@attr@translate{%
  \pgfsysanim@comp@translate%
  \edef\pgfsys@anim@code{\noexpand\pgfsys@transformshift{\pgfsysanim@x@val pt}{\pgfsysanim@y@val pt}}%
}

\def\pgfsysanim@prep@attr@scale{%
  \pgfsysanim@comp@scale%
  \edef\pgfsys@anim@code{\noexpand\pgfsys@transformxyscale{\pgfsysanim@x@val}{\pgfsysanim@y@val}}%
}

\def\pgfsysanim@prep@attr@rotate{%
  \pgfsysanim@comp@scalar%
  {%
    \pgfmathsin@{\pgfsysanim@x@val}%
    \let\pgftransform@sin=\pgfmathresult%
    \pgfmathcos@{\pgfsysanim@x@val}%
    \let\pgftransform@cos=\pgfmathresult%
    \pgf@x=\pgftransform@sin pt%
    \pgf@xa=-\pgf@x%
    \edef\pgfsys@anim@code{\noexpand\pgfsys@transformcm{\pgftransform@cos}{\pgftransform@sin}{\pgf@sys@tonumber{\pgf@xa}}{\pgftransform@cos}{0pt}{0pt}}%
  \expandafter}%
  \expandafter\def\expandafter\pgfsys@anim@code\expandafter{\pgfsys@anim@code}%
}

\def\pgfsysanim@prep@attr@skewx{%
  \pgfsysanim@comp@scalar%
  {%
    \pgfmathatan@{\pgfsysanim@x@val}%
    \edef\pgfsys@anim@code{\noexpand\pgfsys@transformcm{1}{0}{\pgfmathresult}{1}{0pt}{0pt}}%
  \expandafter}%
  \expandafter\def\expandafter\pgfsys@anim@code\expandafter{\pgfsys@anim@code}%
}

\def\pgfsysanim@prep@attr@skewy{%
  \pgfsysanim@comp@scalar%
  {%
    \pgfmathatan@{\pgfsysanim@x@val}%
    \edef\pgfsys@anim@code{\noexpand\pgfsys@transformcm{1}{\pgfmathresult}{0}{1}{0pt}{0pt}}%
  \expandafter}%
  \expandafter\def\expandafter\pgfsys@anim@code\expandafter{\pgfsys@anim@code}%
}

\def\pgfsysanim@prep@attr@motion{%
  \pgfsysanim@comp@motion%
  \pgferror{snapshot of motion not yet implemented}%
}

\def\pgfsysanim@prep@attr@strokecolor{%
  \pgfsysanim@comp@color%
  \pgferror{snapshot of colors not yet implemented}%
}

\def\pgfsysanim@prep@attr@fillcolor{%
  \pgfsysanim@comp@color%
  \pgferror{snapshot of colors not yet implemented}%
}

\def\pgfsysanim@prep@attr@viewbox{%
  \pgfsysanim@comp@viewbox%
  \pgferror{snapshot of viewbox not yet implemented}%
}

\def\pgfsysanim@prep@attr@path{%
  \pgfsysanim@comp@path%
  \pgferror{snapshot of path not yet implemented}%
}


\def\pgfsysanim@comp@scalar{%
  \pgfsysanim@compute@fractions%
  \pgfmathsetmacro\pgfsysanim@x@val{\pgfsysanim@frac@a*\pgfsysanim@prev@val+\pgfsysanim@frac@b*\pgfsysanim@next@val}%
}
\def\pgfsysanim@comp@scale{\pgfmathsetmacro\pgfsysanim@x@val{\pgfsysanim@snaptime}\def\pgfsysanim@y@val{1}}
\def\pgfsysanim@comp@dimension{%
  \pgfsysanim@compute@fractions%
  \pgfmathsetmacro\pgfsysanim@x@val{\pgfsysanim@frac@b*\pgfsysanim@prev@val+\pgfsysanim@frac@a*\pgfsysanim@next@val}%
}
\def\pgfsysanim@comp@dasharray{\pgferror{not yet implemented}}
\def\pgfsysanim@comp@color{\pgferror{not yet implemented}}
\def\pgfsysanim@comp@viewbox{\pgferror{not yet implemented}}
\def\pgfsysanim@comp@path{\pgferror{not yet implemented}}
\def\pgfsysanim@comp@translate{
  \pgfsysanim@compute@fractions%
  \expandafter\expandafter\expandafter\pgfsysanim@comp@translate@\expandafter\pgfsysanim@prev@val\pgfsysanim@next@val%
}
\def\pgfsysanim@comp@translate@#1#2#3#4{%
  \pgfmathsetmacro\pgfsysanim@x@val{\pgfsysanim@frac@b*#1+\pgfsysanim@frac@a*#3}%
  \pgfmathsetmacro\pgfsysanim@y@val{\pgfsysanim@frac@b*#2+\pgfsysanim@frac@a*#4}%  
}
\def\pgfsysanim@comp@motion{\pgferror{not yet implemented}}
\def\pgfsysanim@comp@text{\pgferror{not yet implemented}}

\def\pgfsysanim@compute@fractions{%
  % This must be redone and made numerically stable!
  \ifdim\pgfsysanim@snaptime pt=\pgfsysanim@prev@time pt%
    \def\pgfsysanim@frac@a{1}%
    \def\pgfsysanim@frac@b{0}%
  \else%
    \pgfmathsetmacro{\pgfsysanim@frac@a}{(\pgfsysanim@snaptime-\pgfsysanim@prev@time)/(\pgfsysanim@next@time-\pgfsysanim@prev@time)}%
    \pgfmathsetmacro{\pgfsysanim@frac@b}{1-\pgfsysanim@frac@a}%
  \fi%
}



\endinput

% Copyright 2008 by Till Tantau
%
% This file may be distributed and/or modified
%
% 1. under the LaTeX Project Public License and/or
% 2. under the GNU Public License.
%
% See the file doc/generic/pgf/licenses/LICENSE for more details.

\ProvidesFileRCS[v\pgfversion] $Header$

\usepgfmodule{datavisualization}


% The main \datavisualization command
%
% This command must, as always, be given inside a tikz picture. It
% will add a data visualization to the picture; use "shift" option and
% friends to place the data visualization somewhere other than at the
% origin.
%
% The \datavisualization gets a set of options as (optional) argument,
% which can be used to setup the data visualization. Next come the
% "data block". Then comes the "annotation block". While the data
% block is mandatory, the annotation block is optional.
%
% The data block starts with the keyword "data" or "data sets",
% followed by optional options that are executed in the /pgf/data
% visualization path. If the source is set here, the data block may
% not have a body and the data is read from that source. Otherwise, if
% only "data" is used, the body must contain the data. If the keyword
% "data sets" is used, the body is passed to the command
% \pgfdatavisualizationrender unchanged, allowing multiple data 
% sets to be visualized at once. 
%
% The annotation block must start, if present, with "annotate". It
% contains any normal tikz code that will be executed inside the scope
% of the data visualization.
%
% Examples:
%
% \datavisualization[schoolbook plot]
%   data [source=my_data_file.cvs,format=comma separated columns]
% ;
%
%  \datavisualization[schoolbook plot]
%    data [format=key value pairs]
%    {
%      x=0, y=0
%      x=1, y=1
%      x=2, y=4
%      x=3, y=9
%    };
%
%  \datavisualization[schoolbook plot]
%    data [format=function]
%    {
%      var x = [0:3];
%      func y = \value x*\value x;
%    }
%    annotate
%    {
%      \draw (0,0) -- (3,3);
%    };
%
%  \datavisualization[schoolbook plot]
%    data sets [format=comma separated columns] {
%      \dataset [/data point/label=first experiment,source=file_1]
%      \dataset [/data point/label=second experiment,source=file_2]
%      \dataset [/data point/label=third experiment,source=file_3]
%      \dataset [/data point/label=prediction,format=function]
%        { var x=[0,1]; func y = rand(\value x); }
%    };

\def\tikz@lib@datavisualization{\pgfutil@ifnextchar[\tikz@lib@datavisualization@opt{\tikz@lib@datavisualization@opt[]}}%}

\def\tikz@lib@datavisualization@opt[#1]#2data{%
  \begingroup%
    % Ok, first, start a new data visualization
    \pgfnewdatavisualization
    % Now setup some options
    \tikzset{#1}%
    % Now handle the data
    \pgfutil@ifnextchar s\tikz@lib@dv@handle@datasets
    }%
}

\def\tikz@lib@dv@handle@data[#1]{%
  % Does a brace follow?
  \pgfutil@ifnextchar\bgroup{%
    % Yes. A single data set
    \pgfdatavisualizationrender[/utils/exec=\aftergroup\tikz@lib@dv@cont]\dataset[#1]%
  }
  {% No. Source must have been set
    \pgfdatavisualizationrender[/utils/exec=\aftergroup\tikz@lib@dv@cont]{\dataset[#1];}%
  }%
}%

\def\tikz@lib@dv@handle@datasets sets{\pgfutil@ifnextchar[\tikz@lib@dv@handle@datasets@opt{\tikz@lib@dv@handle@datasets@opt[]}}%}
\def\tikz@lib@dv@handle@datasets@opt[#1]{%
  \pgfdatavisualizationrender[/utils/exec=\aftergroup\tikz@lib@dv@cont,#1]%
}


\def\tikz@lib@dv@cont{%
  \tikzset{every data visualization annotation/.try}%
  \pgfutil@ifnextchar a{% an annotation!
    \tikz@lib@dv@annotation
  }{\tikz@lib@dv@close}%
}
\def\tikz@lib@dv@close{%
  \pgfutil@ifnextchar ;{% Ok!
    \endgroup% done
    \pgfutil@gobble % get rid of semicolon
  }{
    \PackageError{pgf}{Semicolon expected at end of a data
      visualization}{}%
    \endgroup%
  }
}

\def\tikz@lib@dv@annotation annotate{%
  \bgroup\aftergroup\tikz@lib@dv@close\let\pgf@temp=%get rid of \bgroup%
}



\endinput

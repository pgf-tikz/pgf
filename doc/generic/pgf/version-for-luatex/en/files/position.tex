% This file has been generated from the lua sources using LuaDoc.
% To regenerate it call "make genluadoc" in
% doc/generic/pgf/version-for-luatex/en.

\paragraph{pgflibrarygraphdrawing-position.lua}


\begin{luacommand}{{Position.calcCoordsTo}(\meta{posFrom},\meta{posTo})}
Returns a vector between two positions.

Parameters:
\begin{itemize}
	\item[] \meta{posFrom} \subitem Position A.\item[] \meta{posTo} \subitem Position B.
\end{itemize}


Return value:
\begin{itemize} \item[] x- and y-coordinates of the vector between posFrom and posTo. \end{itemize}


\end{luacommand}\begin{luacommand}{{Position:\textunderscore{}\textunderscore{}tostring}()}
Returns a readable string representation of the position.


Return value:
\begin{itemize} \item[] string representation of the position. \end{itemize}


\end{luacommand}\begin{luacommand}{{Position:copy}()}
Creates a copy of this position object.


Return value:
\begin{itemize} \item[] Copy of the position. \end{itemize}


\end{luacommand}\begin{luacommand}{{Position:equals}(\meta{pos})}
Returns a boolean value whether the object is equal to the given position.


Return value:
\begin{itemize} \item[] true if the position is equal to the given position pos. \end{itemize}


\end{luacommand}\begin{luacommand}{{Position:getAbsCoordinates}(\meta{x},\meta{y})}
Computes absolute coordinates of a position.

Parameters:
\begin{itemize}
	\item[] \meta{x} \subitem Just used internally for recrusion.\item[] \meta{y} \subitem Just used internally for recrusion.
\end{itemize}


Return value:
\begin{itemize} \item[] Absolute position. \end{itemize}


\end{luacommand}\begin{luacommand}{{Position:isAbsPosition}()}
Determines if the position is absolute.


Return value:
\begin{itemize} \item[] True if the position is absolute, else false. \end{itemize}


\end{luacommand}\begin{luacommand}{{Position:new}(\meta{values})}
Represents a relative postion.

Parameters:
\begin{itemize}
	\item[] \meta{values} \subitem Values (e.g. x- and y-coordinate) to be merged with the default-metatable of a position.
\end{itemize}


Return value:
\begin{itemize} \item[] A new position object. \end{itemize}


\end{luacommand}\begin{luacommand}{{Position:relateTo}(\meta{pos},\meta{keepAbsPosition})}
Relates a position to the given position.

Parameters:
\begin{itemize}
	\item[] \meta{pos} \subitem The relative position.\item[] \meta{keepAbsPosition} \subitem If true, the coordinates of the position are computed in the relation to the given position pos.
\end{itemize}



\end{luacommand}

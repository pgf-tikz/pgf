% This file has been generated from the lua sources using LuaDoc.
% To regenerate it call "make genluadoc" in
% doc/generic/pgf/version-for-luatex/en.

\paragraph{pgflibrarygraphdrawing-graph.lua}


\begin{luacommand}{{Graph:\textunderscore{}\textunderscore{}tostring}()}
Returns a string representation of this graph including all nodes and edges.


Return value:
\begin{itemize} \item[] Graph as string. \end{itemize}


\end{luacommand}\begin{luacommand}{{Graph:addEdge}(\meta{edge})}
Adds an edge to the graph.

Parameters:
\begin{itemize}
	\item[] \meta{edge} \subitem The edge to be added.
\end{itemize}



\end{luacommand}\begin{luacommand}{{Graph:addNode}(\meta{node})}
Adds a node to the graph.

Parameters:
\begin{itemize}
	\item[] \meta{node} \subitem The node to be added.
\end{itemize}



\end{luacommand}\begin{luacommand}{{Graph:copy}()}
Creates a shallow copy of a graph. That is, without nodes or edges.


Return value:
\begin{itemize} \item[] A copy of the graph. \end{itemize}


\end{luacommand}\begin{luacommand}{{Graph:createEdge}(\meta{nodeA},\meta{nodeB},\meta{direction},\meta{options})}
Creates and adds a new edge to the graph. The edge contains the given nodes and its direction and options are set to the param direction/option.

Parameters:
\begin{itemize}
	\item[] \meta{nodeA} \subitem The first node of the new edge.\item[] \meta{nodeB} \subitem The second node of the new edge.\item[] \meta{direction} \subitem The direction of the new edge.\item[] \meta{options} \subitem The options of the new edge.
\end{itemize}


Return value:
\begin{itemize} \item[] The newly created edge. \end{itemize}


\end{luacommand}\begin{luacommand}{{Graph:deleteEdge}(\meta{edge})}
Like removeEdge, but also removes the edge from the nodes incident with it.

Parameters:
\begin{itemize}
	\item[] \meta{edge} \subitem The edge to be removed.
\end{itemize}


Return value:
\begin{itemize} \item[] The edge or nil. \end{itemize}


\end{luacommand}\begin{luacommand}{{Graph:deleteNode}(\meta{node})}
Like removeNode, but also removes all edges incident to the removed node and for all nodes incident to the removed edges, remove the edges from them, too.

Parameters:
\begin{itemize}
	\item[] \meta{node} \subitem The node to be deleted with its edges.
\end{itemize}


Return value:
\begin{itemize} \item[] The node or nil if the node wasn't contained in the graph. \end{itemize}


\end{luacommand}\begin{luacommand}{{Graph:findNode}(\meta{name})}
Searches the nodes of the graph by the given name.

Parameters:
\begin{itemize}
	\item[] \meta{name} \subitem Name of the node you're looking for.
\end{itemize}


Return value:
\begin{itemize} \item[] The node with the given name or nil if it wasn't contained in the graph. \end{itemize}


\end{luacommand}\begin{luacommand}{{Graph:findNodeIf}(\meta{test})}
Searches the nodes of the graph by the given test-function and returns the first matching node.

Parameters:
\begin{itemize}
	\item[] \meta{test} \subitem A function (with a parameter of node) returning a boolean value.
\end{itemize}


Return value:
\begin{itemize} \item[] The matching node or nil. \end{itemize}


\end{luacommand}\begin{luacommand}{{Graph:getOption}(\meta{name})}
Returns the value of the option defined by name.

Parameters:
\begin{itemize}
	\item[] \meta{name} \subitem Name of the option.
\end{itemize}


Return value:
\begin{itemize} \item[] The stored value of the option or nil. \end{itemize}


\end{luacommand}\begin{luacommand}{{Graph:mergeOptions}(\meta{options})}
Merges the given options into options of the graph.

Parameters:
\begin{itemize}
	\item[] \meta{options} \subitem The options to be merged.
\end{itemize}



See also:
\begin{itemize}
	\item[] |mergeTable|
\end{itemize}

\end{luacommand}\begin{luacommand}{{Graph:new}(\meta{values})}
Creates a new graph.

Parameters:
\begin{itemize}
	\item[] \meta{values} \subitem Values (e.g. options) to be merged with the default-metatable of a graph
\end{itemize}


Return value:
\begin{itemize} \item[] The new graph. \end{itemize}


\end{luacommand}\begin{luacommand}{{Graph:removeEdge}(\meta{edge})}
Removes an edge from the graph, if possible and returns it.

Parameters:
\begin{itemize}
	\item[] \meta{edge} \subitem The edge to be removed.
\end{itemize}


Return value:
\begin{itemize} \item[] The edge or nil. \end{itemize}


\end{luacommand}\begin{luacommand}{{Graph:removeNode}(\meta{node})}
Removes a node from the graph, if possible and returns it.

Parameters:
\begin{itemize}
	\item[] \meta{node} \subitem The node to remove.
\end{itemize}


Return value:
\begin{itemize} \item[] The node or nil if it wasn't contained in the graph. \end{itemize}


\end{luacommand}\begin{luacommand}{{Graph:setOption}(\meta{name},\meta{value})}
Sets the option name to value.

Parameters:
\begin{itemize}
	\item[] \meta{name} \subitem Name of the option to be set.\item[] \meta{value} \subitem Value for the option defined by name.
\end{itemize}



\end{luacommand}\begin{luacommand}{{Graph:subGraph}(\meta{root},\meta{graph},\meta{visited})}
The function returns a new subgraph. The result graph begins at the node root, excludes all nodes and edges which are marked as visited.

Parameters:
\begin{itemize}
	\item[] \meta{root} \subitem Root node where operation starts.\item[] \meta{graph} \subitem Result graph object or nil.\item[] \meta{visited} \subitem Set of already visited nodes/edges or nil; will be modified.
\end{itemize}



\end{luacommand}\begin{luacommand}{{Graph:subGraphParent}(\meta{root},\meta{parent},\meta{graph})}
Creates a new subgraph with the parent marked visited. Useful if the graph is a tree structure (and parent is the parent of root).

Parameters:
\begin{itemize}
	\item[] \meta{parent} \subitem Parent of the recursion step before.
\end{itemize}



See also:
\begin{itemize}
	\item[] |subGraph|
\end{itemize}

\end{luacommand}\begin{luacommand}{{Graph:walkAux}(\meta{root},\meta{visited},\meta{removeIndex})}
Auxiliary function to walk a graph. Does nothing if no nodes exist.

Parameters:
\begin{itemize}
	\item[] \meta{root} \subitem The first node to be visited.  If nil, chooses some node.\item[] \meta{visited} \subitem Set of already seen things (nodes and edges). |visited[v] == true| indicates that the object v was already seen.\item[] \meta{removeIndex} \subitem Is either nil or a numeric value where the objects are removed from the local queues (nil therefore designates queue behaviour, 1 a stack behaviour).
\end{itemize}



See also:
\begin{itemize}
	\item[] |walkDepth|\item[] |walkBreadth|
\end{itemize}

\end{luacommand}\begin{luacommand}{{Graph:walkBreadth}(\meta{root},\meta{visited})}
The function returns an iterator to walk the graph breadth-first. The iterator then returns all edges and nodes one at a time and once only.  Use a filter function to return only edges or nodes.



See also:
\begin{itemize}
	\item[] |iterator.filter|
\end{itemize}

\end{luacommand}\begin{luacommand}{{Graph:walkDepth}(\meta{root},\meta{visited})}
The function returns an iterator to walk the graph depth-first. The iterator then returns all edges and nodes one at a time and once only.  Use a filter function to return only edges or nodes.



See also:
\begin{itemize}
	\item[] |iterator.filter|
\end{itemize}

\end{luacommand}

% Copyright 2010 by Till Tantau
%
% This file may be distributed and/or modified
%
% 1. under the LaTeX Project Public License and/or
% 2. under the GNU Free Documentation License.
%
% See the file doc/generic/pgf/licenses/LICENSE for more details.


\section{Graph Specification Library}
\label{section-library-graphs}

\begin{tikzlibrary}{graph}
  The package must be loaded to use the |graph| path command. It
  offers an easy syntax for specifying the connection between nodes in
  a graph. 
\end{tikzlibrary}


\subsection{Overview}

\tikzname\ offers a powerful path command for specifying how the nodes
in a graph are connected by edges and arcs: The |graph| path
command, which becomes available when you load the |graph| library.

In this section, by \emph{graph} we refer to a set of nodes together
with some edges such as the following:

\begin{codeexample}[]
\tikz \graph [nodes={draw, circle}] { K_n [n=5, clockwise] };
\end{codeexample}

\begin{codeexample}[]
\tikz
  \graph [nodes={draw, circle}, clockwise, radius=.5cm, typeset=, n=5]
  { I_n [name=inner] -- [join=bipartite] I_n [name=outer] };
\end{codeexample}

\begin{codeexample}[]
\tikz
  \graph [nodes={draw, circle}, clockwise, radius=.75cm, typeset=, n=8]
  { C_n [name=inner] -- C_n [name=outer] };
\end{codeexample}


\begin{codeexample}[]
\tikz [x=8mm, y=6mm, font=\footnotesize, circle]
  \graph [nodes={fill=blue!70, text=white}, typeset=, n=8, evenly spaced] {
    I_n [name=A] -- [join = {butterfly={level=4}}]
    I_n [name=B] -- [join = {butterfly={level=2}}]
    I_n [name=C] -- [join = butterfly]
    I_n [name=D] -- 
    I_n [name=E]  
  };
\end{codeexample}

The nodes of a graph are normal \tikzname\ nodes, the edges are
normal lines drawn between nodes. There is nothing in the |graph|
library that you cannot do using the normal |\node| and the |edge|
command. Rather, its purpose is to offer a concise and powerful way of
\emph{specifying} which nodes are present 
and how they are connected. The |graph| library makes little attempt
at trying to ``layout'' the graph in some pretty way, its main
strength is in specifying which nodes and edges are present in
principle. 

The |graph| library uses a syntax that is quite different from the
normal \tikzname\ syntax for specifying nodes. The reason for this is
that for many medium-sized graphs it can become quite cumbersome to
specify all the nodes using |\node| repeatedly and then using a great
number of |edge| command; possibly with complicated |\foreach|
statements. Instead, the syntax of the |graph| library is loosely
inspired by the \textsc{dot} format, which is quite useful for
specifying medium-sized graphs.





%%% Local Variables: 
%%% mode: latex
%%% TeX-master: "pgfmanual-pdftex-version"
%%% End: 

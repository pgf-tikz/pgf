% Copyright 2011 by Jannis Pohlmann
%
% This file may be distributed and/or modified
%
% 1. under the LaTeX Project Public License and/or
% 2. under the GNU Free Documentation License.
%
% See the file doc/generic/pgf/licenses/LICENSE for more details.

\section{Automatic Graph Drawing}
\label{section-library-graphdrawing-algorithms}

{\emph{by Jannis Pohlmann}}

\begin{tikzlibrary}{graphdrawing}
  This package needs to be loaded in order to use any of the automatic
  graph drawing algorithms described in this chapter. It provides the
  \emph{Lua}-based graph drawing framework as well as a number of 
  generic \tikzname\ options to adjust the generated graph drawings.
  The individual graph drawing algorithms are located in their own
  libraries and have to be loaded separately.
\end{tikzlibrary}

\subsection{Orientation of Graph Drawings}

The drawing computed for a graph may be pleasing in general but some
algorithms like the |spring| algorithm show the tendency to generate
slightly rotated drawings. Also, if several graphs are defined in a
single |{tikzpicture}| then these will possibly overlap.
As a consequence, a postprocessing step may be necessary to adjust the
orientation of the graph drawing. This can be achieved in a relatively
simple way using the options described in the following.

\begin{key}{/tikz/orientation=\meta{colon-separated list}}
  With graph drawing algorithms it makes sense with, the |graphdrawing|
  library will, by default and in a postprocessing step, attempt to
  detect the principal axis of the graph automatically and will adjust
  its orientation so that it is parallel to the $x$ axis.

  This behavior may be altered by setting |orientation| using one of
  the following notations:
  |(|\meta{first axis node}|):(|\meta{second axis node}|)| specifies
  two nodes that together define the axis of the graph drawing. The
  drawing is then rotated so that this axis is parallel to the $x$
  axis.
  \begin{codeexample}[]
\tikz \graph [experimental layout,orientation=(a):(c)] {
  a -- { b, c }
};
  \end{codeexample}
  \begin{codeexample}[]
\tikz \graph [experimental layout,orientation=(c):(a)] {
  a -- { b, c }
};
  \end{codeexample}
  The angle to the $x$ axis can be changed easily by using the
  |(|\meta{first axis node}|):|\meta{angle}|:(|\meta{second axis node}|)|
  syntax to define an axis with a specific angle.
  \begin{codeexample}[]
\tikz \graph [experimental layout,orientation=(a):90:(c)] {
  a -- { b, c}
};
  \end{codeexample}
  \begin{codeexample}[]
\tikz \graph [experimental layout,orientation=(a):120:(b)] {
  a -- b
};
  \end{codeexample}
\end{key}

\begin{key}{/tikz/desired at=\meta{coordinate}}
  % TODO: Sets a desired position for the node. In a postprocessing
  % step, the graphdrawing library attempts to move the entire graph 
  % so that the first node with a |desired at| option is moved to
  % its desired location.
\end{key}

%\subsection{Packing of Connected Components}
%
%Graphs may be composed of subgraphs or \emph{components} that are not
%connected to each other. In order to draw these nicely, the 
%|graphdrawing| library splits them up into separate graphs, computes
%their layouts with the same graph drawing algorithm independently and,
%in a postprocessing step, arranges them in a non-uniform grid in the 
%final drawing. This is called \emph{component packing}.
%
%The following options can be used to configure the order and placement
%strategy during component packing.
%
%\begin{key}{/tikz/component packing=\marg{options}}
%  Executes the \meta{options} with the path prefix 
%  |/tikz/component packing|.
%  
%  Defines how to arrange the connected components of the graph after 
%  their individual drawings have been computed.
%\end{key}
%
%\begin{key}{/tikz/component packing/layered=\opt{\meta{boolean}} (default true, initially true)}
%  |layered| arranges the different components in a non-uniform grid
%  starting in the top left corner. Components are placed in these 
%  layers in descending order of their size.
%  \begin{codeexample}[]
%\tikz \graph [experimental layout,component packing={layered}] {
%  a -- b -- c -- a,
%  d -- e -- d,
%  g
%};
%  \end{codeexample}
%\end{key}
%
%\begin{key}{/tikz/component packing/centered=\opt{\meta{boolean}} (default true, initially false)}
%  If set to |true|, arranges the different components clockwise in a 
%  non-uniform grid starting at the center. Components are placed in 
%  descending order of their size.
%  \begin{codeexample}[]
%\tikz \graph [experimental layout,component packing={centered}] {
%  a -- b -- c -- a,
%  d -- e -- d,
%  g
%};
%  \end{codeexample}
%\end{key}
%
%\begin{key}{/tikz/component packing/padding=\meta{dimension} (initially 0pt)}
%  Defines how much padding is used to separate the connected 
%  components.
%  \begin{codeexample}[]
%\tikz \graph [experimental layout,component packing={centered,padding=10pt}] {
%  a -- b -- c -- a,
%  d -- e -- d,
%  g
%};
%  \end{codeexample}
%\end{key}
%
%
%\section{The Spring and Spring-Electrical Drawing Algorithms}
%
%\subsection{Overview}
%
%\subsection{Common Options}
%
%\subsection{Options for the Spring Algorithm}
%
%\subsection{Options for the Spring-Electrical Algorithm}
%
%% TODO
%% Explain the following concepts:
%% - separation of graph drawing options and regular TikZ options
%% - generic graph drawing options:
%%   - component packing
%%   - orientation
%% - pre-defined graph drawing styles
%% - graph drawing options for fine-tuning the different algorithms

%%% Local Variables: 
%%% mode: latex
%%% TeX-master: "pgfmanual-pdftex-version"
%%% End:

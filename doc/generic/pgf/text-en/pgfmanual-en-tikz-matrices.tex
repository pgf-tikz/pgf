% Copyright 2019 by Till Tantau
%
% This file may be distributed and/or modified
%
% 1. under the LaTeX Project Public License and/or
% 2. under the GNU Free Documentation License.
%
% See the file doc/generic/pgf/licenses/LICENSE for more details.


\section{Matrices and Alignment}
\label{section-matrices}

\subsection{Overview}

When creating pictures, one often faces the problem of correctly aligning parts
of the picture. For example, you might wish that the |baseline|s of certain
nodes should be on the same line and some further nodes should be below these
nodes with, say, their centers on a vertical lines. There are different ways of
solving such problems. For example, by making clever use of anchors, nearly all
such alignment problems can be solved. However, this often leads to complicated
code. An often simpler way is to use \emph{matrices}, the use of which is
explained in the current section.

A \tikzname\ matrix is similar to \LaTeX's |{tabular}| or |{array}|
environment, only instead of text each cell contains a little picture or a
node. The sizes of the cells are automatically adjusted such that they are
large enough to contain all the cell contents.

Matrices are a powerful tool and they need to be handled with some care. For
impatient readers who skip the rest of this section: you \emph{must} end
\emph{every} row with |\\|. In particular, the last row \emph{must} be ended
with |\\|.

Many of the ideas implemented in \tikzname's matrix support are due to Mark
Wibrow -- many thanks to Mark at this point!


\subsection{Matrices are Nodes}

Matrices are special in many ways, but for most purposes matrices are treated
like nodes. This means, that you use the |node| path command to create a matrix
and you only use a special option, namely the |matrix| option, to signal that
the node will contain a matrix. Instead of the usual \TeX-box that makes up the
|text| part of the node's shape, the matrix is used. Thus, in particular, a
matrix can have a shape, this shape can be drawn or filled, it can be used in a
tree, and so on. Also, you can refer to the different anchors of a matrix.

\begin{key}{/tikz/matrix=\meta{true or false} (default true)}
    This option can be passed to a |node| path command. It signals that the
    node will contain a matrix.
    %
\begin{codeexample}[]
\begin{tikzpicture}
  \draw[help lines] (0,0) grid (4,2);
  \node [matrix,fill=red!20,draw=blue,very thick] (my matrix) at (2,1)
  {
    \draw (0,0)   circle (4mm); & \node[rotate=10] {Hello};        \\
    \draw (0.2,0) circle (2mm); & \fill[red]   (0,0) circle (3mm); \\
  };

  \draw [very thick,->] (0,0) |- (my matrix.west);
\end{tikzpicture}
\end{codeexample}
    %
    The exact syntax of the matrix is explained in the course of this section.
    %
    \begin{stylekey}{/tikz/every matrix (initially \normalfont empty)}
        This style is used in every matrix.
    \end{stylekey}
    %
    \begin{stylekey}{/tikz/every outer matrix (initially \normalfont empty)}
        While the |every matrix| key also applies to the matrix contents, this
        only applies to the outer node which holds the matrix.
    \end{stylekey}
\end{key}

Even more so than nodes, matrices will often be the only object on a path.
Because of this, there is a special abbreviation for creating matrices:

\begin{command}{\matrix}
    Inside |{tikzpicture}| this is an abbreviation for |\path node[matrix]|.
\end{command}

Even though matrices are nodes, some options do not have the same effect as for
normal nodes:
%
\begin{enumerate}
    \item Rotations and scaling have no effect on a matrix as a whole (however,
        you can still transform the contents of the cells normally). Before the
        matrix is typeset, the rotational and scaling part of the
        transformation matrix is reset.
    \item For multi-part shapes you can only set the |text| part of the node.
    \item All options starting with |text| such as |text width| have no effect.
    \item If you place a matrix on a path, the matrix contents will be
        collected into a macro, which tokenizes them.  This means that |&| will
        lose its meaning as an alignment character, resulting in an error.  If
        you need to place a matrix on a path, use |ampersand replacement| to
        work around that problem.
\end{enumerate}


\subsection{Cell Pictures}
\label{section-tikz-cell-pictures}

A matrix consists of rows of \emph{cells}. Each row (including the last one!)
is ended by the command |\\|. The character |&| is used to separate cells.
Inside each cell, you must place commands for drawing a picture, called the
\emph{cell picture} in the following. (However, cell pictures are not enclosed
in a complete |{pgfpicture}| environment, they are a bit more light-weight. The
main difference is that cell pictures cannot have layers.) It is not necessary
to specify beforehand how many rows or columns there are going to be and if a
row contains less cell pictures than another line, empty cells are
automatically added as needed.


\subsubsection{Alignment of Cell Pictures}

For each cell picture a bounding box is computed. These bounding boxes and the
origins of the cell pictures determine how the cells are aligned. Let us start
with the rows: Consider the cell pictures on the first row. Each has a bounding
box and somewhere inside this bounding box the origin of the cell picture can
be found (the origin might even lie outside the bounding box, but let us ignore
this problem for the moment). The cell pictures are then shifted around such
that all origins lie on the same horizontal line. This may make it necessary to
shift some cell pictures upwards and others downwards, but it can be done and
this yields the vertical alignment of the cell pictures this row. The top of
the row is then given by the top of the ``highest'' cell picture in the row,
the bottom of the row is given by the bottom of the lowest cell picture. (To be
more precise, the height of the row is the maximum $y$-value of any of the
bounding boxes and the depth of the row is the negated minimum $y$-value of the
bounding boxes).
%
\begin{codeexample}[]
\begin{tikzpicture}
  [every node/.style={draw=black,anchor=base,font=\huge}]

  \matrix [draw=red]
  {
    \node {a}; \fill[blue] (0,0) circle (2pt); &
    \node {X}; \fill[blue] (0,0) circle (2pt); &
    \node {g}; \fill[blue] (0,0) circle (2pt); \\
  };
\end{tikzpicture}
\end{codeexample}

Each row is aligned in this fashion: For each row the cell pictures
are vertically aligned such that the origins lie on the same
line. Then the second row is placed below the first row such that the
bottom of the first row touches the top of the second row (unless a
|row sep| is used to add a bit of space). Then the bottom of the
second row touches the top of the third row, and so on. Typically,
each row will have an individual height and depth.
%
\begin{codeexample}[]
\begin{tikzpicture}
  [every node/.style={draw=black,anchor=base}]

  \matrix [draw=red]
  {
    \node {a}; & \node {X}; & \node {g}; \\
    \node {a}; & \node {X}; & \node {g}; \\
  };

  \matrix [row sep=3mm,draw=red] at (0,-2)
  {
    \node {a}; & \node {X}; & \node {g}; \\
    \node {a}; & \node {X}; & \node {g}; \\
  };
\end{tikzpicture}
\end{codeexample}

Let us now have a look at the columns. The rules for how the pictures on any
given column are aligned are very similar to the row alignment: Consider all
cell pictures in the first column. Each is shifted horizontally such that the
origins lie on the same vertical line. Then, the left end of the column is at
the left end of the bounding box that protrudes furthest to the left. The right
end of the column is at the right end of the bounding box that protrudes
furthest to the right. This fixes the horizontal alignment of the cell pictures
in the first column and the same happens the cell pictures in the other
columns. Then, the right end of the first column touches the left end of the
second column (unless |column sep| is used). The right end of the second column
touches the left end of the third column, and so on. (Internally, two columns
are actually used to achieve the desired horizontal alignment, but that is only
an implementation detail.)
%
\begin{codeexample}[]
\begin{tikzpicture}[every node/.style={draw}]
  \matrix [draw=red]
  {
    \node[left]  {Hallo}; \fill[blue] (0,0) circle (2pt); \\
    \node        {X};     \fill[blue] (0,0) circle (2pt); \\
    \node[right] {g};     \fill[blue] (0,0) circle (2pt); \\
  };
\end{tikzpicture}
\end{codeexample}

\begin{codeexample}[]
\begin{tikzpicture}[every node/.style={draw}]
  \matrix [draw=red,column sep=1cm]
  {
    \node {8}; & \node{1}; & \node {6}; \\
    \node {3}; & \node{5}; & \node {7}; \\
    \node {4}; & \node{9}; & \node {2}; \\
  };
\end{tikzpicture}
\end{codeexample}


\subsubsection{Setting and Adjusting Column and Row Spacing}

There are different ways of setting and adjusting the spacing between columns
and rows. First, you can use the options |column sep| and |row sep| to set a
default spacing for all rows and all columns. Second, you can add options to
the |&| character and the |\\| command to adjust the spacing between two
specific columns or rows. Additionally, you can specify whether the space
between two columns or rows should be considered between the origins of cells
in the column or row or between their borders.

\begin{key}{/tikz/column sep=\meta{spacing list}}
    This option sets a default space that is added between every two columns.
    This space can be positive or negative and is zero by default. The
    \meta{spacing list} normally contains a single dimension like |2pt|.
    %
\begin{codeexample}[]
\begin{tikzpicture}
  \matrix [draw,column sep=1cm,nodes=draw]
  {
    \node(a) {123}; & \node (b) {1};   & \node {1}; \\
    \node    {12};  & \node     {12};  & \node {1}; \\
    \node(c) {1};   & \node (d) {123}; & \node {1}; \\
  };
  \draw [red,thick]  (a.east) -- (a.east |- c)
                     (d.west) -- (d.west |- b);
  \draw [<->,red,thick] (a.east) -- (d.west |- b)
    node [above,midway] {1cm};
\end{tikzpicture}
\end{codeexample}
    %
    More generally, the \meta{spacing list} may contain a whole list of
    numbers, separated by commas, and occurrences of the two key words
    |between origins| and |between borders|. The effect of specifying such a
    list is the following: First, all numbers occurring in the list are simply
    added to compute the final spacing. Second, concerning the two keywords,
    the last occurrence of one of the keywords is important. If the last
    occurrence is |between borders| or if neither occurs, then the space is
    inserted between the two columns normally. However, if the last occurs is
    |between origins|, then the following happens: The distance between the
    columns is adjusted such that the difference between the origins of all the
    cells in the first column (remember that they all lie on straight line) and
    the origins of all the cells in the second column is exactly the given
    distance.

    \emph{The }|between origins|\emph{ option can only be used for columns
    mentioned in the first row, that is, you cannot specify this option for
    columns introduced only in later rows.}
    %
\begin{codeexample}[]
\begin{tikzpicture}
  \matrix [draw,column sep={1cm,between origins},nodes=draw]
  {
    \node(a) {123}; & \node (b) {1};   & \node {1}; \\
    \node    {12};  & \node     {12};  & \node {1}; \\
    \node    {1};   & \node     {123}; & \node {1}; \\
  };
  \draw [<->,red,thick] (a.center) -- (b.center) node [above,midway] {1cm};
\end{tikzpicture}
\end{codeexample}
    %
\end{key}

\begin{key}{/tikz/row sep=\meta{spacing list}}
    This option works like |column sep|, only for rows. Here, too, you can
    specify whether the space is added between the lower end of the first row
    and the upper end of the second row, or whether the space is computed
    between the origins of the two rows.
    %
\begin{codeexample}[]
\begin{tikzpicture}
  \matrix [draw,row sep=1cm,nodes=draw]
  {
    \node (a) {123}; & \node {1};   & \node {1}; \\
    \node (b) {12};  & \node {12};  & \node {1}; \\
    \node     {1};   & \node {123}; & \node {1}; \\
  };
  \draw [<->,red,thick] (a.south) -- (b.north) node [right,midway] {1cm};
\end{tikzpicture}
\end{codeexample}
    %
\begin{codeexample}[]
\begin{tikzpicture}
  \matrix [draw,row sep={1cm,between origins},nodes=draw]
  {
    \node (a) {123}; & \node {1};   & \node {1}; \\
    \node (b) {12};  & \node {12};  & \node {1}; \\
    \node     {1};   & \node {123}; & \node {1}; \\
  };
  \draw [<->,red,thick] (a.center) -- (b.center) node [right,midway] {1cm};
\end{tikzpicture}
\end{codeexample}
    %
\end{key}

The row-end command |\\| allows you to provide an optional argument, which must
be a dimension. This dimension will be added to the list in |row sep|. This
means that, firstly, any numbers you list in this argument will be added as an
extra row separation between the line being ended and the next line and,
secondly, you can use the keywords |between origins| and |between borders| to
locally overrule the standard setting for this line pair.
%
\begin{codeexample}[]
\begin{tikzpicture}
  \matrix [row sep=1mm]
  {
    \draw (0,0) circle (2mm); & \draw (0,0) circle (2mm); \\
    \draw (0,0) circle (2mm); & \draw (0,0) circle (2mm); \\[-1mm]
    \draw (0,0) coordinate (a) circle (2mm); &
    \draw (0,0) circle (2mm); \\[1cm,between origins]
    \draw (0,0) coordinate (b) circle (2mm); &
    \draw (0,0) circle (2mm); \\
  };
  \draw [<->,red,thick] (a.center) -- (b.center) node [right,midway] {1cm};
\end{tikzpicture}
\end{codeexample}

The cell separation character |&| also takes an optional argument, which must
also be a spacing list. This spacing list is added to the |column sep| having a
similar effect as the option for the |\\| command for rows.

This optional spacing list can only be given the first time a new column is
started (usually in the first row), subsequent usages of this option in later
rows have no effect.
%
\begin{codeexample}[]
\begin{tikzpicture}
  \matrix [draw,nodes=draw,column sep=1mm]
  {
    \node {8}; &[2mm] \node{1}; &[-1mm] \node {6}; \\
    \node {3}; &      \node{5}; &       \node {7}; \\
    \node {4}; &      \node{9}; &       \node {2}; \\
  };
\end{tikzpicture}
\end{codeexample}
%
\begin{codeexample}[]
\begin{tikzpicture}
  \matrix [draw,nodes=draw,column sep=1mm]
  {
    \node {8}; &[2mm] \node(a){1}; &[1cm,between origins] \node(b){6}; \\
    \node {3}; &      \node   {5}; &                      \node   {7}; \\
    \node {4}; &      \node   {9}; &                      \node   {2}; \\
  };
  \draw [<->,red,thick] (a.center) -- (b.center) node [above,midway] {11mm};
\end{tikzpicture}
\end{codeexample}
%
\begin{codeexample}[]
\begin{tikzpicture}
  \matrix [draw,nodes=draw,column sep={1cm,between origins}]
  {
    \node (a) {8}; & \node (b) {1}; &[between borders] \node (c) {6}; \\
    \node     {3}; & \node     {5}; &                  \node     {7}; \\
    \node     {4}; & \node     {9}; &                  \node     {2}; \\
  };
  \draw [<->,red,thick] (a.center) -- (b.center) node [above,midway] {10mm};
  \draw [<->,red,thick] (b.east) -- (c.west) node [above,midway] {10mm};
\end{tikzpicture}
\end{codeexample}


\subsubsection{Cell Styles and Options}

The following styles and options are useful for changing the appearance of all
cell pictures:

\begin{stylekey}{/tikz/every cell=\marg{row}\marg{column} (initially \normalfont empty)}
    This style is installed at the beginning of each cell picture with the two
    parameters being the current \meta{row} and \meta{column} of the cell. Note
    that setting this style to |draw| will \emph{not} cause all nodes to be
    drawn since the |draw| option has to be passed to each node individually.

    Inside this style (and inside all cells), the current \meta{row} and
    \meta{column} number are also accessible via the counters
    |\pgfmatrixcurrentrow| and |\pgfmatrixcurrentcolumn|.
\end{stylekey}

\begin{key}{/tikz/cells=\meta{options}}
    This key adds the \meta{options} to the style |every cell|. It is mainly
    just a shorthand for the code |every cell/.append style=|\meta{options}.
\end{key}

\begin{key}{/tikz/nodes=\meta{options}}
    This key adds the \meta{options} to the style |every node|. It is mainly
    just a shorthand for the code |every node/.append style=|\meta{options}.

    The main use of this option is the install some options for the nodes
    \emph{inside} the matrix that should not apply to the matrix \emph{itself}.
    %
\begin{codeexample}[]
\begin{tikzpicture}
  \matrix [nodes={fill=blue!20,minimum size=5mm}]
  {
    \node {8}; & \node{1}; & \node {6}; \\
    \node {3}; & \node{5}; & \node {7}; \\
    \node {4}; & \node{9}; & \node {2}; \\
  };
\end{tikzpicture}
\end{codeexample}
    %
\end{key}

The next set of styles can be used to change the appearance of certain rows,
columns, or cells. If more than one of these styles is defined, they are
executed in the below order (the |every cell| style is executed before all of
the below).
%
\begin{stylekey}{/tikz/column \meta{number}}
    This style is used for every cell in column \meta{number}.
\end{stylekey}

\begin{stylekey}{/tikz/every odd column}
    This style is used for every cell in an odd column.
\end{stylekey}

\begin{stylekey}{/tikz/every even column}
    This style is used for every cell in an even column.
\end{stylekey}

\begin{stylekey}{/tikz/row \meta{number}}
    This style is used for every cell in row \meta{number}.
\end{stylekey}

\begin{stylekey}{/tikz/every odd row}
    This style is used for every cell in an odd row.
\end{stylekey}

\begin{stylekey}{/tikz/every even row}
    This style is used for every cell in an even row.
\end{stylekey}

\begin{stylekey}{/tikz/row \meta{row number} column \meta{column number}}
    This style is used for the cell in row \meta{row number} and column
    \meta{column number}.
\end{stylekey}
%
\begin{codeexample}[]
\begin{tikzpicture}
  [row 1/.style={red},
   column 2/.style={green!50!black},
   row 3 column 3/.style={blue}]

  \matrix
  {
    \node {8}; & \node{1}; & \node {6}; \\
    \node {3}; & \node{5}; & \node {7}; \\
    \node {4}; & \node{9}; & \node {2}; \\
  };
\end{tikzpicture}
\end{codeexample}

You can use the |column |\meta{number} option to change the alignment for
different columns.
%
\begin{codeexample}[]
\begin{tikzpicture}
  [column 1/.style={anchor=base west},
   column 2/.style={anchor=base east},
   column 3/.style={anchor=base}]
  \matrix
  {
    \node {123}; & \node{456}; & \node {789}; \\
    \node {12}; & \node{45}; & \node {78}; \\
    \node {1}; & \node{4}; & \node {7}; \\
  };
\end{tikzpicture}
\end{codeexample}

In some cases, it is desirable to include some automation in each column/row
separately. A typical example is to apply stripe-pattern to almost all columns
and treat only a few of them differently. Then, nesting these keys can open up
a lot of possibilities; in the following example a ``feature comparison'' table
is demonstrated. It is intentionally made rather verbose to show how the column
and row settings can be progressively overwritten to create certain effects.

\begin{codeexample}[preamble={\usetikzlibrary{matrix,shapes.misc}}]
\begin{tikzpicture}
  [mycol/.style={column #1/.append style={
     every even row/.style={nodes={fill=gray!10}}}
   }]

\matrix [matrix of nodes, nodes in empty cells,
   nodes={text width=2cm, align=center,
          minimum height=1.5em, anchor=center},
   mycol/.list={1,...,5}, % add mycol style to all cols
   column 1/.style={
     row 1 column 1/.style={nodes={fill=none, draw=none}},
     nodes={fill=yellow, chamfered rectangle, inner sep=0,
            chamfered rectangle corners={north west, south east}},
   },
   row 1/.style={nodes={text depth=0.2ex, text width=2cm, text=white}},
   column 2/.append style={row 1/.append style={nodes={fill=green}}},
   column 3/.append style={every even row/.style={nodes={fill=red!10}},
                           row 1 column 3/.append style={nodes={fill=red}}
                          },
   column 4/.append style={row 1/.append style={nodes={fill=blue!40}}},
   column 5/.append style={row 1/.append style={nodes={fill=blue!90!black}}},
  ] (m)
  {
            & Basic     & Standard   & Professional & Enterprise \\
  Feature A & $\bullet$ & $\bullet$  & $\bullet$    & $\bullet$  \\
  Feature B & $\bullet$ & $\bullet$  & $\bullet$    & $\bullet$  \\
  Feature C &           &            &              & $\bullet$  \\
  Feature D &           & $\bullet$  & $\bullet$    & $\bullet$  \\
  Feature E &           &            & $\bullet$    & $\bullet$  \\
  };

\draw[ultra thick, rounded corners=1mm, red] (m-1-3.north west)
                          rectangle (m-6-3.south east);
\node[anchor=north,
   align=center, text=red] at (m-6-3.south) {Popular\\Choice!};
\end{tikzpicture}
\end{codeexample}


The order in which these styles are applied is configurable.  You can also
install your own styles.  The following styles (in fact, internally they are
|/.code| keys) wrap the styles introduced in the previous paragraph passing the
correct argument and ensuring that they are only called for even or odd rows.
It's not recommended to override these.

\begin{stylekey}{/tikz/matrix/inner style/every cell}
    Wraps |/tikz/every cell|.
\end{stylekey}
\begin{stylekey}{/tikz/matrix/inner style/column}
    Wraps |/tikz/column |\meta{number}.
\end{stylekey}
\begin{stylekey}{/tikz/matrix/inner style/even odd column}
    Wraps |/tikz/every even column| and |/tikz/every odd column|.
\end{stylekey}
\begin{stylekey}{/tikz/matrix/inner style/row}
    Wraps |/tikz/row |\meta{number}.
\end{stylekey}
\begin{stylekey}{/tikz/matrix/inner style/even odd row}
    Wraps |/tikz/every even row| and |/tikz/every odd row|.
\end{stylekey}
\begin{stylekey}{/tikz/matrix/inner style/this cell}
    Wraps |/tikz/row |\meta{number}| column |\meta{number}.
\end{stylekey}

\begin{stylekey}{/tikz/matrix/inner style order}
    The order in which these styles are applied to the matrix cells is
    specified by this key.  By default it is
    %
\begin{codeexample}[code only]
\tikzset{
  matrix/inner style order={
    every cell,
    column,
    even odd column,
    row,
    even odd row,
    this cell,
  }
}
\end{codeexample}
    %
    You can use this to install your own styles here, but only \emph{names} of
    styles are permitted here.  The style specification has to be placed
    outside of |matrix/inner style order| and unless it is installed inside
    |/tikz/matrix/inner style/|, it has to be fully qualified.
    %
\begin{codeexample}[code only]
\tikzset{
  my style/.code={%
    \ifnum\pgfmatrixcurrentcolumn=2
        \tikzset{font=\itshape}%
    \fi
  },
  matrix/inner style order={
      every cell,
      even odd column,
      even odd row,
      column,
      row,
      this cell,
      /tikz/my style,
  },
}
\end{codeexample}
    %
\end{stylekey}

In many matrices all cell pictures have nearly the same code. For example,
cells typically start with |\node{| and end |};|. The following options allow
you to execute such code in all cells:

\begin{key}{/tikz/execute at begin cell=\meta{code}}
    The code will be executed at the beginning of each nonempty cell.
\end{key}
%
\begin{key}{/tikz/execute at end cell=\meta{code}}
    The code will be executed at the end of each nonempty cell.
\end{key}
%
\begin{key}{/tikz/execute at empty cell=\meta{code}}
    The code will be executed inside each empty cell.
\end{key}
%
\begin{codeexample}[]
\begin{tikzpicture}
  [matrix of nodes/.style={
     execute at begin cell=\node\bgroup,
     execute at end cell=\egroup;%
   }]
  \matrix [matrix of nodes]
  {
    8 & 1 & 6 \\
    3 & 5 & 7 \\
    4 & 9 & 2 \\
  };
\end{tikzpicture}
\end{codeexample}
%
\begin{codeexample}[]
\begin{tikzpicture}
  [matrix of nodes/.style={
     execute at begin cell=\node\bgroup,
     execute at end cell=\egroup;,%
     execute at empty cell=\node{--};%
   }]
  \matrix [matrix of nodes]
  {
    8 & 1 &   \\
    3 &   & 7 \\
      &   & 2 \\
  };
\end{tikzpicture}
\end{codeexample}

The |matrix| library defines a number of styles that make use of the above
options.


\subsection{Anchoring a Matrix}

Since matrices are nodes, they can be anchored in the usual fashion using the
|anchor| option. However, there are two ways to influence this placement
further. First, the following option is often useful:

\begin{key}{/tikz/matrix anchor=\meta{anchor}}
    This option has the same effect as |anchor|, but the option applies only to
    the matrix itself, not to the cells inside. If you just say |anchor=north|
    as an option to the matrix node, all nodes inside matrix will also have
    this anchor, unless it is explicitly set differently for each node. By
    comparison, |matrix anchor| sets the anchor for the matrix, but for the
    nodes inside the value of |anchor| remain unchanged.
    %
\begin{codeexample}[]
\begin{tikzpicture}
  \matrix [matrix anchor=west] at (0,0)
  {
    \node {123}; \\ % still center anchor
    \node {12}; \\
    \node {1}; \\
  };
  \matrix [anchor=west] at (0,-2)
  {
    \node {123}; \\ % inherited west anchor
    \node {12}; \\
    \node {1}; \\
  };
\end{tikzpicture}
\end{codeexample}
    %
\end{key}

The second way to anchor a matrix is to use \emph{an anchor of a node inside
the matrix}. For this, the |anchor| option has a special effect when given as
an argument to a matrix:

\begin{key}{/tikz/anchor=\meta{anchor or node.anchor}}
    Normally, the argument of this option refers to anchor of the matrix node,
    which is the node that includes all of the stuff of the matrix. However,
    you can also provide an argument of the form \meta{node}|.|\meta{anchor}
    where \meta{node} must be node defined inside the matrix and \meta{anchor}
    is an anchor of this node. In this case, the whole matrix is shifted around
    in such a way that this particular anchor of this particular node lies at
    the |at| position of the matrix. The same is true for |matrix anchor|.
    %
\begin{codeexample}[]
\begin{tikzpicture}
  \draw[help lines] (0,0) grid (3,2);
  \matrix[matrix anchor=inner node.south,anchor=base,row sep=3mm] at (1,1)
  {
    \node {a}; & \node             {b}; & \node {c}; & \node {d}; \\
    \node {a}; & \node(inner node) {b}; & \node {c}; & \node {d}; \\
    \node {a}; & \node             {b}; & \node {c}; & \node {d}; \\
  };
  \draw (inner node.south) circle (1pt);
\end{tikzpicture}
\end{codeexample}
    %
\end{key}


\subsection{Considerations Concerning Active Characters}

Even though \tikzname\ seems to use |&| to separate cells, \pgfname\ actually
uses a different command to separate cells, namely the command
|\pgfmatrixnextcell| and using a normal |&| character will normally fail. What
happens is that, \tikzname\ makes |&| an active character and then defines this
character to be equal to |\pgfmatrixnextcell|. In most situations this will
work nicely, but sometimes |&| cannot be made active; for instance because the
matrix is used in an argument of some macro or the matrix contains nodes that
contain normal |{tabular}| environments. In this case you can use the following
option to avoid having to type |\pgfmatrixnextcell| each time:

\begin{key}{/tikz/ampersand replacement=\meta{macro name or empty}}
    If a macro name is provided, this macro will be defined to be equal to
    |\pgfmatrixnextcell| inside matrices and |&| will not be made active. For
    instance, you could say |ampersand replacement=\&| and then use |\&| to
    separate columns as in the following example:
    %
\begin{codeexample}[]
\tikz
  \matrix [ampersand replacement=\&]
  {
    \draw (0,0)   circle (4mm); \& \node[rotate=10] {Hello};        \\
    \draw (0.2,0) circle (2mm); \& \fill[red]   (0,0) circle (3mm); \\
  };
\end{codeexample}
    %
\end{key}


\subsection{Examples}

The following examples are adapted from code by Mark Wibrow. The first two
redraw pictures from Timothy van Zandt's PStricks documentation:
%
{\catcode`\|=12
\begin{codeexample}[preamble={\usetikzlibrary{matrix}}]
\begin{tikzpicture}
  \matrix [matrix of math nodes,row sep=1cm]
  {
    |(U)| U &[2mm]                       &[8mm]    \\
            &      |(XZY)| X \times_Z Y  &      |(X)| X \\
            &      |(Y)|   Y             &      |(Z)| Z \\
  };
  \begin{scope}[every node/.style={midway,auto,font=\scriptsize}]
    \draw [double, dashed] (U)   -- node {$x$} (X);
    \draw                  (X)   -- node {$p$} (X -| XZY.east)
                           (X)   -- node {$f$} (Z)
                                 -- node {$g$} (Y)
                                 -- node {$q$} (XZY)
                                 -- node {$y$} (U);
   \end{scope}
\end{tikzpicture}
\end{codeexample}

\begin{codeexample}[
    preamble={\usetikzlibrary{matrix}},
    pre={\definecolor{graphicbackground}{rgb}{0.96,0.96,0.8}},
]
\begin{tikzpicture}[>=stealth,->,shorten >=2pt,looseness=.5,auto]
  \matrix [matrix of math nodes,
           column sep={2cm,between origins},
           row sep={3cm,between origins},
           nodes={circle, draw, minimum size=7.5mm}]
  {
            & |(A)| A &         \\
    |(B)| B & |(E)| E & |(C)| C \\
            & |(D)| D           \\
  };
  \begin{scope}[every node/.style={font=\small\itshape}]
    \draw (A) to [bend left] node [midway]   {g} (B);
    \draw (B) to [bend left] node [midway]   {f} (A);
    \draw (D) --             node [midway]   {c} (B);
    \draw (E) --             node [midway]   {b} (B);
    \draw (E) --             node [near end] {a} (C);
    \draw [-,line width=8pt,draw=graphicbackground]
          (D) to [bend right, looseness=1] (A);
    \draw (D) to [bend right, looseness=1]
            node [near start] {b} node [near end] {e} (A);
  \end{scope}
\end{tikzpicture}
\end{codeexample}

\begin{codeexample}[preamble={\usetikzlibrary{matrix}}]
\begin{tikzpicture}
  \matrix (network)
    [matrix of nodes,%
     nodes in empty cells,
     nodes={outer sep=0pt,circle,minimum size=4pt,draw},
     column sep={1cm,between origins},
     row sep={1cm,between origins}]
  {
                  &                &                 & \\
                  &                &                 & \\
    |[draw=none]| & |[xshift=1mm]| & |[xshift=-1mm]|   \\
  };
  \foreach \a in {1,...,4}{
    \draw (network-3-2) -- (network-2-\a);
    \draw (network-3-3) -- (network-2-\a);
    \draw [-stealth] ([yshift=5mm]network-1-\a.north) -- (network-1-\a);
    \foreach \b in {1,...,4}
      \draw (network-1-\a) -- (network-2-\b);
  }
  \draw [stealth-] ([yshift=-5mm]network-3-2.south) -- (network-3-2);
  \draw [stealth-] ([yshift=-5mm]network-3-3.south) -- (network-3-3);
\end{tikzpicture}
\end{codeexample}

The following example is adapted from code written by Kjell Magne Fauske, which
is based on the following paper: K.~Bossley, M.~Brown, and C.~Harris,
Neurofuzzy identification of an autonomous underwater vehicle,
\emph{International Journal of Systems Science}, 1999, 30, 901--913.
%
\begin{codeexample}[preamble={\usetikzlibrary{arrows,shapes.geometric}}]
\begin{tikzpicture}
  [auto,
   decision/.style={diamond, draw=blue, thick, fill=blue!20,
                    text width=4.5em,align=flush center,
                    inner sep=1pt},
   block/.style   ={rectangle, draw=blue, thick, fill=blue!20,
                    text width=5em,align=center, rounded corners,
                    minimum height=4em},
   line/.style    ={draw, thick, -latex',shorten >=2pt},
   cloud/.style   ={draw=red, thick, ellipse,fill=red!20,
                    minimum height=2em}]

  \matrix [column sep=5mm,row sep=7mm]
  {
    % row 1
      \node [cloud] (expert)   {expert}; &
      \node [block] (init)     {initialize model}; &
      \node [cloud] (system)   {system}; \\
    % row 2
      & \node [block] (identify) {identify candidate model}; & \\
    % row 3
      \node [block] (update)   {update model};  &
      \node [block] (evaluate) {evaluate candidate models}; & \\
    % row 4
      & \node [decision] (decide) {is best candidate}; & \\
    % row 5
      & \node [block] (stop)      {stop}; & \\
  };
  \begin{scope}[every path/.style=line]
    \path          (init)     -- (identify);
    \path          (identify) -- (evaluate);
    \path          (evaluate) -- (decide);
    \path          (update)   |- (identify);
    \path          (decide)   -| node [near start] {yes} (update);
    \path          (decide)   -- node [midway] {no} (stop);
    \path [dashed] (expert)   -- (init);
    \path [dashed] (system)   -- (init);
    \path [dashed] (system)   |- (evaluate);
  \end{scope}
\end{tikzpicture}
\end{codeexample}
}


%%% Local Variables:
%%% mode: latex
%%% TeX-master: "pgfmanual"
%%% End:

% Copyright 2006 by Till Tantau
%
% This file may be distributed and/or modified
%
% 1. under the LaTeX Project Public License and/or
% 2. under the GNU Free Documentation License.
%
% See the file doc/generic/pgf/licenses/LICENSE for more details.


\section{Calendar Library}

\label{section-calender-snakes}

\begin{pgflibrary}{calendar}
  The library defines the |\calendar| command, which can be used to
  typeset calendars. The command relies on the |\pgfcalendar| command
  from the |pgfcalendar| package, which is loaded automatically.

  The |\calendar| command is quite configurable, allowing you to
  produce all kinds of different calendars.
\end{pgflibrary}


\subsection{Calendar Command}

The core command for creating calendars in \tikzname\ is the
following:
\begin{command}{\calendar \meta{calendar specification}|;|}
  The syntax for this command is similar to commands like |\node| or
  |\matrix|. However, it has its complete own parser and only those
  commands described in the following will be recognized, nothing
  else. Note, furthermore, that a \meta{calendar specification} is not
  a path specification, indeed, no path is created for the calendar.

  
\end{command}


\subsection{Day Arrangements}


\subsection{Month Labels}


\subsection{Examples}

In the following, some example calendars are shown that come either
from real applications or are just nice to look at.

Let us start with a year-2100-countdown, in which we cross out dates
as we approach the big celebration. For
this, we set the shape to |strike out| for these dates.

\begin{codeexample}[leave comments]
\begin{tikzpicture}
  \calendar
  [
    dates=2099-12-01 to 2100-01-last,
    week list,inner sep=2pt,month label above centered,
    month text=\%mt \%y0
  ]
  if (at most=2099-12-29) [nodes={strike out,draw}]
  if (weekend)            [black!50,nodes={draw=none}]
  ;
\end{tikzpicture}
\end{codeexample}

The next calendar shows a deadline, which is 10 days in the future
from the current date. The last three days before the deadline are in
red, because we really should be done by then. All days on which we
can no longer work on the project are crossed out.

\begin{codeexample}[leave comments]
\begin{tikzpicture}
  \calendar
  [
    dates=\year-\month-\day+-25 to \year-\month-\day+25,
    week list,inner sep=2pt,month label above centered,
    month text=\textit{\%mt \%y0}
  ]
  if (at least=\year-\month-\day) {}
    else [nodes={strike out,draw}]
  if (at most=\year-\month-\day+7)
    [green!50!black]
  if (between=\year-\month-\day+8 and \year-\month-\day+10)
    [red]
  if (Sunday)
    [gray,nodes={draw=none}]
  ;
\end{tikzpicture}
\end{codeexample}

The following example is a futuristic calendar that is all about circles:

\begin{codeexample}[]
\sffamily

\colorlet{winter}{blue}  
\colorlet{spring}{green!60!black}  
\colorlet{summer}{orange}  
\colorlet{fall}{red}  

% A counter, since TikZ is not clever enough (yet) to handle
% arbitrary angle systems.
\newcount\mycount

\begin{tikzpicture}[transform shape]
  \tikzstyle{every day}=[anchor=mid,font=\fontsize{6}{6}\selectfont]
  \node{\normalsize\the\year};
  \foreach \month/\monthcolor in
    {1/winter,2/winter,3/spring,4/spring,5/spring,6/summer,
     7/summer,8/summer,9/fall,10/fall,11/fall,12/winter}
  {
    % Computer angle:
    \mycount=\month
    \advance\mycount by -1
    \multiply\mycount by 30
    \advance\mycount by -90

    % The actual calendar
    \calendar at (\the\mycount:6.4cm)
    [
      dates=\the\year-\month-01 to \the\year-\month-last,
    ]
    if (day of month=1) {\color{\monthcolor}\tikzmonthcode}
    if (Sunday) [red]
    if (all) 
    {
      % Again, compute angle
      \mycount=1
      \advance\mycount by -\pgfcalendarcurrentday
      \multiply\mycount by 11
      \advance\mycount by 90
      \pgftransformshift{\pgfpointpolar{\mycount}{1.4cm}}
    };
  }
\end{tikzpicture}
\end{codeexample}



%%% Local Variables: 
%%% mode: latex
%%% TeX-master: "pgfmanual-pdftex-version"
%%% End: 

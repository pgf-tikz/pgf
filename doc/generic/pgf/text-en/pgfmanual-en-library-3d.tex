% Copyright 2003 by Till Tantau <tantau@cs.tu-berlin.de>.
%
% This program can be redistributed and/or modified under the terms
% of the LaTeX Project Public License Distributed from CTAN
% archives in directory macros/latex/base/lppl.txt.



\section{Three Dimensional Drawing Library}

\begin{package}{pgflibrarytikz3d}
  This packages, which is still under construction, provides some
  styles and options for drawing three dimensional shapes.

  \textbf{This package is not officially released and commands may
    change or disappear without backward compatibility support.}
\end{package}

This package is not yet documented. Here is, at least, an example of
where this whole thing is supposed to head.

\begin{codeexample}[]
\begin{tikzpicture}[z={(10:10mm)},x={(-45:5mm)}]  
  \def\wave{
    \draw[fill,thick,fill opacity=.2]
     (0,0) sin (1,1) cos (2,0) sin (3,-1) cos (4,0) 
           sin (5,1) cos (6,0) sin (7,-1) cos (8,0)
           sin (9,1) cos (10,0)sin (11,-1)cos (12,0);
    \foreach \shift in {0,4,8}
    {
      \begin{scope}[xshift=\shift cm,thin]
        \draw (.5,0)  -- (0.5,0 |- 45:1cm);
        \draw (1,0)   -- (1,1);
        \draw (1.5,0) -- (1.5,0 |- 45:1cm);
        \draw (2.5,0) -- (2.5,0 |- -45:1cm);
        \draw (3,0)   -- (3,-1);
        \draw (3.5,0) -- (3.5,0 |- -45:1cm);
      \end{scope}
    }
  }
  \begin{scope}[canvas is zy plane at x=0,fill=blue]
    \wave
    \node at (6,-1.5) [transform shape] {magnetic field};
  \end{scope}
  \begin{scope}[canvas is zx plane at y=0,fill=red]
    \draw[help lines] (0,-2) grid (12,2);
    \wave
    \node at (6,1.5) [rotate=180,xscale=-1,transform shape] {electric field};
  \end{scope}
\end{tikzpicture}
\end{codeexample}

\begin{codeexample}[]
\begin{tikzpicture}
  \begin{scope}[canvas is zy plane at x=0]
    \draw (0,0) circle (1cm);
    \draw (-1,0) -- (1,0) (0,-1) -- (0,1);
  \end{scope}

  \begin{scope}[canvas is zx plane at y=0]
    \draw (0,0) circle (1cm);
    \draw (-1,0) -- (1,0) (0,-1) -- (0,1);
  \end{scope}

  \begin{scope}[canvas is xy plane at z=0]
    \draw (0,0) circle (1cm);
    \draw (-1,0) -- (1,0) (0,-1) -- (0,1);
  \end{scope}
\end{tikzpicture}
\end{codeexample}





%%% Local Variables: 
%%% mode: latex
%%% TeX-master: "pgfmanual-pdftex-version"
%%% End: 

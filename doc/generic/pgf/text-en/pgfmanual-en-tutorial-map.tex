% Copyright 2008 by Till Tantau
%
% This file may be distributed and/or modified
%
% 1. under the LaTeX Project Public License and/or
% 2. under the GNU Free Documentation License.
%
% See the file doc/generic/pgf/licenses/LICENSE for more details.

\section{Tutorial: A Lecture Map for Johannes}

In this tutorial we explore the tree and mind map mechanisms of
\tikzname.

Johannes is quite excited: Forthe first time he will be teaching a
course all by himself during the upcoming semester! Unfortunately, the
course is not on his favorite subject, namely Theoretical Immunology,
but rather it is a general Introduction to Computer Science, but as a
young academic Johannes is not likely to complain too loudly.
In order to help the students get a general overview of what is
going to happen during the course as a whole, he intends to draw some
kind of tree or graph containing the basic concepts. He got this idea
from his old professor who seems to be using these ``lecture maps''
with some success. Independently of the success of these maps,
Johannes thinks they look quite neat.


\subsection{Problem Statement}

Johannes wishes to create a lecture map with the following features:
\begin{enumerate}
\item It should contain a tree or graph depicting the main concepts.
\item It should somehow visualize the different lectures that will be
  taught. Note that the lectures are not necessarily the same as teh
  concepts since the graph may contain more concepts than will be
  addressed in lectures and some concepts may be addressed during more
  than one lecture.
\item The map should also contain a calendar showing when the
  individual lectures will be given.
\item The aesthetical reaons, the whole map should have a  visually
  nice and information-rich background.
\end{enumerate}

As always, Johannes will have to include the right libraries and
setup the environment. Since Johannes is going to use the
|mindmap| library and since he wishes to show a calendar, he will need
the |mindmap| and the |calendar| libraries. In order to put something
on a background layer, it seems like a good idea to also include the
|background| library.


\subsection{Introduction to Trees}

The first choice Johannes must make is whether he will organize the
concepts are a tree, with root concepts and concept branches and leaf
concepts, or as a general graph. The tree implicitly organizes the
concepts, while a graph is more flexible. Johannes decides to
compromise: Basically, the concepts will be organized as a
tree. However, he will selectively add connections between concepts
that are related, but which appear on different levels or branches of
the tree.

Johannes starts with a simple tree-like list of concepts that he
intends to talk about: 
\begin{itemize}
\item Applied Computer Science
  \begin{itemize}
  \item Applications
  \item Databases
  \item Operating Systems
  \item World-Wide-Web
  \item Society and Computers
  \end{itemize}
\item Practical Computer Science
  \begin{itemize}
  \item Algorithms
  \item Data Structures
  \item Programming Languages
  \item Software Engineering 
  \end{itemize}
\item Technical Computer Science
  \begin{itemize}
  \item Integrated Circuits
  \item Chip Design
  \item Networks
  \item Hardware Components
  \end{itemize}
\item Theoretical Computer Science
  \begin{itemize}
  \item Complexity Theory
  \item Specification and Verification
  \item Formal Language Theory
  \item Type Theory
  \item Cryptology 
  \end{itemize}
\end{itemize}

Naturally, this list is neither complete nor perfect, but Johannes
will use it for his first planning.




\subsection{Creating the Lecture Map}

\subsection{Adding the Lecture Annotations}

\subsection{Adding the Calendar}

\subsection{Adding the Background}

\subsection{The Complete Code}

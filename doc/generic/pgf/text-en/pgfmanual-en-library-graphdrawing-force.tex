% Copyright 2011 by Jannis Pohlmann
%
% This file may be distributed and/or modified
%
% 1. under the LaTeX Project Public License and/or
% 2. under the GNU Free Documentation License.
%
% See the file doc/generic/pgf/licenses/LICENSE for more details.

\section{Force-Based Graph Drawing Algorithms}
\label{section-library-graphdrawing-force-based}

{\emph{by Jannis Pohlmann}}


\begin{tikzlibrary}{graphdrawing.force}
  Load this package when you wish to use force-based graph drawing
  algorithms. You should load the |graphdrawing| library first.
\end{tikzlibrary}

\ifluatex\relax\else{LuaTeX is required for setting this manual section.}\expandafter\endinput\fi


\subsection{Overview}

% TODO Jannis: Explain ideas and concepts behind force-based graph
% drawing algorithms. Briefly explain the various approaches in that
% specific area of graph drawing algorithms (e.g. spring,
% spring-electrical and multidimensional embedding). 

...

\subsubsection{Spring and Spring-Electrical Layouts}

% TODO Jannis: Explain ideas and concepts behind spring and
% spring-electrical algorithms. Describe the technical as well as visual
% differences between the two techniques (think: no attractive forces
% and no peripheral effects in spring layouts). Explain why they were
% consolidated in the common family 'spring layout'.

% TODO Jannis: Also, somewhere, present the two spring-electrical and 
% the one spring algorithm and explain how they work and to what 
% parameters they respond well (and how in general). Explain how to
% make the best use of either algorithm.

\begin{key}{/graph drawing/spring layout=\meta{options}}
  \keyalias{tikz}\keyalias{tikz/graphs}
  
  Similar to the |>| option, this ``generic'' name for a spring layout
  algorithm is not hardwired to any specific algorithm. Rather, users
  can select an algorithm somewhere at the beginning of their program
  and then just write |\graph[spring layout]| to draw a graph.

  The \meta{options} will be forwarded to the currently selected
  algorithm.
\begin{codeexample}[]
\tikz \graph [spring layout] { a -> {b,c} };    
\end{codeexample}
  
  To change the algorithm, change the following key:
  \begin{key}{/graph drawing/spring layout/default
  algorithm=\meta{algorithm} (initially Hu2006 spring)}

    Set this key to the spring graph drawing algorithm of your choice.
  \end{key}
\end{key}

\begin{key}{/graph drawing/spring electrical layout=\meta{options}}
  \keyalias{tikz}\keyalias{tikz/graphs}

  Similar to the |>| option, this ``generic'' name for a 
  spring-electrical layout algorithm is not hardwired to any specific
  algorithm. Rather, users can select an algorithm somewhere at the
  beginning of their program and then just write 
  |\graph[spring electrical layout]| to draw a graph.

  The \meta{options} will be forwarded to the currently selected
  algorithm.
  \begin{codeexample}[]
\tikz \graph [spring electrical layout] { a -> {b,c} };
  \end{codeexample}

  To change the algorithm, change the following key:
  \begin{key}{/graph drawing/spring electrical layout/default
    algorithm=\meta{algorithm} (initially Hu2006 spring electrical)}

    Set this key to the spring-electrical graph drawing algorithm of 
    your choice.
  \end{key}
\end{key}

\subsection{Common Options}

The spring and and spring-electrical drawing algorithms are very similar
in terms of their parameters and the constraints they can handle. They
thus share a number of common \tikzname\ options for fine-tuning. These
options are split up into \emph{graph options} that can be specified
once for a graph, \emph{node options} that can be specified for each
node and \emph{edge options} that can be specified for each edge.

\subsubsection{Graph Options}

% TODO Jannis: This might be worth implementing. It's not very useful in
% the Hu2006 algorithm as it uses the Barnes-Hut algorithm, but the 
% Walshaw2000 algorithm can benefit from it.
%
%\begin{key}{/tikz/influence cutoff distance=\meta{dimension} (initially
%  0pt)}
%  Specifies a distance beyond which the attractive and repulsive forces 
%  between two nodes are assumed to be virtually non-existent. If 
%  \meta{dimension} is set to |0pt|, the cutoff distance is computed 
%  automatically.
%
%  Depending on the graph drawing algorithm being used, the distance
%  between two nodes is computed either based on the graph distance
%  (spring algorithm) or based on the Euclidean distance
%  (spring-electrical algorithm).
%  \begin{codeexample}[]
%  \end{codeexample}
%\end{key}

\begin{key}{/graph drawing/spring layout/iterations=\meta{number}
  (initially 500)}
  \keyalias{graph drawing/spring electrical layout}

  Limits the number of iterations of algorithms for spring and 
  spring-electrical layouts to \meta{number}.

  Depending on the characteristics of the input graph and the parameters
  chosen for the spring or spring-electrical algorithm, minimizing the
  system energy may require many iterations.

  In these situations it may come in handy to limit the number of
  iterations. This feature can also be useful to draw the same graph
  after different iterations and thereby demonstrate how the spring or
  spring-electrical algorithm improves the drawing step by step.

  The following example shows two drawings generated with the default
  |spring layout| algorithm using two different |iteration| limits:
  \begin{codeexample}[]
\tikz \graph [spring layout={iterations=10}]  { subgraph K_n [n=4] };
\tikz \graph [spring layout={iterations=500}] { subgraph K_n [n=4] };
  \end{codeexample}

  This key also works with |spring electrical layout| algorithms:
  \begin{codeexample}[width=5cm]
\tikz \graph [spring electrical layout={iterations=10}]  
  { subgraph K_n [n=4] };

\tikz \graph [spring electrical layout={iterations=500}] 
  { subgraph K_n [n=4] };
  \end{codeexample}
\end{key}

\begin{key}{/graph drawing/spring layout/random seed=\meta{number} 
  (initially 42)}
  \keyalias{graph drawing/spring electrical layout}
  
  Specifies the seed used for Lua's pseudo-random number generator. If
  set to something other than |0|, the random number sequence generated
  by the pseudo-random number generator will be the same at every run.
  The resulting graph drawings will be reproducible in consecutive runs,
  despite randomized elements used in the algorithm.
  If set to |0|, the results are not guaranteed to be reproducible.
  
  A special application of |random seed| is to unravel visually
  imperfect layouts. Often when drawings feature a few undesired edge
  crossings, this may have been caused by a unfavorable random number
  sequence. Here is an example where a different |random seed| 
  significantly improves the drawing of a graph:
  \begin{codeexample}[width=6.0cm]
\tikz \graph [spring layout={random seed=1}] 
  { 5 -- subgraph C_n [n=4] };

\tikz \graph [spring layout={random seed=4}] 
  { 5 -- subgraph C_n [n=4] };
  \end{codeexample}
%  \begin{codeexample}[]
%% bad layout due to unfavorable random number sequence
%\tikz \graph [/graph drawing/spring layout/default algorithm=Walshaw2000 spring electrical,
%              spring layout={random seed=1, coarsen=false}, orient=7-8] 
%{ subgraph Grid_n [n=8, wrap after=2] };
%
%% better layout thanks to a different random seed
%\tikz \graph [/graph drawing/spring layout/default algorithm=Walshaw2000 spring electrical,
%              spring layout={random seed=2, coarsen=false}, orient=7-8] 
%{ subgraph Grid_n [n=8, wrap after=2] };
%  \end{codeexample}
\end{key}

\begin{key}{/graph drawing/spring layout/natural spring
  dimension=\meta{dimension} (initially 1cm)}
  \keyalias{graph drawing/spring electrical layout}

  Defines the equilibrium length of a spring between two nodes in the
  graph. Both, the |spring layout| and |spring electrical layout|
  algorithms aim at minimizing the energy of the system. The
  natural or equilibrium length of a spring specifies the natural
  distance of any pair of adjacent nodes.

  The |natural spring dimension| option is mostly intended as a scaling
  parameter but it can have other side-effects on the arrangement of
  nodes in the graph as well.

  The following example shows how a simple graph can be scaled by
  changing the |natural spring dimension|:
  \begin{codeexample}[width=5cm]
\tikz \graph [spring layout={natural spring dimension=0.7cm}]
  { subgraph C_n[n=3] };

\tikz \graph [spring layout]
  { subgraph C_n[n=3] };

\tikz \graph [spring layout={natural spring dimension=1.5cm}]
  { subgraph C_n[n=3] };
  \end{codeexample}

  The option works in the very same way with |spring electrical layout|
  algorithms:
  \begin{codeexample}[width=5cm]
\tikz \graph [spring electrical layout={natural spring dimension=0.7cm}]
  { subgraph C_n[n=3] };

\tikz \graph [spring electrical layout]
  { subgraph C_n[n=3] };

\tikz \graph [spring layout={natural spring dimension=1.5cm}]
  { subgraph C_n[n=3] };
  \end{codeexample}
\end{key}

% TODO Jannis: Document this option.
%\begin{key}{/graph drawing/spring layout/cooling factor=\meta{number}
%  (initially 0.95)}
%  \keyalias{graph drawing/spring electrical layout}
%\end{key}

% TODO Jannis: Document this option.
%\begin{key}{/graph drawing/spring layout/convergence
%  tolerance=\meta{number}}
%  \keyalias{graph drawing/spring electrical layout}
%\end{key}

% TODO Jannis: Document this option.
%\begin{key}{/graph drawing/spring layout/initial step 
%  dimension=\meta{dimension} (initially 0)}
%  \keyalias{graph drawing/spring electrical layout}
%\end{key}

\begin{key}{/graph drawing/spring layout/coarsen=\opt{\meta{boolean}}
  (default true, initially true)}
  \keyalias{graph drawing/spring electrical layout}

  Defines whether or not a multilevel approach is used that
  iteratively coarsens the input graph into graphs $G_1,\dots,G_l$ with 
  a smaller and smaller number of nodes. The coarsening stops as soon as
  a minimum number of nodes is reached, as set via the 
  |minimum graph size| option, or if, in the last iteration, the 
  number of nodes was not reduced by at least the ratio specified via 
  |downsize ratio|. 

  A random initial layout is computed for the coarsest graph $G_l$ first.
  Afterwards, it is laid out by computing the attractive and repulsive
  forces between its nodes. 
  
  In the subsequent steps, the previous coarse graph $G_{l-1}$ is 
  restored and its node positions are interpolated from the nodes in 
  $G_l$. $G_{l-1}$ is again laid out by computing the forces between 
  its nodes. These steps are repeated with $G_{l-2},\dots,G_1$ until 
  the original input graph $G_0$ has been restored, interpolated 
  and laid out.

  The idea behind this approach is that, by arranging recursively 
  formed supernodes first and then interpolating and arranging their
  subnodes step by step, the algorithm is less likely to settle in a
  local energy minimum (of which there can be many, particularly for
  large graphs). The quality of the drawings with coarsening enabled is
  expected to be higher than graphics where this feature is not applied.

  There are a number of options to fine-tune the coarsening phase.
  They are consolidated in the |/graph drawing/spring layout/coarsening|
  prefix described below.

  The following example demonstrates how coarsening can improve the
  quality of graph drawings generated with the 
  |Walshaw2000 spring electrical| algorithm.
  \begin{codeexample}[width=5cm]
\pgfkeys{
  /graph drawing/spring electrical layout/default algorithm=%
    Walshaw2000 spring electrical%
}

\tikz \graph [spring electrical layout={coarsen=false}, orient=3|4] 
  { 
    { [clique] 1, 2 } -- 3 -- 4 -- { 5, 6, 7 }
  };

\tikz \graph [spring electrical layout={coarsen}, orient=3|4] 
  { 
    { [clique] 1, 2 } -- 3 -- 4 -- { 5, 6, 7 }
  };
  \end{codeexample}
\end{key}

\begin{key}{/graph drawing/spring layout/coarsening=\marg{options}}
  \keyalias{graph drawing/spring electrical layout}

  Executes the \meta{options} with the path prefix 
  |/graph drawing/spring layout/coarsening|.

  These options can be used to configure the coarsening approach
  described in the documentation of the 
  |/graph drawing/spring layout/coarsen| option.
\end{key}

\begin{key}{/graph drawing/spring layout/coarsening/minimum graph
  size=\meta{number} (initially 2)}
  \keyalias{graph drawing/spring electrical layout}

  Defines the minimum number of nodes down to which the graph is 
  coarsened iteratively. The first graph that has a smaller or equal 
  number of nodes becomes the coarsest graph $G_l$, where $l$ is the 
  number of coarsening steps. The algorithm proceeds with the steps 
  described in the documentation of the 
  |/graph drawing/spring layout/coarsen| option.

  In the following example the same graph is coarsened down to two
  and four nodes, respectively. The layout of the original graph is 
  interpolated from the random initial layout and is not improved
  further because the forces are not computed (0 iterations). Thus, 
  in the two graphs, the nodes are placed at exactly two and four
  coordinates in the final drawing:
  \begin{codeexample}[width=5.5cm]
\tikz \graph [spring layout={
                iterations=0,
                coarsening={minimum graph size=2}
              }] 
{ subgraph C_n [n=8] };

\tikz \graph [spring layout={
                iterations=0,
                coarsening={minimum graph size=4}
              }] 
{ subgraph C_n [n=8] };
  \end{codeexample}
\end{key}

\begin{key}{/graph drawing/spring layout/coarsening/downsize
  ratio=\meta{number} (initially 0.25)}
  \keyalias{graph drawing/spring electrical layout}

  Minimum ratio between 0 and 1 by which the number of nodes between 
  two coarse graphs $G_i$ and $G_{i+1}$ need to be reduced in order for 
  the coarsening to stop and for the algorithm to use $G_{i+1}$ as the 
  coarsest graph $G_l$. Aside from the input graph, the optimal value 
  of |downsize ratio| mostly depends on the coarsening scheme being
  used. Possible schemes are |collapse independent edges| and 
  |connect independent nodes| which are explained later in this
  document.

  Increasing this option possibly reduces the number of coarse
  graphs computed during the coarsening phase as coarsening will stop as
  soon as a coarse graph does not reduce the number of nodes
  substantially. This may speed up the algorithm but if the size of the 
  coarsest graph $G_l$ is much larger than |minimum graph size|, the 
  multilevel approach may not produce drawings as good as with a lower
  |downsize ratio|.
  \begin{codeexample}[width=5cm]
\pgfkeys{
  /graph drawing/spring electrical layout/default algorithm=%
    Walshaw2000 spring electrical%
}

% 1. ratio too high, coarsening stops early, benefits are lost
\tikz \graph [spring electrical layout={
                coarsening={downsize ratio=1.0},
                natural spring dimension=0.7cm,
              }, orient=3|4] 
  { 
    { [clique] 1, 2 } -- 3 -- 4 -- { 5, 6, 7 }
  };

% 2. ratio set to default, coarsening benefits are visible
\tikz \graph [spring electrical layout={
                coarsening={downsize ratio=0.2},
                natural spring dimension=0.7cm,
              }, orient=3|4] 
  { 
    { [clique] 1, 2 } -- 3 -- 4 -- { 5, 6, 7 }
  };
  \end{codeexample}
\end{key}

\begin{key}{/graph drawing/spring layout/coarsening/collapse independent
  edges=\opt{\meta{boolean}} (default true, initially true)}
  \keyalias{graph drawing/spring electrical layout}

  During the coarsening phase, the number of nodes in the graph is
  repeatedly reduced using a coarsening scheme such as 
  |collapse independent edges| or |connect independent nodes|.

  If the scheme |collapse independent edges| is enabled, which happens
  to be the default setting, coarse versions of the input graph are 
  generated by finding a maximal independent edge set and by collapsing
  the edges from this set. Nodes adjacent to each of these edges are
  merged into supernodes by which they are replaced in the next coarse
  version of the graph. Edges and nodes that are not related to the
  maximum independent edge set are maintained in the new graph.

  Collapsing the edges of a maximum independent edge set reduces the
  number of nodes of the graph by up to but never more than 50\,\%. 
  This means that the upper limit for reasonable values of 
  |downsize ratio| to be used in combination with 
  |collapse independent edges| is $0.5$. 
  
  Compared to |connected independent nodes| this is the less aggressive
  coarsening scheme and typically yields better drawings.
\end{key}

%% TODO Jannis: Implement and document this option.
%\begin{key}{/graph drawing/spring layout/coarsening/connected
%  independent nodes=\opt{\meta{boolean}} (default true, initially
%  false)}
%  \keyalias{graph drawing/spring electrical layout}
%\end{key}

\subsubsection{Node Options}

%\end{document}

% TODO Jannis: Explain this one better. Also, compare it to /tikz/at,
% which will move the node after the drawing has been computed, as
% opposed to /graph drawing/desired at, which will only move the node
% while computing the layout.
%
%\begin{key}{/tikz/desired at=\meta{coordinate}}
%  Nails the node down at the specified \meta{coordinate}. It will not
%  move from there despite the repulsive and attractive forces in the
%  system. Note that, while sometimes generating a similar effect, using
%  |at| is very different from altering the orientation of a graph
%  drawing (see section~\ref{subsection-library-graphdrawing-standard-orientation}).
%  Also, if an orientation is specified, it is given priority over
%  the |at| option in that nodes are first fixated at their |at|
%  coordinates but are later moved in order to satisfy the orientation 
%  desired by the user.
%  \begin{codeexample}[width=6.0cm]
%\tikz \graph [spring layout] {
%  1 -- 2 -- 3 -- 4 -- 2
%};
%\tikz \graph [spring layout] {
%  1 [at={(0,0)}] -- 2 [at={(0,1)}] -- 3 -- 4 -- 2
%};
%  \end{codeexample}
%\end{key}


% TODO what about node groups / clusters? This works via color classes
% but how do we define their layouts (cluster, line, circle)?

\subsubsection{Edge Options}

... 

%\begin{key}{/tikz/natural length=\meta{dimension} (default 10pt)}
%  Defines the natural (zero energy) length of the edge. The smaller the
%  length, the stronger the attractive force of the adjacent nodes. The
%  \meta{dimension} has a strong influence of how far the nodes will be
%  placed from each other in the final drawing.
%  \begin{codeexample}[]
%% two examples with the same graph
%% notably change the natural length of one of the edges
%  \end{codeexample}
%\end{key}
%
%\begin{key}{/tikz/stiffness=\meta{number} (default 0.5)}
%  Defines how flexible the spring associated with the edge is. The
%  higher this value is, the closer the final edge length will be to its
%  |natural length|.
%  \begin{codeexample}[]
%% two examples with the same graph
%% notably change the stiffness of one of the edges
%  \end{codeexample}
%\end{key}

\subsection{Options for Spring Layouts}

\subsubsection{Graph Options}

...

\subsubsection{Node Options}

...

\subsubsection{Edge Options}

...

\subsection{Options for Spring-Electrical Layouts}

% TODO Jannis: Document this option.
%\begin{key}{/graph drawing/spring layout/spring constant=\meta{number}}
%  \keyalias{graph drawing/spring electrical layout}
%\end{key}

\subsubsection{Graph Options}

\begin{key}{/graph drawing/spring electrical layout/approximate 
  repulsive forces=\opt{\meta{boolean}} (default true, initially false)}
  
  Computing the repulsive forces of the nodes in a graph requires 
  $\mathcal{O}(n^2)$ operations in each iteration of spring- and
  spring-electrical algorithms, where $n$ is the number of nodes in
  the graph. For $l$ coarse graphs, this may increase the runtime of 
  such an algorithm by up to $l\cdot\mathcal{O}(n^2)$ operations. 

  With |approximate repulsive forces| set to |true|, repulsive forces 
  are approximated using the Barnes-Hut algorithm known from solving the
  so-called $N$-body problem. This reduces the number of operations
  needed to compute the repulsive forces to $\mathcal{O}(n\log n)$ per 
  iteration and can thus lead to a significant improvement of the 
  algorithm runtime.
  
  However, this optimization \emph{can} come at the cost of slightly 
  less appealing drawings which is noticable with small graphs in 
  particular. Even though the differences can be very subtle, the
  often reduced quality of the final drawings is the reason why why 
  this feature is turned off by default. Typically, you only want to
  set |approximate repulsive forces| to |true| to save some time when
  laying out large graphs.

  The following code provides an example where the degraded quality 
  of a drawing with repulsive force approximation can be noticed:
  \begin{codeexample}[width=5.5cm]
% TODO Jannis: find a better example.

% bad layout due to not computing repulsive forces accurately
\tikz \graph [spring electrical layout={approximate repulsive forces}, orient=5-4]
{ subgraph K_n [n=5] };

% better layout thanks to accurate force math
\tikz \graph [spring electrical layout, orient=5-4]
{ subgraph K_n [n=5] };
  \end{codeexample}
\end{key}

% TODO Jannis: Document this option.
%\begin{key}{/graph drawing/spring layout/repulsive force
%  order=\meta{number} (initially 1)}
%  \keyalias{graph drawing/spring electrical layout}
%\end{key}

\subsubsection{Node Options}

\begin{key}{/graph drawing/spring layout/electric charge=\meta{number} 
  (default 1)}
  \keyalias{graph drawing/spring electrical layout}

  Defines the electric charge of the node. The stronger the 
  |electric charge| of a node the stronger the repulsion between the
  node and others in the graph. A negative |electric charge| means that
  other nodes are further attracted to the node rather than repulsed,
  although in theory this effect strongly depends on how the 
  |spring electrical layout| algorithm works.

  Two typcal effects of increasing the |electric charge| are distortion
  of symmetries and an upscaling of the drawings:
  \begin{codeexample}[width=5cm] 
\tikz \graph [spring electrical layout, orient=0-1] 
  { 0 [spring electrical layout={electric charge=1}] -- subgraph C_n [n=10] };

\tikz \graph [spring electrical layout, orient=0-1] 
  { 0 [spring electrical layout={electric charge=5}] -- subgraph C_n [n=10] };

\tikz \graph [spring electrical layout={coarsen=false}, orient=0-1] 
  { [clique] 1 [spring electrical layout={electric charge=5}], 2, 3, 4 };
  \end{codeexample}
\end{key}

\subsubsection{Edge Options}

...

\endinput

%%% Local Variables: 
%%% mode: latex
%%% TeX-master: "pgfmanual-pdftex-version"
%%% End: 

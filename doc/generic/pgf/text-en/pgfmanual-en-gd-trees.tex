% Copyright 2011 by Renée Ahrens, Olof Frahm, Jens Kluttig, Matthias Schulz, Stephan Schuster
% Copyright 2011 by Till Tantau
%
% This file may be distributed and/or modified
%
% 1. under the LaTeX Project Public License and/or
% 2. under the GNU Free Documentation License.
%
% See the file doc/generic/pgf/licenses/LICENSE for more details.

\section{Graph Drawing Layouts: Trees}
\label{section-first-graphdrawing-library-in-manual}
\label{section-library-graphdrawing-trees}

{\noindent {\emph{by Till Tantau}}}

File status: Finish description of Reingold--Tilford

\begin{tikzlibrary}{graphdrawing.trees}
  Load this package when you wish to use layout trees. You should load
  the |graphdrawing| library first. 
\end{tikzlibrary}


\tikzname\ offers several different syntax to specify trees (see
Sections \ref{section-library-graphs}
and~\ref{section-trees}). The job of the graph drawing algorithms from
this library is to turn the specification of trees into beautiful
layouts. 


\subsection{The Reingold--Tilford Tree Layout}

The Reingold--Tilford method is a standard method for drawing
trees. It is described in:
\begin{itemize}
\item
  E.~M.\ Reingold and J.~S.\ Tilford,
  \newblock Tidier drawings of trees,
  \newblock \emph{IEEE Transactions on Software Engineering,}
  7(2), 223--228, 1981.
\end{itemize}
  
The implementation of the |tree layout| key follows, however, the
ideas sketched in the following paper, which includes some useful
extensions of the basic Reingold--Tilford algorithm:
\begin{itemize}
\item
  A.\ Br\"uggemann-Klein, D.\ Wood,
  \newblock Drawing trees nicely with \TeX,
  \emph{Electronic Publishing,} 2(2), 101--115, 1989.
\end{itemize}
For the above paper, the authors have implemented the
Reingold--Tilford algorithm directly in \TeX\ (resulting in the
Tree\TeX\ style). Naturally, our 2012 implementation in the
|graphdrawing.tree| library is much more powerful and cleaner, but I
think it was an impressive achievement to implement this algorithm
back in 1989 directly in \TeX.

\begin{gdalgorithm}{tree layout}{pgf.gd.trees.ReingoldTilford1981}
  The basic idea of the Reingold--Tilford algorithm is to use the
  following rules to position the nodes of a tree (assuming it grown
  downwards, other growth directions are provided by the automatic
  orientation mechanisms of the graph drawing library):
  \begin{enumerate}
  \item For a node, recursively compute a layout for each of its children.
  \item Put the layout of the first child somewhere on the page.
  \item Put the second layout to the right of the first one as near as
    possible so that no two nodes touch (and such that a certain
    padding is achieved).
  \item Repeat for all subsequent children.
  \item Then place the root above the child trees at the middle
    position, that is, at the half-way point between the left-most and
    the right-most child of the node.    
  \end{enumerate}
\begin{codeexample}[]
\tikz [tree layout={significant sep=0}, sibling distance=8mm]
\graph [nodes={circle, draw, inner sep=1.5pt}]{
  1 -- { 2 -- 3 -- { 4 -- 5, 6 -- { 7, 8, 8' }}, 9 -- 10 -- { 11, 12 } }
};
\end{codeexample}
  The basic algorithm of Reingold--Tilford was extended by
  Br\"uggemann-Klein and Wood in two ways:
  \begin{enumerate}
  \item ...
  \end{enumerate}

\begin{codeexample}[]
\tikz [binary tree layout, sibling distance=7mm, level distance=10mm]
\graph [nodes={circle, inner sep=0pt, minimum size=2mm, fill}]{
  / -- { / -- / -- { / -- /, / -- { /, / }}, / -- / -- /[second] }
};
\end{codeexample}
\begin{codeexample}[]
\tikz \graph [binary tree layout, level distance=10mm] {
  Knuth -> {
    Beeton -> Kellermann [second] -> Carnes,
    Tobin -> Plass -> { Lamport, Spivak } 
  }
};\qquad
\tikz \graph [binary tree layout, grow'=right, sibling distance=5mm] {
  Knuth -> {
    Beeton -> Kellermann [second] -> Carnes,
    Tobin -> Plass -> { Lamport, Spivak } 
  }
};
\end{codeexample}
\begin{codeexample}[]
\tikz \graph [binary tree layout, grow'=30, sibling distance=5mm] {
  Knuth -> {
    Beeton -> Kellermann [second] -> Carnes,
    Tobin -> Plass -> { Lamport, Spivak } 
  }
};
\end{codeexample}
\end{gdalgorithm}




\subsection{Spanning Tree Computation}

Although the algorithms of this library are tailored to layout trees,
they will work for any graph as input. First, if the graph is not
connected, it is decomposed into connected components and these are
layed out individiually. Second, for each component, a spanning tree of
the graph is computed first and the layout is computed for this
spanning tree; all other edges will still be drawn, but they have no
impact on the placement of the nodes. If the graph is already a tree,
the spanning tree will be the original graph.

The computation of the spanning tree is a non-trivial process since
a non-tree graph has many different possible spanning trees. You can
choose between different methods for deciding on a spanning tree, it
is even possible to implement new algorithms. (In the future, the
computation of spanning trees and the cylce removal in layered graph
drawing algorithms will be unified, but, currently, they are
implemented differently.) 

\begin{key}{/graph drawing/spanning tree algorithm=\meta{algorithm}}
  \keyalias{tikz}\keyalias{tikz/graphs}
  Selects the (sub)algorithm that is to be used for computing spanning
  trees whenever this is requested by a tree layout algorithm. The
  default algorithm is |pgf.gd.trees.BreadthFirst|.
\begin{codeexample}[]
\tikz \graph [tree layout,
  spanning tree algorithm=pgf.gd.trees.BreadthFirst]
{
  1 -- {2,3,4,5} -- 6;    
};   
\end{codeexample}
\begin{codeexample}[]
\tikz \graph [tree layout,
  spanning tree algorithm=pgf.gd.trees.DepthFirst]
{
  1 --[bend right] {2,3,4,5 [>bend left]} -- 6;    
};   
\end{codeexample} 
\end{key}

Usually, you do not use the |spanning tree subalgorithm| option directly,
but use one of the below keys to select a method.

\begin{gdalgorithm}{breadth first spanning tree}{pgf.gd.trees.BreadthFirst}
  This key selects ``breadth first'' as the (sub)algorithm for
  computing spanning trees. Note that this key does not cause a graph
  drawing scope to start; the key only has an effect in conjunction
  with keys like |tree layout|.

  The algorithm will be called whenever a graph drawing algorithm
  needs a spanning tree on which to operate. It works as follows:
  
  \begin{enumerate}
  \item It looks for a node for which the following graph parameter is
    set:
    \begin{key}{/graph drawing/root}
      \keyalias{tikz}\keyalias{tikz/graphs}
    \end{key}
    If there are several such nodes, the first one is used. If there
    are no such nodes, the first node of indegree 0 (with respect to
    |->| edges) is used. If there is no such node, the first node is
    used.

    Let call the node determined in this way the \emph{root node}.
  \item For every edge, a \emph{priority} is determined, which is a
    number between 1 and 5. How this happens, exactly, will be
    explained in a moment.
  \item Starting from the root node, we now perform a breadth first
    search through the tree, thereby implicitly building a spanning
    tree: Suppose for a moment that all edges have priority~1. Then,
    the algorithm works just the way that a normal breadth first
    search is performed: We keep a queue of to-be-visited nodes and
    while this queue is not empty, we remove its first node. If this
    node has not yet been visited, we add all its neighbors at the
    end of the queue. When a node is taken out of the queue, we make
    it the child of the node whose neighbor it was when it was
    added. Since the queue follows the ``first in, first out''
    principle (it is a fifo queue), the children of the root will be
    all nodes at distance $1$ form the root, their children will be
    all nodes at distance $2$, and so on. 
  \item Now suppose that some edges have a priority different
    from~1, in which case things get more complicated. We now keep
    track of one fifo queue for each of the five possible
    priorities. When we consider the neighbors of a node, we actually
    consider all its incident edges. Each of these edges has a certain
    priority and the neighbor is put into the queue of the edge's
    priority. Now, we still remove nodes normally from the queue for
    priority~1; only if this queue is empty and there is still a node
    in the queue for priority~2 we remove the first element from this
    queue (and proceed as before). If the second queue is also empty,
    we try the third, and so on up to the fifth queue. If all queues
    are empty, the algorithm stops.
  \end{enumerate}

  The effect of the five queues is the following: If the edges of
  priority $1$ span the whole graph, a spanning tree consisting solely
  of these edges will be computed. However, if they do not, once we
  have visited reachable using only priority 1 edges, we will extend
  the spanning tree using a priority 2 edge; but then we once switch
  back to using only priority 1 edges. If neither priority~1 nor
  priority~2 edges suffice to cover the whole graph, priority~3 edges
  are used, and so on.
  
  You can directly specify the priority of an edge using the following
  edge parameter:
  \begin{key}{/graph drawing/edge priority=\meta{number}}
    \keyalias{tikz}\keyalias{tikz/graphs}
    Sets the priority of the edge to \meta{number}, which must be
    |1|, |2|, |3|, |4|, or  |5|.      
  \end{key}
  Remember that priority |1| edges are always preferred over other
  edges. For this reason, there is the following key:
  \begin{key}{/graph drawing/spanning tree edge}
    \keyalias{tikz}\keyalias{tikz/graphs}
    An easy-to-remember shorthand for |edge priority=1|. When this key
    is used with an edge, it will always be preferred over other edges
    when it comes to choosing which edges are part of the spanning tree.    
  \end{key}
  \begin{key}{/graph drawing/no spanning tree edge}
    \keyalias{tikz}\keyalias{tikz/graphs}
    An easy-to-remember shorthand for |edge priority=5|. This causes
    the edge to be used only as a last resort as part of a spanning
    tree. 
  \end{key}
  
  Instead of specifying priorities ``by hand'', you can also let them
  be computed automatically. For all edges without an explicitly
  specified priority the following key is used to determine it:
  \begin{key}{/graph drawing/edge priority method=\meta{method}}
    \keyalias{tikz}\keyalias{tikz/graphs}
    There are currently the following choices for \meta{methods}:
    \begin{itemize}
    \item \declare{|forward first|} (this is the default method)
      
      For two nodes $u$ and $v$ connected by an edge $e$, this method
      assigns the priority 2 to $e$ if it is a |->| edge from $u$
      to~$v$; it assigns $3$ to all undirected and bidirected edges
      (|--| and |<->| edges); and $4$ to all other edges (|<-| edges
      and |->| edge from $v$ to $u$).

      In practice, this means that the algorithm will favor
      forward-going edges and whenever a spanning tree can be computed
      using only the |->| edges, this spanning tree will be used. Only
      when these edges no longer suffice to build a spanning tree, we
      will also consider undirected or bidirected edges. The
      ``backward'' edges are only used as a last resort.
\begin{codeexample}[]
\tikz \graph [tree layout, edge priority method=forward first]        
{
  1 -> 2 -> 3 -- 4, 2 -> 4;
  a -- b -- c -> d, b --[bend right] d;
};
\end{codeexample}
    \item \declare{|undirected first|}
      
      This method assigns priority 2 to |--| and |<->| edges,
      priority~3 to |->| edges, and priority~4 to |<-| edges.
\begin{codeexample}[]
\tikz \graph [tree layout, edge priority method=undirected first]        
{
  1 -> 2 -> 3 -- 4, 2 -> [bend right] 4;
  a -- b -- c -> d, b -- d;
};
\end{codeexample}
      An interesting application of this method is in creating trees
      where siblings are automatically connected by directed edges:
\begin{codeexample}[]
\tikz \graph [tree layout, nodes={draw}, sibling distance=0pt,
              edge priority method=undirected first,
              every group/.style={
                default edge kind=->,
                path=source % connect every group as a path
              }] 
{
  5 -- {
    "1,3" -- {0,2,4},
    11    -- {
      "7,9" -- { 6, 8, 10 }
    }
  }
};
\end{codeexample}     
      Note that in the above example, we could also have used the key
      |no spanning tree edge| inside the |every group|. This would
      also cause the added edges to be ignored:
\begin{codeexample}[]
\tikz \graph [tree layout, nodes={draw}, sibling distance=0pt,
              every group/.style={
                default edge kind=->, no spanning tree edge,
                path=source}] 
{
  5 -> {
    "1,3" -> {0,2,4},
    11    -> {
      "7,9" -> { 6, 8, 10 }
    }
  }
};
\end{codeexample}     
    \item \declare{|directed first|}
      
      Assigns priority 2 to directed edges that ``go along their
      direction''. That is, for an edge between $u$ and $v$, we get
      priority $2$ for |->|, for |<->|, and also for |<-|, but only
      when going from $v$ to $u$. Undirected edges get priority~3, all
      other edges get priority 4.

      This strategy is nice with trees specified with both forward and
      backward edges:
\begin{codeexample}[]
\tikz \graph [tree layout, nodes={draw}, sibling distance=0pt,
              edge priority method=directed first]
{
  3 <- 5[root] -> 8,
  1 <- 3 -> 4,
  7 <- 8 -> 9,
  1 -- 4 -- 7 -- 9
};
\end{codeexample}
    \item \declare{|all the same|}
      
      Assigns priority 3 to all edges, meaning that they are ``all the same.''
    \end{itemize}
  \end{key}
\end{gdalgorithm}

\begin{gdalgorithm}{depth first spanning tree}{pgf.gd.trees.DepthFirst}
  Works exactly like |depth first spanning tree| (same handling of
  priorities by five queues), only the queues are now lifo instead of
  fifo. 
\end{gdalgorithm}


%%% Local Variables: 
%%% mode: latex
%%% TeX-master: "pgfmanual-pdftex-version"
%%% End: 

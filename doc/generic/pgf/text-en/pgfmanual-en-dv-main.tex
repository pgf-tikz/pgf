% Copyright 2008 by Till Tantau
%
% This file may be distributed and/or modified
%
% 1. under the LaTeX Project Public License and/or
% 2. under the GNU Free Documentation License.
%
% See the file doc/generic/pgf/licenses/LICENSE for more details.


\section{The Data Visualization Frontend}
\label{section-dv-main}
\label{section-dv-main-setup}

\subsection{Overview}

The present section explains how a data visualization is created in
\tikzname. For this, you need to include the |datavisualization|
library and then use the command |\datavisualization| whose syntax is
explained in the rest of the present section.

In order to visualize, you basically need to do three things:
\begin{enumerate}
\item You need to select what kind of plot you would like to have (a
  ``school book plot'' or a ``scientific 2d plot'' or a ``scientific
  spherical plot'' etc.). This is done by passing an option to the
  |\datavisualization| command that selects this kind of plot.
\item You need to provide data points, which is done using the |data|
  command.
\item Additionally, you can add options that give you more
  fine-grained control over the way the visualization will look. You
  can configure the number of ticks and grid lines, where the labels
  are placed, the colors, or the fonts. Indeed, since the
  data visualization engine internally uses \tikzname-styles, you can
  have extremely fine-grained control over how a plot will look like. 
\end{enumerate}

The syntax of the |\datavisualization| command is designed in such a
way that if you only need to provide very few options to create plots
that ``look good by default''. 

This section is structured as follows: First, the philosophy behind
concepts like ``axes,'' ``ticks,'' and ``grid lines'' are
explained. Then, the syntax of the |\datavisualization| command is
covered. The reference sections explain which predefined plot kinds
are available.


\subsection{Concept: Data Points and Data Formats}

As explained in Section~\ref{section-dv-intro-data-points}, data
points are the basic entities that are processed by the data
visualization engine. In order to specify data points, you use the
|data| command, whose syntax is explained in more detail in
Section~\ref{section-dv-data-syntax}. The |data| command allows you to
either specify points ``inline,'' directly inside your \TeX-file; or
you can specify the name of file that contains the data points.

Data points can be formatted in different ways. For instance, in the so
called \emph{comma separated values} format, there is one line for
each data point and the different attributes of a data point are
separated by commas. Another common format is to specify data points
using the so called \emph{key-value} format, where on each line the
different attributes of a data point are set using a comma-separated
list of strings of the form
|attribute=value|. Section~\label{section-dv-formats} gives an
in-depth coverage of the available data formats and explains how new
data formats can be defined.



\subsection{Concept: Axes}

Most plots have two or three axes: A horizontal axis usually called
the $x$-axis, a vertical axis called the $y$-axis, and possibly some 
axis pointing in a sloped direction called the $z$-axis. Axes are
usually drawn as lines with \emph{ticks} indicating interesting
positions on the axes. The data visualization engine gives you
detailed control over where these ticks are rendered and how many of
them are used. Great care is taken to ensure that the position of
ticks are chosen well by default.

From the point of view of the data visualization engine, axes are a
somewhat more general concept than ``just'' lines that point ``along''
some dimension: The data visualization engine uses axes to visualize
any change of an attribute by varying the position of data points in the
plane. For instance, in a polar plot, there is an ``axis'' for the
angle and another ``axis'' for the distance if the point from the
center. Clearly these axes vary the position of data points in the
plane according to some attribute of the data points; but just as
clearly they do not point in any ``direction.''

A great benefit of this approach is that the powerful methods for
specifying and automatic inference of ``good'' positions for ticks or
grid lines apply to all sorts of situations. For instance, you can use
it to automatically put ticks and grid lines at well-chosen angles of
a polar plot.

% In general, when you visualize a data point, you can visualize the
% differences between the attributes of different data points in many
% different ways: You can change the brightness or the hue or the
% saturation of data points according to the value of some attribute,
% you can change which shape is used to render the data points, or you
% could vary the pattern used to fill data points. Axes are simply the
% most basic way of varying the rendering of a data points, namely  by
% changing \emph{where} a data point is rendered.



\subsection{Concept: Ticks and Grid Lines}





\subsection{Usage}
\label{section-dv-data-syntax}

For every data visualization some data is needed. This data is
provided to the |\datavisualization| command via the use of the |data|
command, see Section~\ref{section-dv-main} for an introduction. The
|data| command can be used repeatedly in a single visualization and
each time the to-be-read data may have a different format.

\begin{datavisualizationoperation}{data}{\opt{\oarg{options}}\opt{\meta{inline
        data}}}
  
\end{datavisualizationoperation}


\subsection{Reference: Types of Axes}

\subsection{Reference: Types of Plots}

\subsection{Reference: Tick Placement Strategies}

\subsection{Advanced: Creating New Axes}

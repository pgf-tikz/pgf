% Copyright 2007 by Mark Wibrow
%
% This file may be distributed and/or modified
%
% 1. under the LaTeX Project Public License and/or
% 2. under the GNU Free Documentation License.
%
% See the file doc/generic/pgf/licenses/LICENSE for more details.
%

\part{Mathetical Operations}
\pgfname{} supports mathematical operations beyond simple addition,
subtraction, multiplication and division, using both integers and 
non-integers. In addition, \pgfname{} can calculate functions such
as square-roots, sine, cosine, and generate pseudo-random numbers, 
both as ``stand-alone'' macros, and as infix operations in the 
arguments of certain commands.
\newpage

\section{Mathematics in PGF}
Those familiar with the \calcname{} package, know that it is 
possible to parse basic infix mathematical arithmatic operations, 
in the arguments of certain macros. 
However, \calcname{} operates under fairly constrained input. 
For example, only the |+|, |-|, |*| and |/| operators are supported, 
all non-integers must be followed by a \TeX{} dimension, and the 
second operand in a multiplication or division operation can only be
 an integer. 

\pgfname{} provides enhanced functionality, which permits the parsing
of mathematical operations involving integers and non-integers 
with or without units. Futhermore, various functions, including
trigonometic functions and random number generators can also be 
parsed (see Section~\ref{pgfmath-parsing}). 
The \calcname{} macros |\setlength| and friends have \pgfname{} versions 
which can parse these operations and functions 
(see Section~\ref{pgfmath-registers}). Additionally, each operation
and function has an independent \pgfname{} command associated with it
(see Section~\ref{pgfmath-commands}), and can be 
accessed outside the parser.

Currently, the mathematical algorithms are all implemented in \TeX.
This poses some intriguing programming challenges as \TeX{} is a
typesetting language not a mathematical one, and as with any 
programming language, there is a trade-off between accuracy and 
efficiency. Some levels of accuracy may not be your liking. 
However, it is very simple to replace them. In fact, the interface to 
the mathematical operations and functions has been deliberately 
implemented in a transparently enacpsualted manner, for precisely this 
purpose. For more details see Section~\ref{pgfmath-reimplement}.

\subsection{Setting Registers}

\label{pgfmath-registers}

These macros are very similar to their cousins in the \calcname{} 
package. The only difference is that the expressions that than can be
used to determine the values assigned to the registers, can be more
complex mathematical expressions, including functions and 
non-integers, as described in Section~\ref{pgfmath-parsing}. 

\begin{command}{\pgfmathsetlength\marg{dimension register}\marg{expression}}
Sets the length of the \TeX{} \meta{dimension register}, to the value
(in points) specified by \meta{expression}. 
Section~\ref{pgfmath-parsing}, provides details of syntax for
\meta{expression}.
\end{command}

\begin{command}{\pgfmathaddtolength\marg{dimension register}\marg{expression}}
	Adds the value (in points) of \meta{expression} to the \TeX{} 
	\meta{dimension register}.
\end{command}

\begin{command}{\pgfmathsetcount\marg{count register}\marg{expression}}
	Sets the value of the \TeX{} \meta{count register}, to the 
	\emph{truncated} value specified by \meta{expression}. 
\end{command}

\begin{command}{\pgfmathaddtocount\marg{count register}\marg{expression}}
	Adds the \emph{truncated} value  of \meta{expression} to the \TeX{} 
\	meta{count register}.
\end{command}

\begin{command}{\pgfmathsetcounter\marg{counter}\marg{expression}}
	Sets the value of the \meta{counter}, to the \emph{truncated} value 
	specified by \meta{expression}. 
\end{command}

\begin{command}{\pgfmathaddtocounter\marg{counter}\marg{expression}}
	Adds the \emph{truncated} value  of \meta{expression} to 
	\meta{counter}.
\end{command}

\begin{command}{\pgfmathnewcounter\marg{counter}}
	This is simply a version of the \LaTeX{} macro |\newcounter|, 
	implemented	entirely to maintain consistency (consistency is good,
	inconsistency is evil). Considering |\pgfmathnewcounter{foo}|, this
	creates a new count register |\c@foo|, and a macro |\thefoo|, which
	returns the value in |\c@foo|.
\end{command}

\begin{command}{\pgfmathsetmacro\marg{macro}\marg{expression}}
	Defines \meta{macro} as the  value of \meta{expression}. The result
	is a decimal \emph{without} units.
\end{command}

All the fancy mathematical ``bells-and-whistles'' that the parser 
provides, come with an additional processing cost, and in some
instances, such as simply setting a length to |1cm|, with no other
operations involved, the addition processing time is undesirable. 
To overcome this, the following feature is implemented: when no
mathematical operations are required, the value in \meta{expression}
can be preceded by |+|. This will bypass the parsing process and the 
assignment will be orders of magnitude faster. This feature 
\emph{only} works with the macros for setting registers described in
this section.

\begin{codeexample}[code only]
\pgfmathsetlength\mydimen{1cm}  % parsed     : slower.
\pgfmathsetlength\mydimen{+1cm} % not parsed : much faster.
\end{codeexample}

% Copyright 2007 by Mark Wibrow
%
% This file may be distributed and/or modified
%
% 1. under the LaTeX Project Public License and/or
% 2. under the GNU Free Documentation License.
%
% See the file doc/generic/pgf/licenses/LICENSE for more details.
%

\section{Parsing Mathematical Expressions}

\label{pgfmath-parsing}

The support for infix mathematical operations involving 
integers and non-integers, with or without units comes with some 
caveats. Firstly, it should be noted that all
calculations must not exceed $\pm16383.99999$ at \emph{any} point, 
because the underlying algorithms are all implemented by hacking 
around the \TeX{} dimensions. This means that many of the underlying 
algorithms are necessarily approximate (but even native \TeX{} 
arithmatic occasionally throws up some surprises). It also means that
some of the algorithms are not very fast. \TeX{} is, after all, a
typesetting language and not ideally suited to relatively advanced 
mathematical operations. However, it is very simple to change the
algorithms as described in Section~\ref{pgfmath-reimplement}.

\begingroup
% Locally redefine \medskip, so examples are not too far apart.
\let\medskip\smallskip%

\begin{command}{\pgfmathparse\marg{expression}}

	This macro parses \meta{expression} and returns the result in 
	the macro |\pgfmathresult|. 
	
\end{command}

Note the following features (note also, that unlike the rest of the 
manual, the examples in these sections show the result of the 
calculation, not what is displayed in the document).


	\begin{itemize}
	\item the result is a decimal \emph{without units}, this is 
	regardless of whether the \meta{expression} contains any unit 
	specification. But, any units specified will be converted to 
	points first.

\begin{codeexample}[post=\tt\footnotesize\pgfmathresult]
\pgfmathparse{2pt+3.4pt}
\end{codeexample}

\begin{codeexample}[post=\tt\footnotesize\pgfmathresult]
\pgfmathparse{2cm+3.4cm}
\end{codeexample}

	\item if no units are specified \emph{at any point} in the 
	expression, the result will be mulitplied by the value in 
	|\pgfmathresultunitscale|, which can be a number or a dimension 
	(which will be converted to points). By default it is set to 1, 
	but can be changed with |\pgfmathsetresultunitscale|. Note that 
	the result will still be a number \emph{without} units.

\begin{codeexample}[post=\tt\footnotesize\pgfmathresult]
\pgfmathparse{2pt+3.4pt}
\end{codeexample}

\begin{codeexample}[post=\tt\footnotesize\pgfmathresult]
\pgfmathsetresultunitscale{1cm}
\pgfmathparse{2+3.4}
\end{codeexample}

	\pgfmathsetresultunitscale{1pt}

	\item The parser handles numbers with or without units regardless
	of the operation.

\begin{codeexample}[post=\tt\footnotesize\pgfmathresult]
\pgfmathparse{54pt/3cm*2.1} 
\end{codeexample}

	\item the parser can cope with \TeX{} registers. Including those 
	preceded by |\the|.

\begin{codeexample}[post=\tt\footnotesize\pgfmathresult]
\pgf@x=12.34pt
\c@pgf@counta=5
\pgfmathparse{\pgf@x+\c@pgf@counta*6}
\end{codeexample}

\begin{codeexample}[post=\tt\footnotesize\pgfmathresult]
\pgf@x=56.78pt
\pgfmathparse{\pgf@x+\the\pgf@x}
\end{codeexample}

	\item \TeX{} registers can be multiplied without the |*| operator
	by precededing then with a number (\emph{not} a function), or a
	count register.
	 
\begin{codeexample}[post=\tt\footnotesize\pgfmathresult]
\c@pgf@counta=-4
\pgf@x=10pt
\pgfmathparse{.5\pgf@x-\c@pgf@counta\pgf@x}%
\end{codeexample}

	\item parenthesis can be used to group operations.

\begin{codeexample}[post=\tt\footnotesize\pgfmathresult]
\pgfmathparse{(4pt+0.5)*3}
\end{codeexample}

	\item functions are recognized, so it is possible to parse
	|sin(.5*pi r)*60|, which means ``the sine of $0.5$ times $\pi$ 
	radians, multiplied by 60''. The argument of most functions can
	be any expression.

\begin{codeexample}[post=\tt\footnotesize\pgfmathresult]
\pgfmathparse{sin(pi/2 r)*60}
\end{codeexample}

\end{itemize}

\begin{command}{\pgfmathqparse\marg{expression}}

	This macro is similar to |\pgfmathparse|: it parses 
	\meta{expression} and returns the result in the macro 
	|\pgfmathresult|. It differs in that, \emph{every} number in
	\meta{expression} \emph{must} specify a \TeX{} unit. This greatly
	simplfies the problem of parsing of non-integers and as a result
	is about twice as fast as |\pgfmathparse|. Note that the result will 
	still be a number \emph{without} units.	
\end{command}

\begin{command}{\pgfmathsetresultunitscale\marg{number}\textsl{\ or\ }\marg{dimension}}
	Sets the value in |\pgfmathresultunitscale|, which scales the result
	of an expression parsed with |\pgfmathparse|, if that expression
	contains no units	\emph{at any point}. The argument can be an integer,
	non-integer or a dimension, but the result will still be a number 
	\emph{without} units. Note, that this will affect |\pgfmathsetlength| 
	and friends, but not if the |+| expression prefix is used 
	(see Section~\ref{pgfmath-registers}). By default the value in
	|\pgfmathresultunitscale| is 1.
\end{command}

\subsection{Syntax for mathematical expressions}

\label{pgfmath-syntax}

The syntax for the expressions recognised by |\pgfmathparse| has been
based in the syntax recognised by \textsc{matlab}. The following 
operations and functions are currently recognized:

\begin{math-operator}{\mvar{x}\ +\ \mvar{y}}
	Adds \mvar{y} to \mvar{x}.
	
\begin{codeexample}[post=\tt\footnotesize\pgfmathresult]
\pgfmathparse{4+2pt}
\end{codeexample}
\end{math-operator}\begin{math-operator}{\mvar{x}\ -\ \mvar{y}}
	Subtracts \mvar{y} from  \mvar{x}.
	
\begin{codeexample}[post=\tt\footnotesize\pgfmathresult]
\pgfmathparse{155.35-4cm}
\end{codeexample}
\end{math-operator}
\begin{math-operator}{\mvar{x}\ *\ \mvar{y}}
	Multiplies \mvar{x} by  \mvar{y}.
	
\begin{codeexample}[post=\tt\footnotesize\pgfmathresult]
\pgfmathparse{3.9pt*4.56}
\end{codeexample}

\end{math-operator}
\begin{math-operator}{\mvar{x}\ /\ \mvar{y}}
	Divides \mvar{x} by  \mvar{y}.
	
\begin{codeexample}[post=\tt\footnotesize\pgfmathresult]
\pgfmathparse{-31.6pt/17}
\end{codeexample}

\end{math-operator}
\begin{math-operator}{\mvar{x}\ {\char94}\ \mvar{y}} 

Raises \mvar{x} to the power \mvar{y}. \mvar{y} should be an integer, but it can be negative.

\begin{codeexample}[post=\tt\footnotesize\pgfmathresult]
\pgfmathparse{2.3^4}
\end{codeexample}

\begin{codeexample}[post=\tt\footnotesize\pgfmathresult]
\pgfmathparse{2^-4}
\end{codeexample}
\end{math-operator}

\begin{math-operator}{\mvar{x}\ ==\ \mvar{y}} 

	This evaluates to |1| if \mvar{x} equals \mvar{y}, or |0| if \mvar{x}
	does not equal \mvar{y}. 
	Note that equalities (and inequalities) are evaluated left to right, 
	and are only evaluated when another equality (or inequality) operator 
	is scanned, or the end of the current group or parse is reached. So 
	|5+4==3+2==9| results in |0| because |5+4| does not equal |3+2|, 
	resulting in zero, and the second equality is therefore evaluating 
	|0=9|.

\begin{codeexample}[post=\tt\footnotesize\pgfmathresult]
\pgfmathparse{3*5 == 15}
\end{codeexample}

\end{math-operator}


\begin{math-operator}{\mvar{x}\ >\ \mvar{y}} 

	This evaluates to |1| if \mvar{x} is greater than \mvar{y}, or |0| if 
	\mvar{x} is smaller or equal to \mvar{y}.
	
\begin{codeexample}[post=\tt\footnotesize\pgfmathresult]
\pgfmathparse{17 > 4.2*1.97+4}
\end{codeexample}

\end{math-operator}

\begin{math-operator}{\mvar{x}\ <\ \mvar{y}}

	This evaluates to |1| if \mvar{x} is smaller than \mvar{y}, or |0| if
	\mvar{x} is greater or equal to \mvar{y}.
	
\begin{codeexample}[post=\tt\footnotesize\pgfmathresult]
\pgfmathparse{2 < -5.2/-3.6-2}
\end{codeexample}

\end{math-operator}

\begin{math-function}{mod(\mvar{x},\mvar{y})}
	This evaluates \mvar{x} modulo \mvar{y}. This function cannot be 
	nested inside itself or the functions |max|, |min| or |veclen|.

\begin{codeexample}[post=\tt\footnotesize\pgfmathresult]
\pgfmathparse{mod(20,6)}
\end{codeexample}

\end{math-function}

\begin{math-function}{max(\mvar{x},\mvar{y})}
	This evaluates to the maximum of \mvar{x} or \mvar{y}. This function 
	cannot be nested inside itself or the functions |min|, |mod| or 
	|veclen|.

\begin{codeexample}[post=\tt\footnotesize\pgfmathresult]
\pgfmathparse{max(17,23)}
\end{codeexample}

\end{math-function}

\begin{math-function}{min(\mvar{x},\mvar{y})}
	This evaluates to the minimum of \mvar{x} or \mvar{y}. This function 
	cannot be nested inside itself or the functions |max|, |mod| or 
	|veclen|.

\begin{codeexample}[post=\tt\footnotesize\pgfmathresult]
\pgfmathparse{min(17,23)}
\end{codeexample}

\end{math-function}

\begin{math-function}{abs(\mvar{x})}

	Evaluates the absolute value of $x$.
	
\begin{codeexample}[post=\tt\footnotesize\pgfmathresult]
\pgfmathparse{abs(-5)}
\end{codeexample}

\begin{codeexample}[post=\tt\footnotesize\pgfmathresult]
\pgfmathparse{-abs(4*-3)}
\end{codeexample}

\end{math-function}

\begin{math-function}{round(\mvar{x})}

	Rounds \mvar{x} to the nearest integer. It uses ``asymmetric half-up'' 
	rounding. So |1.5| is rounded to |2|, but |-1.5| is rounded to |-2| 
	(\emph{not} |0|).

\begin{codeexample}[post=\tt\footnotesize\pgfmathresult]
\pgfmathparse{round(1.5)}
\end{codeexample}

\begin{codeexample}[post=\tt\footnotesize\pgfmathresult]
\pgfmathparse{round(-1.5)}
\end{codeexample}

\begin{codeexample}[post=\tt\footnotesize\pgfmathresult]
\pgfmathparse{round(32.5/17)}
\end{codeexample}

\begin{codeexample}[post=\tt\footnotesize\pgfmathresult]
\pgfmathparse{round(398/12)}
\end{codeexample}

\end{math-function}

\begin{math-function}{floor(\mvar{x})}

	Rounds \mvar{x} down to the nearest integer. 
	
\begin{codeexample}[post=\tt\footnotesize\pgfmathresult]
\pgfmathparse{floor(32.5/17)}
\end{codeexample}

\begin{codeexample}[post=\tt\footnotesize\pgfmathresult]
\pgfmathparse{floor(398/12)}
\end{codeexample}

\end{math-function}

\begin{math-function}{ceil(\mvar{x})}

	Rounds \mvar{x} up to the nearest integer. 

\begin{codeexample}[post=\tt\footnotesize\pgfmathresult]
\pgfmathparse{ceil(32.5/17)}
\end{codeexample}

\begin{codeexample}[post=\tt\footnotesize\pgfmathresult]
\pgfmathparse{ceil(398/12)}
\end{codeexample}

\end{math-function}

\begin{math-function}{exp(\mvar{x})}

	Maclaurin series for $e^\textrm{\mvar{x}}$. 
	
\begin{codeexample}[post=\tt\footnotesize\pgfmathresult]
\pgfmathparse{exp(1)}
\end{codeexample}

\begin{codeexample}[post=\tt\footnotesize\pgfmathresult]
\pgfmathparse{exp(2.34)}
\end{codeexample}

\end{math-function}

\begin{math-function}{sqrt(\mvar{x})}

 A Newton-Raphson approximation of $\sqrt{\textrm{\mvar{x}}}$.  

\begin{codeexample}[post=\tt\footnotesize\pgfmathresult]
\pgfmathparse{sqrt(10)}
\end{codeexample}

\begin{codeexample}[post=\tt\footnotesize\pgfmathresult]
\pgfmathparse{sqrt(8765.432)}
\end{codeexample}


\end{math-function}

\begin{math-function}{veclen(\mvar{x},\mvar{y})}

	Evaluates the Euclidean distance from |(0,0)| to |(|\mvar{x}|,|\mvar{y}|)|. 
	This function cannot be nested inside itself or the functions |max|,
	|min| or |mod|.

\begin{codeexample}[post=\tt\footnotesize\pgfmathresult]
\pgfmathparse{veclen(15,14)}
\end{codeexample}

\begin{codeexample}[post=\tt\footnotesize\pgfmathresult]
\pgfmathparse{veclen(3,4)}
\end{codeexample}

\end{math-function}

\begin{math-operator}{\mvar{x}\ r}

	This converts \mvar{x} from radians to degrees. Note that |r| will 
	evaluate any preceding series of multiplication or division 
	\emph{before} conversion, but not other operations. So |3*4/6r| 
	converts 2 radians to degrees, but |3-4+6r|, converts 6 radians to
	degrees and adds the result to |-1|.

\begin{codeexample}[post=\tt\footnotesize\pgfmathresult]
\pgfmathparse{2*pi r-pi r}
\end{codeexample}

\begin{codeexample}[post=\tt\footnotesize\pgfmathresult]
\pgfmathparse{2*pi/8 r}
\end{codeexample}

\begin{codeexample}[post=\tt\footnotesize\pgfmathresult]
\pgfmathparse{sin(3*pi/2r)*60}
\end{codeexample}

\end{math-operator}

\begin{math-constant}{pi}

	The constant $\pi=3.14159$.
	
\begin{codeexample}[post=\tt\footnotesize\pgfmathresult]
\pgfmathparse{pi}
\end{codeexample}

\begin{codeexample}[post=\tt\footnotesize\pgfmathresult]
\pgfmathparse{pi r}
\end{codeexample}

\end{math-constant}

\begin{math-function}{sin(\mvar{x})}

	Sine of \mvar{x}. By employing the |r| operator, \mvar{x} can be in 
	radians.
	
\begin{codeexample}[post=\tt\footnotesize\pgfmathresult]
\pgfmathparse{sin(30)}
\end{codeexample}

\begin{codeexample}[post=\tt\footnotesize\pgfmathresult]
\pgfmathparse{sin(pi/3 r)}
\end{codeexample}

\end{math-function}

\begin{math-function}{cos(\mvar{x})}

	Cosine of \mvar{x}. By employing the |r| operator, \mvar{x} can be in 
	radians.

\begin{codeexample}[post=\tt\footnotesize\pgfmathresult]
\pgfmathparse{cos(60)}
\end{codeexample}

\begin{codeexample}[post=\tt\footnotesize\pgfmathresult]
\pgfmathparse{cos(pi/6 r)}
\end{codeexample}

\end{math-function}

\begin{math-function}{tan(\mvar{x})}

	Tangent of \mvar{x}. By employing the |r| operator, \mvar{x} can be in 
	radians.
	
\begin{codeexample}[post=\tt\footnotesize\pgfmathresult]
\pgfmathparse{tan(45)}
\end{codeexample}

\begin{codeexample}[post=\tt\footnotesize\pgfmathresult]
\pgfmathparse{tan(2*pi/8 r)}
\end{codeexample}

\end{math-function}

\begin{math-function}{asin(\mvar{x})}

	Arcsine of \mvar{x}. The result is in degrees.

\begin{codeexample}[post=\tt\footnotesize\pgfmathresult]
\pgfmathparse{asin(0.7071)}
\end{codeexample}

\end{math-function}

\begin{math-function}{acos(\mvar{x})}

	Arccosine of \mvar{x} in degrees.  

\begin{codeexample}[post=\tt\footnotesize\pgfmathresult]
\pgfmathparse{acos(0.5)}
\end{codeexample}

\end{math-function}

\begin{math-function}{atan(\mvar{x})}

	Arctangent of $x$ in degrees. 

\begin{codeexample}[post=\tt\footnotesize\pgfmathresult]
\pgfmathparse{atan(1)}
\end{codeexample}

\end{math-function}

\begin{math-function}{rnd}

	Generates a pseudo-random number between 0 and 1.

\begin{codeexample}[post=\tt\footnotesize\pgfmathresult]
\pgfmathparse{rnd}
\end{codeexample}

\begin{codeexample}[post=\tt\footnotesize\pgfmathresult]
\pgfmathparse{2*rnd}
\end{codeexample}

\begin{codeexample}[post=\tt\footnotesize\pgfmathresult]
\pgfmathparse{-rnd+5}
\end{codeexample}

\end{math-function}

\begin{math-function}{rand}

	Generates a pseudo-random number between -1 and 1.

\begin{codeexample}[post=\tt\footnotesize\pgfmathresult]
\pgfmathparse{rand}
\end{codeexample}

\begin{codeexample}[post=\tt\footnotesize\pgfmathresult]
\pgfmathparse{rand*15}
\end{codeexample}

\end{math-function}

\endgroup\medskip\par\leavevmode

% Copyright 2007 by Mark Wibrow
%
% This file may be distributed and/or modified
%
% 1. under the LaTeX Project Public License and/or
% 2. under the GNU Free Documentation License.
%
% See the file doc/generic/pgf/licenses/LICENSE for more details.

\section{Evaluating Mathematical Operations}

\label{pgfmath-commands}

Instead of parsing and evaluating complex expressions, you can also
use the mathematical engine to evaluate a single mathematical
operation. The macros used for these computations are described in the
following. 


\subsection{Basic Operations and Functions}

\label{pgfmath-operations}

\begin{command}{\pgfmathadd\marg{x}\marg{y}}  
	Defines |\pgfmathresult| as $\meta{x}+\meta{y}$.
\end{command}

\begin{command}{\pgfmathsubtract\marg{x}\marg{y}}      
	Defines |\pgfmathresult| as $\meta{x}-\meta{y}$.                                       
\end{command}

\begin{command}{\pgfmathmultiply\marg{x}\marg{y}}      
	Defines |\pgfmathresult| as $\meta{x}\times\meta{y}$.                                
\end{command}

\begin{command}{\pgfmathdivide\marg{x}\marg{y}}        
	Defines |\pgfmathresult| as $\meta{x}\div\meta{y}$. An error will
	result if \meta{y} is	|0|, or if the result of the division is
	too big for the mathematical engine.
	Please remember	when using this command that accurate (and reasonably 
	quick) division of non-integers is particularly tricky in \TeX{}. 	
	There are three different forms of division used in this command:
	\begin{itemize}
		\item 
		If \meta{y} is an integer then the native |\divide| operation of 
		\TeX{} is used.
		\item
		If \vrule\meta{y}\vrule$<1$, then |\pgfmathreciprocal| is employed.
		\item
		For all other values of \meta{y} an optimised long division 
		algorithm is used. In theory this should be accurate
		to any finite precision, but in practice it is constrained by the
		limits of \TeX{}'s native mathematics.
	\end{itemize}
	                             
\end{command}

\begin{command}{\pgfmathreciprocal\marg{x}}         
	Defines |\pgfmathresult| as $1\div\meta{x}$.                            
\end{command}

\begin{command}{\pgfmathgreaterthan\marg{x}\marg{y}}   
	Defines |\pgfmathresult| as 1.0 if \meta{x} $>$ \meta{y}, but 0.0 otherwise.                 
\end{command}

\begin{command}{\pgfmathlessthan\marg{x}\marg{y}} 
	Defines |\pgfmathresult| as 1.0 if \meta{x} $<$ \meta{y}, but 0.0 otherwise.             
\end{command}
	
\begin{command}{\pgfmathequalto\marg{x}\marg{y}}       
	Defines |\pgfmathresult| 1.0 if \meta{x} $=$ \meta{y}, but 0.0 otherwise.                    
\end{command}

\begin{command}{\pgfmathround\marg{x}}              
	Defines |\pgfmathresult| as $\left\lfloor\textrm{\meta{x}}\right\rceil$.	
	This uses asymmetric	half-up rounding.                          
\end{command}

\begin{command}{\pgfmathfloor\marg{x}}              
	Defines |\pgfmathresult| as $\left\lfloor\textrm{\meta{x}}\right\rfloor$.
\end{command}

\begin{command}{\pgfmathceil\marg{x}}               
	Defines |\pgfmathresult| as $\left\lceil\textrm{\meta{x}}\right\rceil$.                           
\end{command}
	
\begin{command}{\pgfmathpow\marg{x}\marg{y}}         
	Defines |\pgfmathresult| as $\meta{x}^{\meta{y}}$.  For greatest 
	accuracy \mvar{y} should be an integer. If \mvar{y} is not an integer 
	the actual calculation will be an approximation of $e^{y\ln(x)}$.
\end{command}

\begin{command}{\pgfmathmod\marg{x}\marg{y}}           
	Defines |\pgfmathresult| as \meta{x} modulo \meta{y}.                       
\end{command}

\begin{command}{\pgfmathmax\marg{x}\marg{y}}           
	Defines |\pgfmathresult| as the maximum of \meta{x} or \meta{y}.                       
\end{command}

\begin{command}{\pgfmathmin\marg{x}\marg{y}}           
	Defines |\pgfmathresult| as the minimum \meta{x} or \meta{y}.                       
\end{command}
	
\begin{command}{\pgfmathabs\marg{x}}                
	Defines |\pgfmathresult| as  absolute value of \meta{x}.                                 
\end{command}
	
\begin{command}{\pgfmathexp\marg{x}}                
	Defines |\pgfmathresult| as $e^{\meta{x}}$. Here, \meta{x} can be a 
	non-integer. The algorithm	uses a Maclaurin series.               
\end{command}

\begin{command}{\pgfmathln\marg{x}}                
	Defines |\pgfmathresult| as $\ln{\meta{x}}$. This uses an algorithm
	due to Rouben Rostamian, and coefficients suggested by
	Alain Matthes.             
\end{command}
	
\begin{command}{\pgfmathsqrt\marg{x}} 
	Defines |\pgfmathresult| as $\sqrt{\meta{x}}$. 
\end{command}
	
\begin{command}{\pgfmathveclen\marg{x}\marg{y}}        
	Defines |\pgfmathresult| as $\sqrt{\meta{x}^2+\meta{y}^2}$. This uses
	a polynomial approximation, based on ideas due to Rouben Rostamian.                                    
\end{command}

\subsection{Trignometric Functions}

\label{pgfmath-trigonmetry}

\begin{command}{\pgfmathpi}
  	Defines |\pgfmathresult| as $3.14159$.
\end{command}
   
\begin{command}{\pgfmathdeg{\marg{x}}} 
	Defines |\pgfmathresult| as \meta{x} (given in radians) converted to 
	degrees. 
\end{command}

\begin{command}{\pgfmathrad{\marg{x}}} 
	Defines |\pgfmathresult| as \meta{x} (given in degrees) converted to 
	radians. 
\end{command}

\begin{command}{\pgfmathsin{\marg{x}}}  
	Defines |\pgfmathresult| as the sine of \meta{x}.  
\end{command}

\begin{command}{\pgfmathcos{\marg{x}}}
	Defines |\pgfmathresult| as the cosine of \meta{x}.
\end{command}

\begin{command}{\pgfmathtan{\marg{x}}}  
	Defines |\pgfmathresult| as the tangant of \meta{x}.  
\end{command}

\begin{command}{\pgfmathsec{\marg{x}}}
	Defines |\pgfmathresult| as the secant of \meta{x}.
\end{command}

\begin{command}{\pgfmathcosec{\marg{x}}}  
	Defines |\pgfmathresult| as the cosecant of \meta{x}.  
\end{command}

\begin{command}{\pgfmathcot{\marg{x}}}  
	Defines |\pgfmathresult| as the cotangant of \meta{x}.  
\end{command}

\begin{command}{\pgfmathasin{\marg{x}}}
	Defines |\pgfmathresult| as the arcsine of \meta{x}. 
	The result will be in the range $\pm90^\circ$.
\end{command}

\begin{command}{\pgfmathacos{\marg{x}}}
	Defines |\pgfmathresult| as the arccosine of \meta{x}.
	The result will be in the range $\pm90^\circ$.
\end{command}

\begin{command}{\pgfmathatan{\marg{x}}}
 	Defines |\pgfmathresult| as the arctangent of \meta{x}.
\end{command}



\subsection{Pseudo-Random Numbers}

\label{pgfmath-random}


\begin{command}{\pgfmathgeneratepseudorandomnumber}
	Defines |\pgfmathresult| as a pseudo-random integer between 1 and 
	$2^{31}-1$. This uses a linear congruency generator, based on ideas
	due to Erich Janka.
\end{command}

\begin{command}{\pgfmathrnd}
	Defines |\pgfmathresult| as a pseudo-random number between |0| and |1|.
\end{command}

\begin{command}{\pgfmathrand}
	Defines |\pgfmathresult| as a pseudo-random number between |-1| and |1|.
\end{command}

\begin{command}{\pgfmathrandominteger\marg{macro}\marg{maximum}\marg{minimum}}
	This defines \meta{macro} as a pseudo-randomly generated integer from 
	the range \meta{maximum} to \meta{minimum} (inclusive).
	
\begin{codeexample}[]
\begin{pgfpicture}
   \foreach \x in {1,...,50}{
      \pgfmathrandominteger{\a}{1}{50}
      \pgfmathrandominteger{\b}{1}{50}
      \pgfpathcircle{\pgfpoint{+\a pt}{+\b pt}}{+2pt}
      \color{blue!40!white}
      \pgfsetstrokecolor{blue!80!black}
      \pgfusepath{stroke, fill}
   }	  
\end{pgfpicture}
\end{codeexample}
\end{command}

\begin{command}{\pgfmathdeclarerandomlist\marg{list name}\{\marg{item-1}\marg{item 2}...\}}
	This creates a list of items with the name \meta{list name}.
\end{command}

\begin{command}{\pgfmathrandomitem\marg{macro}\marg{list name}}
	Select an item from a random list \meta{list name}. The
	selected item is placed in \meta{macro}.
\end{command}

\begin{codeexample}[]
\begin{pgfpicture}
   \pgfmathdeclarerandomlist{color}{{red}{blue}{green}{yellow}{white}}
   \foreach \a in {1,...,50}{
      \pgfmathrandominteger{\x}{1}{85}
      \pgfmathrandominteger{\y}{1}{85}
      \pgfmathrandominteger{\r}{5}{10}
      \pgfmathrandomitem{\c}{color}
      \pgfpathcircle{\pgfpoint{+\x pt}{+\y pt}}{+\r pt}
      \color{\c!40!white}
      \pgfsetstrokecolor{\c!80!black}
      \pgfusepath{stroke, fill}
   }	  
\end{pgfpicture}
\end{codeexample}

\begin{command}{\pgfmathsetseed\marg{integer}}
  Explicitly set seed for the pseudo-random number generator. By
  default it is set to the value of |\time|$\times$|\year|.
\end{command}


      
\subsection{Conversion Between Bases}
	
\label{pgfmath-bases}

\pgfname{} provides limited support for conversion between 
\emph{representations} of numbers. Currently the numbers must be
positive integers in the range $0$ to $2^{31}-1$, and the bases in the
range $2$ to $36$. All digits representing numbers greater than 9 (in
base ten), are alphabetic, but may be upper or lower case. 

\begin{command}{\pgfmathbasetodec\marg{macro}\marg{number}\marg{base}}
	Defines \meta{macro} as the result of converting \meta{number} from
	base \meta{base} to base 10. Alphabetic digits can be upper or lower
	case.

\medskip{\def\medskip{}

\begin{codeexample}[]
\pgfmathbasetodec\mynumber{107f}{16} \mynumber
\end{codeexample}


\begin{codeexample}[]
\pgfmathbasetodec\mynumber{33FC}{20} \mynumber
\end{codeexample}

}\medskip

\end{command}

\begin{command}{\pgfmathdectobase\marg{macro}\marg{number}\marg{base}}
	Defines \meta{macro} as the result of converting \meta{number} from
	base 10 to base \meta{base}. Any resulting alphabetic digits are in
	\emph{lower case}.
	
\begin{codeexample}[]
\pgfmathdectobase\mynumber{65535}{16} \mynumber
\end{codeexample}

\end{command}

\begin{command}{\pgfmathdectoBase\marg{macro}\marg{number}\marg{base}}
	Defines \meta{macro} as the result of converting \meta{number} from
	base 10 to base \meta{base}. Any resulting alphabetic digits are in
	\emph{upper case}.
	
\begin{codeexample}[]
\pgfmathdectoBase\mynumber{65535}{16} \mynumber
\end{codeexample}

\end{command}

\begin{command}{\pgfmathbasetobase\marg{macro}\marg{number}\marg{base-1}\marg{base-2}}
	Defines \meta{macro} as the result of converting \meta{number} from
	base \meta{base-1} to base \meta{base-2}. Alphabetic digits in 
	\meta{number} can be upper or lower case, but any resulting 
	alphabetic digits are in \emph{lower case}.
	
\begin{codeexample}[]
\pgfmathbasetobase\mynumber{11011011}{2}{16} \mynumber
\end{codeexample}

\end{command}

\begin{command}{\pgfmathbasetoBase\marg{macro}\marg{number}\marg{base-1}\marg{base-2}}
	Defines \meta{macro} as the result of converting \meta{number} from
	base \meta{base-1} to base \meta{base-2}. Alphabetic digits in 
	\meta{number} can be upper or lower case, but any resulting 
	alphabetic digits are in \emph{upper case}.
	
\begin{codeexample}[]
\pgfmathbasetoBase\mynumber{121212}{3}{12} \mynumber
\end{codeexample}

\end{command}


\begin{command}{\pgfmathsetbasenumberlength\marg{integer}}
	Set the number of digits in the result of a base conversion to 
	\meta{integer}. If the result of a conversion has less digits
	than this number it is prefixed with zeros.

\begin{codeexample}[]
\pgfmathsetbasenumberlength{8}
\pgfmathdectobase\mynumber{15}{2} \mynumber
\end{codeexample}

\end{command}

% Copyright 2007 by Mark Wibrow
%
% This file may be distributed and/or modified
%
% 1. under the LaTeX Project Public License and/or
% 2. under the GNU Free Documentation License.
%
% See the file doc/generic/pgf/licenses/LICENSE for more details.


\section[Reimplementing the Computations of the Mathematical Engine]
  {Reimplementing the Computations of the\\ Mathematical Engine}

\label{pgfmath-reimplement}

Perhaps you are not satisfied with the Maclaurin series for
$e^x$. Perhaps you have a fantastically more accurate
and efficient way of calculating the sine or cosine of angles. Perhaps
 you would like the library to interface with a package such as |fp| 
 for fixed-point arithmetic (but you may find that exclusively
 using |fp| can cause a significant increase in compile time for
 documents involving many hundreds of calculations).
In these cases you will want to replace the current implementations of
the computations done by the mathematical engine by your own code. 

The mathematical engine was designed with such a replacement in
mind. For this reason, the operations and functions like |\pgfmathadd|
are implemented in the following manner: 

\begin{itemize}
\item |\pgfmath|\meta{function name} 

  This macro is the ``public'' interface for the function
  \meta{function name}. All arguments passed to this macro are 
  evaluated using |\pgfmathparse| and then passed on to the following
  function:
  
\item |\pgfmath|\meta{function name}|@|
  
  This macro is the ``non-public'' implementation of the functions 
  algorithm (but note that, for speed, the parser calls this macro 
  rather than the ``public'' one). Arguments passed to this macro 
  are expected to be numbers \emph{without units}. This is the macro 
  which should be rewritten with your prize-winning new algorithm.

  Note, furthermore, that if the function takes more than one
  argument, the second argument should not involve the dimensions
  |\pgfmath@x| nor |\pgfmath@xa| nor |\pgf@x| nor |\pgf@xa| since
  these may be set to the value of the first argument when the
  second argument is parsed.
\end{itemize}

The effect of |\pgfmath|\meta{function name}|@| should be to set the
macro |\pgfmathresult| to the correct value (namely to the result of
the computation without units). Furthermore, the function should have
no other side effects, that is, it should not change any global
values. One way to achieve this is to use the following code:

\begin{codeexample}[code only]
\def\pgfmath...@#1#2...{%
   \begingroup%
      ... code for algorithm ...
      \pgfmath@returnone\pgfmath@x%
   \endgroup%
}
\end{codeexample}


The macro |\pgfmath@returnone|\meta{code} must be directly followed by an
|\endgroup| and will save result of the algorithm, by defining
|\pgfmathresult| as the  expansion of \meta{code} \emph{without units}
outside the group. The \meta{code} should expand to a dimension
register or to a dimension. By performing the algorithm within a
\TeX{} group, \pgfname{} registers such as |\pgf@x|, |\pgf@y| and 
|\c@pgf@counta|, |\c@pgfcountb|, and so forth, can be used at will.

\pgfname{} uses the last known definition of a function within the
prevailing scope, so it is possible for a function to be redefined 
or |\let| to an alternative definition locally.
You should also remember that any |.sty| or |.tex| file contatining any
re-implementions should be loaded \emph{after} \pgfname-Math.

% Copyright 2007 by Mark Wibrow
%
% This file may be distributed and/or modified
%
% 1. under the LaTeX Project Public License and/or
% 2. under the GNU Free Documentation License.
%
% See the file doc/generic/pgf/licenses/LICENSE for more details.

\section{Conversion Between Bases}
	
\label{pgfmath-bases}

\pgfname{} provides limited support for conversion between 
\emph{representations} of numbers. Currently the numbers must be
positive integers in the range $0$ to $2^{31}-1$, and the bases in the
range $2$ to $36$. All digits representing numbers greater than 9 (in
base ten), are alphabetic, but may be upper or lower case. Note, that
again, examples in this section, show the result of the calculation
\emph{not} what is shown on screen.

\begin{command}{\pgfmathbasetodec\marg{macro}\marg{number}\marg{base}}
	Defines \meta{macro} as the result of converting \meta{number} from
	base \meta{base} to base 10. Alphabetic digits can be upper or lower
	case.

\medskip{\def\medskip{}

\begin{codeexample}[post=\tt\footnotesize\mynumber]
\pgfmathbasetodec\mynumber{107f}{16}
\end{codeexample}


\begin{codeexample}[post=\tt\footnotesize\mynumber]
\pgfmathbasetodec\mynumber{33FC}{20}
\end{codeexample}

}\medskip

\end{command}

\begin{command}{\pgfmathdectobase\marg{macro}\marg{number}\marg{base}}
	Defines \meta{macro} as the result of converting \meta{number} from
	base 10 to base \meta{base}. Any resulting alphabetic digits are in
	\emph{lower case}.
	
\begin{codeexample}[post=\tt\footnotesize\mynumber]
\pgfmathdectobase\mynumber{65535}{16}
\end{codeexample}

\end{command}

\begin{command}{\pgfmathdectoBase\marg{macro}\marg{number}\marg{base}}
	Defines \meta{macro} as the result of converting \meta{number} from
	base 10 to base \meta{base}. Any resulting alphabetic digits are in
	\emph{upper case}.
	
\begin{codeexample}[post=\tt\footnotesize\mynumber]
\pgfmathdectoBase\mynumber{65535}{16}
\end{codeexample}

\end{command}

\begin{command}{\pgfmathbasetobase\marg{macro}\marg{number}\marg{base-1}\marg{base-2}}
	Defines \meta{macro} as the result of converting \meta{number} from
	base \meta{base-1} to base \meta{base-2}. Alphabetic digits in 
	\meta{number} can be upper or lower case, but any resulting 
	alphabetic digits are in \emph{lower case}.
	
\begin{codeexample}[post=\tt\footnotesize\mynumber]
\pgfmathbasetobase\mynumber{11011011}{2}{16}
\end{codeexample}

\end{command}

\begin{command}{\pgfmathbasetoBase\marg{macro}\marg{number}\marg{base-1}\marg{base-2}}
	Defines \meta{macro} as the result of converting \meta{number} from
	base \meta{base-1} to base \meta{base-2}. Alphabetic digits in 
	\meta{number} can be upper or lower case, but any resulting 
	alphabetic digits are in \emph{upper case}.
	
\begin{codeexample}[post=\tt\footnotesize\mynumber]
\pgfmathbasetoBase\mynumber{121212}{3}{12}
\end{codeexample}

\end{command}


\begin{command}{\pgfmathsetbasenumberlength\marg{integer}}
	Set the number of digits in the result of a base conversion to 
	\meta{integer}. If the result of a conversion has less than this 
	number it is prefixed with zeros.

\begin{codeexample}[post=\tt\footnotesize\mynumber]
\pgfmathsetbasenumberlength{8}
\pgfmathdectobase\mynumber{15}{2}
\end{codeexample}

\end{command}
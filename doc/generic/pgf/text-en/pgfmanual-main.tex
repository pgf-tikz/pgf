% Copyright 2006 by Till Tantau
%
% This file may be distributed and/or modified
%
% 1. under the LaTeX Project Public License and/or
% 2. under the GNU Free Documentation License.
%
% See the file doc/generic/pgf/licenses/LICENSE for more details.

% pgf version is defined in \pgfversion in file
% generic/pgf/utilities/pgfrcs.code.tex 

\def\xcolorversion{2.00}
\def\xkeyvalversion{1.8}

\usepackage[version=0.96]{pgf}
\usepackage{tikz}
\usepackage{pgflibraryarrows}
\usepackage{pgflibraryshapes}
\usepackage{pgflibraryplotmarks}
\usepackage{pgflibrarytikzbackgrounds}
\usepackage{pgflibrarytikztrees}
\usepackage[left=2.25cm,right=2.25cm,top=2.5cm,bottom=2.5cm,nohead]{geometry}
\usepackage{amsmath,amssymb}
\usepackage{xxcolor}
\usepackage{pifont}
\usepackage{makeidx}
\usepackage[latin1]{inputenc}
\usepackage{amsmath}

\input{../../macros/pgfmanual-macros}

\makeindex

\makeatletter
\renewcommand*\l@subsection{\@dottedtocline{2}{1.5em}{2.8em}}
\renewcommand*\l@subsubsection{\@dottedtocline{3}{4.3em}{3.2em}}
\makeatother

%\includeonly{pgfmanual-libraries}

% Global styles:
\tikzstyle{every plot}=[prefix=plots/pgf-]
\tikzstyle{shape example}=[color=black!30,draw,fill=yellow!30,line width=.5cm,inner xsep=2.5cm,inner ysep=0.5cm]

\index{Options for graphics|see{Graphic options}}
\index{Options for packages|see{Package options}}
\index{File|see{Packages and files}}
\index{Layout|see{Page layout}}

\begin{document}

{
  \parindent0pt
\vbox{}
\vskip 3.5cm
\Huge
\tikzname\ and \pgfname

\Large
Manual for Version \pgfversion

\vskip 3cm 

\begin{codeexample}[graphic=white]
\tikz[rotate=30]
  \foreach \x / \xcolor in {0/blue,1/cyan,2/green,3/yellow,4/red}
    \foreach \y / \ycolor in {0/blue,1/cyan,2/green,3/yellow,4/red}
      \shade[ball color=\xcolor!50!\ycolor] (\x,\y) circle (7.5mm);
\end{codeexample}
\vskip 0cm plus 1.5fill
\vbox{}         
\clearpage
}

{
  \vbox{}
  \vskip0pt plus 1fill
  F�r meinen Vater, damit er noch viele sch�ne \TeX-Graphiken erschaffen kann.
  \vskip0pt plus 3fill
  \vbox{}
  \clearpage
}


\title{The \tikzname\ and \pgfname\ Packages\\
  Manual for Version \pgfversion\\[1mm]
\large\href{http://sourceforge.net/projects/pgf}{\texttt{http://sourceforge.net/projects/pgf}}}
\author{Till Tantau\\
  \href{mailto:tantau@users.sourceforge.net}{\texttt{tantau@users.sourceforge.net}}}

\maketitle

\tableofcontents

\clearpage

\part{Getting Started}

This part is intended to help you get started with the \pgfname\
package. First, the installation process is explained; however, the
system will typically be already installed on your system, so this can
often be skipped. Next, a short tutorial is given that explains the
most often used commands and concepts of \tikzname, without going into
any of the glorious details. At the end of this section you will find
some, hopefully useful, hints on how to create ``good'' graphics in
general. The information in this section is not specific to
\pgfname. 

\vskip3cm

\begin{codeexample}[graphic=white,width=0pt]
\tikz \draw[thick,rounded corners=8pt]
  (0,0) -- (0,2) -- (1,3.25) -- (2,2) -- (2,0) -- (0,2) -- (2,2) -- (0,0) -- (2,0);
\end{codeexample}

\include{pgfmanual-introduction}
\include{pgfmanual-installation}
\include{pgfmanual-tutorial}
\include{pgfmanual-guidelines}
\include{pgfmanual-drivers}


\part{Ti\emph{k}Z ist \emph{kein} Zeichenprogramm}
\label{part-tikz}

\vskip3cm
\begin{codeexample}[graphic=white]
\begin{tikzpicture}
  \draw[fill=yellow] (0,0) -- (60:.75cm) arc (60:180:.75cm);
  \draw(120:0.4cm) node {$\alpha$};

  \draw[fill=green!30] (0,0) -- (right:.75cm) arc (0:60:.75cm);
  \draw(30:0.5cm) node {$\beta$};

  \begin{scope}[shift={(60:2cm)}]
    \draw[fill=green!30] (0,0) -- (180:.75cm) arc (180:240:.75cm);
    \draw (30:-0.5cm) node {$\gamma$};

    \draw[fill=yellow] (0,0) -- (240:.75cm) arc (240:360:.75cm);
    \draw (-60:0.4cm) node {$\delta$};
  \end{scope}

  \begin{scope}[thick]
    \draw  (60:-1cm) node[fill=white] {$E$} -- (60:3cm) node[fill=white] {$F$};
    \draw[red]                   (-2,0) node[left] {$A$} -- (3,0) node[right]{$B$};
    \draw[blue,shift={(60:2cm)}] (-3,0) node[left] {$C$} -- (2,0) node[right]{$D$};
  
    \draw[shift={(60:1cm)},xshift=4cm]
    node [right,text width=6cm,rounded corners,fill=red!20,inner sep=1ex]
    {
      When we assume that $\color{red}AB$ and $\color{blue}CD$ are
      parallel, i.\,e., ${\color{red}AB} \mathbin{\|} \color{blue}CD$,
      then $\alpha = \delta$ and $\beta = \gamma$.
    };
  \end{scope}
\end{tikzpicture}
\end{codeexample}



\include{pgfmanual-tikz-design}
% Copyright 2003 by Till Tantau <tantau@cs.tu-berlin.de>.
%
% This program can be redistributed and/or modified under the terms
% of the LaTeX Project Public License Distributed from CTAN
% archives in directory macros/latex/base/lppl.txt.


\section[Hierarchical Structures: Package, Environments, Scopes, and Styles]
{Hierarchical Structures:\\
  Package, Environments, Scopes, and Styles}

The present section explains how your files should be structured when
you use \tikzname. On the top level, you need to include the |tikz|
package. In the main text, each graphi needs to be put in a
|{tikzpicture}| environment. Inside these environments, you can use
|{scope}| environments to create internal groups. Inside the scopes,
you use |\path| commands to actually draw something. On all levels
(except for the package level), graphic options can be given that
apply to everything within the environment.



\subsection{Loading the Package}

\begin{package}{tikz}
  This package does not have any options.
  
  This will automatically load the \pgfname\ package and several other
  stuff that \tikzname\ needs (like the |xkeyval| package).

  \pgfname\ needs to know what \TeX\ driver you are intending to use. In
  most cases, \pgfname\ is clever enough to find out the correct driver
  for you; this is true in particular if you \LaTeX. Currently, the only
  situation where \pgfname\ cannot know the driver ``by itself'' is when
  you use plain \TeX\ or Con\TeX\ together with |dvipdfm|. In this case,
  you have to write |\def\pgfsysdriver{pgfsys-dvipdfm.def}|
  \emph{before} you input |tikz.tex|. 
\end{package}


\subsection{The Main Picture Environment}

The ``outermost'' scope of \pgfname\ and \tikzname\ is the |{tikzpicture}| 
environment. You may give drawing commands only inside this
environment, giving them outside (as is possible in many other
packages) will result in chaos.

In \tikzname, most of how a graphic ``looks like'' is governed by graphic
options. For example, there is an option for setting the color used
for drawing, another for setting the color used for filling, and some
more obscure one like the option  for setting the prefix used in the
filenames of temporary files written while plotting functions using an
external program. The graphic options are nearly always specified in a
so-called key-value style. (The ``nearly always'' refers to the name
of nodes, which can also be specified differently.) All graphic
options are local to the |{tikzpicture}| to which they apply.

\begin{environment}{{tikzpicture}\opt{\oarg{options}}}
  All \tikzname\ and nearly all \pgfname\ commands should be given inside
  this environment. Unlike other packages, it is not possible to use,
  say, |\pgfpathmoveto| outside this environment and will result in
  chaos. For \tikzname, commands like |\path| are only defined inside this
  environment, so there is little chance that you will do something
  wrong here. 

  When this environment is encountered, the \meta{options} are
  parsed. All options given here will apply to the whole
  picture. Giving options makes sense only when \tikzname\ is loaded, the
  \pgfname\ core does not define any options.

  Next, the contents of the environment is processed and the graphic
  commands therein are put into a box. Non-graphic text is suppressed
  as well as possible, but non-\pgfname\ commands inside a
  |{tikzpicture}| environment should not produce any ``output'' since
  this may totally scramble the positioning system of the backend
  drivers. The suppressing of normal text, by the way, is done by
  temporarily switching the font to |\nullfont|. You can, however,
  ``escape back'' to normal \TeX\ typesetting. This happens, for
  example, when you specify a node.

  At the end of the environment, \pgfname\ tries to make a good guess
  at a good guess at the size of the bounding box of the graphic and
  then resizes the box such that the box has this size. To ``make its
  guess,'' everytime \pgfname\ encounters a coordinate, it updates the
  bound box's size such that it encompasses all these
  coordinates. This will usually give a good 
  approximation at the bounding box, but will not always be
  accurate. First, the line thickness is not taken into
  account. Second, controls points of a curve often lie far
  ``outside'' the curve and make the bounding box too large. In this
  case, you should use the |[use as bounding box]| option.

  The following option influences the baseline of the resulting
  picture:
  \begin{itemize}
    \itemoption{baseline}\opt{|=|\meta{dimension}}
    Normally, the lower end of the picture is put on the baseline of
    the surrounding text. For example, when you give the code
    |\tikz\draw(0,0)circle(.5ex);|, \pgfname\ will find out that the
    lower end of the picture is at $-.5\mathrm{ex}$ and that the upper
    end is at $.5\mathrm{ex}$. Then, the lower end will be put on the
    baseline, resulting in the following: \tikz\draw(0,0)circle(.5ex);.

    Using this option, you can specify that the picture should be
    raised or lowered such that the height \meta{dimension} is on the
    baseline. For example, |tikz[baseline=0pt]\draw(0,0)circle(.5ex);|
    yields \tikz[baseline=0pt]\draw(0,0)circle(.5ex); since, now, the
    baseline is on the height of the $x$-axis. If you omit the
    \meta{dimensions}, |0pt| is assumed as default.

    This options is often useful for ``inlined'' graphics as in
\begin{codeexample}[]
$A \mathbin{\tikz[baseline] \draw[->>] (0pt,.5ex) -- (3ex,.5ex);} B$
\end{codeexample}
  \end{itemize}
  
  All options ``end'' at the end of the picture. To set an option
  ``globally'' you can use the following style:
  \begin{itemize}
    \itemstyle{every picture}
    This style is installed at the beginning of each picture.
\begin{codeexample}[code only]
\tikzstyle{every picture}=[semithick]
\end{codeexample}
\end{itemize}
\end{environment}

In plain \TeX, you should use instead the following commands:

\begin{plainenvironment}{{tikzpicture}\opt{\oarg{options}}}
\end{plainenvironment}

The following two commands are used for ``small'' graphics.

\begin{command}{\tikz\opt{\oarg{options}}\marg{commands}}
  This little command places the \meta{commands} inside a
  |{tikzpicture}| environment and adds a semicolon at the end. This is
  just a convenience.

  The \meta{commands} may not contain a paragraph (an empty
  line). This is a precaution to ensure that users really use this
  command only for small graphics.

  \example |\tikz{\draw (0,0) rectangle (2ex,1ex)}| yields
  \tikz{\draw (0,0) rectangle (2ex,1ex);} 
\end{command}


\begin{command}{\tikz\opt{\oarg{options}}\meta{text}|;|}
  If the \meta{text} does not start with an opening brace, the end of
  the \meta{text} is the next semicolon that is encountered.

  \example |\tikz \draw (0,0) rectangle (2ex,1ex);| yields
  \tikz \draw (0,0) rectangle (2ex,1ex);
\end{command}


\subsection{Scopes}

Inside a |{tikzpicture}| environment, you can create ``subscopes''
using the |{scope}| environment. This environment is available only
inside the |{tikzpicture}| environment, so once more, there is little
chance of doing anything wrong.

\begin{environment}{{scope}\opt{\oarg{options}}}
  All \meta{options} are local to the \meta{environment
  contents}. Furthermore, the clipping path is also local to the
  environment, that is, any clipping done inside the environment
  ``ends'' at its end.

  \example
\begin{tikzpicture}
  \begin{scope}[red]
    \draw (0mm,0mm) -- (10mm,0mm);
    \draw (0mm,1mm) -- (10mm,1mm);
  \end{scope}
  \draw (0mm,2mm) -- (10mm,2mm);
  \begin{scope}[green]
    \draw (0mm,3mm) -- (10mm,3mm);
    \draw (0mm,4mm) -- (10mm,4mm);
    \draw[blue] (0mm,5mm) -- (10mm,5mm);
  \end{scope}
\end{tikzpicture}
\begin{verbatim}
\begin{tikzpicture}
  \begin{scope}[red]
    \draw (0mm,5mm) -- (10mm,5mm);
    \draw (0mm,4mm) -- (10mm,4mm);
  \end{scope}
  \draw (0mm,3mm) -- (10mm,3mm);
  \begin{scope}[green]
    \draw (0mm,2mm) -- (10mm,2mm);
    \draw (0mm,1mm) -- (10mm,1mm);
    \draw[blue] (0mm,0mm) -- (10mm,0mm);
  \end{scope}
\end{tikzpicture}
\end{verbatim}
  
  The following style influences scopes:
  \begin{itemize}
    \itemstyle{every scope}
    This style is installed at the beginning of every scope. I do not
    know really know what this might be good for, but who knows?
  \end{itemize}
\end{environment}


In plain \TeX, you use the following commands instead:

\begin{plainenvironment}{{scope}\opt{\oarg{options}}}
\end{plainenvironment}



\subsection{Path Scopes}

The |\path| command, which is described in much more detail in later
sections, also takes graphic options. These options are local to the
path. Furthermore, it is possible to create local scopes withing a
path simply by using curly braces as in
\begin{codeexample}[]
\tikz \draw (0,0) -- (1,1)
           {[rounded corners] -- (2,0) -- (3,1)}
           -- (3,0) -- (2,1);
\end{codeexample}

Note that many options apply only to the path as a whole and cannot be
scoped in this way. For example, it is not possible to scope the
|color| of the path. See the explanations in the section on paths for
more details.

Finally, certain elements that you specify in the argument to the
|\path| command also take local options. For example, a node
specification takes options. In this case, the options apply only to
the node, not to the surrounding path.


\subsection{Styles}

There is a way of organizing sets of graphic options ``orthogonally''
to the normal scoping mechanism. For example, you might wish all your
``help lines'' to be drawn in a certain way like, say, gray and thin
(do \emph{not} dash them, that distracts). For this, you can use
\emph{styles}.

A style is simply a set of graphic options that is predefined at some
point. Once a style has been defined, it can be used anywhere using
the |style| option:

\begin{itemize}
  \itemoption{style}|=|\meta{style name}
  invokes all options that are currently set in the \meta{style
    name}. An example of a style is the predefined |help lines| style,
  which you should use for lines in the background like grid lines or
  construction lines. You can easily define new styles and modify
  existing ones.
\begin{codeexample}[]
\begin{tikzpicture}
  \draw                   (0,0) grid +(2,2);
  \draw[style=help lines] (2,0) grid +(2,2);
\end{tikzpicture}
\end{codeexample}
\end{itemize}


\begin{command}{\tikzstyle\meta{style name}\opt{|+|}|=[|\meta{options}|]|}
  This command defines the style \meta{style name}. Whenever it is
  used using the |style=|\meta{style name} command, the \meta{options}
  will be invoked. It is permissible that a style invokes another
  style using the |style=| command inside the \meta{options}, which
  allows you to build hierarchies of styles. Naturally, you should
  \emph{not} create cyclic dependencies.

  If the style already has a predefined meaning, it will
  uncermimoniously be redefined without a warning.
\begin{codeexample}[]
\tikzstyle help lines=[blue!50,very thin]
\begin{tikzpicture}
  \draw                   (0,0) grid +(2,2);
  \draw[style=help lines] (2,0) grid +(2,2);
\end{tikzpicture}
\end{codeexample}

  If the optional |+| is given, the options are \emph{added} to the
  existing defintion:
\begin{codeexample}[]
\tikzstyle help lines+=[dashed] % aaarghhh!!!
\begin{tikzpicture}
  \draw                   (0,0) grid +(2,2);
  \draw[style=help lines] (2,0) grid +(2,2);
\end{tikzpicture}
\end{codeexample}
\end{command}

% Copyright 2003 by Till Tantau <tantau@cs.tu-berlin.de>.
%
% This program can be redistributed and/or modified under the terms
% of the LaTeX Project Public License Distributed from CTAN
% archives in directory macros/latex/base/lppl.txt.


\section{Specifying Coordinates}


\subsection{Coordinates and Coordinate Options}

A \emph{coordinate} is a position in a picture. \tikzname\ uses a
special syntax for specifying coordinates. Coordinates are always put
in round brackets. The general syntax is
\declare{|(|\opt{|[|\meta{options}|]|}\meta{coordinate  specification}|)|}. 

It is possible to give options that apply only to a single
coordinate, although this makes sense for transformation options
only. To give transformation options for a single coordinate, give
these options at the beginning in brackets:
\begin{codeexample}[]
\begin{tikzpicture}
  \draw[style=help lines] (0,0) grid (3,2);
  \draw      (0,0) -- (1,1);
  \draw[red] (0,0) -- ([xshift=3pt] 1,1);
  \draw      (1,0) -- +(30:2cm);
  \draw[red] (1,0) -- +([shift=(135:5pt)] 30:2cm);
\end{tikzpicture}
\end{codeexample}

\subsection{Simple Coordinates}

The simplest way to specify coordinates is as a comma-separated pair
of \TeX\ dimensions as in |(1cm,2pt)| or |(2cm,\textheight)|. As can
be seen, different units can be mixed. The coordinate specified in
this way means ``1cm to the right and 2pt up from the origin of the
picture.'' You can also write things like |(1cm+2pt,2pt)| since the
|calc| package is used. 


\subsection{Polar Coordinates}

You can also specify coordinates in polar coordinates. In this case,
you specify an angle and a distance, separated by a colon as in
|(30:1cm)|. The angle must always be given in degrees and should be
between $-360$ and $720$. 

\begin{codeexample}[]
\tikz \draw    (0cm,0cm) -- (30:1cm) -- (60:1cm) -- (90:1cm)
            -- (120:1cm) -- (150:1cm) -- (180:1cm);
\end{codeexample}

Instead of an angle given as a number you can also use certain
words. For example, |up| is the same as |90|, so that you can write
|\tikz \draw (0,0) -- (2ex,0pt) -- +(up:1ex);|
and get \tikz \draw (0,0) -- (2ex,0pt) -- +(up:1ex);. Apart from |up|
you can use |down|, |left|, |right|, |north|, |south|, |west|, |east|,
|north east|, |north west|, |south east|, |south west|, all of which
have their natural meaning.



\subsection{Xy- and Xyz-Coordinates}

You can specify coordinates in \pgfname's $xy$-coordinate system. In
this case, you provide two unit-free numbers, separated by a comma as
in |(2,-3)|. This means ``add twice the current \pgfname\ $x$-vector and
subtract three times the $y$-vector.'' By default, the $x$-vector
points 1cm to the right, the $y$-vector points 1cm upwards, but this
can be changed arbitrarily using the |x| and~|y| graphic options.

Similarly, you can specify coordinates in the $xyz$-coordinate
system. The only difference to the $xy$-coordinates is that you
specify three numbers separated by commas as in |(1,2,3)|. This is
interpreted as ``once the $x$-vector plus twice the $y$-vector plus
three times the $z$-vector.'' The default $z$-vector points to
$\bigl(-\frac{1}{\sqrt2}
\textrm{cm},-\frac{1}{\sqrt2}\textrm{cm}\bigr)$. Consider the
following example: 

\begin{codeexample}[]
\begin{tikzpicture}[->]
  \draw (0,0,0) -- (1,0,0);
  \draw (0,0,0) -- (0,1,0);
  \draw (0,0,0) -- (0,0,1);
\end{tikzpicture}
\end{codeexample}


\subsection{Node Coordinates}
\label{section-node-coordinates}

In \pgfname\ and in \tikzname\ it is quite easy to define a node that you
wish to reference at a later point. Once you have defined a node,
there are different ways of referencing points of the node.


\subsubsection{Named Anchor Coordinates}

An \emph{anchor coordinate} is a point in a node that you have
previously defined using the node operation. The syntax is
|(|\meta{node name}|.|\meta{anchor}|)|, where \meta{node name} is
the name that was previously used to name the node using the
|name=|\meta{node name} option or the special node name syntax. Here is
an example: 

\begin{codeexample}[]
\begin{tikzpicture}
  \node (shape)   at (0,2)  [draw] {|class Shape|};
  \node (rect)    at (-2,0) [draw] {|class Rectangle|};
  \node (circle)  at (2,0)  [draw] {|class Circle|};
  \node (ellipse) at (6,0)  [draw] {|class Ellipse|};

  \draw (circle.north) |- (0,1);
  \draw (ellipse.north) |- (0,1);
  \draw[-open triangle 90] (rect.north) |- (0,1) -| (shape.south);
\end{tikzpicture}
\end{codeexample}

Section~\ref{section-the-shapes} explain which anchors are available
for the basic shapes. 




\subsubsection{Angle Anchor Coordinates}

In addition to the named anchors, it is possible to use the syntax
\meta{node name}|.|\meta{angle} to name a point of the node's
border. This point is the coordinate where a ray shot from the center
in the given angle hits the border. Here is an example:

\begin{codeexample}[]
\begin{tikzpicture}
  \node (start) [draw,shape=ellipse] {start};
  \foreach \angle in {-90, -80, ..., 90}
    \draw (start.\angle) .. controls +(\angle:1cm) and +(-1,0) .. (2.5,0);
  \end{tikzpicture}
\end{codeexample}


\subsubsection{Anchor-Free Node Coordinates}

It is also possible to just ``leave out'' the anchor and have \tikzname\
calculate an appropriate border position for you. Here is an example:

\begin{codeexample}[]
\begin{tikzpicture}[fill=blue!20]
  \draw[style=help lines] (-1,-2) grid (6,3);
  \path (0,0)  node(a) [ellipse,rotate=10,draw,fill]    {An ellipse}
        (3,-1) node(b) [circle,draw,fill]               {A circle}
        (2,2)  node(c) [rectangle,rotate=20,draw,fill]  {A rectangle}
        (5,2)  node(d) [rectangle,rotate=-30,draw,fill] {Another rectangle};
  \draw[thick] (a) -- (b) -- (c) -- (d);
  \draw[thick,red,->] (a) |- +(1,3) -| (c) |- (b);       
  \draw[thick,blue,<->] (b) .. controls +(right:2cm) and +(down:1cm) .. (d);       
\end{tikzpicture}
\end{codeexample}

\tikzname\ will be reasonably clever at determining the border points that
you ``mean,'' but, naturally, this may fail in some situations. If
\tikzname\ fails to determine an appropriate border point, the center will
be used instead.

Automatic computation of anchors works only with the line-to operations
|--|, the vertical/horizontal versions \verb!|-! and \verb!-|!, and
with the curve-to operation |..|. For other path commands such as
|parabola| or |plot|, the center will be used. If this is not desired,
you should give a named anchor or an angle anchor.

Note that if you use an automatic coordinate for both the start and
the end of a line-to, as in |--(b)--|, then \emph{two} border
coordinates are computed with a move-to between them. This is usually
exactly what you want.

If you use relative coordinates together with automatic anchor
coordinates, the relative coordinates are always computed relative to
the node's center, not relative to the border point. Here is an
example:

\begin{codeexample}[]
\tikz \draw (0,0) node(x) [draw] {Text}
            rectangle (1,1)
            (x) -- +(1,1);
\end{codeexample}

Similarly, in the following examples both control points are $(1,1)$:

\begin{codeexample}[]
\tikz \draw (0,0) node(x) [draw] {X}
            (2,0) node(y) {Y}
            (x) .. controls +(1,1) and +(-1,1) .. (y);
\end{codeexample}

           
\subsection{Intersection Coordinates}


\subsubsection{Intersection of Two Lines}

Often you wish to specify a point that is on the
intersection of two lines. The first way to specify such an
intersection is the following: You can use the special syntax
\declare{|(intersection of |\meta{$p_1$}|--|\meta{$p_2$}%
  | and |\meta{$q_1$}|--|\meta{$q_2$}|)|}. This will yield the
intersection point of the line going through $p_1$ and $p_2$ and the
line through $q_1$ and $q_2$. If the lines do not meet or if they are
identical and arithmetical overflow error will result.

\begin{codeexample}[]
\begin{tikzpicture}
  \draw[help lines] (0,0) grid (3,2);
  \draw (0,0) coordinate (A) -- (3,2) coordinate (B)
        (1,2)                -- (3,0);

  \fill[red] (intersection of A--B and 1,2--3,0) circle (2pt);
\end{tikzpicture}
\end{codeexample}

\subsubsection{Intersection of Horizontal and Vertical Lines}

A frequent special case of intersections is the intersection of a
vertical line going through a point $p$ and a horizontal line going
through some other point $q$. For this situation there is a special,
shorter, syntax: You can say either
\declare{|(|\meta{p}\verb! |- !\meta{q}|)|} or
\declare{|(|\meta{q}\verb! -| !\meta{p}|)|}.

For example, \verb!(2,1 |- 3,4)! and  \verb!(3,4 -| 2,1)! both yield
the same as \verb!(2,4)! (provided the $xy$-coordinate system has not
been modified). 

The most useful application of the syntax is to draw a line up to some
point on a vertical or horizontal line. Here is an example:

\begin{codeexample}[]
\begin{tikzpicture}
  \path (30:1cm) node(p1) {$p_1$}   (75:1cm) node(p2) {$p_2$};

  \draw (-0.2,0) -- (1.2,0) node(xline)[right] {$q_1$};
  \draw (2,-0.2) -- (2,1.2) node(yline)[above] {$q_2$};

  \draw[->] (p1) -- (p1 |- xline);
  \draw[->] (p2) -- (p2 |- xline);
  \draw[->] (p1) -- (p1 -| yline);
  \draw[->] (p2) -- (p2 -| yline);
\end{tikzpicture}
\end{codeexample}



\subsection{Relative and Incremental Coordinates}

You can prefix coordinates by |++| to make them ``relative.'' A
coordinate such as |++(1cm,0pt)| means ``1cm to the right of the
previous position.'' Relative coordinates are often useful in
``local'' contexts:

\begin{codeexample}[]
\begin{tikzpicture}
  \draw (0,0)     -- ++(1,0) -- ++(0,1) -- ++(-1,0) -- cycle;
  \draw (2,0)     -- ++(1,0) -- ++(0,1) -- ++(-1,0) -- cycle;
  \draw (1.5,1.5) -- ++(1,0) -- ++(0,1) -- ++(-1,0) -- cycle;
\end{tikzpicture}
\end{codeexample}

Instead of |++| you can also use a single |+|. This also specifies a
relative coordinate, but it does not ``update'' the current point for
subsequent usages of relative coordinates. Thus, you can use this
notation to specify numerous points, all relative to the same
``initial'' point:

\begin{codeexample}[]
\begin{tikzpicture}
  \draw (0,0)     -- +(1,0) -- +(1,1) -- +(0,1) -- cycle;
  \draw (2,0)     -- +(1,0) -- +(1,1) -- +(0,1) -- cycle;
  \draw (1.5,1.5) -- +(1,0) -- +(1,1) -- +(0,1) -- cycle;
\end{tikzpicture}
\end{codeexample}

There is one special situation, where relative coordinates are
interpreted differently. If you use a relative coordinate as a control
point of a B�zier curve, the following rule applies: First, a relative
first control point is taken relative to the beginning of the
curve. Second, a relative second control point is taken relative to
the end of the curve. Third, a relative end point of a curve is taken
relative to the start of the curve.

This special behavior makes it easy to specify that a curve should
``leave or arrives from a certain direction'' at the start or end. In
the following example, the curve ``leaves'' at $30^\circ$ and
``arrives'' at $60^\circ$: 

\begin{codeexample}[]
\begin{tikzpicture}
  \draw (1,0) .. controls +(30:1cm) and +(60:1cm) .. (3,-1);
  \draw[gray,->] (1,0) -- +(30:1cm);
  \draw[gray,<-] (3,-1) -- +(60:1cm);
\end{tikzpicture}
\end{codeexample}

% Copyright 2005 by Till Tantau <tantau@cs.tu-berlin.de>.
%
% This program can be redistributed and/or modified under the terms
% of the LaTeX Project Public License Distributed from CTAN
% archives in directory macros/latex/base/lppl.txt.


\section{Syntax for Path Specifications}

A \emph{path} is a series of straight and curved line segments. It is
specified following a |\path| command and the specification must
follow a special syntax, which is described in the subsections of the
present section.


\begin{command}{\path\meta{specification}|;|}
  This command is available only inside a |{tikzpicture}| environment
  and  only if the \tikzname\ package is loaded.

  The \meta{specification} is a long stream of \emph{path
  operations}. Most of these path operations tell \tikzname\ how the path
  is build. For example, when you write |--(0,0)|, you use a
  \emph{lineto operation} and it means ``continue the path from
  wherever you are to the origin.''

  At any point where \tikzname\ expects a path operation, you can also
  give some graphic options, which is a list of options in brackets,
  such as |[rounded cornders]|. These options can have different
  effects:
  \begin{enumerate}
  \item
    Some options take ``immediate'' effect and apply to all subsequent
    path operations on the path. For example, the |rounded corners|
    option will round all following corners, but not the corners
    ``before'' and if the |sharp corners| is given later on the path
    (in a new set of brackets), the rounding effect will end.

\begin{codeexample}[]
\tikz \draw (0,0) -- (1,1)
           [rounded corners] -- (2,0) -- (3,1)
           [sharp corners] -- (3,0) -- (2,1);
\end{codeexample}
    Another example are the transformation options, which also apply
    only to subsequent coordinates.
  \item
    The options that have immediate effect can be ``scoped'' by
    putting part of a path in curly braces. For example, the above
    example could also be written as follows:

\begin{codeexample}[]
\tikz \draw (0,0) -- (1,1)
           {[rounded corners] -- (2,0) -- (3,1)}
           -- (3,0) -- (2,1);
\end{codeexample}
  \item
    Some options only apply to the path as a whole. For example, the
    |color=| option for determining the color used for, say, drawing
    the path always applies to all parts of the path. If several
    different colors are given for different parts of the path, only
    the last one (on the outermost scope) ``wins'':
 
\begin{codeexample}[]
\tikz \draw (0,0) -- (1,1)
           [color=red] -- (2,0) -- (3,1)
           [color=blue] -- (3,0) -- (2,1);
\end{codeexample}

    Most options are of this type. In the above example, we would have
    had to ``split up'' the path into several |\path| commands:
\begin{codeexample}[]
\tikz{\draw (0,0) -- (1,1);
      \draw [color=red] (2,0) -- (3,1);
      \draw [color=blue] (3,0) -- (2,1);}
\end{codeexample}
  \end{enumerate}

  By default, the |\path| command does ``nothing'' with the
  path, it just ``throws it away.'' Thus, if you write
  |\path(0,0)--(1,1);|, nothing is drawn 
  in your picture. The only effect is that the area occupied by the
  picture is (possibly) enlarged so that the path fits inside the
  area. To actuall ``do'' something with the path, an option like
  |draw| or |fill| must be given somewhere on the path. Commands like
  |\draw| do this implicitly.
  
  Finally, it is also possible to give \emph{node specifications} on a
  path. Such specifications can come at different locations, but they
  are always allowed when a normal path operation could follow. A node
  specification starts with |node|. Basically, the effect is to
  typeset the node's text as normal \TeX\ text and to place
  it at the ``current location'' on the path. The details are exlained
  in Section~\ref{section-nodes}.

  Note, however, that the nodes are \emph{not} part of the path in any
  way. Rather, after everything has been done with the path what is
  specified by the path options (like filling and drawing the path due
  to a |fill| and a |draw| option somewhere in the
  \meta{specification}), the nodes are added in a post-processing
  step.   
  
  The following style influences scopes:
  \begin{itemize}
    \itemstyle{every path}
    This style is installed at the beginning of every path. This can
    be useful for (temporarily) adding, say, the |draw| option to
    everything in a scope.
\begin{codeexample}[]
\begin{tikzpicture}[fill=yellow!50!black] % only sets the color
  \tikzstyle{every path}=[draw]           % all paths are drawn
  \fill  (0,0) rectangle +(1,1);
  \shade (2,0) rectangle +(1,1);
\end{tikzpicture}
\end{codeexample}
  \end{itemize}
\end{command}




\subsection{The Move-To Operation}

A \emph{move-to operation} normally starts a path at a certain
point. This does not cause a line segment to be created, but it
specifies the starting point of the next segment. If a path is
already under construction, that is, if several segments have
already been created, a move-to operation will start a new part of the
path that is not connected to any of the previous segments.

The syntax for specifying a move-to operation is easy:
\declare{\meta{coordinate}}. Simply provide a coordinate in round
brackets. For example, the following command (when used inside a
|{tikzpicture}| environment) draws two lines: 

\begin{codeexample}[]
\begin{tikzpicture}
  \draw (0,0) --(2,0) (0,1) --(2,1);
\end{tikzpicture}
\end{codeexample}

In the specification |(0,0) --(2,0) (0,1) --(2,1)| two move-to
operations are specified: |(0,0)| and |(0,1)|. The other two
operations, namely |--(2,0)| and |--(2,1)| are line-to operations,
described next.


\subsection{The Line-To Operation}

A \emph{line-to operation} extends the current path from the current
point in a straight line to the given coordinate. The ``current
point'' is the endpoint of the previous drawing command or the point
specified by a prior moveto operation.

Syntax of the line-to operation: \declare{|--|\meta{coordinate}}. You
use two minus signs followed by a coordinate in round brackets. You
can add spaces before and after the~|--|.

When a line-to operation is used and some path segment has just been
constructed, for example by another line-to operation, the two line
segments become joined. This means that if they are drawn, the point
where they meet is ``joined'' smoothly. To appreciate the difference,
consider the following two examples: In the left example, the path
consists of two path segments that are not joined, but that happen to
share a point, while in the right example a smooth join is shown.

\begin{codeexample}[]
\begin{tikzpicture}[line width=10pt]
  \draw (0,0) --(1,1)  (1,1) --(2,0);
  \draw (3,0) -- (4,1) -- (5,0);
\end{tikzpicture}
\end{codeexample}



\subsection{Horizontal/Vertical Line-To Operations}

Sometimes you want to connect two points via straight lines that are
only horizontal and vertical. For this, you can use the path
construction commands \verb!-|! and \verb!|-!. The first means
``first horizontal, then vertical,'' the second means ``first
vertical, then horizontal.''

Syntax of these versions of the line-to operation:
\declare{\verb!-|!\meta{coordinate}} and
\declare{\verb!|-!\meta{coordinate}}. You can add spaces for clarity. 

Here is an example:

\begin{codeexample}[]
\begin{tikzpicture}
  \draw (0,0) node(a) [draw] {A}  (1,1) node(b) [draw] {B};
  \draw (a.north) |- (b.west);
  \draw[color=red] (a.east) -| (2,1.5) -| (b.north);
\end{tikzpicture}
\end{codeexample}


\subsection{The Curveto Operation}

A \emph{curveto operation} extends the current path from the current
point, let us call it $x$, via a curve to a new current point, let us
call it $y$. The curve is a so-called Bezi�r curve. For such a curve,
apart from $y$, you also specify two control points $c$ and $d$. The
idea is that the curve starts at $x$, ``heading'' in the direction
of~$c$. Mathematically spoken, the tangent of the curve at $x$ goes
through $c$. Similarly, the curve ends at $y$, ``coming from'' the
other control point,~$d$. The larger the distance between $x$ and~$c$
and between $d$ and~$y$, the larger the curve will be.

Syntax of the line-to operation:
\declare{|..controls|\meta{c}}\opt{|and|\meta{d}}\declare{|..|\meta{y}}. If
\meta{d} is not given, $d$ is assumed to be equal to $c$. 

\begin{codeexample}[]
\begin{tikzpicture}
  \draw[line width=10pt] (0,0) .. controls (1,1) .. (4,0)
                               .. controls (5,0) and (5,1) .. (4,1);
  \draw[color=gray] (0,0) -- (1,1) -- (4,0) -- (5,0) -- (5,1) -- (4,1);
\end{tikzpicture}
\end{codeexample}

As with the line-to operation, it makes a difference whether two curves
are joined because they resulted from consecutive curve-to or line-to
operations, or whether they just happen to have the same ending:

\begin{codeexample}[]
\begin{tikzpicture}[line width=10pt]
  \draw (0,0) -- (1,1) (1,1) .. controls (1,0) and (2,0) .. (2,0);
  \draw (3,0) -- (4,1) .. controls (4,0) and (5,0) .. (5,0);
\end{tikzpicture}
\end{codeexample}


\subsection{The Cycle Operation}

A \emph{cycle operation} adds a straight line from the current
point to the last point specified by a move-to operation. Note that
this need not be the beginning of the path. Furthermore, a smooth join
is created between the first segment created after the last move-to
operation and the straight line appended by the cycle operation.

Syntax of the line-to operation:
\declare{|--cycle|}. You can add a space between |--| and |cycle| for
clarity.  

Consider the following example. In the left example, two triangles are
created using three straight lines, but they are not joined at the
ends. In the second example cycle operations are used.

\begin{codeexample}[]
\begin{tikzpicture}[line width=10pt]
  \draw (0,0) -- (1,1) -- (1,0) -- (0,0) (2,0) -- (3,1) -- (3,0) -- (2,0);
  \draw (5,0) -- (6,1) -- (6,0) -- cycle (7,0) -- (8,1) -- (8,0) -- cycle;
\end{tikzpicture}
\end{codeexample}


\subsection{Rounding Corners}

All of the path construction commands mentioned up to now are
influenced by the following option:
\begin{itemize}
  \itemoption{rounded corners}\opt{|=|\meta{inset}}
  When this option is in force, all corners (places where a line is
  continued either via line-to or a curve-to operation) are replaced by
  little arcs so that the corner becomes smooth. 

\begin{codeexample}[]
\tikz \draw [rounded corners] (0,0) -- (1,1)
           -- (2,0) .. controls (3,1) .. (4,0);
\end{codeexample}

  The \meta{inset} describes how big the corner is. Note that the
  \meta{inset} is \emph{not} scaled along if you use a scaling option
  like |scale=|. 

\begin{codeexample}[]
\begin{tikzpicture}
  \draw[color=gray,very thin] (10pt,15pt) circle (10pt);
  \draw[rounded corners=10pt] (0,0) -- (0pt,25pt) -- (40pt,25pt);
\end{tikzpicture}
\end{codeexample}

  You can switch the rounded corners on and off ``in the middle of
  path'' and different corners in the same path can have different
  corner radii:

\begin{codeexample}[]
\begin{tikzpicture}
  \draw (0,0) [rounded corners=10pt] -- (1,1) -- (2,1)
                     [sharp corners] -- (2,0)
               [rounded corners=5pt] -- cycle;
\end{tikzpicture}
\end{codeexample}

\example Here is a rectangle with rounded corners:
\begin{codeexample}[]
\tikz \draw[rounded corners=1ex] (0,0) rectangle (20pt,2ex);
\end{codeexample}

  You should be aware, that there are several pitfalls when using this
  option. First, the rounded corner will only be an arc (part of a
  circle) if the angle is $90^\circ$. In other cases, the rounded
  corner will still be round, but ``not as nice.''

  Second, if there are very short line segements in a path, the
  ``rounding'' may cause inadverted effects. In such case it may be
  necessary to temporarily switch off the rounding using
  |sharp corners|. 

  \itemoption{sharp corners}
  This options switches off any rounding on subsequent corners of the
  path.   
\end{itemize}



\subsection{The Rectangle Operation}

A rectangle can obviously be created using four straight lines and a
cycle operation. However, since rectangles are needed so often, a
special syntax is available for them. When the rectangle operation is
used, one corner will be the current point, another corner will be
given by the specified point. The specified point becomes the new
current point.

Syntax of the rectanlge operation: \declare{|rectangle|\meta{corner}}. You
can add a space.

\begin{codeexample}[]
\begin{tikzpicture}
  \draw (0,0) rectangle (1,1);
  \draw (.5,1) rectangle (2,0.5) (3,0) rectangle (3.5,1.5) -- (2,0);
\end{tikzpicture}
\end{codeexample}


\subsection{The Circle and Ellipse Operations}

A circle can be approximated well using four Bezi�r curves. However,
it is difficult to do so correctly. For this reason, a special syntax
is available for adding such an approximation of a circle to the
current path. The center of the circle is given by the current
point. The new current point of the path will remain to be the center
of the circle. 

Syntax of the circle operation:
\declare{|circle(|\meta{radius}|)|}. You can add spaces.

Syntax of the ellipse operation:
\declare{|ellipse(|\meta{half width}| and |\meta{half height}|)|}. You
can add spaces after |ellipse| and you have to place spaces around |and|.  

\begin{codeexample}[]
\begin{tikzpicture}
  \draw (1,0) circle (.5cm);
  \draw (3,0) ellipse (1cm and .5cm) -- ++(3,0) circle (.5cm)
    -- ++(2,-.5) circle (.25cm);
\end{tikzpicture}
\end{codeexample}


\subsection{The Arc Operation}

The \emph{arc operation} allows you to add an arc to the current
path. You provide a start angle, an end angle, and a radius. The arc
operation will then add a part of a circle of the given radius between
the given angles. The arc will start at the current point and will end
at the end of the arc.

Syntax for the arc operation: \declare{|arc(|\meta{start
    angle}|:|\meta{end
    angle}|:|\meta{radius}\opt{|/|\meta{half height}}|)|}.

\begin{codeexample}[]
\begin{tikzpicture}
  \draw (0,0) arc (180:90:1cm) -- (2,.5) arc (90:0:1cm);
  \draw (4,0) -- +(30:1cm) arc (30:60:1cm) -- cycle;
  \draw (8,0) arc (0:270:1cm/.5cm) -- cycle;
\end{tikzpicture}
\end{codeexample}

\begin{codeexample}[]
\begin{tikzpicture}
  \draw (-1,0) -- +(3.5,0);
  \draw (1,0) ++(210:2cm) -- +(30:4cm);
  \draw (1,0) +(0:1cm) arc (0:30:1cm);      
  \draw (1,0) +(180:1cm) arc (180:210:1cm);
  \path (1,0) ++(15:.75cm) node{$\alpha$};
  \path (1,0) ++(15:-.75cm) node{$\beta$};
\end{tikzpicture}
\end{codeexample}


\subsection{The Grid Operation}

You can add a grid to the current path using the |grid| path
operation. 

Syntax for the grid operation: \declare{|grid|\meta{corner
    coordinate}}.

The effect of the grid operation is the following: It will draw a grid
filling a rectangle whose two corners are given by \meta{corner
  coordinate} and by the previous coordinate. Thus, the typical way in
which a grid is drawn is |\draw (1,1) grid (3,3);|, which yields a
grid filling the rectangle whose corners are at $(1,1)$ and $(3,3)$.
All coordinate transformations apply to the grid.

\begin{codeexample}[]
\tikz[rotate=30] \draw[step=1mm] (0,0) grid (2,2);
\end{codeexample}

The stepping of the grid is governed by the following options:

\begin{itemize}
  \itemoption{step}|=|\meta{dimension} sets the stepping in both the
  $x$ and $y$-direction.
  \itemoption{xstep}|=|\meta{dimesion} sets the stepping in the
  $x$-direction. 
  \itemoption{ystep}|=|\meta{dimesion} sets the stepping in the
  $y$-direction. 
\end{itemize}

It is important to note that the grid is always ``phased'' such that
it contains the point $(0,0)$ if that point happens to be inside the
rectangle. Thus, the grid does \emph{not} always have an intersection
at the corner points; this occurs only if the corner points are
multiples of the stepping. Note that due to rounding errors, the
``last'' lines of a grid may be ommitted. In this case, you have to
add an epsilon to the corner points.

The following styles are useful for drawing grids:
\begin{itemize}
  \itemstyle{help lines}
  This style makes lines ``subdued''by using thin gray lines for
  them. However, this style is not installed automatically and you
  have to say for example:
\begin{codeexample}[]
\tikz \draw[style=help lines] (0,0) grid (3,3);
\end{codeexample}
\end{itemize}


\subsection{The Parabola Operation}

The |parabola| path command continues the current path with a
parabola. A parabola is a (shifted and scaled) curve defined by the
equation $f(x) = x^2$ and looks like this: \tikz \draw (-1ex,1.5ex)
parabola (1ex,0ex);.

The basic syntax of the parabola command is
\declare{|parabola|\meta{coordinate}}. This will draw a parabola that
``fills'' the rectangle whose corners are the old current point and
\meta{coordinate}. Here is an example:

\begin{codeexample}[]
\tikz \draw (0,0) rectangle (1,1)  (0,0) parabola (1,1);
\end{codeexample}

In detail, the parabola has the following properties: The old current
point lies in one corner of the rectangle and the parabola ``goes
through'' this point. The parabola will \emph{not} end at the given
\meta{coordinate}. Rather, it will have its bend on the 
$y$-coordinate of \meta{coordinate}. It will end on the $x$-coordinate
of \meta{coordinate} and the $y$-coordinate of the old current point.

Sometimes you may want to draw only a ``part'' of a parabola. For
this, you can  use |parabola left| and |parabola right|.

The syntax of the first command is
\declare{|parabola left|\meta{coordinate}}. This will draw the ``left
part'' of a parabola. More precisely, a parabola is drawn that goes
through the cold current point and has its bend at
\meta{coordinate}. Here is an example:
\tikz \draw (-1ex,1.5ex) parabola left (0,0);, which is obtained through the command
|\tikz \draw (-1ex,1.5ex) parabola left (0,0);|.

Similarly, there exists a \declare{|parabola right|\meta{coordinate}}
command. Here, the parabola has its bend at the old current point and
goes through the new \meta{coordinate}.

You can use |parabola left| and |parabola right| to draw parts of
parabolas. However, it is not possible to draw only a part of a
parabola that does not contain the bend. For this you need to use a
|plot| command.

\begin{codeexample}[]
\tikz \draw[scale=0.25] (-1,1) parabola left (0,0) parabola right (2,4);
\end{codeexample}

\begin{codeexample}[]
\tikz \draw[rotate=-90] (-1.5,2.25) node[right]{$\sqrt x$} parabola left(0,0);
\end{codeexample}

\subsection{The Sine and Cosine Operation}

The |sin| and |cos| operations are similar to the |parabola|
command. They, too, can be used to draw (parts of) a sine or cosine
curve.

Syntax: \declare{|sin|\meta{coordiante}} and
\declare{|cos|\meta{coordinate}}.

The effect of |sin| is to draw a scaled and shifted version of a sine
curve in the interval $[0,\pi/2]$. The scaling and shifting is done in
such a way that the start of the sine curve in the interval is at the
old current point and that the end of the curve in the interval is at
\meta{coordinate}. Here is an example that should clarify this:

\begin{codeexample}[]
\tikz \draw (0,0) rectangle (1,1)     (0,0) sin (1,1)
           (2,0) rectangle +(1.57,1) (2,0) sin +(1.57,1);
\end{codeexample}

The |cos| operation works similarly, only a cosine in the interval
$[0,\pi/2]$ is drawn. By correctly alternating |sin| and |cos|
operations, you can create a complete sine or cosine curve:

\begin{codeexample}[]
\begin{tikzpicture}[xscale=1.57]
  \draw (0,0) sin (1,1) cos (2,0) sin (3,-1) cos (4,0) sin (5,1);
  \draw[color=red] (0,1.5) cos (1,0) sin (2,-1.5) cos (3,0) sin (4,1.5) cos (5,0);
\end{tikzpicture}
\end{codeexample}


Note that there is no way to (conveniently) draw an interval on a sine
or cosine curve whose end points are not multiples of $\pi/2$ (or $90^\circ$).



\subsection{The Plot Operation}

The |plot| operation can be used to append a line or curve to the path
that goes through a large numbor of coordiantes. These coordinates are
either given in a simple list of coordinates or they are read from
some file.

The syntax of the |plot| comes in different versions. First, the
``beginning'' of a |plot| command can be
\begin{enumerate}
\item
  \declare{|plot|} or
\item
  \declare{|--plot|} (space may be added for clarity)
\end{enumerate}

The difference between the first and the second case is that |plot|
will start plotting at the first coordinate by ``moving'' to that
point. The |--plot| will instead perform a |--| (line-to) operation to
the first point of the plot.

Next, there are three ways of specifying the coordinates of the points
to be plotted:

\begin{enumerate}
\item
  \opt{|--|}|plot|\oarg{local options}\declare{|coordinates{|\meta{coordinate
    1}\meta{coordinate 2}\dots\meta{coordinate $n$}|}|}
\item
  \opt{|--|}|plot|\oarg{local options}\declare{|file{|\meta{filename}|}|}
\item
  \opt{|--|}|plot|\oarg{local options}\declare{|function{|\meta{gnuplot formula}|}|}
\end{enumerate}

These different ways are explained in the following.


\subsubsection{Plotting Points Given Inline}

In the first two cases, the points are given directly in the \TeX-file
as in the following example:

\begin{codeexample}[]
\tikz \draw plot coordinates {(0,0) (1,1) (2,0) (3,1) (2,1) (10:2cm)};
\end{codeexample}

Here is an example showing the difference between |plot| and |--plot|:

\begin{codeexample}[]
\begin{tikzpicture}
  \draw (0,0) -- (1,1) plot coordinates {(2,0)  (4,0)};
  \draw[color=red,xshift=5cm]
        (0,0) -- (1,1) -- plot coordinates {(2,0)  (4,0)};
\end{tikzpicture}
\end{codeexample}


\subsubsection{Plotting Points Read From an External File}

In the third and fourth form the points reside in an external
file named \meta{filename}. Currently, the only file format that
\tikzname\ allows is the following: Each line of the \meta{filename}
should contain one line starting with two numbers, separated by a
space. Anything following the two numbers on the line is
ignored. Also, lines starting with a |%| or a |#| are ignored as well
as empty lines. (This is exactly the format that \textsc{gnuplot}
produces when you say |set terminal table|.) If necessary, more
formats will be supported in the future, but it is usually easy to
produce a file containing data in this form.

\begin{codeexample}[]
\tikz \draw plot[mark=x,smooth] file {plots/pgfmanual-sine.table};
\end{codeexample}

The file |plots/pgfmanual-sine.table| reads:
\begin{codeexample}[code only]
#Curve 0, 20 points
#x y type
0.00000 0.00000  i
0.52632 0.50235  i
1.05263 0.86873  i
1.57895 0.99997  i
...
9.47368 -0.04889  i
10.00000 -0.54402  i
\end{codeexample}
It was produced from the following source, using |gnuplot|:
\begin{codeexample}[code only]
set terminal table
set output "../plots/pgfmanual-sine.table"
set format "%.5f"
set samples 20
plot [x=0:10] sin(x)
\end{codeexample}

The \meta{local options} of the |plot| command are local to each
plot and do not affect other plots ``on the same path.'' For example,
|plot[yshift=1cm]| will locally shift the plot 1cm upward. Remember,
however, that most options can only be applied to paths as a
whole. For example, |plot[red]| does not have the effect of making the
plot red. After all, you are trying to ``locally'' make part of the
path red, which is not possible.

\subsubsection{Plotting a Function}
\label{section-tikz-gnuplot}

Often, you will want to plot points that are given via a function like
$f(x) = x \sin x$. Unfortunately, \TeX\ does not really have enough
computational power to generate the points on such a function
efficiently (it is a text processing program, after all). However,
if you allow it, \TeX\ can try to call external programs that can
easily produce the necessary points. Currently, \pgfname\ knows how to
call \textsc{gnuplot}.

When \tikzname\ encounters your command
|plot[id=|\meta{id}|] function{x*sin(x)}| for 
the first time, it will create a file called
\meta{prefix}\meta{id}|.gnuplot|, where \meta{prefix} is |\jobname.| by
default, that is, the name of you main |.tex| file. If no \meta{id} is
given, it will be empty, which is allright, but it is better when each
plot has a unique \meta{id} for reasons explained in a moment. Next,
\tikzname\ writes some initialization code into this file followed by
|plot x*sin(x)|. The initialization code sets up things 
such that the |plot| command will write the coordinates into another
file called \meta{prefix}\meta{id}|.table|. Finally, this table file
is read as if you had said |plot file{|\meta{prefix}\meta{id}|.table}|. 

For the plotting mechansim to work, two conditions must be met:
\begin{enumerate}
\item
  You must have allowed \TeX\ to call external programs. This is often
  switched off by default since this is a securtiy risk (you might,
  without knowing, run a \TeX\ file that calls all sorts of ``bad''
  commands). To enable this ``calling external programs'' a command
  line option must be given to the \TeX\ program. Usually, it is
  called something like |shell-escape| or |enable-write18|. For
  example, for my |pdflatex| the option |--shell-escape| can be
  given.
\item
  You must have installed the |gnuplot| program and \TeX\ must find it
  when compiling your file.
\end{enumerate}

Unfortunately, these conditions will not always be met. Especially if
you pass some source to a coauthor and the coauthor does not have
\textsc{gnuplot} installed, he or she will have trouble compiling your
files.

For this reason, \tikzname\ behaves differently when you compile your
graphic for the second time: If upon reaching
|plot[id=|\meta{id}|] function{...}| the file \meta{prefix}\meta{id}|.table|
already exists \emph{and} if the \meta{prefix}\meta{id}|.gnuplot| file
contains what \tikzname\ thinks that it ``should'' contain, the |.table|
file is immediately read without trying to call a |gnuplot|
program. This approach has the following advantages: 
\begin{enumerate}
\item
  If you pass a bundle of your |.tex| file and all |.gnuplot| and
  |.table| files to someone else, that person can \TeX\ the |.tex|
  file without having to have |gnuplot| installed.
\item
  If the |\write18| feature is switched off for security reasons (a
  good idea), then, upon the first compilation of the |.tex| file, the
  |.gnuplot| will still be generated, but not the |.table|
  file. You can then simply call |gnuplot| ``by hand'' for each
  |.gnuplot| file, which will produce all necessary |.table| files.
\item
  If you change the function that you wish to plot or its
  domain, \tikzname\ will automatically try to regenerate the |.table|
  file.
\item
  If, out of laziness, you do not provide an |id|, the same |.gnuplot|
  will be used for different plots, but this is not a problem since
  the |.table| will automatically be regenerated for each plot
  on-the-fly. \emph{Note: If you intend to share your files with
  someone else, always use an id, so that the file can by typset
  without having \textsc{gnuplot} installed.} Also, having unique ids
  for each plot will improve compilation speed since no external
  programs need to be called, unless it is really necessary.
\end{enumerate}

When you use |plot function{|\meta{gnuplot formula}|}|, the \meta{gnuplot
  formula} must be given in the |gnuplot| syntax, whose details are
beyond the scope of this manual. Here is the ultra-condensed
essence: Use |x| as the variable and use the C-syntax for normal
plots, use |t| as the variable for parametric plots. Here are some examples:

\begin{codeexample}[]
\begin{tikzpicture}[domain=0:4]
  \draw[very thin,color=gray] (-0.1,-1.1) grid (3.9,3.9);
  
  \draw[->] (-0.2,0) -- (4.2,0) node[right] {$x$};
  \draw[->] (0,-1.2) -- (0,4.2) node[above] {$f(x)$};
  
  \draw[color=red]    plot[id=x]   function{x}           node[right] {$f(x) =x$};
  \draw[color=blue]   plot[id=sin] function{sin(x)}      node[right] {$f(x) = \sin x$};
  \draw[color=orange] plot[id=exp] function{0.05*exp(x)} node[right] {$f(x) = \frac{1}{20} \mathrm e^x$};
\end{tikzpicture}
\end{codeexample}


The following options influence the plot:

\begin{itemize}
  \itemoption{samples}|=|\meta{number}
  sets the number of samples used in the plot. The default is 25.
  \itemoption{domain}|=|\meta{start}|:|\meta{end}
  sets the domain between which the samples are taken. The default is
  |-5:5|. 
  \itemoption{parametric}\opt{|=|\meta{true or false}}
  sets whether the plot is a parameteric plot. If true, then |t| must
  be used instead of |x| as the parameter and two comma-separated
  functions must be given in the \meta{gnuplot formula}. An example is
  the following:
\begin{codeexample}[]
\tikz \draw[scale=0.5,domain=-3.141:3.141,smooth]
  plot[parametric,id=parametric-example] function{t*sin(t),t*cos(t)};
\end{codeexample}
  
  \itemoption{id}|=|\meta{id}
  sets the identifier of the current plot. This should be a unique
  identifier for each plot (though things will also work if it is not,
  but not as well, see the explanations above). The \meta{id} will be
  part of a filename, so it should not contain anything fancy like |*|
  or |$|.%$
  \itemoption{prefix}|=|\meta{prefix}
  is put before each plot file name. The default is |\jobname.|, but
  if you have many plots, it might be better to use, say |plots/| and
  have all plots placed in a directory. You have to create the
  director yourself.
  \itemoption{raw gnuplot}
  causes the \meta{gnuplot formula} to be passed on to
  \textsc{gnuplot} without setting up the samples or the |plot|
  command. Thus, you could write
\begin{codeexample}[code only]
plot[raw gnuplot,id=raw-example] function{set samples 25; plot sin(x)}
\end{codeexample}
  This can be 
  useful for complicated things that need to be passed to
  \textsc{gnuplot}. However, for really complicated situations you
  should create a special external generating \textsc{gnuplot} file
  and use the |file|-syntax to include the table ``by hand.''
\end{itemize}

The following styles influence the plot:
\begin{itemize}
  \itemstyle{every plot}
  This style is installed in each plot, that is, as if you always said
\begin{codeexample}[code only]
  plot[style=every plot,...]
\end{codeexample}
 This is most useful for globally setting a prefix for all plots by saying:
\begin{codeexample}[code only]
\tikzstyle{every plot}=[prefix=plots/]
\end{codeexample}
\end{itemize}



\subsubsection{Placing Marks on the Plot}

As we saw already, it is possible to add \emph{marks} to a plot using
the |mark| option. When this option is used, a copy of the plot
mark is placed on each point of the plot. Note that the marks are
placed \emph{after} the whole path has been drawn/filled/shaded. In
this respect, they are handled like text nodes. 

In detail, the following options govern how marks are drawn:
\begin{itemize}
  \itemoption{mark}|=|\meta{mark mnemonic}
  Sets the mark to a mnemonic that has previously been defined using
  the |\newpgfplotmark|. By default, |*|, |+|, and |x| are available,
  which draw a filled circle, a plus, and a cross as marks. Many more
  marks become available when the library |pgflibraryplotmarks| is
  loaded. Section~\ref{section-plot-marks} lists the available plot
  marks.

  One plot mark is special: the |ball| plot mark is available only
  it \tikzname. The |ball color| determines the balls's color. Do not use
  this option with large number of marks since it will take very long
  to render in PostScript.
  
  \begin{tabular}{lc}
    Option & Effect \\\hline \vrule height14pt width0pt
    \plotmarkentrytikz{ball}
  \end{tabular}
  
  \itemoption{mark size}|=|\meta{dimension}
  Sets the size of the plot marks. For circular plot marks,
  \meta{dimension} is the radius, for other plot marks
  \meta{dimension} should be about half the width and height.

  This option is not really necessary, since you achieve the same
  effect by specifying |scale=|\meta{factor} as a local option, where
  \meta{factor} is the quotient of the desired size and the default
  size. However, using |mark size| is a bit faster and more natural. 

  \itemoption{mark options}|=|\meta{options}
  These options are applied to marks when they are drawn. For example,
  you can scale (or otherwise transform) the plot mark or set its
  color. 
\begin{codeexample}[]
\tikz \fill[fill=blue!20]
  plot[mark=triangle*,mark options={color=blue,rotate=180}]
    file{plots/pgfmanual-sine.table} |- (0,0);
\end{codeexample}
\end{itemize}



\subsubsection{Smooth Plots, Sharp Plots, and Comb Plots}

There are different things the |plot| command can do with the points
it reads from a file or from the inlined list of points. By default,
it will connect these points by straight lines. However, you can also
use options to change the behaviour of |plot|.

\begin{itemize}
  \itemoption{sharp plot}
  This is the default and causes the points to be connected by
  straight lines. This option is included only so that you can
  ``switch back'' if you ``globally'' install, say, |smooth|.
  
  \itemoption{smooth}
  This option causes the points on the path to be connected using a
  smooth curve:

\begin{codeexample}[]
\tikz\draw plot[smooth] file{plots/pgfmanual-sine.table};
\end{codeexample}

  Note that the smoothing algorithm is not very intelligent. You will
  get the best results if the bending angles are small, that is, less
  than about $30^\circ$ and, even more importantly, if the distances
  between points are about the same all over the plotting path.

  \itemoption{tension}|=|\meta{value}
  This option influences how ``tight'' the smoothing is. A lower value
  will result in sharper corners, a higher value in more ``round''
  curves. The default is $0.15$. The ``correct'' value depends on the
  details of plot.
  
\begin{codeexample}[]
\begin{tikzpicture}[smooth cycle]
  \draw                 plot[tension=0.1]
    coordinates{(0,0) (1,1) (2,0) (1,-1)};
  \draw[yshift=-2.25cm] plot[tension=0.2]
    coordinates{(0,0) (1,1) (2,0) (1,-1)};
  \draw[yshift=-4.5cm]  plot[tension=0.275]
    coordinates{(0,0) (1,1) (2,0) (1,-1)};
\end{tikzpicture}
\end{codeexample}
  
  \itemoption{smooth cycle}
  This option causes the points on the path to be connected using a
  closed smooth curve. 

\begin{codeexample}[]
\tikz[scale=0.5]
  \draw plot[smooth cycle] coordinates{(0,0) (1,0) (2,1) (1,2)}
        plot               coordinates{(0,0) (1,0) (2,1) (1,2)} -- cycle;
\end{codeexample}

  \itemoption{ycomb}
  This option causes the |plot| command to interpret the plotting
  points differently. Instead of connecting them, for each point of
  the plot a straight line is added to the path from the $x$-axis to the point,
  resulting in a sort of ``comb'' or ``bar diagram.''

\begin{codeexample}[]
\tikz\draw[ultra thick] plot[ycomb,thin,mark=*] file{plots/pgfmanual-sine.table};
\end{codeexample}

\begin{codeexample}[]
\begin{tikzpicture}[ycomb]
  \draw[color=red,line width=6pt]
    plot coordinates{(0,1) (.5,1.2) (1,.6) (1.5,.7) (2,.9)};
  \draw[color=red!50,line width=4pt,xshift=3pt]
    plot coordinates{(0,1.2) (.5,1.3) (1,.5) (1.5,.2) (2,.5)};
\end{tikzpicture}
\end{codeexample}

  \itemoption{xcomb}
  This option works like |ycomb| except that the bars are horizontal. 

\begin{codeexample}[]
\tikz \draw plot[xcomb,mark=x] coordinates{(1,0) (0.8,0.2) (0.6,0.4) (0.2,1)};
\end{codeexample}

  \itemoption{polar comb}
  This option causes a line from the origin to the point to be added
  to the path for each plot point.

\begin{codeexample}[]
\tikz \draw plot[polar comb,
     mark=pentagon*,mark options={fill=white,draw=red},mark size=4pt]
   coordinates {(0:1cm) (30:1.5cm) (160:.5cm) (250:2cm) (-60:.8cm)};
\end{codeexample}


  \itemoption{only marks}
  This option causes only marks to be shown; no path segments are
  added to the actual path. This can be useful for quickly adding some
  marks to a path.

\begin{codeexample}[]
\tikz \draw (0,0) sin (1,1) cos (2,0)
  plot[only marks,mark=x] coordinates{(0,0) (1,1) (2,0) (3,-1)};
\end{codeexample}
\end{itemize}



  

\subsection{The Scoping Operation}

When \tikzname\ encounters and opening or a closing brace (|{| or~|}|) at
some point where a path operation should come, it will open or close a
scope. All options that can be applied ``locally'' will be scoped
inside the scope. For example, if you apply a transformation like
|[xshift=1cm]| inside the scoped area, the shifting only applies to
the scope. On the other hand, a command like |color=red| does not have
any effect inside a scope since it can only be applied to the path as
a whole. 


\subsection{The Node Operation}

You can add nodes to a path using the |node| operation. Since this
operation is quite complex and since the nodes are not really part of
the path itself, there is a separate section dealing with nodes, see
Section~\ref{section-nodes}. 

\include{pgfmanual-tikz-actions}
\include{pgfmanual-tikz-shapes}
% Copyright 2005 by Till Tantau <tantau@cs.tu-berlin.de>.
%
% This program can be redistributed and/or modified under the terms
% of the LaTeX Project Public License Distributed from CTAN
% archives in directory macros/latex/base/lppl.txt.


\section{Watching Trees Grow}

\label{section-trees}


\subsection{Introduction to the  Child Operation}

\emph{Trees} are a common way of visualizing hierarchical
structures. A simple tree looks like this:
\begin{codeexample}[]
\begin{tikzpicture}
  \node {root}
    child {node {left}}
    child {node {right}
      child {node {child}}
      child {node {child}}
    };
\end{tikzpicture}
\end{codeexample}

Admittedly, in reality trees are more likely to grow \emph{upward} and
not downward as above. This is easy enough to specify in \tikzname:

\begin{codeexample}[]
\begin{tikzpicture}
  \node {root} [grow'=up]
    child {node {left}}
    child {node {right}
      child {node {child}}
      child {node {child}}
    };
\end{tikzpicture}
\end{codeexample}

(You can tell whether the author of a paper is a mathematician or a
computer scientist by looking at the direction their trees grow. A
computer scientist's trees will grow downward while a mathematician's
tree will upward. Naturally, the correct way is the mathematician's
way.)

In \tikzname, trees are specified by adding \emph{child nodes} to a
node on a path. The syntax for the child operation is the following:

\begin{pathoperation}{child}{\opt{\oarg{options}}\opt{\marg{child path}}}
  This operation should directly follow a completed |node| operation
  or another |child| operation, although it is permissible that the
  first |child| operation is preceded by options (we will come to
  that).

  The exact effects of this operation are described in the rest of
  this present section.
\end{pathoperation}





\subsection{Where Children and Their Options Are Specified}

When a |node| operation like |node {X}| is followed by |child|,
\tikzname\ starts counting the number of child nodes that follow the
original |node {X}|. For this, it scans the input and stores away each
|child| and its arguments until it reaches a path operation that is
not a |child|. Note that this will fix the character codes or any
text inside the child arguments, which means, in essence, that you
cannot use verbatim text inside the nodes inside a |child|. Sorry. 

Once the children have been collected and counted, \tikzname\ starts
generating the nodes of the children.

Each |child| may have its own \meta{options}, which apply to ``the
whole child,'' including all of its grandchildren. Here is an
example:

\begin{codeexample}[]
\begin{tikzpicture}[thick,sibling distance=10mm on level 2]
  \coordinate
    child[red]   {child child}
    child[green] {child child[blue]};
\end{tikzpicture}
\end{codeexample}

The options of the root node have no effect on the children since
the options of a node are always ``local'' to that node. Because of
this, the edges in the following tree are black, not red.
  
\begin{codeexample}[]
\begin{tikzpicture}[thick]
  \node [red] {root}
    child
    child;
\end{tikzpicture}
\end{codeexample}
  This raises the problem of how to set options for \emph{all}
  children. Naturally, you could always set options for the whole path
  as in |\path [red] node {root} child child;| but this is bothersome
  in some situations. Instead, it is easier to give the options
  \emph{before the first child} as follows:
\begin{codeexample}[]
\begin{tikzpicture}[thick]
  \node [red] {root}
    [green] % option applies to all children
    child
    child;
\end{tikzpicture}
\end{codeexample}

To sum up: Options for the whole tree are given before the root
node. Options for the root node are given directly to the |node|
operation of the root. Options for all children can be given between
the root node and the first child. Options applying to a specific
child are given as options to the child.


\subsection{The Shapes and Content Child Nodes}

For each |child| of a root node, a node is generated and placed
somewhere (the placement rules will be discussed later). The shape of
the child node depends on the \meta{child path} of the child.

In the easiest case, the \meta{child path} is completely missing
(including the curly braces). An example would be
|\node {x} child child;| where both children miss their \meta{child
  path}. In this case the shape is simply a |coordinate|. Thus, the
child node has no extend and no text. 
\begin{codeexample}[]
\tikz \node {root} child child;
\end{codeexample}

Next, the \meta{child path} may \emph{start} with a |node| or a
|coordinate| specification. An example is
|\node {x} child {node {y}};| where the \meta{child path} consists
of the node specification  |node {y}|. In this case, this first node
on the path becomes the child node. As for any normal node, you can
give this child node a name, shift it around, or use options to
influence how it is rendered.
\begin{codeexample}[]
\begin{tikzpicture}
  \node[rectangle,draw] {root}
    child {node[circle,draw] (left node) {left}}
    child {node[ellipse,draw] (right node) {right}};
  \draw[dashed,->] (left node) -- (right node);
\end{tikzpicture}
\end{codeexample}

A third case occurs when the \meta{child path} exists, but does not
start with a |node| or |coordinate| as in
|child {child};|, where the \meta{child path} start with |child|
itself. In this case, a node of shape |coordinate| is automatically
added at the beginning of the path. 

\subsection{The Placement of Child Nodes}

\subsection{The Edge From the Parent Node}





%%% Local Variables: 
%%% mode: latex
%%% TeX-master: "pgfmanual"
%%% End: 

\include{pgfmanual-tikz-transformations}



\part{Libraries and Utilities}
\label{part-libraries}

In this part the library and utility packages are documented. The
library packages provide additional predefined graphic objects like
new arrow heads, or new plot marks. These are not loaded by default
since many users will not need them.

The utility packages are not directly involved in creating graphics,
but you may find them useful nonetheless. All of them either directly
depend on \pgfname\ or they are designed to work well together with
\pgfname\ even though they can be used in a stand-alone way.
\vskip2cm
\medskip
\noindent
\begin{codeexample}[graphic=white]
\begin{tikzpicture}[scale=2]
  \shade[top color=blue,bottom color=gray!50] (0,0) parabola (1.5,2.25) |- (0,0);
  \draw (1.05cm,2pt) node[above] {$\displaystyle\int_0^{3/2} \!\!x^2\mathrm{d}x$};
  
  \draw[style=help lines] (0,0) grid (3.9,3.9)
       [step=0.25cm]      (1,2) grid +(1,1);

  \draw[->] (-0.2,0) -- (4,0) node[right] {$x$};
  \draw[->] (0,-0.2) -- (0,4) node[above] {$f(x)$};

  \foreach \x/\xtext in {1/1, 1.5/1\frac{1}{2}, 2/2, 3/3}
    \draw[shift={(\x,0)}] (0pt,2pt) -- (0pt,-2pt) node[below] {$\xtext$};

  \foreach \y/\ytext in {1/1, 2/2, 2.25/2\frac{1}{4}, 3/3}
    \draw[shift={(0,\y)}] (2pt,0pt) -- (-2pt,0pt) node[left] {$\ytext$};
    
  \draw (-.5,.25) parabola bend (0,0) (2,4) node[below right] {$x^2$};
\end{tikzpicture}
\end{codeexample}

% Copyright 2003 by Till Tantau <tantau@cs.tu-berlin.de>.
%
% This program can be redistributed and/or modified under the terms
% of the LaTeX Project Public License Distributed from CTAN
% archives in directory macros/latex/base/lppl.txt.


\section{Libraries}

\subsection{Arrow Tip Library}
\label{section-library-arrows}

\begin{package}{pgflibraryarrows}
  The package defines additional arrow tips, which are described
  below. See page~\pageref{standard-arrows} for the arrows tips that
  are defined by default. Note that neither the standard packages nor
  this package defines an arrow name containing |>| or |<|. These are
  left for the user to defined as he or she sees fit.
\end{package}

\subsubsection{Triangular Arrow Tips}

\begin{tabular}{ll}
  \symarrow{latex'} \\
  \symarrow{latex' reversed}  \\
  \symarrow{stealth'} \\
  \symarrow{stealth' reversed}\\
  \symarrow{triangle 90} \\
  \symarrow{triangle 90 reversed}   \\
  \symarrow{triangle 60} \\
  \symarrow{triangle 60 reversed}   \\
  \symarrow{triangle 45} \\
  \symarrow{triangle 45 reversed}   \\
  \symarrow{open triangle 90} \\
  \symarrow{open triangle 90 reversed}   \\
  \symarrow{open triangle 60} \\
  \symarrow{open triangle 60 reversed}   \\
  \symarrow{open triangle 45} \\
  \symarrow{open triangle 45 reversed}   \\
\end{tabular}

\subsubsection{Barbed Arrow Tips}

\begin{tabular}{ll}
  \symarrow{angle 90} \\
  \symarrow{angle 90 reversed}   \\
  \symarrow{angle 60} \\
  \symarrow{angle 60 reversed}   \\
  \symarrow{angle 45} \\
  \symarrow{angle 45 reversed}   \\
  \symarrow{hooks} \\
  \symarrow{hooks reversed} \\
\end{tabular}


\subsubsection{Bracket-Like Arrow Tips}

\begin{tabular}{ll}
  \sarrow{[}{]} \\
  \sarrow{]}{[} \\
  \sarrow{(}{)} \\
  \sarrow{)}{(}
\end{tabular}

\subsubsection{Circle and Diamond Arrow Tips}


\begin{tabular}{ll}
  \symarrow{o} \\
  \symarrow{*} \\
  \symarrow{diamond} \\
  \symarrow{open diamond}   \\
\end{tabular}


\subsubsection{Partial Arrow Tips}

\begin{tabular}{ll}
  \symarrow{left to} \\
  \symarrow{left to reversed} \\
  \symarrow{right to} \\
  \symarrow{right to reversed} \\
  \symarrow{left hook} \\
  \symarrow{left hook reversed} \\
  \symarrow{right hook} \\
  \symarrow{right hook reversed}
\end{tabular}


\subsubsection{Line Caps}

\begin{tabular}{ll}
  \carrow{round cap} \\
  \carrow{butt cap} \\
  \carrow{triangle 90 cap} \\
  \carrow{triangle 90 cap reversed} \\
  \carrow{fast cap} \\
  \carrow{fast cap reversed} \\
\end{tabular}



\subsection{Plot Handler Library}
\label{section-library-plothandlers}

\begin{package}{pgflibraryplothandlers}
  This library packages defines additional plot handlers, see
  Section~\ref{section-plot-handlers} for an introduction to plot
  handlers. The additional handlers are described in the following. 
\end{package}


\subsubsection{Curve Plot Handlers}
  
\begin{command}{\pgfplothandlercurveto}
  This handler will issue a |\pgfpathcurveto| command for each point of
  the plot, \emph{except} possibly for the first. As for the line-to
  handler, what happens with the first point can be specified using
  |\pgfsetmovetofirstplotpoint| or |\pgfsetlinetofirstplotpoint|.

  Obviously, the |\pgfpathcurveto| command needs, in addition to the
  points on the path, some control points. These are generated
  automatically using a somewhat ``dumb'' algorithm: Suppose you have
  three points $x$, $y$, and $z$ on the curve such that $y$ is between
  $x$ and $z$:
\begin{codeexample}[]
\begin{tikzpicture}    
  \draw[gray] (0,0) node {x} (1,1) node {y} (2,.5) node {z};
  \pgfplothandlercurveto
  \pgfplotstreamstart
  \pgfplotstreampoint{\pgfpoint{0cm}{0cm}}
  \pgfplotstreampoint{\pgfpoint{1cm}{1cm}}
  \pgfplotstreampoint{\pgfpoint{2cm}{.5cm}}
  \pgfplotstreamend
  \pgfusepath{stroke}
\end{tikzpicture}
\end{codeexample}

  In order to determine the control points of the curve at the point
  $y$, the handler computes the vector $z-x$ and scales it by the
  tension factor (see below). Let us call the resulting vector
  $s$. Then $y+s$ and $y-s$ will be the control points around $y$. The
  first control point at the beginning of the curve will be the
  beginning itself, once more; likewise the last control point is the
  end itself.
\end{command}

\begin{command}{\pgfsetplottension\marg{value}}
  Sets the factor used by the curve plot handlers to determine the
  distance of the control points from the points they control. The
  default is $0.15$. The higher the curvature of the curve points, the
  higher this value should be.

\begin{codeexample}[]
\begin{tikzpicture}    
  \draw[gray] (0,0) node {x} (1,1) node {y} (2,.5) node {z};
  \pgfsetplottension{0.75}
  \pgfplothandlercurveto
  \pgfplotstreamstart
  \pgfplotstreampoint{\pgfpoint{0cm}{0cm}}
  \pgfplotstreampoint{\pgfpoint{1cm}{1cm}}
  \pgfplotstreampoint{\pgfpoint{2cm}{0.5cm}}
  \pgfplotstreamend
  \pgfusepath{stroke}
\end{tikzpicture}
\end{codeexample}
\end{command}


\begin{command}{\pgfplothandlerclosedcurve}
  This handler works like the curve-to plot handler, only it will
  add a new part to the current path that is a closed curve through
  the plot points.
\begin{codeexample}[]
\begin{tikzpicture}    
  \draw[gray] (0,0) node {x} (1,1) node {y} (2,.5) node {z};
  \pgfplothandlerclosedcurve
  \pgfplotstreamstart
  \pgfplotstreampoint{\pgfpoint{0cm}{0cm}}
  \pgfplotstreampoint{\pgfpoint{1cm}{1cm}}
  \pgfplotstreampoint{\pgfpoint{2cm}{0.5cm}}
  \pgfplotstreamend
  \pgfusepath{stroke}
\end{tikzpicture}
\end{codeexample}
\end{command}


\subsubsection{Comb Plot Handlers}

There are three ``comb'' plot handlers. There name stems from the fact
that the plots they produce look like ``combs'' (more or less).

\begin{command}{\pgfplothandlerxcomb}
  This handler converts each point in the plot stream into a line from
  the $y$-axis to the point's coordinate, resulting in a ``horizontal
  comb.''

  
\begin{codeexample}[]
\begin{tikzpicture}    
  \draw[gray] (0,0) node {x} (1,1) node {y} (2,.5) node {z};
  \pgfplothandlerxcomb
  \pgfplotstreamstart
  \pgfplotstreampoint{\pgfpoint{0cm}{0cm}}
  \pgfplotstreampoint{\pgfpoint{1cm}{1cm}}
  \pgfplotstreampoint{\pgfpoint{2cm}{0.5cm}}
  \pgfplotstreamend
  \pgfusepath{stroke}
\end{tikzpicture}
\end{codeexample}
\end{command}


\begin{command}{\pgfplothandlerycomb}
  This handler converts each point in the plot stream into a line from
  the $x$-axis to the point's coordinate, resulting in a ``vertical
  comb.''
  
  This handler is useful for creating ``bar diagrams.''
\begin{codeexample}[]
\begin{tikzpicture}    
  \draw[gray] (0,0) node {x} (1,1) node {y} (2,.5) node {z};
  \pgfplothandlerycomb
  \pgfplotstreamstart
  \pgfplotstreampoint{\pgfpoint{0cm}{0cm}}
  \pgfplotstreampoint{\pgfpoint{1cm}{1cm}}
  \pgfplotstreampoint{\pgfpoint{2cm}{0.5cm}}
  \pgfplotstreamend
  \pgfusepath{stroke}
\end{tikzpicture}
\end{codeexample}
\end{command}

\begin{command}{\pgfplothandlerpolarcomb}
  This handler converts each point in the plot stream into a line from
  the origin to the point's coordinate.
  
\begin{codeexample}[]
\begin{tikzpicture}    
  \draw[gray] (0,0) node {x} (1,1) node {y} (2,.5) node {z};
  \pgfplothandlerpolarcomb
  \pgfplotstreamstart
  \pgfplotstreampoint{\pgfpoint{0cm}{0cm}}
  \pgfplotstreampoint{\pgfpoint{1cm}{1cm}}
  \pgfplotstreampoint{\pgfpoint{2cm}{0.5cm}}
  \pgfplotstreamend
  \pgfusepath{stroke}
\end{tikzpicture}
\end{codeexample}
\end{command}

\subsubsection{Mark Plot Handler}

\label{section-plot-marks}

\begin{command}{\pgfplothandlermark\marg{mark code}}
  This command will execute the \meta{mark code} for each point of the
  plot, but each time the coordinate transformation matrix will be
  setup such that the origin is at the position of the point to be
  plotted. This way, if the \meta{mark code} draws a little circle
  around the origin, little circles will be drawn at each point of the
  plot.
  
\begin{codeexample}[]
\begin{tikzpicture}    
  \draw[gray] (0,0) node {x} (1,1) node {y} (2,.5) node {z};
  \pgfplothandlermark{\pgfpathcircle{\pgfpointorigin}{4pt}\pgfusepath{stroke}}
  \pgfplotstreamstart
  \pgfplotstreampoint{\pgfpoint{0cm}{0cm}}
  \pgfplotstreampoint{\pgfpoint{1cm}{1cm}}
  \pgfplotstreampoint{\pgfpoint{2cm}{0.5cm}}
  \pgfplotstreamend
  \pgfusepath{stroke}
\end{tikzpicture}
\end{codeexample}

  Typically, the \meta{code} will be |\pgfuseplotmark{|\meta{plot mark
      name}|}|, where \meta{plot mark name} is the name of a
  predefined plot mark.
\end{command}

\begin{command}{\pgfuseplotmark\marg{plot mark name}}
  Draws the given \meta{plot mark name} at the origin. The \meta{plot
    mark name} must previously have been declared using
  |\pgfdeclareplotmark|. 

\begin{codeexample}[]
\begin{tikzpicture}    
  \draw[gray] (0,0) node {x} (1,1) node {y} (2,.5) node {z};
  \pgfplothandlermark{\pgfuseplotmark{pentagon}}
  \pgfplotstreamstart
  \pgfplotstreampoint{\pgfpoint{0cm}{0cm}}
  \pgfplotstreampoint{\pgfpoint{1cm}{1cm}}
  \pgfplotstreampoint{\pgfpoint{2cm}{0.5cm}}
  \pgfplotstreamend
  \pgfusepath{stroke}
\end{tikzpicture}
\end{codeexample}
\end{command}

\begin{command}{\pgfdeclareplotmark\marg{plot mark name}\marg{code}}
  Declares a plot mark for later used with the |\pgfuseplotmark|
  command.

\begin{codeexample}[]
\pgfdeclareplotmark{my plot mark}
  {\pgfpathcircle{\pgfpoint{0cm}{1ex}}{1ex}\pgfusepathqstroke}  
\begin{tikzpicture}    
  \draw[gray] (0,0) node {x} (1,1) node {y} (2,.5) node {z};
  \pgfplothandlermark{\pgfuseplotmark{my plot mark}}
  \pgfplotstreamstart
  \pgfplotstreampoint{\pgfpoint{0cm}{0cm}}
  \pgfplotstreampoint{\pgfpoint{1cm}{1cm}}
  \pgfplotstreampoint{\pgfpoint{2cm}{0.5cm}}
  \pgfplotstreamend
  \pgfusepath{stroke}
\end{tikzpicture}
\end{codeexample}
\end{command}


\begin{command}{\pgfsetplotmarksize\marg{dimension}}
  This command sets the \TeX\ dimension |\pgfplotmarksize| to
  \meta{dimension}. This dimension is a ``recommendation'' for plot
  mark code at which size the plot mark should be drawn; plot mark
  code may choose to ignore this \meta{dimension} altogether. For
  circles, \meta{dimension} should  be the radius, for other shapes it
  should be about half the width/height.

  The predefined plot marks all take this dimension into account.

\begin{codeexample}[]
\begin{tikzpicture}    
  \draw[gray] (0,0) node {x} (1,1) node {y} (2,.5) node {z};
  \pgfsetplotmarksize{1ex}
  \pgfplothandlermark{\pgfuseplotmark{*}}
  \pgfplotstreamstart
  \pgfplotstreampoint{\pgfpoint{0cm}{0cm}}
  \pgfplotstreampoint{\pgfpoint{1cm}{1cm}}
  \pgfplotstreampoint{\pgfpoint{2cm}{0.5cm}}
  \pgfplotstreamend
  \pgfusepath{stroke}
\end{tikzpicture}
\end{codeexample}
\end{command}

\begin{textoken}{\pgfplotmarksize}
  A \TeX\ dimension that is a ``recommendation'' for the size of plot
  marks.
\end{textoken}

The following plot marks are predefined (the filling color has been
set to yellow):

\medskip
\begin{tabular}{lc}
  \plotmarkentry{*}
  \plotmarkentry{x}
  \plotmarkentry{+}
\end{tabular}


\subsection{Plot Mark Library}

\begin{package}{pgflibraryplotmarks}
  When this package is loaded, the following plot marks are defined in
  addition to |*|, |x|, and |+| (the filling color has been set to a
  dark yellow):

  \catcode`\|=12
  \medskip
  \begin{tabular}{lc}
    \plotmarkentry{-}
    \index{*vbar@\protect\texttt{\protect\myvbar} plot mark}%
    \index{Plot marks!*vbar@\protect\texttt{\protect\myvbar}}
    \texttt{\char`\\pgfuseplotmark\char`\{\declare{|}\char`\}} &
    \tikz\draw[color=black!25] plot[mark=|,mark options={fill=yellow,draw=black}]
    coordinates {(0,0) (.5,0.2) (1,0) (1.5,0.2)};\\
    \plotmarkentry{o}
    \plotmarkentry{asterisk}
    \plotmarkentry{star}
    \plotmarkentry{oplus}
    \plotmarkentry{oplus*}
    \plotmarkentry{otimes}
    \plotmarkentry{otimes*}
    \plotmarkentry{square}
    \plotmarkentry{square*}
    \plotmarkentry{triangle}
    \plotmarkentry{triangle*}
    \plotmarkentry{diamond}
    \plotmarkentry{diamond*}
    \plotmarkentry{pentagon}
    \plotmarkentry{pentagon*}
  \end{tabular}
\end{package}

\subsection{Shape Library}

\begin{shape}{ellipse}
  This shape is an ellipse tightly fitting the text box, if no inner
  separation is given. The following figure shows the anchors this
  shape defines; the anchors |10| and |130| are example of border anchors.
\begin{codeexample}[]
\Huge
\begin{tikzpicture}
  \node[name=s,shape=ellipse,style=shape example] {Ellipse\vrule width 1pt height 2cm};
  \foreach \anchor/\placement in
    {north west/above left, north/above, north east/above right, 
     west/left, center/above, east/right, 
     mid west/right, mid/above, mid east/left, 
     base west/left, base/below, base east/right, 
     south west/below left, south/below, south east/below right, 
     text/left, 10/right, 130/above}
     \draw[shift=(s.\anchor)] plot[mark=x] coordinates{(0,0)}
       node[\placement] {\scriptsize\texttt{(s.\anchor)}};
\end{tikzpicture}
\end{codeexample}
\end{shape}

%%% Local Variables: 
%%% mode: latex
%%% TeX-master: "pgfmanual"
%%% End: 

\include{pgfmanual-pgffor}
\include{pgfmanual-pages}
\include{pgfmanual-xxcolor}



\part{The Basic Layer}

\vskip1cm
\begin{codeexample}[graphic=white]
\begin{tikzpicture}
  \draw[gray,very thin] (-1.9,-1.9) grid (2.9,3.9)
          [step=0.25cm] (-1,-1) grid (1,1);
  \draw[blue] (1,-2.1) -- (1,4.1); % asymptote
                
  \draw[->] (-2,0) -- (3,0) node[right] {$x(t)$};
  \draw[->] (0,-2) -- (0,4) node[above] {$y(t)$};

  \foreach \pos in {-1,2}
    \draw[shift={(\pos,0)}] (0pt,2pt) -- (0pt,-2pt) node[below] {$\pos$};

  \foreach \pos in {-1,1,2,3}
    \draw[shift={(0,\pos)}] (2pt,0pt) -- (-2pt,0pt) node[left] {$\pos$};

  \fill (0,0) circle (0.064cm);
  \draw[thick,parametric,domain=0.4:1.5,samples=200]
    % The plot is reparameterised such that there are more samples
    % near the center.
    plot[id=asymptotic-example] function{(t*t*t)*sin(1/(t*t*t)),(t*t*t)*cos(1/(t*t*t))}
    node[right] {$\bigl(x(t),y(t)\bigr) = (t\sin \frac{1}{t}, t\cos \frac{1}{t})$};

  \fill[red] (0.63662,0) circle (2pt)
    node [below right,fill=white,yshift=-4pt] {$(\frac{2}{\pi},0)$};
\end{tikzpicture}
\end{codeexample}


\include{pgfmanual-base-design}
\include{pgfmanual-base-scopes}
\include{pgfmanual-base-points}
\include{pgfmanual-base-paths}
\include{pgfmanual-base-actions}
\include{pgfmanual-base-arrows}
\include{pgfmanual-base-nodes}
\include{pgfmanual-base-transformations}
\include{pgfmanual-base-images}
\include{pgfmanual-base-shadings}
\include{pgfmanual-base-plots}
\include{pgfmanual-base-layers}
\include{pgfmanual-base-quick}




\part{The System Layer}
\label{part-system}

This part describes the low-level interface of \pgfname, called the
\emph{system layer}. This interface provides a complete abstraction of
the internals of the underlying drivers. 

Unless you intend to port \pgfname\ to another driver or unless you intend
to write your own optimized frontend, you need not read this part.

In the following it is assumed that you are familiar with the basic
workings of the |graphics| package and that you know what
\TeX-drivers are and how they work.

\vskip1cm
\begin{codeexample}[graphic=white]
\begin{tikzpicture}[shorten >=1pt,->]
  \tikzstyle{vertex}=[circle,fill=black!25,minimum size=17pt,inner sep=0pt]
  
  \foreach \name/\x in {s/1, 2/2, 3/3, 4/4, 15/11, 16/12, 17/13, 18/14, 19/15, t/16}
    \node[vertex] (G-\name) at (\x,0) {$\name$};

  \foreach \name/\angle/\text in {P-1/234/5, P-2/162/6, P-3/90/7, P-4/18/8, P-5/-54/9}
    \node[vertex,xshift=6cm,yshift=.5cm] (\name) at (\angle:1cm) {$\text$};
  
  \foreach \name/\angle/\text in {Q-1/234/10, Q-2/162/11, Q-3/90/12, Q-4/18/13, Q-5/-54/14}
    \node[vertex,xshift=9cm,yshift=.5cm] (\name) at (\angle:1cm) {$\text$};

  \foreach \from/\to in {s/2,2/3,3/4,3/4,15/16,16/17,17/18,18/19,19/t}
    \draw (G-\from) -- (G-\to);  

  \foreach \from/\to in {1/2,2/3,3/4,4/5,5/1,1/3,2/4,3/5,4/1,5/2}
    { \draw (P-\from) -- (P-\to); \draw (Q-\from) -- (Q-\to); }

  \draw (G-3) .. controls +(-30:2cm) and +(-150:1cm) .. (Q-1);
  \draw (Q-5) -- (G-15);
\end{tikzpicture}
\end{codeexample}

% Copyright 2003 by Till Tantau <tantau@cs.tu-berlin.de>.
%
% This program can be redistributed and/or modified under the terms
% of the LaTeX Project Public License Distributed from CTAN
% archives in directory macros/latex/base/lppl.txt.

\section{Design of the System Layer}

\makeatletter


\subsection{Driver Files}
\label{section-pgfsys}

The \pgfname\ system layer consists of a large number of commands
starting with |\pgfsys@|. These commands will be called \emph{system
  commands} in the following. The higher layers ``interface'' with the
system layer by calling these commands. The higher layers should never
use |\special| commands directly or even check whether |\pdfoutput| is 
defined. Instead, all drawing requests should be ``channeled'' through
the system commands.

The system layer is loaded and setup by the following package:

\begin{package}{pgfsys}
  This file provides ``default implementations'' of all system
  commands, but most simply produce a warning that they are not
  implemented. The actual implementations of the system commands for a
  particular driver like, say, |pdftex| reside in files called
  |pgfsys-pdftex.sty|. These will be called \emph{driver files} in the
  following.

  When |pgfsys.sty| is loaded, it will try to determine which driver
  is used by loading |pgf.cfg|. This file should setup the macro
  |\pgfsysdriver| appropriately. The, |pgfsys.sty| will input the
  appropriate |pgfsys-|\meta{drivername}|.sty|. 
\end{package}

\begin{command}{\pgfsysdriver}
  This macro should expand to the name of the driver to be used by
  |pgfsys|. The default from |pgf.cfg| is |pgfsys-\Gin@driver|. This
  is very likely to be correct if you are using \LaTeX. For plain
  \TeX, the macro will be set to |pgfsys-pdftex.def| if |pdftex| is
  used und to |pgfsys-dvips.def| otherwise.
\end{command}

\begin{filedescription}{pgf.cfg}
  This file should setup the command |\pgfsysdriver| correctly. If
  |\pgfsysdriver| is already set to some value, the driver normally
  should not change it. Otherwise, it should make a ``good guess'' at
  which driver will be appropriate.
\end{filedescription}



\subsection{System Commands Shared Between Different Drivers}

Some definitions of system layer commands can be ``shared'' between
different drivers. For example, the literal text needed to stroke a
path in pdf is |S|, independently of the driver. For this reason,
the drivers for |pfdtex| and for |dvipdfm|, both of which produce
|.pdf| in the end, both include the file |pgfsys-common-pdf.def|,
which defines all common commands. Similarly, all PostScript based
drivers can used |pgfsys-common-postscript.def| for the ``standard''
postscript commands.


\subsection{Existing Driver Files}

With the current version of \pgfname, the following drivers are
implemented:

\subsection{Supported Drivers}

\begin{filedescription}{pgfsys-pdftex.def}
  This is a driver file for use with pdf\TeX, that is, with the
  |pdftex| or |pdflatex| command. It includes
  |pgfsys-common-pdf.def|. This driver has the most functionality. 
\end{filedescription}

\begin{filedescription}{pgfsys-dvipdfm.def}
  This is a driver file for use with (|la|)|tex| followed by |dvipdfm|. It
  includes |pgfsys-common-pdf.def|. This driver uses |graphicx| for the
  graphics inclusion and does not support masking. It does not
  support image inclusion in plain \TeX\ mode.
\end{filedescription}

\begin{filedescription}{pgfsys-dvips.def}
  This is a driver file for use with (|la|)|tet| followed by
  |dvips|. It includes |pgfsys-common-postscript.def|. This driver
  uses |graphicx| for the graphics inclusion and does not support
  masking. Shading is implemented, but the results will not be
  as good as with a driver producing |.pdf| as output. It does not
  support image inclusion in plain \TeX\ mode.
\end{filedescription}


\subsection{Common Definition Files}

Some drivers share many |\pgfsys@| commands. For the reason, files
defining these ``common'' commands are available. These files are
\emph{not} usable alone.

\begin{filedescription}{pgfsys-common-postscript}
  This file defines |\pgfsys@| commands so that they produce
  appropriate PostScript code.
\end{filedescription}

\begin{filedescription}{pgfsys-common-pdf}
  This file defines |\pgfsys@| commands so that they produce
  appropriate \textsc{pdf} code.
\end{filedescription}


\include{pgfmanual-pgfsys-commands}
\include{pgfmanual-pgfsys-paths}
\include{pgfmanual-pgfsys-protocol}



\part{References and Index}

\vskip1cm
\begin{codeexample}[graphic=white]
\begin{tikzpicture}
  \draw[line width=0.3cm,color=red!30,cap=round,join=round] (0,0)--(2,0)--(2,5);
  \draw[help lines] (-2.5,-2.5) grid (5.5,7.5);
  \draw[very thick] (1,-1)--(-1,-1)--(-1,1)--(0,1)--(0,0)--
    (1,0)--(1,-1)--(3,-1)--(3,2)--(2,2)--(2,3)--(3,3)--
    (3,5)--(1,5)--(1,4)--(0,4)--(0,6)--(1,6)--(1,5)
    (3,3)--(4,3)--(4,5)--(3,5)--(3,6)
    (3,-1)--(4,-1);
  \draw[below left] (0,0) node(s){$s$};
  \draw[below left] (2,5) node(t){$t$};
  \fill (0,0) circle (0.06cm) (2,5) circle (0.06cm);
  \draw[->,rounded corners=0.2cm,shorten >=2pt]
    (1.5,0.5)-- ++(0,-1)-- ++(1,0)-- ++(0,2)-- ++(-1,0)-- ++(0,2)-- ++(1,0)--
    ++(0,1)-- ++(-1,0)-- ++(0,-1)-- ++(-2,0)-- ++(0,3)-- ++(2,0)-- ++(0,-1)--
    ++(1,0)-- ++(0,1)-- ++(1,0)-- ++(0,-1)-- ++(1,0)-- ++(0,-3)-- ++(-2,0)--
    ++(1,0)-- ++(0,-3)-- ++(1,0)-- ++(0,-1)-- ++(-6,0)-- ++(0,3)-- ++(2,0)--
    ++(0,-1)-- ++(1,0);
\end{tikzpicture}
\end{codeexample}

\printindex

\end{document}



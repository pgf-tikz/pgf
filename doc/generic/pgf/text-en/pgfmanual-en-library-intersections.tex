% Copyright 2008 by Mark Wibrow
%
% This file may be distributed and/or modified
%
% 1. under the LaTeX Project Public License and/or
% 2. under the GNU Free Documentation License.
%
% See the file doc/generic/pgf/licenses/LICENSE for more details.


\section{Intersections Library}

{\bf\emph{This library is experimental and likely to change,
move, or disappear, without warning.}}

\begin{pgflibrary}{intersections}
  This library enables the calculation of intersections of
  two arbitrary paths. However, due to the low accuracy of
  \TeX, the paths should not be ``too complicated''.
  In particular, you should not try to intersect paths consisting 
  lots of very small segments such as plots or decorated paths.
\end{pgflibrary}

\subsection{Intersecting Two Paths in \tikzname}

  To find the intersections of two paths in \tikzname, they must be
  ``named''. A ``named path" is, quite simply, a path that has been 
  named using the following key:
  
\begin{key}{/tikz/path name=\meta{name}}
	The effect of this key is that, after the path has been constructed, 
	just before it is used, it is associated with \meta{name}. This 
	association survives beyond the final semi-colon of the path 
	but not the end of the surrounding scope. Any paths created by
	nodes on the (main) path are ignored, unless this key is explicitly 
	used. If the same \meta{name} is used for the main path and the
	node path(s), then they will all be considered together.
	
	
\begin{codeexample}[code only]
\begin{scope}
  \path [path name=a] ... ;       % Path named "a"   
  
  \node [path name=b] {};         % Node path named "b"
  
  \path [path name=c]  ...        % Main path named "c"
    node {};                      % Node path unnamed  
  
  \path [path name=d] ...         % Main Path named "d"
    node [path name=e] {} ...     % Node path named "e"
    node [path name=f] {};        % Node path named "f"
                                                        
  \path [path name=g] ...
    node [path name=g] {}...
    node [path name=g] {};        % Main path and node paths named "g"
\end{scope}

% No named paths available

\end{codeexample}
\end{key}

  To find the intersection of named paths, the following key is used:

\begin{key}{/tikz/name intersections=\marg{options}}
  
  This key chainges the key path to |/tikz/intersect| and processes
  \meta{options}. These options determine, among other things,
  which paths to use for the intersection. Having procesed the 
  options, any intersections are then found. A coordinate is created 
  at each intersection, which by default, will be named 
  |intersection-1|, |intersection-2|, and so on. Unfortunately, there
  can be no guarantee of a ``heplful'' ordering of solutions.
  Optionally, the prefix |intersection| can be changed, and the 
  total number of intersections stored in a \TeX-macro. 

\begin{codeexample}[]
\begin{tikzpicture}[every node/.style={opacity=1, black, above left}]
  \draw [help lines] grid (3,2);
  \draw [path name=ellipse] (2,0.5) ellipse (0.75cm and 1cm);
  \draw [path name=rectangle, rotate=10] (0.5,0.5) rectangle +(2,1);
  \fill [red, opacity=0.5, name intersections={of=ellipse and rectangle}]
    (intersection-1) circle (2pt) node {1}
    (intersection-2) circle (2pt) node {2};
\end{tikzpicture}
\end{codeexample}

\end{key}

  The following keys can be used in \meta{options}:
  
\begin{key}{/tikz/intersection/of=\meta{path name 1}| and |\meta{path name 2}}
  This key is used to specify the names of the paths to use for
  the intersection.
\end{key}

\begin{key}{/tikz/intersection/name=\meta{prefix}}
  This key specifies the prefix name for the coodinate nodes placed
  at each intersection.
\end{key}

\begin{key}{/tikz/intersection/total=\meta{macro}}
  This key will mean than the total number of intersections found
  will be stored in macro.
\end{key}

\begin{codeexample}[]
\begin{tikzpicture}
  \clip (-2,-2) rectangle (2,2);
  \draw [path name=curve 1] (-2,-1) .. controls (8,-1) and (-8,1) .. (2,1);
  \draw [path name=curve 2] (-1,-2) .. controls (-1,8) and (1,-8) .. (1,2);
  
  \fill [name intersections={of=curve 1 and curve 2, name=i, total=\t}]
        [red, opacity=0.5, every node/.style={above left, black, opacity=1}] 
        \foreach \s in {1,...,\t}{(i-\s) circle (2pt) node {\footnotesize\s}};
\end{tikzpicture}
\end{codeexample}

\subsection{Intersecting Two Paths in PGF}
  
  To find the intersections of two paths in \pgfname, the following 
  command is provided:
   
\begin{command}{\pgfintersectionofpaths\marg{path 1}\marg{path 2}}
  This command finds the intersection points on the paths 
  \meta{path 1} and \meta{path 2}. The number of intersection points
  (``solutions'') that are found will be stored, and each point 
  can be accessed afterward. The code for \meta{path 1} and 
  \meta{path 2} is executed within a \TeX{} group and so can contain
  transformations (which will be in addition to any existing
  transformations). The code should not use the path in any way, 
  unless the path is saved first and restored afterward.
  \pgfname{} will regard solutions as ``a bit
  special'', in that the points returned  will be ``absolute'' and 
  unaffected by any further transformations.

\begin{codeexample}[]
\begin{pgfpicture}
\pgfintersectionofpaths
{
  \pgfpathellipse{\pgfpointxy{0}{0}}{\pgfpointxy{1}{0}}{\pgfpointxy{0}{2}}
  \pgfgetpath\temppath
  \pgfusepath{stroke}
  \pgfsetpath\temppath
}
{
  \pgftransformrotate{-30}
  \pgfpathrectangle{\pgfpointorigin}{\pgfpointxy{2}{2}}
  \pgfgetpath\temppath
  \pgfusepath{stroke}
  \pgfsetpath\temppath
}
\foreach \s in {1,...,\pgfintersectionsolutions}
  {\pgfpathcircle{\pgfpointintersectionsolution{\s}}{2pt}}
\pgfusepath{stroke}
\end{pgfpicture}
\end{codeexample}

\end{command}

\begin{command}{\pgfintersectionsolutions}
  After using the |\pgfintersectionofpaths| command, this \TeX-counter
  will indicate the number of solutions found.
\end{command}

\begin{command}{\pgfpointintersectionsolution\marg{number}}
  After using the |\pgfintersectionofpaths| command, this command
  will return the point for solution \meta{number} or the origin
  if this solution was not found. Unfortunately
  there can be no guarantee of a ``helpful'' ordering of solutions.
\end{command}
 

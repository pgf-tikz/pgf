% Copyright 2008 by Mark Wibrow
%
% This file may be distributed and/or modified
%
% 1. under the LaTeX Project Public License and/or
% 2. under the GNU Free Documentation License.
%
% See the file doc/generic/pgf/licenses/LICENSE for more details.

\section{Decoration Library}
\label{section-library-decorations}

\begin{pgflibrary}{decorations}
  This library package defines a number of decorations. Also, the
  \tikzname\ version of the library is needed in order to use
  decorations in \tikzname\ at all.
\end{pgflibrary}


\subsection{Overview}

The decoration library defines a number of (more or less) useful
decorations that can be applied to paths. The usage of decorations is
not covered in the present section, please consult 
Sections~\ref{section-tikz-snakes-and-decorations}, which explains how
decorations are used in \tikzname, and
\ref{section-base-snakes-and-decorations}, which  explains how new
decorations can be defined. 

\subsubsection{Classification of Decorations}

The present library is structured according to the basic effect the
decorations have on the path:
\begin{enumerate}
\item \emph{Path deforming} decorations deform the path in the sense
  that what used to be a straight  line might afterwards be a squiggly
  line or might have bumps. However, a line is still and a line and
  path deforming decorations do not change the number of subpaths.
\item \emph{Path chopping} decorations deform a path and,
  additionally, may break up a line into numerous subpath. An example
  is a decoration that replaces a line by, say, little triangles.
\item \emph{Path removing} decorations completely remove the
  path. Thus, they have no effect on the main path that is being
  constructed. However, they typically have numerous \emph{side
    effects}. For instance, they might ``write some text'' along the
  (removed) path or they might place nodes along this path. Note that
  for such decorations the path usage command for the main path have
  no influence on how the decoration looks like.
\end{enumerate}

Furthermore, at the end of this section some fractal decorations are
documented separately (even though they are either path deforming or
path chopping).


\subsubsection{Decoration Options}

The decorations are influenced by a number of parameters that can be
set using the |decoration options| option. These parameters are
typically shared between different decorations. In the following, the
general options are documented, special-purpose keys are documented
with the decoration that uses it.

\begin{key}{/pgf/decoration options/amplitude=\meta{dimension} (initially 2.5pt)}
  This key determines the ``desired height'' (or amplitude) of 
  decorations for which this makes sense. For instance, the initial
  value of |2.5pt| means that deforming decorations should deform a
  path by up to 2.5pt away from the original path.
\end{key}

\begin{key}{/pgf/decoration options/meta-amplitude=\meta{dimension} (initially 2.5pt)}
  This key determines the amplitude for a meta-decoration. 
\end{key}

\begin{key}{/pgf/decoration options/segment length=\meta{dimension} (initially 10pt)}
  Many decorations are made up of small segments. This key determines
  the desired length of such segments. 
\end{key}

\begin{key}{/pgf/decoration options/meta-segment length=\meta{dimension} (initially 1cm)}
  This determined the length of the meta-segments from which a
  meta-snake is made up.
\end{key}

\begin{key}{/pgf/decoration options/angle=\meta{degree} (initially 45)}
  The way some decorations look like depends on a configurable angle. For
  instance, a |wave| decoration consists of arcs and the opening angle
  of these arcs is given by the |angle|.
\end{key}

\begin{key}{/pgf/decoration options/aspect=\meta{factor} (initially 0.5)}
  For some decorations there is a natural aspect ratio. For instance,
  for a |brace| decoration the aspect ratio determines where the brace
  point will be.
\end{key}

\begin{key}{/pgf/decoration options/shape width=\meta{dimension}
    (initially \normalfont same as |amplitude|)}
  For decorations that are made up from shapes, this key determines
  the (desired) width of these shapes.
\end{key}

\begin{key}{/pgf/decoration options/text=\meta{text}
    (initially \normalfont empty)}
  For decorations that need text, this option sets the text.
\end{key}

\begin{key}{/pgf/decoration options/text color=\meta{color}
    (initially \normalfont black)}
  For decorations that need colored text, this option sets this color.
\end{key}

\begin{key}{/pgf/decoration options/shape=\meta{shape name} (initially circle)}
  For decorations that use shapes, this is the shape name that is used.
\end{key}

\begin{key}{/pgf/decoration options/anchor=\meta{anchor}    (initially center)}
  For decorations that use shapes and that need to anchor then, this
  is the anchor that is used.
\end{key}



\subsection{Path Deforming Decorations}

A \emph{path deforming decorations} deform the to-be-decorated
path. This means that what used to be a straight line might afterwards
be a snaking curve and have bumps. However, a line is still and a line
and path deforming decorations do not change the number of
subpaths. For instance, if the path used to consist of two circles and
an open arc, the path will after the decoration process still consist
of two closed subpath and one open subpath.


\subsubsection{Deformation by Straight Lines}

The following deformations use only straight lines in order to deform
the paths.


\begin{decoration}{lineto}
  This decoration replaces the path by straight lines. For each curve,
  the path simply goes directly from the start point to the end point.
  In the following example, the arc actually consist of two
  subcurves. 
\begin{codeexample}[]
\begin{tikzpicture}[decoration=lineto]
  \draw [help lines] grid (3,2);
  \draw [decorate] (0,0) -- (3,1) arc (0:180:1.5 and 1);
\end{tikzpicture}
\end{codeexample}
\end{decoration}


\begin{decoration}{line zigzag}
  This (meta-)decoration decorates the path by alternating between 
  |line along| and |zigzag| decorations. It always finishes
  with the |line along| decoration. The following parameters influence
  the decoration:
  \begin{itemize}
  \item |amplitude|
    determines how much the zig-zag lines raises above and falls below
    a straight line to the target point.
  \item |segment length|
    determines the length of a complete ``up-down'' cycle.
  \item	|meta-segment length|
    determines the length of the |line along| and the |zigzag| decorations.
  \end{itemize}

\begin{codeexample}[]
\begin{tikzpicture}[decoration=line zigzag,
                    decoration options={meta-segment length=1.1cm}]
  \draw [help lines] grid (3,2);
  \draw [decorate] (0,0) -- (3,1) arc (0:180:1.5 and 1);
\end{tikzpicture}
\end{codeexample}
\end{decoration}


\begin{decoration}{random steps}
  This snake consists of straight line segments. The line segments
  head towards the target, but each step is randomly shifted a little
  bit. The following parameters influence the snake:
  \begin{itemize}
  \item |segment length|
    determines the basic length of each step.
  \item |amplitude|
    The end of each step is perturbed both in $x$- and in
    $y$-direction by two values drawn uniformly from the interval
    $[-d,d]$, where $d$ is the value of |amplitude|.
  \end{itemize}
\begin{codeexample}[]
\begin{tikzpicture}[decoration=random steps,
                    decoration options={segment length=2mm}]
  \draw [help lines] grid (3,2);
  \draw [decorate] (0,0) -- (3,1) arc (0:180:1.5 and 1);
\end{tikzpicture}
\end{codeexample}
\end{decoration}


\begin{decoration}{saw}
  This decoration looks like the blade of a saw. The following parameters
  influence the decoration:
  \begin{itemize}
  \item |amplitude|
    determines how much each spike raises above the straight line.
  \item |segment length|
    determines the length each spike.
  \end{itemize}
\begin{codeexample}[]
\begin{tikzpicture}[decoration=saw]
  \draw [help lines] grid (3,2);
  \draw [decorate] (0,0) -- (3,1) arc (0:180:1.5 and 1);
\end{tikzpicture}
\end{codeexample}
\end{decoration}


\begin{decoration}{zigzag}
  This decoration looks like a zig-zag line. The following parameters
  influence the decoration:
  \begin{itemize}
  \item |amplitude|
    determines how much the zig-zag lines raises above and falls below
    a straight line to the target point.
  \item |segment length|
    determines the length of a complete ``up-down'' cycle.
  \end{itemize}
\begin{codeexample}[]
\begin{tikzpicture}[decoration=zigzag]
  \draw [help lines] grid (3,2);
  \draw [decorate] (0,0) -- (3,1) arc (0:180:1.5 and 1);
\end{tikzpicture}
\end{codeexample}
\end{decoration}




\subsubsection{Deformation by Curved Lines}


\begin{decoration}{bent}
  This decoration adds a slightly bent line from the start to the
  target. The amplitude of the bent is given |amplitude|
  (an amplitude of zero gives a straight line). 
  \begin{itemize}
  \item |amplitude|
    determines the amplitude of the bent.
  \item |aspect|
    determines how tight the bent is. A good value is around |0.3|. 
  \end{itemize}
  Note that this decoration makes only little sense for curves. You
  should apply it only to straight lines.
\begin{codeexample}[]
\begin{tikzpicture}[decoration=bent]
  \draw [help lines] grid (3,2);
  \draw [decorate] (0,0) -- (3,1) -- (1.5,2) -- (0,1);
\end{tikzpicture}
\end{codeexample}
\begin{codeexample}[]
\begin{tikzpicture}[decoration options={aspect=.3},decoration=bent]
  \node[circle,draw] (A) at (.5,.5) {A};
  \node[circle,draw] (B) at (3,1.5) {B};
  \draw[->,decorate] (A) -- (B);
  \draw[->,decorate] (B) -- (A);

  \draw [decorate] (0,0) rectangle (3.5,2);
\end{tikzpicture}
\end{codeexample}
\end{decoration}


\begin{decoration}{bumps}
  This decoration replaces the path by little half ellipses. The
  following parameters influence itL
  \begin{itemize}
  \item |amplitude|
    determines the height of the half ellipse.
  \item |segment length|
    determines the width of the half ellipse.
  \end{itemize}
\begin{codeexample}[]
\begin{tikzpicture}[decoration=bumps]
  \draw [help lines] grid (3,2);
  \draw [decorate] (0,0) -- (3,1) arc (0:180:1.5 and 1);
\end{tikzpicture}
\end{codeexample}
\end{decoration}


\begin{decoration}{coil}
  This decoration replaces the path by a coiled line. To understand how this works,
  imagine a three-dimensional spring. The spring's axis points along
  the path toward the target. Then, we ``view'' the spring from a
  certain angle. If we look ``straight from the side'' we will see a
  perfect sine curve, if we look ``more from the front'' we will see a
  coil. The following parameters influence the snake:  
  \begin{itemize}
  \item |amplitude|
    determines how much the coil rises above the path and falls below
    it. Thus, this is the radius of the coil.
  \item |segment length|
    determines the distance between two consecutive ``curls.'' Thus,
    when the spring is see ``from the side'' this will be the wave
    length of the sine curve. 
  \item |aspect|
    determines the ``viewing direction.'' A value of |0| means
    ``looking from the side'' and a value of |0.5|, which is the
    default, means ``look more from the front.'' 
  \end{itemize}
\begin{codeexample}[]
\begin{tikzpicture}[decoration=coil]
  \draw [help lines] grid (3,2);
  \draw [decorate] (0,0) -- (3,1) arc (0:180:1.5 and 1);
\end{tikzpicture}
\end{codeexample}
\begin{codeexample}[]
\begin{tikzpicture}[decoration=coil,decoration options=
                    {aspect=0.3,segment length=3mm,amplitude=3mm}]
  \draw [help lines] grid (3,2);
  \draw [decorate] (0,0) -- (3,1) arc (0:180:1.5 and 1);
\end{tikzpicture}
\end{codeexample}
\end{decoration}



\begin{decoration}{line along}
  This decoration simply yields a line following the original
  path. This means that (ideally) it does not change the path. In
  reality, due to the internals of how decorations are implemented,
  this decoration actually replaces the path by numerous small
  straight lines.

  This decoration is useful only in conjunction with
  meta-decorations. 

\begin{codeexample}[]
\begin{tikzpicture}[decoration=line along]
  \draw [help lines] grid (3,2);
  \draw [decorate] (0,0) -- (3,1) arc (0:180:1.5 and 1);
\end{tikzpicture}
\end{codeexample}
\end{decoration}



\begin{decoration}{snake}
  This decoration replaces the path by a line that looks like a snake
  seen from above. More precisely, the snake is a sine wave with a
  ``softened'' start and ending. The following parameters influence
  the snake: 
  \begin{itemize}
  \item |amplitude|
    determines the sine wave's amplitude.
  \item |segment length|
    determines the sine wave's wave length.
  \end{itemize}
\begin{codeexample}[]
\begin{tikzpicture}[decoration=snake]
  \draw [help lines] grid (3,2);
  \draw [decorate] (0,0) -- (3,1) arc (0:180:1.5 and 1);
\end{tikzpicture}
\end{codeexample}
\end{decoration}



  
\subsection{Path Chopping Decorations}

Still missing...

\subsection{Path Removing Decorations}

Still missing...

\subsection{Fractal Decorations}

\begin{decoration}{Koch curve type 1}
  This decoration replaces a straight line by a ``rectangular bump.''
  By repeatedly applying this replacement, different levels of the
  Koch curve fractal can be created. Its Hausdorff dimension is $\log
  5/\log 3$.
\begin{codeexample}[]
\begin{tikzpicture}[decoration=Koch curve type 1]
  \draw decorate{ (0,0) -- (3,0) };
  \draw decorate{ decorate{ (0,-1.5) -- (3,-1.5) }};
  \draw decorate{ decorate{ decorate{ (0,-3) -- (3,-3) }}};
\end{tikzpicture}
\end{codeexample}
\end{decoration}


\begin{decoration}{Koch curve type 2}
  This decoration replaces a straight line by a ``rectangular sine.''
  Its Hausdorff dimension is $3/2$.
\begin{codeexample}[]
\begin{tikzpicture}[decoration=Koch curve type 2]
  \draw decorate{ (0,0) -- (3,0) };
  \draw decorate{ decorate{ (0,-2) -- (3,-2) }};
  \draw decorate{ decorate{ decorate{ (0,-4) -- (3,-4) }}};
\end{tikzpicture}
\end{codeexample}
\end{decoration}

\begin{decoration}{Koch snowflake}
  This decoration replaces a straight line by a ``line with a spike.''
  Koch's snowflake's Hausdorff dimension is $\log 4/\log 3$.
\begin{codeexample}[]
\begin{tikzpicture}[decoration=Koch snowflake]
  \draw decorate{ (0,0) -- (3,0) };
  \draw decorate{ decorate{ (0,-1) -- (3,-1) }};
  \draw decorate{ decorate{ decorate{ (0,-2) -- (3,-2) }}};
  \draw decorate{ decorate{ decorate{ decorate{ (0,-3) -- (3,-3) }}}};
\end{tikzpicture}
\end{codeexample}
\end{decoration}

\begin{decoration}{Cantor set}
  This decoration replaces a straight line by a ``line with a whole in
  the middle.'' The Hausdorff dimension of the Cantor set is $\log
  2/\log 3$. 
\begin{codeexample}[]
\begin{tikzpicture}[decoration=Cantor set,very thick]
  \draw decorate{ (0,0) -- (3,0) };
  \draw decorate{ decorate{ (0,-.5) -- (3,-.5) }};
  \draw decorate{ decorate{ decorate{ (0,-1) -- (3,-1) }}};
  \draw decorate{ decorate{ decorate{ decorate{ (0,-1.5) -- (3,-1.5) }}}};
\end{tikzpicture}
\end{codeexample}
\end{decoration}










\endinput





\begin{decoration}{border}
  This snake adds straight lines the path that are at a specific angle
  to the line toward the target. The idea is to add these little lines
  to indicate the ``border'' or an area. The following parameters
  influence the snake:  
  \begin{itemize}
  \item |/pgf/segment length|
    determines the distance between consecutive ticks.
  \item |/pgf/segment amplitude|
    determines the length of the ticks.
  \item |/pgf/segment angle|
    determines the angle between the ticks and the line toward the
    target. 
  \end{itemize}
\begin{codeexample}[]
\tikz{\draw (0,0) rectangle (3,1)
            [snake=border,segment angle=-45] (0,0) rectangle (3,1);}
\end{codeexample}
\end{decoration}


\begin{decoration}{brace}
  This snake adds a long brace to the path. The left and right end of
  the brace will be exactly on the start and endpoint of the
  snake. The following parameters influence the snake:  
  \begin{itemize}
  \item |/pgf/segment amplitude|
    determines how much the brace rises above the path.
  \item |/pgf/segment aspect|
    determines the fraction of the total length where the ``middle
    part'' of the brace will be.  
  \end{itemize}
\begin{codeexample}[]
\tikz{\draw[snake=brace,segment aspect=0.25] (0,0) -- (3,0);}
\end{codeexample}
\end{decoration}


\begin{decoration}{expanding waves}
  This snake adds arcs to the path that get bigger along the line
  towards the target. The following parameters influence the snake:
  \begin{itemize}
  \item |/pgf/segment length|
    determines the distance between consecutive arcs.
  \item |/pgf/segment angle|
    determines the opening angle below and above the path. Thus, the
    total opening angle is twice this angle.
  \end{itemize}
\begin{codeexample}[]
\tikz{\draw[snake=expanding waves] (0,0) -- (3,0);}
\end{codeexample}
\end{decoration}



\begin{decoration}{ticks}
  This snake adds straight lines  the path that are orthogonal to the
  line toward the target. The following parameters influence the snake: 
  \begin{itemize}
  \item |/pgf/segment length|
    determines the distance between consecutive ticks.
  \item |/pgf/segment amplitude|
    determines half the length of the ticks.
  \end{itemize}
\begin{codeexample}[]
\tikz{\draw[snake=ticks] (0,0) -- (3,0);}
\end{codeexample}
\end{decoration}

\begin{decoration}{triangles}
  This snake adds triangles to the path that point toward the
  target. The following parameters influence the snake: 
  \begin{itemize}
  \item |/pgf/segment length|
    determines the distance between consecutive triangles.
  \item |/pgf/segment amplitude|
    determines half the length of the triangle side that is orthogonal
    to the path.
  \item |/pgf/segment object length|
    determines the height of the triangle.
  \end{itemize}
\begin{codeexample}[]
\tikz{\draw[snake=triangles] (0,0) -- (3,0);}
\end{codeexample}
\end{decoration}

\begin{decoration}{crosses}
  This snake adds (diagonal) crosses to the path. The following
  parameters influence the snake:  
  \begin{itemize}
  \item |/pgf/segment length|
    determines the distance between consecutive crosses.
  \item |/pgf/segment amplitude|
    determines half the hieght of each cross.
  \item |/pgf/segment object length|
    determines width of each cross.
  \end{itemize}
\begin{codeexample}[]
\tikz{\draw[snake=crosses] (0,0) -- (3,0);}
\end{codeexample}
\end{decoration}


\begin{decoration}{waves}
  This snake adds arcs to the path that have a constant size. The
  following parameters influence the snake: 
  \begin{itemize}
  \item |/pgf/segment length|
    determines the distance between consecutive arcs.
  \item |/pgf/segment angle|
    determines the opening angle below and above the path. Thus, the
    total opening angle is twice this angle.
  \item |/pgf/segment amplitude|
    determines the radius of each arc.
  \end{itemize}
\begin{codeexample}[]
\tikz{\draw[snake=waves] (0,0) -- (3,0);}
\end{codeexample}
\end{decoration}


\begin{decoration}{shape snake}
  This snake adds a succession of shapes to the path. The shape must
  have been defined by |\pgfdeclareshape| and must have defined a 
  background path. Please note that the shapes in a snake are not 
  nodes. They cannot have text inside them, be named, or referred to. 
  The snake simply adds the background path of the shape to the ongoing 
  snaked path.

\begin{codeexample}[]
\tikzset{paint/.style={snake=shape snake, draw=#1!50!black, fill=#1!50}}
\begin{tikzpicture}
  \draw [shape snake shape=dart,      paint=red]    (0,1.5) -- (3,1.5);
  \draw [shape snake shape=diamond,   paint=green]  (0,1)   -- (3,1);
  \draw [shape snake shape=rectangle, paint=blue]   (0,0.5) -- (3,0.5);
  \draw [shape snake shape=circle,    paint=yellow] (0,0)   -- (3,0);
\end{tikzpicture}
\end{codeexample}

  All shapes are positioned by their center anchor (as this is the only
  anchor that all shapes must define). A shape is drawn at the start 
  point of the path and, if the distance between the shapes is 
  appropriate, at the end point of the path.
	
\begin{codeexample}[]
\begin{tikzpicture}[snake=shape snake, shape snake shape=regular polygon]
  \draw [help lines] grid (3,2);
  \draw [thick] (0,0) -- (2,2) (1,0) -- (3,0);
  \draw [very thick, red!50, snake, shape snake sep=.5cm]  (1,0) -- (3,0);
  \draw [very thick, blue!50, snake, shape snake sep=.5cm] (0,0) -- (2,2);
\end{tikzpicture}
\end{codeexample}

  Keys for cusomizing specific shapes can be specified (e.g., 
  |star points|, |cloud puffs|, |kite angles|, and so on). However, you
  should be aware that the size of each shape is enforced using a 
  coodinate transformation, which may mean that settings involving 
  angles and distances may not appear entirely accurate. More general
  options such as |inner sep| and |minimum size| will be ignored, 
  but transformations can be applied to each segment as described
  below.
  
\begin{codeexample}[]
\tikzset{
  paint/.style={snake=shape snake, draw=#1!50!black, fill=#1!50},
  my star/.style={shape snake shape=star, star points=#1}
}
\begin{tikzpicture}[shape snake sep=.5cm, shape snake start size=.5cm]
  \draw [my star=9, paint=red]                            (0,1.5) -- (3,1.5);
  \draw [my star=5, paint=blue]                           (0,.75) -- (3,.75);
  \draw [my star=5, paint=yellow, shape border rotate=30] (0,0) -- (3,0);
\end{tikzpicture}
\end{codeexample}

  There are various keys to control the drawing of the shape snake.

\begin{key}{/pgf/shape snake shape=\meta{shape} (initially circle)}
  \keyalias{tikz}
  Set the shape for the snake. If \meta{shape} is defined in a shape
  library which has not been loaded then an error will result.
\end{key}

\begin{key}{/pgf/shape snake sep=\meta{spacing} (initially {.25cm, between centers})}
  \keyalias{tikz}
  Set the spacing between the shapes on the snaked path. This can be
  just a distance on its own, but the additional keywords 
  |between centers|, and |between borders| (which must be preceded by a 
  comma), specify that the distance	is between the center anchors of 
  the shapes or between the edges of the \emph{boundaries} of
  the shape borders.
	
\begin{codeexample}[]
\begin{tikzpicture}[snake=shape snake, shape snake start size=.5cm,
    paint/.style={snake, draw=#1!50!black, fill=#1!50},
    shape snake shape=signal, signal from=west, signal to=east]
  \draw [help lines] grid (3,2);
  \draw [paint=red, shape snake sep=.5cm]                    (0,0) -- (3,0);
  \draw [paint=green, shape snake sep={1cm, between center}] (0,1) -- (3,1);
  \draw [paint=blue, shape snake sep={1cm, between borders}] (0,2) -- (3,2);
\end{tikzpicture}
\end{codeexample}

\end{key}


  
\begin{key}{/pgf/shape snake evenly spread=\meta{number}}
  \keyalias{tikz}
  This key overides the |shape snake sep| key and forces the snake to
  fit \meta{number} shapes evenly across the path. 
  If \meta{number} is less than |1|, then no shapes will be drawn. 
  If \meta{number} equals |1|, then one shape is drawn in the middle 
  of the path. 
  The additional keywords |by centers| (the default, if no keyword is
  specified) and |by borders| can be used (both preceded by a comma), 
  to specify how the distance between shapes is determined. These
  keywords will only have a noticable effect if the snake is scaled.
  
\begin{codeexample}[]
\tikzset{my snake/.style={%
  snake=shape snake, shape snake shape=rectangle, shape snake start size=.5cm,
  draw=#1!50!black, fill=#1!50}
}
\begin{tikzpicture}
  \fill [shape snake evenly spread={5, by borders}, 
         my snake=green, shape snake scaled]           (0,2)   -- (3,2);
  \fill [shape snake evenly spread={5, by centers},
         my snake=blue, shape snake scaled]            (0,1.5) -- (3,1.5);   
  \fill [my snake=red, shape snake evenly spread=5]    (0,1)   -- (3,1);
  \fill [my snake=orange, shape snake evenly spread=4] (0,.5)  -- (3,.5);
  \fill [my snake=gray, shape snake evenly spread=1]   (0,0)   -- (3,0);
\end{tikzpicture}
\end{codeexample}

\end{key}

\begin{key}{/pgf/shape snake sloped=\meta{boolean} (default true)}
  \keyalias{tikz}
  By default, shapes are rotated to the slope of the snaked path. If 
  \meta{boolean} is the value |false|, then this rotation is turned 
  off. Internally this sets the \TeX-if |\ifpgfshapesnakesloped|
  appropriately.

\begin{codeexample}[]
\tikzset{
  shape snake start width=.65cm, shape snake start height=.45cm,
  shape snake shape=isosceles triangle, shape snake sep=.75cm,
  paint/.style={snake, draw=#1!50!black, fill=#1!50}
}
\begin{tikzpicture}[snake=shape snake]
  \draw [help lines] grid (3,2);
  \draw [paint=red] (0,0) -- (2,2);
  \draw [paint=blue, shape snake sloped=false] (1,0) -- (3,2);
\end{tikzpicture}
\end{codeexample}

\end{key}%

It is possible to scale the width and height of the shapes across the
length of the snaked path. The shapes are scaled between the starting
size and the ending size. The following keys customize the way the
snake shapes are scaled:
	
\begin{key}{/pgf/shape snake scaled=\meta{boolean} (default true)}
  \keyalias{tikz}
  Internally this sets the \TeX-if |\ifpgfshapesnakescaled| 
  appropriately.
	
\begin{codeexample}[]
\tikzset{
  bigger/.style={shape snake start size=.125cm, shape snake end size=.5cm},
  smaller/.style={shape snake start size=.5cm, shape snake end size=.125cm},
  shape snake sep={.25cm, between borders}
}
\begin{tikzpicture}[snake=shape snake]
  \draw [help lines] grid (3,2);
  \fill [snake, shape snake scaled, bigger, red!50]   (0,1) -- (3,2);
  \fill [snake, shape snake scaled, smaller, blue!50] (0,0) -- (3,1);
\end{tikzpicture}
\end{codeexample}

\end{key}

\begin{key}{/pgf/shape snake start width=\meta{length} (initially .25cm)}
  \keyalias{tikz}
  The starting width of the shape.
\end{key}%

\begin{key}{/pgf/shape snake start height=\meta{length} (initially .25cm)}
  \keyalias{tikz}
  The starting height of the shape.
\end{key}%

\begin{stylekey}{/pgf/shape snake start size=\meta{length}}
  \keyalias{tikz}
  Set both the the start height and start width simultaneously.
\end{stylekey}%

\begin{key}{/pgf/shape snake end width=\meta{length} (initially .125cm)}
  \keyalias{tikz}
  The recommended ending width of the shape. Note, that this is the
  width that a shape will take only if it is drawn exactly at the end
  of the path.
		
\begin{codeexample}[]
\tikzset{
  bigger/.style={shape snake start size=.25cm, shape snake end size=1cm},
  smaller/.style={shape snake start size=1cm, shape snake end size=.25cm},
  shape snake scaled, snake=shape snake
}
\begin{tikzpicture}
  \draw [help lines]grid(3,2);
  \fill [snake, bigger,  shape snake sep={.25cm, between borders}, blue!50] 
    (0,1.5) -- (3,1.5);
  \fill [snake, smaller, shape snake sep={1cm, between centers},   red!50]  
    (0,.5)  -- (3,.5);
  \draw [gray, dotted] (0,1.625) -- (3,2)    (0,1.375) -- (3,1) 
                       (0,1)     -- (3,.625) (0,0)     -- (3,.375); 
\end{tikzpicture}
\end{codeexample}

\end{key}%

\begin{key}{/pgf/shape snake end height=\meta{length}}
  \keyalias{tikz}
  The recommended ending height of the shape.
\end{key}%

\begin{stylekey}{/pgf/shape snake end size=\meta{length}}
  \keyalias{tikz}
  Set both the the end height and end width simultaneously.
\end{stylekey}

There is an additional TikZ key for the shape snake:

\begin{stylekey}{/tikz/shape snake tranform=\meta{keys}}
  \keyalias{tikz}
  This key parses \meta{keys}, which should be things like |rotate|, or
  |yshift|, and so on. The resulting transformation is applied to each 
  segment as it is drawn. It is analogous to the pgf command
  |\pgfsetsnakesegmenttransformation| (and in fact, uses it internally).

\begin{codeexample}[]
\tikzset{my snake/.style={%
  snake=shape snake, shape snake shape=rectangle, shape snake sep=.5cm,
  very thick, draw=#1!50}
}
\begin{tikzpicture}
  \draw [help lines] grid (3,2);
  \draw [thick] (0,0.5) -- (3,1.5);
  \draw [my snake=red,  shape snake transform={yshift=7.5pt}] 
     (0,0.5) -- (3,1.5);
  \draw [my snake=blue, shape snake transform={yshift=-7.5pt, rotate=45}] 
     (0,0.5) -- (3,1.5);
\end{tikzpicture}
\end{codeexample}
\end{stylekey}

\end{decoration}

\begin{decoration}{triangles}
	This decoration adds triangles to the path that point toward the
  target. The following parameters influence the decoration: 
  \begin{itemize}
  \item |/pgf/decoration segment length|
    determines the distance between consecutive triangles.
  \item |/pgf/decoration segment amplitude|
    determines half the length of the triangle side that is orthogonal
    to the path.
  \item |/pgf/decoration segment object length|
    determines the height of the triangle.
  \end{itemize}
\begin{codeexample}[]
\begin{tikzpicture}[decoration=triangles]
  \draw [help lines] grid (3,2);
  \draw [decorate] (0,0) .. controls (0,2) and (3,0) .. (3,2);
\end{tikzpicture}
\end{codeexample}
\end{decoration}

\begin{decoration}{crosses}
	This decoration adds (diagonal) crosses to the path. The following
  parameters influence the decoration:  
  \begin{itemize}
  \item |/pgf/decoration segment length|
    determines the distance between consecutive crosses.
  \item |/pgf/decoration segment amplitude|
    determines half the hieght of each cross.
  \item |/pgf/decoration segment object length|
    determines width of each cross.
  \end{itemize}
\begin{codeexample}[]
\begin{tikzpicture}[decoration=crosses]
  \draw [help lines] grid (3,2);
  \draw [decorate] (0,0) .. controls (0,2) and (3,0) .. (3,2);
\end{tikzpicture}
\end{codeexample}
\end{decoration}

\begin{decoration}{ticks}
  This decoration adds straight lines  the path that are orthogonal to 
  the line toward the target. The following parameters influence the 
  decoration: 
  \begin{itemize}
  \item |/pgf/decoration segment length|
    determines the distance between consecutive ticks.
  \item |/pgf/decoration segment amplitude|
    determines half the length of the ticks.
  \end{itemize}
\begin{codeexample}[]
\begin{tikzpicture}[decoration=ticks]
  \draw [help lines] grid (3,2);
  \draw [decorate] (0,0) .. controls (0,2) and (3,0) .. (3,2);
\end{tikzpicture}
\end{codeexample}
\end{decoration}

\begin{decoration}{text}
  This decoration decorates the path with text.
	
\begin{codeexample}[]
\catcode`\|12
\begin{tikzpicture}
  \draw [help lines] grid (3,2);
  \draw [red, dashed, postaction={decoration=text, decorate,
    decoration text={Some text along a curve}}] 
    (0,0) .. controls (0,2) and (3,0) .. (3,2);
\end{tikzpicture}
\end{codeexample}

  \pgfname{} ``does its best'' to typeset the text, however you
  should note the following points:
  \begin{itemize}
  \item
    Each character in the text is typeset in a separate |\hbox|. This
    means that if you want fancy things like kerning or ligatures you
    will have to manually annotate the characters in the decoration 
    text within a group, for example, |W{\kern-1ptA}TER|. 
  \item
    Each character is positioned using the center of its baseline. To
    move the text vertcally (relative to the path), the additional
    transform key should be used.
  \item
    No attempt is made to ensure characters do not overlap when
    the angle between segments is considerably less than 180\textdegree{}
    (this is tricky to do in \TeX{} without a huge processing
    overhead). In general this should not be too much of a problem, 
    but, once again, kerning can be used in most cases to overcome 
    any undesirable	effects.
  \item			
    It is only possible to typeset text in math mode under considerable
    restrictions. Math mode is entered and exited using any character	
    of category code 3 (e.g., in plain \TeX{} this is |$|). %$
    Math subscripts and superscripts need to be	contained within braces 
    (e.g., |{^y_i}|) as do commands like |\times| or |\cdot|. 
    However, even modestly complex mathematical	typesetting is unlikely 
    to be sucessful along a path (or even desirable).
  \item
    Some inaccuracies in positioning may be particularly apparent
    at subpath boundaries. This can (unfortunately) only be solved 
    on case by case basis	by individually kerning the offending 
    characters within a group.
  \end{itemize}
  
  The following keys are used by the |text| decoration:
  
  \begin{key}{/pgf/decoration text=\marg{text} (initially \char`\{\char`\})}
    Set the text to typeset along the curve. 
    Consecutive spaces are ignored, so |\ | (or |\space| in \LaTeX) 
    should be used to insert multiple spaces.	It is possible to
    format the text using normal formating commands, such as |\it|, |\bf|
    and |\color|, within customisable delimiters. Initially these
    delimiters are both {\tt\char`\|} (however, care will be needed 
    regarding	the category codes of delimiters --- see below). 

{\catcode`\|12
\begin{codeexample}[]
\catcode`\|12
\begin{tikzpicture}
  \draw [help lines] grid (3,2);
  \path [decorate, decoration=text,	
   decoration text={a big |\color{green}|green|| juicy apple.}] 
    (0,0) .. controls (0,2) and (3,0) .. (3,2);
\end{tikzpicture}
\end{codeexample}
}
  By following the first delimiter
  with |+|, the formatting commands are added to any exisiting 
  formatting.

{\catcode`\|12
\begin{codeexample}[]
\begin{tikzpicture}
  \draw [help lines] grid (3,2);
  \path [decorate, decoration=text,	
     decoration text={a |\large|big |+\bf\color{red}|red|| juicy apple.}] 
    (0,0) .. controls (0,2) and (3,0) .. (3,2);
\end{tikzpicture}
\end{codeexample}
}
	
  Internally, the text is stored in the macro |\pgfdecorationtext|. 
  Any characters that have not been typeset when the end of the 
  path has been reached will be stored in |\pgfdecorationrestoftext|.

\end{key}

{\catcode`\|12
\begin{key}{/pgf/decoration text format delimiters=\marg{before}\marg{after} (initially \char`\{|\char`\}\char`\{\char`\})}

  \catcode`\|13
	
  Set the characters that the text decoration will use to parse 
  formatting commands. 
  If \meta{after} is empty, then \meta{before} will be used for both
  delimiters.
  In general you should stick to characters	whose category codes are 
  |11| or |12|.
  As |+| is used to indicate that the specifed format commands 
  are added	to any exisiting ones, you should avoid using |+| as
  a delimiter. 

\begin{codeexample}[]
\begin{tikzpicture}
  \draw [help lines] grid (3,2);
  \path [decorate, decoration=text, decoration text format delimiters={[}{]}, 
  decoration text={A big [\color{red}]red[] and [\color{green}]green[] apple.}] 
    (0,0) .. controls (0,2) and (3,0) .. (3,2);
\end{tikzpicture}
\end{codeexample}
\end{key}
}

\begin{key}{/pgf/decoration text color=\meta{color} (initially black)}
  Set the color for the text.
\end{key}
\end{decoration}


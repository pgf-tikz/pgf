% Copyright 2006 by Till Tantau
%
% This file may be distributed and/or modified
%
% 1. under the LaTeX Project Public License and/or
% 2. under the GNU Free Documentation License.
%
% See the file doc/generic/pgf/licenses/LICENSE for more details.


\section{Transparency}

\label{section-tikz-transparency}


\subsection{Overview}

Still missing...

Document |fade|, |fading|, |transparency group|, |fading angle|,
|scope fading| and |scope fading bounding box|.

% Normally, when you paint something using any of \pgfname's commands
% (this includes stroking, filling, shading, patterns, and images), the
% newly painted objects totally obscure whatever was painted earlier in
% the same area.

% You can change this behaviour by using something that can be thought
% of as ``(semi)transparent colors.'' Such colors do not completely
% obscure the background, rather they blend the background with the new
% color. At first sight, using such semitransparent colors might seem quite
% straightforward, but the math going on in the background is quite
% involved and the correct handling of transparency fills some 64 pages
% in the PDF specification. 

% In the present section, we start with the different ways of specifying
% ``how transparent'' newly drawn objects should be. The simplest way is
% to just specify a percentage like ``60\% transparent.'' A much more
% general way is to use something that I call a \emph{fading,} also
% known as a soft mask or a mask. Finally, it is possible to use an
% image to specify such a mask in a special way.

% At the end of the section we adress the problem of creating so-called
% \emph{transparency groups}. This problem arises when you paint over a
% position several times with a semitransparent color. Sometimes you
% want the effect to accumulate, sometimes you do not.


% \emph{Note:} Transparency is best supported by the pdf\TeX\
% driver. The \textsc{svg} driver also has some support. For PostScript
% output, opacity is rendered correctly only with the most recent
% versions of GhostScript. Printers and other programs will typically
% ignore the opacity setting. 


\subsection{Specifying a Uniform Opacity}

% Specifying a stroke and/or fill opacity is quite easy.

% \begin{command}{\pgfsetstrokeopacity\marg{value}}
%   Sets the opacity of stroking operations. The \meta{value} should be
%   a number between |0| and |1|, where |1| means ``fully opaque'' and
%   |0| means ``fully transparent.'' A value like |0.5| will cause paths
%   to be stroked in a semitransparent way.
  
% \begin{codeexample}[]
% \begin{pgfpicture}
%   \pgfsetlinewidth{5mm}
%   \color{red}
%   \pgfpathcircle{\pgfpoint{0cm}{0cm}}{10mm} \pgfusepath{stroke}
%   \color{black}
%   \pgfsetstrokeopacity{0.5}
%   \pgfpathcircle{\pgfpoint{1cm}{0cm}}{10mm} \pgfusepath{stroke}
% \end{pgfpicture}
% \end{codeexample}
% \end{command}


% \begin{command}{\pgfsetfillopacity\marg{value}}
%   Sets the opacity of filling operations. As for stroking, the
%   \meta{value} should be a number between |0| and~|1|.

%   The ``filling transparency'' will also be used for text and images.  
  
% \begin{codeexample}[]
% \begin{tikzpicture}
%   \pgfsetfillopacity{0.5}
%   \fill[red]   (90:1cm)  circle (11mm);
%   \fill[green] (210:1cm) circle (11mm);
%   \fill[blue]  (-30:1cm) circle (11mm);
% \end{tikzpicture}
% \end{codeexample}
% \end{command}

% Note the following effect: If you setup a certain opacity for stroking
% or filling and you stroke or fill the same area twice, the effect
% accumulates:

% \begin{codeexample}[]
% \begin{tikzpicture}
%   \pgfsetfillopacity{0.5}
%   \fill[red] (0,0) circle (1);
%   \fill[red] (1,0) circle (1);
% \end{tikzpicture}
% \end{codeexample}

% Often, this is exactly what you intend, but not always. You can use
% transparency groups, see the end of this section, to change this.


\subsection{Specifying a Fading}


% After having declared a fading, we can use it. As for shadings, there
% are two different commands for using fadings:

% \begin{command}{\pgfsetfading\marg{name}\marg{transformations}}
%   This command sets the graphic state parameter ``fading'' to a
%   previously defined fading \meta{name}. This graphic state works like
%   other graphic states, that is, is persists till the end of the
%   current scope or until a different transparency setting is chosen.

%   When the fading is installed, it will be centered on the origin with
%   its natural size. Anything outside the fading pictures's original
%   bounding box will be transparent and, thus, the fading effectively
%   clips against this bounding box.

%   The \meta{transformations} are applied to the fading before it is
%   used. They contain normal \pgfname\ transformation commands like
%   |\pgftransformshift|. You can also scale the fading using this
%   command. Note, however, that the transformation needs to be inverted
%   internally, which may result in inaccuracies and the following
%   graphics may be slightly distorted if you use a strong
%   \meta{transformation}.
% \begin{codeexample}[]
% \pgfdeclarefading{fading2}
% {\tikz \shade[left color=pgftransparent!0,
%               right color=pgftransparent!100] (0,0) rectangle (2,2);}    
% \begin{tikzpicture}
%   \fill [black!20] (0,0) rectangle (2,2);
%   \fill [black!30] (0,0) arc (180:0:1);
%   \pgfsetfading{fading2}{}
%   \fill [red] (0,0) rectangle (2,2);
% \end{tikzpicture}
% \end{codeexample}
% \begin{codeexample}[]
% \begin{tikzpicture}
%   \fill [black!20] (0,0) rectangle (2,2);
%   \fill [black!30] (0,0) arc (180:0:1);
%   \pgfsetfading{fading2}{\pgftransformshift{\pgfpoint{1cm}{1cm}}
%                          \pgftransformrotate{20}}
%   \fill [red] (0,0) rectangle (2,2);
% \end{tikzpicture}
% \end{codeexample}
% \end{command}

% \begin{command}{\pgfsetfadingforcurrentpath\marg{name}\marg{transformations}}
%   This command works like |\pgfsetfading|, but the fading is scaled
%   are transformed according to the following rules:
%   \begin{enumerate}
%   \item
%     It is assumed that the fading has a size of 100bp times 100bp.
%   \item
%     The fading is resized and shiften (using appropriate
%     transformations) such that the position
%     $(25\mathrm{bp},25\mathrm{bp})$ lies at the lower-left corner of
%     the current path and the position $(75\mathrm{bp},75\mathrm{bp})$
%     lies at the upper-right corner of the current path.
%   \end{enumerate}
%   Note that these rules are the same as the ones used in
%   |\pgfshadepath| for shadings. After these transformations, the
%   \meta{transformations} are executed (typically a rotation).
% \begin{codeexample}[]
% \pgfdeclarehorizontalshading{shading}{100bp}
% { color(0pt)=(transparent!0);    color(25bp)=(transparent!0);
%   color(75bp)=(transparent!100); color(100bp)=(transparent!100)}

% \pgfdeclarefading{fading}{\pgfuseshading{shading}}

% \begin{tikzpicture}
%   \fill [black!20] (0,0) rectangle (2,2);
%   \fill [black!30] (0,0) arc (180:0:1);

%   \pgfpathrectangle{\pgfpointorigin}{\pgfpoint{2cm}{1cm}}
%   \pgfsetfadingforcurrentpath{fading}{}
%   \pgfusepath{discard}
  
%   \fill [red] (0,0) rectangle (2,1);

%   \pgfpathrectangle{\pgfpoint{0cm}{1cm}}{\pgfpoint{2cm}{1cm}}
%   \pgfsetfadingforcurrentpath{fading}{\pgftransformrotate{90}}
%   \pgfusepath{discard}

%   \fill [red] (0,1) rectangle (2,2);
% \end{tikzpicture}
% \end{codeexample}

% \end{command}

\subsection{Transparency Groups}

% Consider the following cross and sign. They ``look wrong'' because we
% can see how they were constructed, while this is not really part of
% the desired effect. 

% \begin{codeexample}[]
% \begin{tikzpicture}
%   \pgfsetstrokeopacity{0.5}
%   \draw [line width=5mm] (0,0) -- (2,2);
%   \draw [line width=5mm] (2,0) -- (0,2);
% \end{tikzpicture}
% \end{codeexample}

% \begin{codeexample}[]
% \begin{tikzpicture}
%   \node at (0,0) [forbidden sign,line width=2ex,draw=red,fill=white] {Smoking};
%   \pgfsetstrokeopacity{0.5}
%   \pgfsetfillopacity{0.5}
%   \node at (2,0) [forbidden sign,line width=2ex,draw=red,fill=white] {Smoking};
% \end{tikzpicture}
% \end{codeexample}

% Transparency groups are used to render them correctly:

% \begin{codeexample}[]
% \begin{tikzpicture}
%   \pgfsetfillopacity{0.5}
%   \pgftransparencygroup
%     \draw [line width=5mm] (0,0) -- (2,2);
%     \draw [line width=5mm] (2,0) -- (0,2);
%   \endpgftransparencygroup
% \end{tikzpicture}
% \end{codeexample}

% \begin{codeexample}[]
% \begin{tikzpicture}
%   \node at (0,0) [forbidden sign,line width=2ex,draw=red,fill=white] {Smoking};
%   \pgfsetfillopacity{0.5}
%   \pgftransparencygroup
%     \node at (2,0) [forbidden sign,line width=2ex,draw=red,fill=white]
%       {Smoking};
%   \endpgftransparencygroup
% \end{tikzpicture}
% \end{codeexample}


% \begin{environment}{{pgftransparencygroup}}
%   This environment should only be used inside a |{pgfpicture}|. It has
%   the following effect:
%   \begin{enumerate}
%   \item The \meta{environment contents} is stroked/filled
%     ``ignoring any outside transparency.'' This means, all previous
%     transparency settings are ignored (you can still set transparency
%     inside the group, but never mind). This means that if in the
%     \meta{environment contents} you stroke a pixel three times in
%     black, it is just black. Stroking it white afterwards yields a
%     white pixel, and so on.
%   \item When the group is finished, it is painted as a whole. The 
%     \emph{fill} transparency settings are now applied to the resulting
%     picutre. For instance, the pixel that has been painted three times
%     in black and once in white is just white at the end, so this white
%     color will be blended with whatever is ``behind'' the group on the
%     page.
%   \end{enumerate}

%   Note that, depending on the driver, \pgfname\ may have to guess the
%   size of the contents of the transparency group (because such a group
%   is put in an XForm in \textsc{pdf} and a bounding box must be
%   supplied). \pgfname\ will use normally use the size of the picture's
%   bounding box at the end of the transparency group plus a safety
%   margin of 1cm. Under normal circumstances, this will work nicely
%   since the picture's bounding box contains everything
%   anyway. However, if you have switched off the picture size tracking
%   or if you are using canvas transformations, you may have to make
%   sure that the bounding box is big enough. The trick is to locallly
%   create a picture that is ``large enough'' and then insert this
%   picture into the main picture while ignoring the size. The following
%   example shows how this is done:

  
% \begin{codeexample}[]
% \begin{tikzpicture}
%   \draw [help lines] (0,0) grid (2,2);

%   % Stuff outside the picture, but still in a transparency group.
%   \node [left,overlay] at (0,1) {
%     \begin{tikzpicture}
%       \pgfsetfillopacity{0.5}
%       \pgftransparencygroup
%       \node at (2,0) [forbidden sign,line width=2ex,draw=red,fill=white]
%         {Smoking};
%       \endpgftransparencygroup
%     \end{tikzpicture}  
%   };
% \end{tikzpicture}
% \end{codeexample}


% \begin{plainenvironment}{{pgftransparencygroup}}
%   Plain \TeX\ version of the |{pgftransparencygroup}| environment.
% \end{plainenvironment}

% \begin{contextenvironment}{{pgftransparencygroup}}
%   This is the Con\TeX t version of the environment.
% \end{contextenvironment}

% \end{environment}



%%% Local Variables: 
%%% mode: latex
%%% TeX-master: "pgfmanual"
%%% End: 

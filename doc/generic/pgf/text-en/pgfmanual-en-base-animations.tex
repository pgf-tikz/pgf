% Copyright 2015 by Till Tantau
%
% This file may be distributed and/or modified
%
% 1. under the LaTeX Project Public License and/or
% 2. under the GNU Free Documentation License.
%
% See the file doc/generic/pgf/licenses/LICENSE for more details.


\section{Animations}

\label{section-base-animations}

\begin{pgfmodule}{animations}
  This module contains the basic layer support of animations, which is
  documented in the following. 
\end{pgfmodule}


\subsection{Overview}

An \emph{animation} changes the way some part of a graphic looks like
over time. The archetypical animation is, of course, a \emph{movement}
of node, but a change of, say, the opacity of a path is also an
animation. \pgfname\ allows you to specify such animations using a set
of commands and keys that are documented in the following.

Unlike other packages, the animations created by \pgfname\ are not
precomputed sequences of pictures that are displayed in rapid
succession. Rather, an animation created by \pgfname\ consists mainly
of an annotation in the output that a certain attribute of a certain
object should change over time in some specific way when the object is
displayed. It is the job of the document viewer application to
actually compute and display the animation. Interestingly, this means
that animations neither increase the size of the ouput files noticably
nor does it put a special burden on \TeX. The hard and complicated
calculations are done by the viewer application, not by \TeX\ and
\pgfname. 

Only few viewer applicaitons and formats are currently ``up to the
job'' of displaying animations.  In particular, 
the popular \textsc{pdf} format does \emph{not} allow one to specify
animations in this way (one can partly ``fake'' animations at the
high price of including a great number of precomputed pictures and
using JavaScript in special viewers, but this is really not the same
thing as what \pgfname\ does). Indeed, currently only the  \textsc{svg}
format allows one to specify animations in a sensible way. Thus,
\pgfname's animations will only be displayed when \textsc{svg} is used
as output format.

Creating an animation is done using the following command:

\begin{command}{\pgfanimateattribute\marg{attribute}\marg{options}}
  Adds an animation of the \meta{attribute} of a future \emph{object}
  to the current graphic. Attributes are things like the ``fill
  opacity'' or the transformation matrix or the line width.

  The \meta{options} are keys that configure how the attribute changes
  over time. Using the |entry| key multiple times you specify which
  value the chosen attribute should have at different points in
  time. Other keys, like |repeats|, allow you to specify how the
  animation behaves ``as a whole''. These keys are documented later in
  this section.

\begin{codeexample}[]
\tikz {
  \pgfanimateattribute{opacity}{
    whom = node, begin on = {click}, entry = {0s}{1}, entry = {2s}{0} }
  \node (node) [fill = blue!20, draw = blue, very thick, circle] {Click me!};
}
\end{codeexample}

  \medskip
  \textbf{The Attributes}
  
  In detail, the |\pgfanimateattribute| command opens a \TeX-scope,
  looks up the \emph{type} values of the specified \meta{attribute}
  have (if you wish to animate the |opacity| of an object, the type is
  ``scalar'' meaning that entries must be scalar numbers; when you
  animate the |fill| atttribute, the type is ``color'' and values 
  must be colors, and so on), and the executed the \meta{options} with
  the path prefix |/pgf/animation|. Finally, an appropriate system
  layer command |\pgfsysanimate...| is called to create the actual
  animation.

  The following \meta{attributes} are permissible:
  
  \begin{tabular}{ll}
    \emph{Attribute} & \emph{Type} \\
    |draw|, |fill|              & color \\
    |line width|                & dimension \\
    |motion|                    & scalar \\
    |opacity|, |fill opacity|, |draw opacity|              & scalar \\
    |path|                      & path \\
    |rotate|                    & scalar \\
    |scale|                     & scaling \\
    |softpath|                  & softpath \\
    |translate|                 & point \\
    |view|                      & viewbox \\
    |visible|                   & text \\
    |xskew|, |yskew|            & scalar \\
  \end{tabular}

  These attributes are detailed in
  Sections \ref{section-base-animation-painting}
  to~\ref{section-base-animation-views}, but here is a quick overview:
  \begin{itemize}
  \item |draw| and |fill| refer to the color used to
    draw (stroke) and fill paths in an object, respectively. Typical
    values for this attribute are |red| or |black!10|.
  \item |line width| is, of course, the line width used in an
    object. Typical values are |0.4pt| or |1mm|. Note that you
    (currently) cannot use keys like |thin| or |thick| here, but this
    may change in the future.
  \item |motion| is a slightly special attribute: It allows you to
    specify a path along which the object should be moved (using the
    |along| key). The values given to the |entry| key for this
    attribute refer to a \emph{fraction of the distance along the
      path}. See the |along| key for details.
  \item |opacity| and the variants |fill opacity| and |draw opacity|
    animate the opacity of an object. Allowed values range between 0
    and 1.
  \item |path| allows you to animate a graphs path (it will
    morph). The ``values'' are now paths themselves. See
    Section~\ref{section-base-animation-paths} for details.
  \item |rotate| refers to a rotation of the object. Values for the
    |entry| key are the rotation angles like |0| or |90|.
  \item |scale| refers to the scaling of the object. Values are either
    single scalars values (like |1| or  |1.5|) or two numbers
    separated by a comma (lile |1,1.5| or |0.5,2|), referring to the
    $x$-scaling and $y$-scaling.
  \item |softpath| is a special case of the |path| attribute, see
    Section~\ref{section-base-animation-paths} once more.
  \item |translate| shifts the object by a certain vector. Values are
    points like |\pgfpoint{1cm}{2cm}|.
  \item |view| allows you to animate the view box of a view, see
    Section~\ref{section-base-animation-views} for details.
  \item |visible| refers to the visibility of an object. Allowed
    values are |visible| and |hidden|.
  \item |xskew| and |yskew| skew the object. Attributes are angles
    like |0| or |45| or even |90|.
  \end{itemize}

  \medskip
  \textbf{The Target Object}

  As stated earlier, the \meta{options} are used to specify the object
  whose attribute for which an animation should be added to the
  picture. Indeed, you \emph{must} specify the object explicitly using
  the |whom| key and you must do so \emph{before} the object is
  created. Note that, in contrast, in \textsc{svg} you can specify an
  animation more or less anywhere and then use hyper-references to
  link the animation to the to-be-animated object; \pgfname\ insists
  that you specify the animation before the object. This is a bit of a
  bother in some situations, but it is the only way to ensure that
  \pgfname\ has a fighting chance to attach some additional code to
  the object (which is necessary for almost all animations of the
  transformation matrix).

  \begin{key}{/pgf/animation/whom=\meta{id}\opt{|.|\meta{type}}}
    You \emph{must} use this key once which each call of the
    |\pgfanimateattribute| command. The \meta{id} and the optional
    \meta{type} (which is whatever follows the first dot) will be
    passed to |\pgfidrefnextuse|, see that command for details. 
  \end{key}

  As explained in the introduction of this chapter, an ``animation''
  is just a bit of special text in the output document asking a viewer
  application to animate the object at some later time. The
  |\pgfanimateattribute| command inserts this special text immediately,
  even though it refers to an object created only later on. Normally,
  this is not a problem, but the special text should be on the same
  page as the to-be-animated object. To ensure this, it suffices to
  call |\pgfanimateattribute| no earlier than the beginning of the
  |pgfpicture| containing the object.
\end{command}


\subsubsection{Animating Color, Opacity, and Visibility}
\label{section-base-animation-painting}

You can animate the color of the target object of an animation using
the attributes |fill| or |draw|, which animate the fill color and the
drawing (stroking) color, respectively. To animate both the fill and
draw color, you need to create two animations, one for each.

\begin{codeexample}[width=2.3cm]
\tikz {
  \pgfanimateattribute{fill}{
    whom = node.background, begin on = {click}, entry = {0s}{white}, entry = {2s}{red} }
  \node (node) [fill = blue!20, draw = blue, very thick, circle] {Click me!}; 
}
\end{codeexample}

\begin{codeexample}[width=2.3cm]
\tikz {
  \pgfanimateattribute{draw}{
    whom = node.background, begin on = {click}, entry = {0s}{white}, entry = {2s}{red} }
  \node (node) [fill = blue!20, draw = blue, very thick, circle] {Click me!}; 
}
\end{codeexample}

When the target of a color animation is a scope, you animate
the color ``used in this scope'' for filling or stroking. However,
when an object inside the scope has its color set explicitly, this
color overrules the color of the scope:

\begin{codeexample}[]
\tikz {
  \pgfanimateattribute{fill}{
    whom = example, begin on = {click, of next=node},
    entry = {0s}{white}, entry = {2s}{red} }
  \node (node) [fill = blue!20, draw = blue, very thick, circle] {Click me!}; 
  \begin{scope}[name = example]
    \fill (1.1,0) rectangle ++ (1,1);
    \fill [blue] (1.5,0.5) rectangle ++ (1,1);
  \end{scope}
}
\end{codeexample}

Note that in certain cases, a graphic scope may contain graphic
objects with their colors set explicitly ``in places where you do not
expect it'': In particular, a node normally consists at least of a
background path and a text. For both the text and for the background
path, colors will be set for the text and also for the path
explicitly. This means that when you pick the fill attribute of a node
as the target of an animation, you will \emph{not} animate the color
of the background path in case this color has been set
explicitly. Instead, you must choose the background path of the node
as the target of the animation. Fortunately, this is easy to achieve
since when the background path of a node is created, the identifier
type is set to |background|, which in turn allows you to access it as
\meta{node}|.background| through the |whom| key.

The text of a node also gets it color set explicitly, which means that
a change of the node's scope's color has no effect on the text
color. Instead, you must choose \meta{name}|.text| as the target (or,
if the node has more parts, use the name of the part as the identifier
type instead of |text|).

\begin{codeexample}[]
\tikz {
  \pgfanimateattribute{fill}{
    whom = example, begin on = {click, of next=node},
    entry = {0s}{white}, entry = {2s}{red} }
  \node (node) [fill = blue!20, draw = blue, very thick, circle] {Click me!}; 
  \node at (2,0) (example) [fill = blue!20, circle] {No effect}; }
\end{codeexample}


\begin{codeexample}[]
\tikz {
  \pgfanimateattribute{fill}{
    whom = example.background, begin on = {click, of next=node},
    entry = {0s}{white}, entry = {2s}{red} }
  \node (node) [fill = blue!20, draw = blue, very thick, circle] {Click me!}; 
  \node at (2,0) (example) [fill = blue!20, circle] {Effect}; }
\end{codeexample}


\begin{codeexample}[]
\tikz {
  \pgfanimateattribute{fill}{
    whom = example.text, begin on = {click, of next=node},
    entry = {0s}{white}, entry = {2s}{red} }
  \node (node) [fill = blue!20, draw = blue, very thick, circle] {Click me!}; 
  \node at (2,0) (example) [fill = blue!20, circle, font=\huge] {Text}; }
\end{codeexample}

Similarly to the color, you can also set the opacity used for filling
and for drawing:

\begin{codeexample}[width=2.3cm]
\tikz {
  \pgfanimateattribute{fill opacity}{
    whom = node, begin on = {click}, entry = {0s}{1}, entry = {2s}{0} }
  \node (node) [fill = blue!20, draw = blue, very thick, circle] {Click me!}; 
}
\end{codeexample}

\begin{codeexample}[width=2.3cm]
\tikz {
  \pgfanimateattribute{draw opacity}{
    whom = node, begin on = {click}, entry = {0s}{1}, entry = {2s}{0} }
  \node (node) [fill = blue!20, draw = blue, very thick, circle] {Click me!}; 
}
\end{codeexample}

Unlike colors, where there is no joint attribute for filling and
stroking, there is a single |opacity| attribute in addition to the above
two attributes. If supported by the driver, it treats the graphic
object to which it is applied as a transparency group. In essence,
``this attribute does what you want'' at least in most situations.

\begin{codeexample}[width=2.3cm]
\tikz {
  \pgfanimateattribute{opacity}{
    whom = node, begin on = {click}, entry = {0s}{1}, entry = {2s}{0} }
  \node (node) [fill = blue!20, draw = blue, very thick, circle] {Click me!}; 
}
\end{codeexample}

The last attribute that concerns colors and visibility is the
Boolean |visible| attribute. The difference to an opacity of |0| is
that an invisible object cannot be clicked and does not need to be
rendered. 

\begin{codeexample}[width=2.3cm]
\tikz {
  \pgfanimateattribute{visible}{
    whom = node, begin on = {click}, entry = {0s}{false}, entry = {2s}{false} }
  \node (node) [fill = blue!20, draw = blue, very thick, circle] {Click me!}; 
}
\end{codeexample}


\subsubsection{Animating Paths and their Rendering}
\label{section-base-animation-paths}

You can animate the appearance of a path in the following ways:

\begin{itemize}
\item You can animate the line width:
\begin{codeexample}[width=2.3cm]
\tikz [very thick] {
  \pgfanimateattribute{line width}{
    whom = node, begin on = {click}, entry = {0s}{1pt}, entry = {2s}{5mm} }
  \node (node) [fill = blue!20, draw = blue, circle] {Click me!}; 
}
\end{codeexample}
\item You can animate the dash phase:
\begin{codeexample}[width=2.3cm]
\tikz {
  \pgfanimateattribute{dash phase}{
    whom = node.background, begin on = {click}, entry = {0s}{0pt}, entry = {2s}{10pt} }
  \node (node) [dashed, fill = blue!20, draw = blue, very thick, circle] {Click me!}; 
}
\end{codeexample}
\item You can animate the dash pattern:
\begin{codeexample}[width=2.3cm]
\tikz {
  \pgfanimateattribute{dash pattern}{
    whom = node, begin on = {click}, entry = {0s}{{10pt}{1pt}}, entry = {2s}{{1pt}{10pt}} }
  \node (node) [fill = blue!20, draw = blue, very thick, circle] {Click me!}; 
}
\end{codeexample}
\item Finally, you can animate the path itself:
\begin{codeexample}[width=2.3cm]
\tikz {
  \pgfanimateattribute{path}{
    whom = node.background.path, begin on = {click, of next=node},
    entry = {0s}{\pgfpathellipse{\pgfpointorigin}{\pgfpointxy{2}{0}}{\pgfpointxy{0}{1}}},
    entry = {2s}{\pgfpathellipse{\pgfpointxy{1}{0}}{\pgfpointxy{1}{1}}{\pgfpointxy{0.25}{.75}}}}
  \node (node) [fill = blue!20, draw = blue, very thick, circle] {Click me!}; 
}
\end{codeexample}
\end{itemize}










\subsubsection{Animating Transformations  and Views}
\label{section-base-animation-views}


\subsubsection{Commands for Specifying Timelines: Specifying Times}

\subsubsection{Commands for Specifying Timelines: Specifying Values}

\subsubsection{Commands for Specifying Timing: Repeats}

\subsubsection{Commands for Specifying Timing: Beginning and Ending}

\subsubsection{Commands for Specifying Timing: Restart Behaviour}

\subsubsection{Commands for Specifying Accumulation}

\endinput



%%% Local Variables: 
%%% mode: latex
%%% TeX-master: "pgfmanual"
%%% End: 

% Copyright 2015 by Till Tantau
%
% This file may be distributed and/or modified
%
% 1. under the LaTeX Project Public License and/or
% 2. under the GNU Free Documentation License.
%
% See the file doc/generic/pgf/licenses/LICENSE for more details.


\section{Animations}

\label{section-base-animations}

\begin{pgfmodule}{animations}
  This module contains the basic layer support of animations, which is
  documented in the following. 
\end{pgfmodule}


\subsection{Overview}

An \emph{animation} changes the way some part of a graphic looks like
over time. The archetypical animation is, of course, a \emph{movement}
of node, but a change of, say, the opacity of a path is also an
animation. \pgfname\ allows you to specify such animations using a set
of commands and keys that are documented in the following.

Differently from other packages, the animations created by \pgfname\ are not
precomputed sequences of pictures that are displayed in rapid
succession. Rather, an animation created by \pgfname\ consists mainly
of an annotation in the output that a certain attribute of a certain
object should change over time in some specific way when the object is
displayed. It is the job of the document viewer application to
actually compute and display the animation. Interestingly, this means
that animations neither increase the size of the ouput files noticably
nor does it put a special burden on \TeX. The hard and complicated
calculations are done by the viewer application, not by \TeX\ and
\pgfname. 

Only few viewer applicaitons and formats are currently ``up to the
job'' of displaying animations.  In particular, 
the popular \textsc{pdf} format does \emph{not} allow one to specify
animations in this way (one can partly ``fake'' animations at the
high price of including a great number of precomputed pictures and
using JavaScript in special viewers, but this is really not the same
thing as what \pgfname\ does). Indeed, currently only the  \textsc{svg}
format allows one to specify animations in a sensible way. Thus,
\pgfname's animations will only be displayed when \textsc{svg} is used
as output format.

Creating an animation is done using the following command:

\begin{command}{\pgfanimateattribute\marg{attribute}\marg{options}}
  Adds an animation of the \meta{attribute} of a future \emph{object}
  to the current graphic. Attributes are things like the ``fill
  opacity'' or the transformation matrix or the line width.

  The \meta{options} are keys that configure how the attribute changes
  over time. Using the |entry| key multiple times you specify which
  value the chosen attribute should have at different points in
  time. Other keys, like |repeats|, allow you to specify how the
  animation behaves ``as a whole''. These keys are documented later in
  this section.

\begin{codeexample}[]
\tikz {
  \pgfanimateattribute{opacity}{
    whom = node, begin on = {click}, entry = {0s}{1}, entry = {2s}{0} }
  \node (node) [fill = blue!20, draw = blue, very thick, circle] {Click me!};
}
\end{codeexample}

  \medskip
  \textbf{The Attributes}
  
  In detail, the |\pgfanimateattribute| command opens a \TeX-scope,
  looks up the \emph{type} values of the specified \meta{attribute}
  have (if you wish to animate the |opacity| of an object, the type is
  ``scalar'' meaning that entries must be scalar numbers; when you
  animate the |fill| atttribute, the type is ``color'' and values 
  must be colors, and so on), and then executes the \meta{options} with
  the path prefix |/pgf/animation|. Finally, an appropriate system
  layer command |\pgfsysanimate...| is called to create the actual
  animation and the scope is closed.

  The following \meta{attributes} are permissible:
  
  \begin{tabular}{ll}
    \emph{Attribute} & \emph{Type} \\
    |draw|, |fill|              & color \\
    |line width|                & dimension \\
    |motion|                    & scalar \\
    |opacity|, |fill opacity|, |draw opacity|              & scalar \\
    |path|                      & path \\
    |rotate|                    & scalar \\
    |scale|                     & scaling \\
    |softpath|                  & softpath \\
    |translate|                 & point \\
    |view|                      & viewbox \\
    |visible|                   & boolean \\
    |xskew|, |yskew|            & scalar \\
  \end{tabular}

  These attributes are detailed in
  Sections \ref{section-base-animation-painting}
  to~\ref{section-base-animation-views}, but here is a quick overview:
  \begin{itemize}
  \item |draw| and |fill| refer to the color used to
    draw (stroke) and fill paths in an object, respectively. Typical
    values for this attribute are |red| or |black!10|.
  \item |line width| is, of course, the line width used in an
    object. Typical values are |0.4pt| or |1mm|. Note that you
    (currently) cannot use keys like |thin| or |thick| here, but this
    may change in the future.
  \item |motion| is a slightly special attribute: It allows you to
    specify a path along which the object should be moved (using the
    |along| key). The values given to the |entry| key for this
    attribute refer to a \emph{fraction of the distance along the
      path}. See the |along| key for details.
  \item |opacity| and the variants |fill opacity| and |draw opacity|
    animate the opacity of an object. Allowed values range between 0
    and 1.
  \item |path| allows you to animate a graphs path (it will
    morph). The ``values'' are now paths themselves. See
    Section~\ref{section-base-animation-paths} for details.
  \item |rotate| refers to a rotation of the object. Values for the
    |entry| key are the rotation angles like |0| or |90|.
  \item |scale| refers to the scaling of the object. Values are either
    single scalars values (like |1| or  |1.5|) or two numbers
    separated by a comma (lile |1,1.5| or |0.5,2|), referring to the
    $x$-scaling and $y$-scaling.
  \item |softpath| is a special case of the |path| attribute, see
    Section~\ref{section-base-animation-paths} once more.
  \item |translate| shifts the object by a certain vector. Values are
    points like |\pgfpoint{1cm}{2cm}|.
  \item |view| allows you to animate the view box of a view, see
    Section~\ref{section-base-animation-views} for details.
  \item |visible| refers to the visibility of an object. Allowed
    values are |visible| and |hidden|.
  \item |xskew| and |yskew| skew the object. Attributes are angles
    like |0| or |45| or even |90|.
  \end{itemize}

  \medskip
  \textbf{The Target Object}

  As stated earlier, the \meta{options} are used to specify the object
  whose attribute for which an animation should be added to the
  picture. Indeed, you \emph{must} specify the object explicitly using
  the |whom| key and you must do so \emph{before} the object is
  created. Note that, in contrast, in \textsc{svg} you can specify an
  animation more or less anywhere and then use hyper-references to
  link the animation to the to-be-animated object; \pgfname\ insists
  that you specify the animation before the object. This is a bit of a
  bother in some situations, but it is the only way to ensure that
  \pgfname\ has a fighting chance to attach some additional code to
  the object (which is necessary for almost all animations of the
  transformation matrix).

  \begin{key}{/pgf/animation/whom=\meta{id}\opt{|.|\meta{type}}}
    You \emph{must} use this key once which each call of the
    |\pgfanimateattribute| command. The \meta{id} and the optional
    \meta{type} (which is whatever follows the first dot) will be
    passed to |\pgfidrefnextuse|, see that command for details. 
  \end{key}

  As explained in the introduction of this chapter, an ``animation''
  is just a bit of special text in the output document asking a viewer
  application to animate the object at some later time. The
  |\pgfanimateattribute| command inserts this special text immediately,
  even though it refers to an object created only later on. Normally,
  this is not a problem, but the special text should be on the same
  page as the to-be-animated object. To ensure this, it suffices to
  call |\pgfanimateattribute| no earlier than the beginning of the
  |pgfpicture| containing the object.
\end{command}


\begin{command}{\pgfanimateattributecode\marg{attribute}\marg{code}}
  The command works like |\pgfanimateattribute|, only instead of
  \meta{options} you specify some \meta{code} whose job is to setup
  the options. 
\end{command}


\subsubsection{Specifying the Timeline}

The core key for specifying how an attribute varies over time is the
|entry| key:
\begin{key}{/pgf/animation/entry=\marg{time}\marg{value}}
  You use this key repeatedly to specify the different values that
  the \meta{attribute} should have over time. At the \meta{time}
  specified, the \meta{attribute} will have the value specified as
  \meta{value}:
  
\begin{codeexample}[]
\tikz {
  \pgfanimateattribute{rotate}{
    whom = node, begin on = {click},
    entry = {0s}{0}, entry = {1s}{90}, entry = {1.1s}{45}, entry = {2s}{90}
  }
  \node (node) [fill = blue!20, draw = blue, very thick, circle] {Click me!};
}
\end{codeexample}

  You need to call |entry| once for each time in the timeline for
  which you want to specify a \meta{value} explicitly. Between these
  times, the values get interpolated (see below for details). You need
  to specify the \meta{time}s in non-decreasing order (it is
  permissible and sometimes also necessary to specify the same time
  twice, namely to create a ``jump'' of the value of some attribute).

  The \meta{time} is parsed using the command |\pgfparsetime|
  described later.

  \medskip
  \textbf{Start and end of the timeline.}

  The first and last times of the timeline are a bit special: The
  timeline starts on the first time and the duration of the timeline is
  the difference between the first and last time. ``Starting'' on the
  start time actually means that any beginnings (see the |begin| and
  |end| keys) get offset by the start time; similarly end times are
  offset by this value. 

  \medskip
  \textbf{Syntax of the values.}

  The syntax of the \meta{value} varies according to the type of the
  \meta{attribute}. In detail, these are:

  \begin{tabular}{lp{12cm}}
    \emph{Type} & \emph{Syntax} \\
    color & Standard color syntax like |red| or |black!10| \\
    scalar & A value parsed using |\pgfmathparse| \\
    dimension & A dimension parsed using |\pgfmathparse| \\
    path & A sequence of path construction commands \\
    softpath & A sequence of soft path construction commands \\
    scaling & A scalar value or a pair of scalar values separated by a
              comma \\
    point & A \pgfname-point like |\pgfpoint{1cm}{5mm}| \\
    viewbox & Two \pgfname-points \\
    boolean & |true| or |false| \\
  \end{tabular}  

  \medskip
  \textbf{Interpolation between key times.}

  You use the |entry| key repeatedly, namely once for each ``key
  time,'' which is a time point for which you specify the value of the
  attribute explicitly. Between these key times, the attribute's value
  is interpolated.
  
  Suppose that an animation is supposed to interpolate a attribute's
  value between the two values |50| and |100| over a time of
  10\,s. The   simplest way of doing so is to do a linear
  interpolation, where the value as, say, 1\,s is 55, at 2\,s it is
  60, and so on. Unfortunately, the linear interpolation does not
  ``look'' nice in many cases since the acceleration of a linear
  interpolation is zero  during the animation, but infinite at the
  beginning and at the end; which looks ``jerky.''

  To avoid this, you can specify that the time--attribute curve should 
  not be a straight line, but rather a curve. You specify this curve
  using a spline. The most logical ``coordinate rectangle'' used for
  this spline in our example would be |(0s,50)| and |(10s,100)| and we
  would like to specify something like
\begin{codeexample}[code only]
  (0s,50) .. controls (5s,50) and (9s,100) .. (10s,100)
\end{codeexample}
  This would result in a time--attribute curve where
  the attribute at |50| changes slowly at 0\,s and also arrives slowly
  at |100| at 10\,s, but speeds up between these values.

  We call the first control point |(5s,50)| the ``exit control'' and
  call |(9s,100)| the ``entry control'': The first control dictates
  how quickly or slowly a time point is left, the second dictates how
  quickly or slowly we enter the next one.

  The control points are, however, not specified in the coordinate
  system indicated above. Rather, the rectangle |(0s,50)| to
  |(10s, 100)| gets normalized to |(0,0)| to |(1,1)|. The control
  point |(5s,50)| would thus become |(0.5,0)| and |(9s,100)| becomes
  |(0.9,1)|. 
  
  \begin{key}{/pgf/animations/exit control=\marg{time
        fraction}\marg{value fraction}}
    Specifies an exit control using two values as above. The spline
    from above would be specified as follows:

\begin{codeexample}[code only]
exit control={0.5}{0},
entry control={0.9}{1},
entry={0s}{50},
entry={10s}{100}
\end{codeexample}

    Note that the curve specified using exit and entry controls must
    be ``well-behaved'' in the sense that exactly one value must be
    specified for each point in time in the time interval.

    In the next three example, we first specify a ``smooth'' exit from
    the start position, then a smooth arrival at the end position,
    and, finally both.
    
\begin{codeexample}[width=2.3cm]
\tikz {
  \pgfanimateattribute{translate}{
    whom = node, begin on = {click},
    exit control={0.5}{0},
    entry = {0s}{\pgfpointorigin},
    linear, 
    entry = {1s}{\pgfpoint{0cm}{-1cm}},
    entry = {2s}{\pgfpoint{0cm}{-2cm}} }
  \node (node) [fill = blue!20, draw = blue, very thick, circle] {Click me!}; 
}
\end{codeexample}

\begin{codeexample}[width=2.3cm]
\tikz {
  \pgfanimateattribute{translate}{
    whom = node, begin on = {click},
    entry = {0s}{\pgfpointorigin},
    entry = {1s}{\pgfpoint{0cm}{-1cm}},
    entry control={.5}{1},
    entry = {2s}{\pgfpoint{0cm}{-2cm}} }
  \node (node) [fill = blue!20, draw = blue, very thick, circle] {Click me!}; 
}
\end{codeexample}

\begin{codeexample}[width=2.3cm]
\tikz {
  \pgfanimateattribute{translate}{
    whom = node, begin on = {click},
    exit control={0.5}{0},
    entry = {0s}{\pgfpointorigin},
    linear, % revert to default
    entry = {1s}{\pgfpoint{0cm}{-1cm}},
    entry control={.5}{1},
    entry = {2s}{\pgfpoint{0cm}{-2cm}} }
  \node (node) [fill = blue!20, draw = blue, very thick, circle] {Click me!}; 
}
\end{codeexample}
  \end{key}

  \begin{key}{/pgf/animations/entry control=\marg{time
        fraction}\marg{value fraction}}
    Works like |exit control|.
  \end{key}

  \begin{key}{/pgf/animations/linear}
    A shorthand for |exit control={0}{0}, entry control={1}{1}|. This
    will (re)install a linear curve.
  \end{key}

  The opposite of having a smooth curve between two values, is to have
  a ``jump'' from one value to the next. There are two keys for this:


  \begin{key}{/pgf/animations/stay}
    Specifies that inside the time interval the value ``stays put'' at
    the first value till the end of the interval, where it will jump
    to the second value. This is similar to an exit control where the
    curve is ``infinitely flat.''
    
\begin{codeexample}[width=2.3cm]
\tikz {
  \pgfanimateattribute{translate}{
    whom = node, begin on = {click},
    entry = {0s}{\pgfpointorigin},
    stay,
    entry = {1s}{\pgfpoint{0cm}{-1cm}},
    linear,
    entry = {2s}{\pgfpoint{0cm}{-2cm}},
    entry = {3s}{\pgfpoint{0cm}{-3cm}} }
  \node (node) [fill = blue!20, draw = blue, very thick, circle] {Click me!}; 
}
\end{codeexample}
  \end{key}
  
  \begin{key}{/pgf/animations/jump}
    Works like the |stay| key, but will cause the value to ``jump to''
    the new value right at the beginning of the time interval. It is
    similar to an entry control specifying a ``flat'' curve.

\begin{codeexample}[width=2.3cm]
\tikz {
  \pgfanimateattribute{translate}{
    whom = node, begin on = {click},
    entry = {0s}{\pgfpointorigin},
    jump,
    entry = {1s}{\pgfpoint{0cm}{-1cm}},
    linear,
    entry = {2s}{\pgfpoint{0cm}{-2cm}} }
  \node (node) [fill = blue!20, draw = blue, very thick, circle] {Click me!}; 
}
\end{codeexample}
  \end{key}
 
\end{key}


\begin{command}{\pgfparsetime\marg{time}}
  This command works like |\pgfmathparse| (indeed, it calls is
  internally), but returns the result in the macro |\pgftimeresult|
  rather than |\pgfmathresult|. Furthermore, the following changes are
  installed:

  \begin{itemize}
  \item 
    The postfix operator |s| is added, which has no effect.
  \item The postfix operator |ms| is added, which devides a number by
    1.000, so |2ms| equals 0.002s.
  \item
    The postfix operator |min| is added, which multiplies a number by
    60.
  \item The postfix operator |h| is added, which multiplies a number by
    3.600.
  \item The infix operator |:| is redefined, so that it multiplies its
    first argument by 60 and adds the second. This implies that
    |1:20| equals 80s and |01:00:00| equals 3600s.
  \item The parsing of octal numbers is switched off to allow things
    like |01:08| for 68s.
  \end{itemize}
\end{command}



\subsubsection{Animating Color, Opacity, and Visibility}
\label{section-base-animation-painting}

You can animate the color of the target object of an animation using
the attributes |fill| or |draw|, which animate the fill color and the
drawing (stroking) color, respectively. To animate both the fill and
draw color, you need to create two animations, one for each.

\begin{codeexample}[width=2.3cm]
\tikz {
  \pgfanimateattribute{fill}{
    whom = node.background, begin on = {click}, entry = {0s}{white}, entry = {2s}{red} }
  \node (node) [fill = blue!20, draw = blue, very thick, circle] {Click me!}; 
}
\end{codeexample}

\begin{codeexample}[width=2.3cm]
\tikz {
  \pgfanimateattribute{draw}{
    whom = node.background, begin on = {click}, entry = {0s}{white}, entry = {2s}{red} }
  \node (node) [fill = blue!20, draw = blue, very thick, circle] {Click me!}; 
}
\end{codeexample}

When the target of a color animation is a scope, you animate
the color ``used in this scope'' for filling or stroking. However,
when an object inside the scope has its color set explicitly, this
color overrules the color of the scope:

\begin{codeexample}[]
\tikz {
  \pgfanimateattribute{fill}{
    whom = example, begin on = {click, of next=node},
    entry = {0s}{white}, entry = {2s}{red} }
  \node (node) [fill = blue!20, draw = blue, very thick, circle] {Click me!}; 
  \begin{scope}[name = example]
    \fill (1.1,0) rectangle ++ (1,1);
    \fill [blue] (1.5,0.5) rectangle ++ (1,1);
  \end{scope}
}
\end{codeexample}

Note that in certain cases, a graphic scope may contain graphic
objects with their colors set explicitly ``in places where you do not
expect it'': In particular, a node normally consists at least of a
background path and a text. For both the text and for the background
path, colors will be set for the text and also for the path
explicitly. This means that when you pick the fill attribute of a node
as the target of an animation, you will \emph{not} animate the color
of the background path in case this color has been set
explicitly. Instead, you must choose the background path of the node
as the target of the animation. Fortunately, this is easy to achieve
since when the background path of a node is created, the identifier
type is set to |background|, which in turn allows you to access it as
\meta{node}|.background| through the |whom| key.

The text of a node also gets it color set explicitly, which means that
a change of the node's scope's color has no effect on the text
color. Instead, you must choose \meta{name}|.text| as the target (or,
if the node has more parts, use the name of the part as the identifier
type instead of |text|).

\begin{codeexample}[]
\tikz {
  \pgfanimateattribute{fill}{
    whom = example, begin on = {click, of next=node},
    entry = {0s}{white}, entry = {2s}{red} }
  \node (node) [fill = blue!20, draw = blue, very thick, circle] {Click me!}; 
  \node at (2,0) (example) [fill = blue!20, circle] {No effect}; }
\end{codeexample}


\begin{codeexample}[]
\tikz {
  \pgfanimateattribute{fill}{
    whom = example.background, begin on = {click, of next=node},
    entry = {0s}{white}, entry = {2s}{red} }
  \node (node) [fill = blue!20, draw = blue, very thick, circle] {Click me!}; 
  \node at (2,0) (example) [fill = blue!20, circle] {Effect}; }
\end{codeexample}


\begin{codeexample}[]
\tikz {
  \pgfanimateattribute{fill}{
    whom = example.text, begin on = {click, of next=node},
    entry = {0s}{white}, entry = {2s}{red} }
  \node (node) [fill = blue!20, draw = blue, very thick, circle] {Click me!}; 
  \node at (2,0) (example) [fill = blue!20, circle, font=\huge] {Text}; }
\end{codeexample}

Similarly to the color, you can also set the opacity used for filling
and for drawing. You specify the opacity using a number between 0
(transparent) and 1 (opaque). 

\begin{codeexample}[width=2.3cm]
\tikz {
  \pgfanimateattribute{fill opacity}{
    whom = node, begin on = {click}, entry = {0s}{1}, entry = {2s}{0} }
  \node (node) [fill = blue!20, draw = blue, very thick, circle] {Click me!}; 
}
\end{codeexample}

\begin{codeexample}[width=2.3cm]
\tikz {
  \pgfanimateattribute{draw opacity}{
    whom = node, begin on = {click}, entry = {0s}{1}, entry = {2s}{0} }
  \node (node) [fill = blue!20, draw = blue, very thick, circle] {Click me!}; 
}
\end{codeexample}

Unlike colors, where there is no joint attribute for filling and
stroking, there is a single |opacity| attribute in addition to the above
two attributes. If supported by the driver, it treats the graphic
object to which it is applied as a transparency group. In essence,
``this attribute does what you want'' at least in most situations.

\begin{codeexample}[width=2.3cm]
\tikz {
  \pgfanimateattribute{opacity}{
    whom = node, begin on = {click}, entry = {0s}{1}, entry = {2s}{0} }
  \node (node) [fill = blue!20, draw = blue, very thick, circle] {Click me!}; 
}
\end{codeexample}

The last attribute that concerns colors and visibility is the
Boolean |visible| attribute. The difference to an opacity of |0| is
that an invisible object cannot be clicked and does not need to be
rendered. The (only) two possible values for this attribute are
|false| and |true|.

\begin{codeexample}[width=2.3cm]
\tikz {
  \pgfanimateattribute{visible}{
    whom = node, begin on = {click}, entry = {0s}{false}, entry = {2s}{false} }
  \node (node) [fill = blue!20, draw = blue, very thick, circle] {Click me!}; 
}
\end{codeexample}


\subsubsection{Animating Paths and their Rendering}
\label{section-base-animation-paths}

You can animate the appearance of a path in the following ways:

\begin{itemize}
\item You can animate the line width:
\begin{codeexample}[width=2.3cm]
\tikz [very thick] {
  \pgfanimateattribute{line width}{
    whom = node, begin on = {click}, entry = {0s}{1pt}, entry = {2s}{5mm} }
  \node (node) [fill = blue!20, draw = blue, circle] {Click me!}; 
}
\end{codeexample}
  The possible values passed to the |entry| key are, of course,
  dimensions. 
\item You can animate the dash phase:
\begin{codeexample}[width=2.3cm]
\tikz {
  \pgfanimateattribute{dash phase}{
    whom = node.background, begin on = {click}, entry = {0s}{0pt}, entry = {2s}{10pt} }
  \node (node) [dashed, fill = blue!20, draw = blue, very thick, circle] {Click me!}; 
}
\end{codeexample}
  Again, the possible values are dimensions.
\item You can animate the dash pattern:
\begin{codeexample}[width=2.3cm]
\tikz {
  \pgfanimateattribute{dash pattern}{
    whom = node, begin on = {click}, entry = {0s}{{10pt}{1pt}}, entry = {2s}{{1pt}{10pt}} }
  \node (node) [fill = blue!20, draw = blue, very thick, circle] {Click me!}; 
}
\end{codeexample}
  To specify the dash pattern, you specify a sequence of ``on and
  off'' dimensions; see the second argument of |\pgfsetdash| for
  details. Note that you \emph{must} specify the same number of
  elements in all patterns of a timeline: You cannot specify that the
  dash pattern for |1s| is |{1pt}{2pt}| and for |2s| is
  |{1pt}{3pt}{2pt}| since the number of elements would differ. In
  particular, you cannot (sensibly) use |current value| for the first
  entry since this corresponds to an empty dash pattern (even when you
  have specified a dash pattern for the target object: this pattern
  will not be attached to the to-be-animated scope or object but to a
  surrounding scope and, thus, the to-be-animated scope will not have
  any dash pattern installed).
\item Finally, you can animate the path itself:
\begin{codeexample}[width=2.3cm]
\tikz {
  \pgfanimateattribute{path}{
    whom = node.background.path, begin on = {click, of next=node},
    entry = {0s}{\pgfpathellipse{\pgfpointorigin}{\pgfpointxy{2}{0}}{\pgfpointxy{0}{1}}},
    entry = {2s}{\pgfpathellipse{\pgfpointxy{1}{0}}{\pgfpointxy{1}{1}}{\pgfpointxy{0.25}{.75}}}}
  \node (node) [fill = blue!20, draw = blue, very thick, circle] {Click me!}; 
}
\end{codeexample}
  The path is specified by giving path construction commands as in the
  above example. They will be executed in a special protected scope to
  ensure that they have only little side effects.

  As for the dash pattern, you must ensure that all paths in the
  timeline have the same structure (same sequence of path construction
  commands); only the coordinates may differ. In particular, you
  cannot say that the path at |1s| is a rectangle using
  |\pgfpathrectangle| and at |2s| is a circle using
  |\pgfpathcircle|. Instead, you would have to ensure that at both
  times that path consists of appropriate Bézier curves.

  Unlike the dash pattern, the to-be-animated object is, indeed, the
  path itself and not some special scope. This means that you can use
  the |current value| for the start path. However, this also means
  that you really must pick \emph{the path object} as the target of
  the animation. In conjunction with \tikzname, this will be an object
  of type |path| as in the above example.
\end{itemize}



\subsubsection{Animating Transformations  and Views}
\label{section-base-animation-views}


In order to animate the canvas transformation matrix, you do not
animate an attribute called ``|transform|'' (or something
similar). Rather, there are several keys that all manipulate the
canvas transformation matrix in 
different ways. These keys, taken in appropriate combination, allow
you to achieve any particular canvas transformation matrix.

Let us start with the basic keys that allow you to change the
canvas transformation matrix directly:

\begin{itemize}
\item The |scale| attribute adds an animation of the scaling:
\begin{codeexample}[width=2.3cm]
\tikz [very thick] {
  \pgfanimateattribute{scale}{
    whom = node, begin on = {click}, entry = {0s}{1}, entry = {2s}{1.5,2} }
  \node (node) [fill = blue!20, draw = blue, circle] {Click me!}; 
}
\end{codeexample}
  The values passed to the |entry| key must either be single scalar
  values or a pair of such numbers separated by a comma (which then
  refer to the $x$- and $y$-scaling).
\item The |rotate| key adds an animation of the rotation:
\begin{codeexample}[width=2.3cm]
\tikz [very thick] {
  \pgfanimateattribute{rotate}{
    whom = node, begin on = {click}, entry = {0s}{45}, entry = {2s}{90} }
  \node (node) [fill = blue!20, draw = blue, circle] {Click me!}; 
}
\end{codeexample}
  The values are scalar values representing a rotation in degrees.
\item The |xskew| and |yskew| keys (and also |skew x| and |skew y|,
  which are aliases) add an animation of the skew (given
  in degrees, not as a slant):
\begin{codeexample}[width=2.3cm]
\tikz [very thick] {
  \pgfanimateattribute{xskew}{
    whom = node, begin on = {click}, entry = {0s}{0}, entry = {2s}{45} }
  \node (node) [fill = blue!20, draw = blue, circle] {Click me!}; 
}
\end{codeexample}
  The values are scalar values.
\item The |translate| key adds an animation of the translation (shift):
\begin{codeexample}[width=2.3cm]
\tikz [very thick] {
  \pgfanimateattribute{translate}{
    whom = node, begin on = {click},
    entry = {0s}{\pgfpointorigin}, entry = {2s}{\pgfpoint{1cm}{1cm}} }
  \node (node) [fill = blue!20, draw = blue, circle] {Click me!}; 
}
\end{codeexample}
  The values are \pgfname-points.
\end{itemize}

For all of these attributes, the following key is of importance:
\begin{key}{/pgf/animation/origin=\meta{pgf point}}
  An animation of the canvas transformation is added to all other
  transformations from surrounding or interior scopes. This means
  that, in particular, the origin of a canvas transformation is, by
  default, the origin of the canvas of the scope surrounding the
  transformation object.

  For some canvas animations, like a rotation or a scaling, you will
  typically wish to use a different origin (like the center of an
  object that is to be rotated or scaled). You can achieve this effect
  by surrounding the object by a scope that shifts the canvas to the
  desired origin, followed by a scope whose transformation matrix you
  animate, followed by a scope that shifts back the canvas.

  The |origin| key simplifies this process by allowing you to specify
  the origin of the transformation directly. Internally, however, all
  this key does is to create the above-mentioned scopes with the
  necessary shifts.
  
\begin{codeexample}[width=2.3cm]
\tikz [very thick] {
  \pgfanimateattribute{rotate}{
    whom = node, begin on = {click},
    origin = \pgfpoint{1cm}{1cm}, entry = {0s}{0}, entry = {2s}{90} }
  \node (node) [fill = blue!20, draw = blue, circle] {Click me!}; 
}
\end{codeexample}
\end{key}


A second way of changing the canvas transformation matrix is to use
the |motion| key:

\begin{codeexample}[width=2.3cm]
\tikz [very thick] {
  \pgfanimateattribute{motion}{
    whom = node, begin on = {click},
    along = \pgfpathcircle{\pgfpointorigin}{5mm},
    entry = {0s}{0}, entry = {2s}{1} }
  \node (node) [fill = blue!20, draw = blue, circle] {Click me!}; }
\end{codeexample}


\begin{key}{/pgf/animation/along=\meta{path}}
  This key (or |rotate along|) must be used with |motion| attribute to
  specify a path along which the transformation matrix will be
  ``moved'' (that is, a shift transformation will be added to the
  different points on the path).

  The values passed to the |entry| key specify fractions of the
  distance along the \meta{path}. That means, when you provide a value
  of |0|, you reference the start point of the path, a value of |1|
  references the end of the path and |0.5| referenced the point
  halfway along the path.

\begin{codeexample}[width=2.3cm]
\tikz [very thick] {
  \pgfanimateattribute{motion}{
    whom = node, begin on = {click},
    along = \pgfpathcircle{\pgfpointorigin}{5mm},
    entry = {0s}{0}, entry = {1s}{0.5}, entry = {2s}{0.25}, entry={3s}{1} }
  \node (node) [fill = blue!20, draw = blue, circle] {Click me!}; }
\end{codeexample}
\end{key}

  
\begin{key}{/pgf/animation/rotate along=\meta{Boolean} (default true)}
  When set to |true|, the |along| key additionally adds a rotation
  that varies in such a way that a tangent to the path always points right. 

\begin{codeexample}[width=2.3cm]
\tikz [very thick] {
  \pgfanimateattribute{motion}{
    whom = node, begin on = {click},
    rotate along = true,
    along = \pgfpathmoveto {\pgfpointorigin}
            \pgfpathcurveto{\pgfpoint{1cm}{0cm}}{\pgfpoint{1cm}{0cm}}{\pgfpoint{1cm}{1cm}},
    entry = {0s}{0}, entry = {2s}{1} }
  \node (node) [fill = blue!20, draw = blue, circle] {Click me!}; }
\end{codeexample}  
\end{key}

The final method of changing the transformation matrix is to animate 
a \emph{view.} A view is a canvas transformation that shifts and
scales the canvas in such a way that a certain rectangle ``matches''
another rectangle: The idea is that you ``look through'' a ``window''
(the view) and ``see'' a certain area of the canvas. View animation do
not allow you to do anything that cannot also be done using the
|translate| and |scale| keys in combination, but it often much more
natural to animate which area of a graphic you wish to see than to
compute and animate a scaling and shift explicitly.

In order to use a view, you first need to create a view, which is done
using a |{pgfviewboxscope}|, see Section~\ref{section-base-view},
which is used by the |view| library internally. You can then animate
the view using the |view| attribute. The values passed to the |entry|
key must be two \pgfname-points, each surrounded by parentheses.

\begin{codeexample}[width=2cm]
\tikz [very thick] {
  \pgfanimateattribute{view}{
    whom = me.view, begin on = {click, of next=n1}, freeze at end,
    entry = {0s}{current value}, entry = {2s}{{\pgfpoint{5mm}{5mm}}{\pgfpoint{15mm}{20mm}}} }
  \pgfanimateattribute{view}{
    whom = me.view, begin on = {click, of next=n2}, freeze at end,
    entry = {0s}{current value}, entry = {2s}{{\pgfpoint{10mm}{10mm}}{\pgfpoint{15mm}{15mm}}} }
  \node (n1) at (0,0) [fill = blue!20, draw = blue, circle] {Zoom blue}; 
  \node (n2) at (2,0) [fill = blue!20, draw = blue, circle] {Zoom red};

  \draw [green!50!black] (4,0) rectangle (6,2);
  \begin{scope}[name = me, meet = {view (0,0) (2,2) at (4,0) (6,2)}]
    \draw [blue] (5mm,5mm) rectangle (15mm,20mm);
    \node at (5mm,5mm) [circle, fill=blue, text=white] {blue};
    
    \draw [red] (10mm,10mm) rectangle (15mm,15mm);
    \node at (10mm,10mm) [circle, fill=red, text=white, font=\tiny] {red};
  \end{scope}
}
\end{codeexample}  


\subsubsection{Commands for Specifying Timing: Beginning and Ending}

Using the |entry| key repeatedly, you specify a timeline: The
\meta{time} used with the first use of the |entry| key in a timeline
is the start time and the \meta{time} in the last |entry| key is the
stop time. However, this leaves open then question of when the whole
timeline is to be started: The moment the document is opened? When the
page is displayed? When the user scrolls to the to-be-animated object?
When some other object is clicked? The key |begin|, and also the key
|end|, allow you to specify answers to these questions.

\begin{key}{/pgf/animation/begin=\meta{time}}
  This key specifies when the ``moment |0s|'' should be relative to
  the moment when the current graphic is first displayed. You can use
  this key multiple times, in this case the timeline is restarted for
  each of the times specified (if it is already running, it will be
  reset). If no |begin| key is given at all, the effect is the same as
  if |begin=0s| had been specified.

  It is permissible to set \meta{time} to a negative value.
\end{key}

\begin{key}{/pgf/animation/end=\meta{time}}
  This key will truncate the timeline so that it ends \meta{time}
  after the display of the graphic, provided the timeline begins
  before the specified end time. For instance, if you specify a
  timeline starting at 2\,s and ending at 5\,s and you set |begin| to
  1\,s and |end| to 4\,s, the timeline will run, relative to the moment
  when the graphic is displayed from 3\,s to 4\,s.
  
\begin{codeexample}[width=3cm]
\tikz [very thick] {
  \pgfanimateattribute{rotate}{
    whom = node, begin = 2s, end = 4s,
    entry = {1s}{0}, entry = {2s}{90}, entry = {3s}{180}, entry = {4s}{270} }
  \node (node) [fill = blue!20, draw = blue, circle] {Turn after 3s!}; }
\end{codeexample}
\end{key}

Instead of specifying the beginning of the timeline relative to the
moment to to-be-animated graphic is displayed, you can also set the
``moment |0s|'' to the moment a specific \emph{event} happens using
the following key:

\begin{key}{/pgf/animation/begin on=\meta{options}}
  The \meta{options} will be executed with the path
  |/pgf/animation/events| and will cause a new beginning to be added
  to the list of possible beginnings for the timeline (so the uses of
  this key accumulate). Each ``beginning'' is just another possible
  ``moment |0s|'' for the timeline. For instance, when the
  \meta{options} are set to |click|, then each time the graph is
  clicked a moment |0s| starts for the timeline.

  Most events are ``caused'' or ``happen to'' some object. For
  instance, the |click| event happens when you click on a certain
  object. In order to specify this object, use the following two keys
  inside the \meta{options}: |of| and |of next|. If neither of these
  keys are given, the to-be-animated object is used.
  
  \begin{key}{/pgf/animation/event/of=\meta{id}\opt{|.|\meta{type}}}
    This specifies a graphic object id in the same way as the |whom|
    key, also with an optional \meta{type}. This is the object that
    ``causes'' the event to happen.

    Unlike the |whom| key, which always refers to a not-yet-existing
    object, this key always refers to an already existing
    object, namely to the most recent use of the \meta{id}. In the
    following example, the referenced object is the node with the
    label |2| since it is the most recently referenced node with
    \meta{id} |X|.
\begin{codeexample}[width=3cm]
\tikz [very thick] {
  \node (X) at (1,1.2)  [fill = blue!20, draw = blue, circle] {1};
  \node (X) at (1,0.4)  [fill = blue!20, draw = blue, circle] {2};
  \pgfanimateattribute{rotate}{
    whom = node, begin on = {click, of = X}, entry = {0s}{0}, entry = {2s}{90} }
  \node (node) [fill = red!20, draw = red, rectangle] {Anim}; 
  \node (X) at (1,-0.4) [fill = blue!20, draw = blue, circle] {3};
  \node (X) at (1,-1.2) [fill = blue!20, draw = blue, circle] {4};
}
\end{codeexample}
  \end{key}
  
  \begin{key}{/pgf/animation/event/of next=\meta{id}\opt{|.|\meta{type}}}
    This key works like the |of| key, only it refers to a future
    (actually, the next) object with the given \meta{id}, not to a
    previous one. This, in the next example, the referenced node is
    the one with label |3|.
\begin{codeexample}[width=3cm]
\tikz [very thick] {
  \node (X) at (1,1.2)  [fill = blue!20, draw = blue, circle] {1};
  \node (X) at (1,0.4)  [fill = blue!20, draw = blue, circle] {2};
  \pgfanimateattribute{rotate}{
    whom = node, begin on = {click, of next = X}, entry = {0s}{0}, entry = {2s}{90} }
  \node (node) [fill = red!20, draw = red, rectangle] {Anim}; 
  \node (X) at (1,-0.4) [fill = blue!20, draw = blue, circle] {3};
  \node (X) at (1,-1.2) [fill = blue!20, draw = blue, circle] {4};
}
\end{codeexample}
  \end{key}

  The following key allows you to specify the event that should cause
  the animation to start:
  \begin{key}{/pgf/animation/event=\meta{event name}}
    Specifies the name of the event whose occurrence should start the
    timeline. Which events are supported depends on the device on
    which the animation is displayed as well as the output format
    (\textsc{svg} or some other format), but here is a list of events
    supported by \textsc{svg}: |click|, |focusin|, |focusout|,
    |mousedown|,  |mouseup|, |mouseover|, |mousemove|,
    |mouseout|, |begin|, |end|. However, the following keys make using
    these events simpler:
    \begin{key}{/pgf/animate/event/click}
      This is a shorthand for |event=click|. This event gets triggered
      when the user clicks on the triggering object with a mouse (or
      something equivalent).
\begin{codeexample}[width=2cm]
\tikz [very thick] {
  \pgfanimateattribute{rotate}{
    whom = node, begin on = {click}, entry = {0s}{0}, entry = {2s}{90} }
  \node (node) [fill = blue!20, draw = blue, circle] {Here!}; 
}
\end{codeexample}
    \end{key}
  \end{key}
\end{key}

\subsubsection{Commands for Specifying Timing: Repeats}

Normally, a timeline is displayed once and then ends.


\subsubsection{Commands for Specifying Timing: Restart Behaviour}

\subsubsection{Commands for Specifying Accumulation}

\endinput



%%% Local Variables: 
%%% mode: latex
%%% TeX-master: "pgfmanual"
%%% End: 

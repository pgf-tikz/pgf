% Copyright 2006 by Till Tantau
%
% This file may be distributed and/or modified
%
% 1. under the LaTeX Project Public License and/or
% 2. under the GNU Free Documentation License.
%
% See the file doc/generic/pgf/licenses/LICENSE for more details.


\section{Pattern Library}
\label{section-library-patterns}

\begin{pgflibrary}{patterns}
    The package defines patterns for filling areas.
\end{pgflibrary}


\newcommand\patternindex[1]{%
  \index{#1@\protect\texttt{#1} pattern}%
  \index{Patterns!#1@\protect\texttt{#1}}%
  \texttt{#1}&
  \begin{tikzpicture}[baseline=.5ex]

    % Background
    \pattern [path fading=west,pattern=checkerboard light gray]
      (0,0) rectangle (5cm,2em);

    \pattern [pattern=#1,pattern color=black] (0,0) rectangle +(1.5cm,2em);
    \pattern [pattern=#1,pattern color=blue] (1.75,0) rectangle +(1.5cm,2em);
    \pattern [pattern=#1,pattern color=red] (3.5,0) rectangle +(1.5cm,2em);
  \end{tikzpicture} \\[1ex]
}

\newcommand\patternindexinherentlycolored[1]{%
  \index{#1@\protect\texttt{#1} pattern}%
  \index{Patterns!#1@\protect\texttt{#1}}%
  \texttt{#1}&
  \begin{tikzpicture}[baseline=.5ex]

    % Background
    \pattern [path fading=west,pattern=checkerboard light gray]
      (0,0) rectangle (5cm,2em);

    \pattern [pattern=#1,pattern color=blue] (0,0) rectangle +(5cm,2em);
  \end{tikzpicture} \\[1ex]
}


\subsection{Form-Only Patterns}

\begin{tabular}{ll}
    \emph{Pattern name}
        & \emph{Example (pattern in black, blue, and red on faded checkerboard)} \\
    \patternindex{horizontal lines}
    \patternindex{vertical lines}
    \patternindex{north east lines}
    \patternindex{north west lines}
    \patternindex{grid}
    \patternindex{crosshatch}
    \patternindex{dots}
    \patternindex{crosshatch dots}
    \patternindex{fivepointed stars}
    \patternindex{sixpointed stars}
    \patternindex{bricks}
    \patternindex{checkerboard}
\end{tabular}


\subsection{Inherently Colored Patterns}

\begin{tabular}{ll}
    \emph{Pattern name} & \emph{Example} \\
    \patternindexinherentlycolored{checkerboard light gray}
    \patternindexinherentlycolored{horizontal lines light gray}
    \patternindexinherentlycolored{horizontal lines gray}
    \patternindexinherentlycolored{horizontal lines dark gray}
    \patternindexinherentlycolored{horizontal lines light blue}
    \patternindexinherentlycolored{horizontal lines dark blue}
    \patternindexinherentlycolored{crosshatch dots gray}
    \patternindexinherentlycolored{crosshatch dots light steel blue}
\end{tabular}

\subsection{User-Defined Patterns}
\label{section-library-patterns-meta}

\begin{pgflibrary}{patterns.meta}
    Define your own patterns with a syntax similar to |arrows.meta|.
\end{pgflibrary}

\begin{codeexample}[preamble={\usetikzlibrary{patterns.meta}}]
\pgfdeclarepattern{
  name=hatch,
  parameters={\hatchsize,\hatchangle,\hatchlinewidth},
  bottom left={\pgfpoint{-.1pt}{-.1pt}},
  top right={\pgfpoint{\hatchsize+.1pt}{\hatchsize+.1pt}},
  tile size={\pgfpoint{\hatchsize}{\hatchsize}},
  tile transformation={\pgftransformrotate{\hatchangle}},
  code={
    \pgfsetlinewidth{\hatchlinewidth}
    \pgfpathmoveto{\pgfpoint{-.1pt}{-.1pt}}
    \pgfpathlineto{\pgfpoint{\hatchsize+.1pt}{\hatchsize+.1pt}}
    \pgfpathmoveto{\pgfpoint{-.1pt}{\hatchsize+.1pt}}
    \pgfpathlineto{\pgfpoint{\hatchsize+.1pt}{-.1pt}}
    \pgfusepath{stroke}
  }
}

\tikzset{
  hatch size/.store in=\hatchsize,
  hatch angle/.store in=\hatchangle,
  hatch line width/.store in=\hatchlinewidth,
  hatch size=5pt,
  hatch angle=0pt,
  hatch line width=.5pt,
}

\begin{tikzpicture}
\foreach \r in {1,...,4}
  \draw [pattern=hatch, pattern color=red]
    (\r*3,0) rectangle ++(2,2);

\foreach \r in {1,...,4}
  \draw [pattern=hatch, pattern color=green, hatch size=2pt]
    (\r*3,3) rectangle ++(2,2);

\foreach \r in {1,...,4}
  \draw [pattern=hatch, pattern color=blue, hatch size=10pt, hatch angle=21]
    (\r*3,6) rectangle ++(2,2);

\foreach \r in {1,...,4}
  \draw [pattern=hatch, pattern color=orange, hatch line width=2pt]
    (\r*3,9) rectangle ++(2,2);
\end{tikzpicture}
\end{codeexample}

\begin{codeexample}[preamble={\usetikzlibrary{patterns.meta}}]
\tikzdeclarepattern{
  name=flower,
  type=colored,
  bottom left={(-.1pt,-.1pt)},
  top right={(10.1pt,10.1pt)},
  tile size={(10pt,10pt)},
  code={
    \tikzset{x=1pt,y=1pt}
    \path [draw=green] (5,2.5) -- (5, 7.5);
    \foreach \i in {0,60,...,300}
      \path [fill=pink, shift={(5,7.5)}, rotate=-\i]
        (0,0) .. controls ++(120:4) and ++(60:4) .. (0,0);
    \path [fill=red] (5,7.5) circle [radius=1];
    \foreach \i in {-45,45}
      \path [fill=green, shift={(5,2.5)}, rotate=-\i]
        (0,0) .. controls ++(120:4) and ++(60:4) .. (0,0);
  }
}

\tikz\draw [pattern=flower] circle [radius=1];
\end{codeexample}

\begin{codeexample}[preamble={\usetikzlibrary{patterns.meta}}]
\tikzdeclarepattern{
  name=Stars,
  type=uncolored,
  bounding box={(-5pt,-5pt) and (5pt,5pt)},
  tile size={(\tikztilesize,\tikztilesize)},
  parameters={\tikzstarpoints,\tikzstarradius,\tikzstarrotate,\tikztilesize},
  tile transformation={rotate=\tikzstarrotate},
  defaults={
    points/.store in=\tikzstarpoints,points=5,
    radius/.store in=\tikzstarradius,radius=3pt,
    rotate/.store in=\tikzstarrotate,rotate=0,
    tile size/.store in=\tikztilesize,tile size=10pt,
  },
  code={
    \pgfmathparse{180/\tikzstarpoints}\let\a=\pgfmathresult
    \fill (90:\tikzstarradius) \foreach \i in {1,...,\tikzstarpoints}{
      -- (90+2*\i*\a-\a:\tikzstarradius/2) -- (90+2*\i*\a:\tikzstarradius)
    } -- cycle;
  }
}

\begin{tikzpicture}
  \draw[pattern=Stars,pattern color=blue]               (0,0) rectangle ++(2,2);
  \draw[pattern={Stars[points=7,tile size=15pt]}]       (2,0) rectangle ++(2,2);
  \draw[pattern={Stars[rotate=45]},pattern color=red]   (0,2) rectangle ++(2,2);
  \draw[pattern={Stars[rotate=30,points=4,radius=5pt]}] (2,2) rectangle ++(2,2);
\end{tikzpicture}
\end{codeexample}

%%% Local Variables:
%%% mode: latex
%%% TeX-master: "pgfmanual-pdftex-version"
%%% End:

% Copyright 2010 by Renée Ahrens, Olof Frahm, Jens Kluttig, Matthias Schulz, Stephan Schuster
% Copyright 2011 by Till Tantau
% Copyright 2011 by Jannis Pohlmann
%
% This file may be distributed and/or modified
%
% 1. under the LaTeX Project Public License and/or
% 2. under the GNU Free Documentation License.
%
% See the file doc/generic/pgf/licenses/LICENSE for more details.


\section{Implementing Graph Drawing Algorithms: The Lua Side}
\label{section-gd-implementation-first}

\noindent{\emph{by Till Tantau}}
\label{section-gd-own-algorithm}
\label{section-library-graphdrawing-ownAlgorithm}

File status: I'm working on it...


\subsection{Overview}

The present section is the first of three sections addressed at
readers interested in implementing new graph drawing
algorithms. Obviously, in order to do so, you need to have an
algorithm in mind and also some programming skills; but no deep
knowledge of \TeX\ will be needed. All graph drawing algorithms can
and must be implemented entirely in the Lua programming language,
which is a small, easy-to-learn (and quite beautiful) language
integrated into current versions of \TeX. The present section focuses
on the Lua side of the graph drawing engine; the next section
addresses the interface between this Lua side and \TeX\ side; the last
of the three sections presents complete example code of simple graph
drawing algorithms. 

In the following, after a small ``hello world'' example of graph
drawing and a discussion of technical details like
how to name files so that \TeX\ will find them, we have a look at the
main parts of the graph drawing engine:

\begin{itemize}
\item Section~\ref{section-gd-models} explains the object-oriented
  model of graphs used throughout the graph drawing engine. Graph
  drawing algorithms do not get the ``raw'' specification used by the
  user to specify a graph (like |{a -> {b,c}}| in the |graph|
  syntax). Instead, what a graph drawing algorithm sees is ``just'' a
  graph object that provides methods for accessing the vertices and
  arcs. 
\item Section~\ref{section-gd-syntactic-digraph} introduces the
  notion of ``syntactic digraphs''. While most graph drawing
  algorithms are not really interested in the ``details'' of how a
  graph was specified, for some algorithms it makes a big difference
  whether you write |a -> b| or |b <- a| in your specification of the
  graph. These algorithms can access the ``fine details'' of how the
  input graph was specified through the syntactic digraph; all other
  algorithms can access their |digraph| or |ugraph| fields and do not
  have to worry about the difference between |a -> b| and |b <- a|.
\item Section~\ref{section-gd-layout-pipeline} is devoted to what I
  call the \emph{layout pipeline.} When the layout of a graph needs to
  be computed, only very few algorithms will actually be able to
  compute positions for the nodes of \emph{every} graph. For instance,
  most algorithms implicitly assume that the input graph is
  connected; algorithms for computing layouts for trees assume that
  the input is, well, a tree; and so on. The job of the layout
  pipeline is to apply numerous pre- and postprocessing operations
  (like decomposing a graph into connected components or computing a
  spanning tree) so that algorithms do not need to worry about these
  steps. As we will see, the layout pipeline can vastly simplify the
  code of algorithms.
\item Section~\ref{section-gd-events} surveys ``events'', a concept
  that enriches the syntactic digraph by ``even more information''
  concerning the specification of the graph.
\item Finally, Section~\ref{section-gd-libs} documents the different
  libraries functions that come with the graph drawing engine. For
  instance, there are library functions for computing the (path)
  distance of nodes in a graph; a parameter that is needed by some
  algorithms. 
\end{itemize}



\subsection{Getting Started}

In this section, a ``hello world'' example of a graph
drawing algorithm  is given, followed by an overview of the
organization of the whole engine.



\subsubsection{The Hello World of Graph Drawing}

Let us start our tour of the Lua side of \tikzname's graph drawing
engine with a ``hello world'' version of graph drawing: Let us
implement an algorithms that simply places all nodes of a graph in a
circle of a fixed radius. Naturally, this is not a particularly
impressive or intelligent graph drawing algorithm; but neither is the
classical ``hello world''\dots\ Here is the complete code:

\begin{codeexample}[code only]
local SimpleDemo = pgf.gd.new_algorithm_class {}

function SimpleDemo:run()
  local alpha = (2 * math.pi) / #self.ugraph.vertices
  for i,vertex in ipairs(self.ugraph.vertices) do
    vertex.pos.x = math.cos(i * alpha) * 25
    vertex.pos.y = math.sin(i * alpha) * 25
  end
end

return SimpleDemo  
\end{codeexample}

We can employ this algorithm as follows:

\begin{codeexample}[pre={
    \pgfkeys{
      /graph drawing/radius=25,
      /graph drawing/algorithm/.code={\pgfgdgraphparameter{/graph drawing/algorithm}{"pgf.gd.examples.##1"}}}}]
\tikz [layout=SimpleDemo]
  \graph { f -> c -> e -> a -> {b -> {c, d, f}, e -> b}};
\end{codeexample}

It turns out, that our little algorithm is already more powerful than
one might expect. Consider the following example:
\begin{codeexample}[pre={
    \pgfkeys{
      /graph drawing/radius=25,
      /graph drawing/algorithm/.code={\pgfgdgraphparameter{/graph drawing/algorithm}{"pgf.gd.examples.##1"}}}}]
\tikz [layout=SimpleDemo, componentwise]
  \graph {
    1 -> 2 ->[orient=right] 3 -> 1;
    a -- b --[orient=45]  c -- d -- a;
  };
\end{codeexample}

Note that, in our algorithm, we ``just'' put all nodes on a circle
around the origin. Nevertheless, the graph gets decomposed into two
connected components, the components are rotated so that the edge from
node |2| to node |3| goes from left to right and the edge from |b| to
|c| goes up at an angle of $45^\circ$, and the components are placed
next to each other so that some spacing is achieved.

The ``magic'' that achieves all this behind the scenes is called the
``layout pipeline.'' It will heavily pre- and postprocess the input
and output of graph drawing algorithms to achieve the above results.

Naturally, some algorithms may not wish their inputs and/or
outputs to be ``tampered'' with by the pipeline. As we will see in the
chapter on the layout pipeline, an algorithm can easily configure
which steps of the pipeline should be applied, namely by passing
appropriate options to the |new_algorithm_class| function.



\subsubsection{Namespaces}

In order to typeset this manual, I had to cheat a little in the
``hello world'' example: There is no file named |SimpleDemo.lua| in
the \pgfname\ distribution, so writing |algortihm=SimpleDemo| would
result in an error. In reality, the code of |SimpleDemo| class resides
in a file with the somewhat lengthy name
|pgf.gd.examples.SimpleDemo.lua| and one actually has to write
\begin{codeexample}[code only]
..., algorithm=pgf.gd.examples.SimpleDemo, ...
\end{codeexample}
\noindent
to use it. The reason for the lengthy file name is that all parts of
the graph drawing library reside in the Lua ``namespace'' |pgf.gd|,
which is itself a ``sub-namespace'' of |pgf|. For your own algorithms,
you are free to place them in whatever namespace you like, so the
|SimpleDemo| example would work as advertised; only for the official
distribution of \pgfname\ everything has been put into the correct
namespace.

Let us now have a more detailed look at these namespaces. A namespace
is just a Lua table, and sub-namespaces are just subtables of
namespace tables. Following the Java convention, namespaces are in
lowercase letters. The following namespaces are part of the core of
the graph drawing engine:
\begin{itemize}
\item |pgf| This namespace is the main namespace of \pgfname. Other
  parts of \pgfname\ and \tikzname\ that also employ Lua should put an
  entry into this table. Since, currently, only the graph drawing
  engine adheres to this rule, this namespace is declared inside the
  graph drawing directory, but this will change.

  The |pgf| table is the \emph{only} entry into the global table of
  Lua generated by the graph drawing engine (or, \pgfname, for that
  matter). If you intend to extend the graph drawing engine, do not
  even \emph{think} of polluting the global namespace. You will be
  fined.
\item |pgf.gd| This namespace is main namespace of the graph drawing
  engine, including the object-oriented models of graphs and the
  layout pipeline. Algorithms that are part of the
  distribution are also inside this namespace, but if you write your
  own algorithms you do not need place them inside this
  namespace. (Indeed, you probably should not.)
\item |pgf.gd.model| This namespace contains all Lua classes that are
  part of the object-oriented model of graphs employed
  throughout the graph drawing engine. For readers familiar with the
  model--view--controller pattern: This is the namespace containing
  the model-part of this pattern. 
\item |pgf.gd.control| As the name suggests, this namespace contains
  the ``control'' classes and functions. These classes manage the
  communication with the \TeX\ layer and stage the layout
  pipeline. For instance, the whole communication between the \TeX\
  part and the Lua part of the graph drawing engine is done entirely
  inside the class |pgf.gd.control.TeXInterface|.

  Readers still familiar with the model--view--controller pattern will
  have recognized the ``control'' part of the pattern at this point.
\item |pgf.gd.lib| Numerous useful classes that ``make an algorithm's
  your life easier'' are collected in this namespace. Examples are a
  class for decomposing a graph into connected components or a class
  for computing the ideal distance between two sibling nodes in a
  tree, taking all sorts of rotations and separation parameters into
  account. 
\item |pgf.gd.trees| This namespace contains classes that are useful
  for dealing with graphs that are trees. In particular, it contains a
  class for computing a spanning tree of an arbitrary connected graph;
  an operation that is an important preprocessing step for many
  algorithms.

  In addition to providing ``utility functions for trees,'' the
  namespace \emph{also} includes actual algorithms for computing graph
  layouts like |pgf.gd.trees.ReingoldTilford1981|. It may seem to be a
  bit of an ``impurity'' that a namespace mixes utility classes and
  ``real'' algorithms, but experience has shown that it is better to
  keep things together in this way.

  Concluding the analogy to the model--view--controller pattern, a
  graph drawing algorithm is, in a loose sense, the ``view'' part of
  the pattern.
\item |pgf.gd.layered| This namespace provides classes and functions
  for ``layered'' layouts; the Sugiyama layout method being the most
  well-known one. Again, the namespace contains both algorithms to be
  used by a user and utility functions.
\item |pgf.gd.force| Collects force-based algorithms and, again, also
  utility functions and classes.
\item |pgf.gd.examples| Contains some example algorithms. They are
  \emph{not} intended to be used directly, rather they should serve as
  inspirations for readers wishing to implement their own algorithms.
\end{itemize}

There are further namespaces that also reside in the |pgf.gd|
namespace, these namespaces are used to organize different graph
drawing algorithms into categories.


\subsubsection{Defining and Using Namespaces and Classes}

There are a number of rules concerning the structure and naming of
namespaces as well as the naming of files. Let us start with the
rules for naming namespaces, classes, and functions. They follow the
``Java convention'':

\begin{enumerate}
\item A namespace is a short lowercase |word|.
\item A function in a namespace is in |lowercase_with_underscores_between_words|.
\item A class name is in |CamelCaseWithAnUppercaseFirstLetter|.
\item A class method name is in |camelCaseWithALowercaseFirstLetter|.
\end{enumerate}

From Lua's point of view, every namespace and every class is just a
table. However, since these tables will be loaded using Lua's
|require| function, each namespace and each class must be placed
inside a separate file (unless you modify the |package.loaded| table,
but, then, you know what you are doing anyway). Inside such a file, you
should first declare a local variable whose name is the name of the
namespace or class that you intend to define and then assign a
(possibly empty) table to this variable:
\begin{codeexample}[code only]
-- File pgf.gd.example.SomeClass.lua:
local SomeClass = {}
\end{codeexample}
Next, you should add your class to the encompassing namespace. This is
achieved as follows:
\begin{codeexample}[code only]
require("pgf.gd.example").SomeClass = SomeClass
\end{codeexample}
The reason this works is that the |require| will return the table that
is the namespace |pgf.gd.example|. So, inside this namespace, the
|SomeClass| field will be filled with the table stored in the local
variable of the same name -- which happens to be the table
representing the class.

At the end of the file, you must write
\begin{codeexample}[code only]
return SomeClass  
\end{codeexample}
This ensures that the table that is defined in this file gets stored
by Lua in the right places. Note that you need and should not use
Lua's |module| command. The reason is that this command has
disappeared in the new version of Lua and that it is not really
needed. 

Users of your class can import and use your class by writing:
\begin{codeexample}[code only]
...
local SomeClass = require "pgf.gd.examples.SomeClass"
...  
\end{codeexample}


\subsubsection{File Names of Lua Classes and Namespaces}

In Lua, similarly to Java, when a class |SomeClass| is part of, say,
the namespace |pgf.gd.example|, it is customary to put the class's
code in a file |SomeClass.lua| and then put this class in a directory
|example|, that is a subdirectory of a directory |gd|, which is in
turn a subdirectory of a directory |pgf|. When you write
\texttt{require "pgf.gd.example.SomeClass"} it is the job of a
so-called \emph{loader} to turn this into a request for the file
\texttt{pgf/gd/example/SomeClass.lua} (for Unix systems).

Unfortunately, this approach does not work well with (Lua)\TeX\ due to
the way \TeX\ looks for files. Although these days a typical \TeX\
installation consists of thousands of files scattered over hundreds of
sub-sub-sub-directories, from \TeX's point of view the directory
structure is in some sense irrelevant: If there are two files
|foo.lua| in two different directories, there is really no way of
knowing which one will be used when you request it; trying to prefix a
file request with a path prefix in a portable way is difficult to
impossible, especially when different |texmf| trees are involved.

For these reasons the files of the graph drawing engine are named and
organized in a somehwat redundant way: A class like
|pgf.gd.examples.SomeClass| will be placed in a file called
\texttt{pgf.gd.examples.SomeClass.\penalty0lua} (so the complete namespace path is
part of the file name). \emph{Additionally}, the file is placed in a
directory path is named according to the ``usual'' way, namely
|pgf/gd/examples/|. Note, however, that the file could just as well
reside in any other directory that is part of some |texmf| tree.

The bottom line of all this is: Filenames of Lua files of the graph
drawing engine contain the complete namespace path and, additionally,
they are placed in a directory path also representing the complete
namespace path. 


\subsection{The Model Classes}

\label{section-gd-models}

All that a graph drawing algorithm will ``see'' of the graph specified
by the user is a ``graph object.'' Such an object is an
object-oriented model of the user's graph that no longer encodes the
specific way in which the user specified the graph; it only encodes
which nodes and edges are present. For instance, the graph
specification 
\begin{codeexample}[code only]
graph { a -- {b, c} }
\end{codeexample}
\noindent and the graph specification
\begin{codeexample}[code only]
node (a) { a }
child { node (b) {b} }
child { node (c) {c} }
\end{codeexample}
will generate exactly the same graph object.

\begin{luanamespace}{pgf.gd.}{model}
  This namespace contains the classes modeling graphs,
  nodes, and edges. Also, the |Coordinate| class is found here, since
  coordinates are also part of the modeling.  
\end{luanamespace}

\subsubsection{Directed Graphs (Digraphs)}

Inside the graph drawing engine, the only model of a graph that is 
available treats graphs as
\begin{enumerate}
\item directed (all edges have a designated head and a designated
  tail) and
\item simple (there can be at most one edge between any pair of
  nodes). 
\end{enumerate}
These two properties may appear to be somewhat at odds with what users
can specify as graphs and with what some graph drawing algorithms
might expect as input. For instance, suppose a user writes
\begin{codeexample}[code only]
graph { a -- b --[red] c, b --[green, bend right] c }
\end{codeexample}
In this case, it seems that the input graph for a graph drawing
algorithm should actually be a \emph{undirected} graph in which there
are \emph{multiple} edges (namely $2$) between |b| and~|c|.
Nevertheless, the graph drawing engine will turn the user's input a
directed simple graph in ways described later. You do not need to
worry that information gets lost during this process: The
\emph{syntactic digraph,} which is available to graph drawing
algorithms on request, stores all the information about which edges
are present in the original input graph.

The main reason for only considering directed, simple graphs is speed
and simplicity: The implementation of these graphs has been optimized so
that all operations on these graphs have a guaranteed running time
that is small in practice. 

\includeluadocumentationof{pgf.gd.model.Digraph}

\subsubsection{Vertices}

\includeluadocumentationof{pgf.gd.model.Vertex}

\subsubsection{Arcs}
\label{section-gd-arc-model}

\includeluadocumentationof{pgf.gd.model.Arc}

\subsubsection{Coordinates and Transformations}

\includeluadocumentationof{pgf.gd.model.Coordinate}
\includeluadocumentationof{pgf.gd.lib.Transform}

\subsubsection{Storing Data in Vertices, Arcs, and Digraphs}

\includeluadocumentationof{pgf.gd.lib.Storage}


\subsection{The Syntactic Digraph}

\label{section-gd-syntactic-digraph}


\subsubsection{Accessing the Syntactic Digraph: Finding Nodes and Edges}

\subsubsection{Accessing the Syntactic Digraph: Node Parameters}

\subsubsection{Accessing the Syntactic Digraph: Graph Parameters}

\subsubsection{Accessing the Syntactic Digraph: Edge Parameters}

\subsubsection{Modifying the Syntactic Digraph: Generating Options}

\subsubsection{Modifying the Syntactic Digraph: Adding Edges}

\subsubsection{Modifying the Syntactic Digraph: Adding Nodes}



\subsection{The Layout Pipeline}

\label{section-gd-layout-pipeline}

\subsubsection{Clusters}

\subsubsection{Anchoring}

\subsubsection{Packing}

\subsubsection{Orientation}

\subsubsection{Spanning Trees}

\subsubsection{Spanning Dags}


\subsection{Events}

\label{section-gd-events}

\includeluadocumentationof{pgf.gd.lib.Event}


\subsection{Support Libraries}

\label{section-gd-libs}

\subsubsection{Lookup Tables}

\includeluadocumentationof{pgf.gd.lib.LookupTable}

\subsubsection{Computing Distances in Graphs}

\includeluadocumentationof{pgf.gd.lib.PathLengths}

\subsubsection{Priority Queues}

\includeluadocumentationof{pgf.gd.lib.PriorityQueue}




































\endinput

BEGIN OLD STUFF...



\begin{key}{/graph drawing/spring layout/random seed=\meta{number} 
  (initially 42)}
  \keyalias{graph drawing/spring electrical layout}
  
  Specifies the seed used for Lua's pseudo-random number generator. If
  set to something other than |0|, the random number sequence generated
  by the pseudo-random number generator will be the same at every run.
  The resulting graph drawings will be reproducible in consecutive runs,
  despite randomized elements used in the algorithm.
  If set to |0|, the results are not guaranteed to be reproducible.
  
  A special application of |random seed| is to unravel visually
  imperfect layouts. Often when drawings feature a few undesired edge
  crossings, this may have been caused by a unfavorable random number
  sequence. Here is an example where a different |random seed| 
  significantly improves the drawing of a graph:
  \begin{codeexample}[width=6.0cm]
\tikz \graph [spring layout={random seed=1}] 
  { 5 -- subgraph C_n [n=4] };

\tikz \graph [spring layout={random seed=4}] 
  { 5 -- subgraph C_n [n=4] };
  \end{codeexample}
%  \begin{codeexample}[]
%% bad layout due to unfavorable random number sequence
%\tikz \graph [/graph drawing/spring layout/default algorithm=Walshaw2000 spring electrical,
%              spring layout={random seed=1, coarsen=false}, orient=7-8] 
%{ subgraph Grid_n [n=8, wrap after=2] };
%
%% better layout thanks to a different random seed
%\tikz \graph [/graph drawing/spring layout/default algorithm=Walshaw2000 spring electrical,
%              spring layout={random seed=2, coarsen=false}, orient=7-8] 
%{ subgraph Grid_n [n=8, wrap after=2] };
%  \end{codeexample}
\end{key}


\subsection{A First Example}

This section presents a simple example of how a graph drawing
algorithm can be implemented. For each graph drawing algorithm
there must be a class of the name given to the |algorithm| key. This
class should usually reside in a file called
|pgfgd-algorithm-|\meta{algorithm name}. This class must provide (at
least) the two methods |new| and |run|. Each time a layout needs to
be computed for a graph, a new object of this algorithm class is
instantiated using the class's |new| method. For the newly created
object, an attribute |graph| will be set to an object representing the
graph. Then, the |constructor| method of the object is called,
provided it exists. Then, the |run| method is called, which should do
the actual work. (The separation into a constructor and a run method
is purely for convenience.) The |run| method should modify the
coordinates of the nodes of its |graph| attribute.

To simplify the creating of classes and constructors, the graph
drawing engine provides the function |pgf.gd.new_algorithm_class|, which
takes a table of infos about the algorithm as input and will create a
class and a |new| method for you.

As a complete example, the following code fragment implements a
trivial graph drawing algorithm that just places all nodes on a
fixed-size circle.  It is accessed with the name 
|SimpleDemo|.

\pgfgddeclareforwardedkeys{/graph drawing}{
  radius/.graph parameter=number,
  node radius/.node parameter=number
}

\begin{codeexample}[code only]
-- File SimpleDemo.lua

local SimpleDemo = pgf.gd.new_algorithm_class()

function SimpleDemo:run()
  local radius = 28.908  -- this is 1cm in points
  local alpha = (2 * math.pi) / #self.ugraph.vertices
  for i,vertex in ipairs(self.ugraph.vertices) do
    vertex.pos.x = math.cos(i * alpha) * radius
    vertex.pos.y = math.sin(i * alpha) * radius
  end
end

return SimpleDemo  
\end{codeexample}

The algorithm computes a circular layout like in the following.

\tikz [graph drawing scope, /graph drawing/algorithm=pgf.gd.examples.SimpleDemo]
  \graph { f -> c -> e -> a -> {b -> {c, d, f}, e -> b}};

\begin{codeexample}[code only]
\tikz [graph drawing scope, /graph drawing/algorithm=SimpleDemo]
  \graph { f -> c -> e -> a -> {b -> {c, d, f}, e -> b}};
\end{codeexample}


\subsection{Lua Layer: Overview}

All of the graph drawing engine resides in the directory
|graphdrawing| of |pgf|. Inside, there are the following
subdirectories:

\begin{itemize}
\item |tex|, with the \pgfname\ and \tikzname\ library files and
\item |lua|, with all the Lua code, including both support files and
  actual algorithms.
\end{itemize}

TODO: Details...

drawing algorithms are implemented in Lua, these directories also
contain mainly Lua files, ...

The job of this library is to make the graph
parameters of the algorithms visible to \pgfname, so this file
typically just contains calls of |\pgfgddeclarealgorithmkey| and
|\pgfgddeclareforwardedkeys|.

In the following, we first describe which steps are necessary to
implement a new graph drawing algorithm. We then have a look at the
classes made available to graph drawing algorithms by the
engine. Finally, the section concludes with a class and function
reference.



\subsection{Lua Layer: Installing Graph Drawing Algorithms}
\label{section-gd-implementing-algorithms}

In the following we describe in detail how a new graph drawing
algorithm can be implemented and installed. 


\subsubsection{Starting the Graph Drawing Engine}

First, before any graph drawing algorithms can be used, the graph
drawing engine needs to be loaded. This is done by loading the
\pgfname\ library |pgflibrarygraphdrawing|. This will
initialise the Lua graph drawing subsystem by invoking the Lua loader
class.   

In the most basic cases, no further \TeX\ code needs to be written to
use a new graph drawing algorithm; but we will see later on, that a
small entry in an appropriate \pgfname\ library of graph drawing
algorithms will make the use of the algorithm somewhat simpler.


\subsubsection{Main File of the Graph Drawing Algorithm}

As indicated at in the description of the |algorithm| key on
page~\ref{section-gd-algorithm-key}, each graph drawing algorithm  
must be implemented in a class. The class name will be the value
passed to the |algorithm| key, albeit without any spaces. This class
is a normal class in the sense of Lua and it will be loaded by calling
|require| on the value. The class file must return the class.

During |\pgfgdendscope|, the algorithm's class will be loaded, if
necessary. Then, the following things happen, in turn, in normal
operation mode: (Let \meta{class} by the class name.)

\begin{enumerate}
\item Some preprocessing is done, if the ``static'' attributes of the
  class specify this.
\item A new instance of the class is created by calling
  \meta{class}|:new(graph)|, where |graph| is a variable holding the
  graph object. 
\item The |run| method of the instance is called. This method should now
  compute ``good'' positions for the nodes in the graph represented by
  the |graph|.
\item Post layout operations are performed, namely orienting the
  graph and then anchoring the graph. Both operations are performed
  automatically, but it is possible to configure them.
\end{enumerate}

TODO: Work on this:

Let us now look at the |graph_drawing_algorithm| function in more
detail. It takes a single parameter \meta{info}, which must be a
table, and does the following:

\begin{enumerate}
\item It declares a new class.
\item It declares a |new| method for this class, which takes two
  parameters |g| and |algo| and returns a new \meta{instance} of the
  class. The first parameter will be installed in the attribute
  |graph| of \meta{instance}, the second in the |parent_algorithm|
  attribute. The |new| method tests whether a key called 
  |graph_options| is defined in the \meta{info} table. If so, the
  value for this key must be a table of \meta{options}. This table is
  processed as follows: For each pair \meta{key} |=| \meta{value}
  inside the \meta{options} table, an attribute \meta{key} is created
  in \meta{instance}. The \meta{value} is then used to set \meta{key}
  to 
  \begin{quote}
    |graph:getOption(|\meta{value}|)|
  \end{quote}
\item If the key |properties| is defined inside the \meta{info} table,
  its value should be table of ``static'' or ``default'' values for
  the class. More precisely, this table is used as the metatable of
  the class.
\end{enumerate}

Let us have a look at an example: We redo our |SimpleDemo|, but this
time using the full power of the |graph_drawing_algorithm| function:

\begin{codeexample}[code only]
local SimpleDemo = pgf.gd.new_algorithm_class = {
  
  -- Declare a property:
  properties = {
    -- Ensures, that the graph is always connected when the graph
    -- drawing algorithm is called
    works_only_on_connected_graphs = true
  },

  -- Declare a graph parameter:
  graph_parameters = {
    label  = '/graph drawing/label',
    radius = '/graph drawing/radius',
  }
}
\end{codeexample}

The code is equivalent to the following:
\begin{codeexample}[code only]
local SimpleDemo = { works_only_on_connected_graphs = true }
SimpleDemo.__index = SimpleDemo

function SimpleDemo:new(g)
  local obj = { graph = g }
  setmetatable(obj, SimpleDemo)
  
  obj.label  = g:getOption('/graph drawing/label')
  obj.radius = g:getOption('/graph drawing/radius')
  
  return obj  
end  
\end{codeexample}



\subsubsection{Coordinate Systems in Lua}

\label{section-gd-lua-coordinates}

The main job of a graph drawing algorithm is to modify the
coordinates of the nodes of the graph object in the |graph|
attribute. Before we have a look at how this can be done, let us 
first clarify how the different coordinate systems of \pgfname\
interact with the graph drawing engine.

Let us start with the case that there is no special transformation
matrix setup is setup. In this case, all coordinates inside the Lua
layer are pairs of numbers that will be interpreted as dimensions in
\TeX\ points (one \TeX\ point equals 1/72.27 inches). The first number
will be interpreted as the $x$-coordinate (going right) and the second
number will be interpreted as the $y$-coordinate (going up). This is
true both for the bounding boxes of the nodes that are passed down to
the Lua layer, but also also for the coordinates that are computed by
the algorithms inside the Lua layer.

When graph parameters are set using the |number|
syntax, the dimensions will already have been converted into this
coordinate system. For instance, when a user writes
|node distance=1in|, then |getOption('/graph drawing/node distance')|
will yield the string |'72.27'|.

When a transformation matrix is
set, such as a shift by 1cm to the right and a rotation by
30$^\circ$, the following happens: At the beginning of a
graph drawing scope, the transformation matrix is reset. Thus, for
instance all nodes created inside the graph drawing scope for which no
scaling or shifting is setup will be centered on the origin. When
|\pgfgdendscope| is reached, the transformation matrix is immediately
restored, \emph{prior} to inserting the nodes at the computed
positions. This means, in particular, that the coordinates computed by
the graph drawing algorithms will be transformed by the transformation
matrix that was in force at the beginning of the graph drawing
scope. Continuing the example, all coordinates computed by the graph
drawing algorithms would be shifted by 1cm and then rotated by
30$^\circ$.

The bottom line is that graph drawing algorithms do not need to worry
about \pgfname's transformation matrix.



\subsubsection{Example of a Graph Drawing Algorithm's Code}

The following code fragment (taken and slightly altered
from the file |pgfgd-algorithm-SimpleDemo.lua|)
implements a trivial graph drawing algorithm that just places all
nodes on a fixed-size circle.  

\pgfgddeclareforwardedkeys{/graph drawing}{
  radius/.graph parameter=number
}


\begin{codeexample}[code only]
local SimpleDemo = pgf.gd.new_algorithm_class {}

function SimpleDemo:run()
  local radius = 28.908  -- this is 1cm in points

  local alpha = (2 * math.pi) / #self.graph.nodes
  for i,node in ipairs(self.graph.nodes) do
    -- the interesting part...
    node.pos.x = radius * math.cos(i * alpha)
    node.pos.y = radius * math.sin(i * alpha)
  end
end

return SimpleDemo
\end{codeexample}


\subsubsection{Setting Up a Key for Selecting the Algorithm}

Users will typically wish to write something shorter than
|graph drawing scope...| in order to run a graph drawing algorithm on
a graph. For this reason, you should setup a style on the \TeX\ side
that calls the above keys. For instance, you could create a small
\tikzname\ library and place the following in the library:

\begin{codeexample}[code only]
\tikzset{circle layout/.style={
    graph drawing scope,
    /graph drawing/algorithm=SimpleDemo}}
\end{codeexample}

However, there is a better command for this:

\begin{codeexample}[code only]
% Place this in a file like pgflibrarygraphdrawing.circular.code.tex
\pgfgddeclarealgorithmkey
{circle layout}
{circle layout}
{algorithm=SimpleDemo}
\end{codeexample}
\pgfgddeclarealgorithmkey
{circle layout}
{circle layout}
{algorithm=pgf.gd.examples.SimpleDemo}

The |\pgfgddeclarealgorithmkey| takes care of setting up your style
key in appropriate ways (including some ways you will not have thought
of) and installs some additional handlers.


\subsubsection{Setting Up Graph Parameters}

Returning to the algorithm, it would be better if we could
``configure'' the radius of the circle. The graph drawing engine
provides for this case: You can declare certain \pgfname\ keys to be
so-called ``graph parameters''. When a key is declared as a graph
parameter, it will be available inside the algorithm: 

\begin{codeexample}[code only]
local radius = tonumber(self.graph:getOption("/graph drawing/radius"))
\end{codeexample}

Using the |getOption| method we obtain the value of the
graph parameter, but we must first register the key on the \TeX\ side
as follows: 

\begin{codeexample}[code only]
\pgfgddeclareforwardedkeys{/graph drawing}{
  radius/.graph parameter=number,
  radius/.parameter initial=1cm
}
\end{codeexample}

The |number| tells \TeX\ that whenever you assign
something to the |radius| option, the mathematical expression should
be evaluated and the result should be passed down to the graph drawing
algorithm. For instance, when you write |radius=20pt+3.5pt|, the
algorithm will get the value |23.5| as a result to calling
|getOption|. The |getOption| function will return a |nil| value for
keys that have not been set. While this sometimes is desired
behaviour, in our example we would want the radius to be set to a
default value (1cm in this case) when nothing has been specified. This
is achieved by the second line. The result of the above modifications
can be seen in the following example:

\begin{codeexample}[]
\tikz \graph [circle layout, radius=1.5cm]
  {f -> c -> e ->[bend right] a -> {b -> {c, d, f}, e -> b}};
\end{codeexample}

Since graph parameters are used quite frequently, there is special
support for them: In the declaration of class via
|pgf.gd.new_algorithm_class|, you can provide a key
|graph_parameters|. This key will take a table of key--value pairs,
where the key is interpreted as an attribute of the algorithm object
and the value is a string possibly suffixed by a type of the parameter
in square brackets. The string is interpreted as the name of a graph
parameter (without the |/graph drawing/| part of the path) and the
attribute will be setup to the value of this graph parameter, possibly
after a typecast. For our example, we would write:

\begin{codeexample}[code only]
local SimpleDemo = pgf.gd.new_algorithm_class {
  graph_parameters = { radius = "radius [number]" }
}
\end{codeexample}
In the main program, we can now write |self.radius|.

In addition to graph parameters, we can also have \emph{node
  parameters}. These are setup similarly to |.graph parameter|, but
with |.node parameter| and they are then accessed via the
|node:getOption| function.

As a slightly artificial example, let us introduce a |node radius|
key, which allows us to change the radius of a single node. For this,
we check for a node whether its radius key is set:

\begin{codeexample}[code only]
-- In SimpleDemo.lua:

  for i,node in ipairs(self.graph.nodes) do
    -- the interesting part...
    local node_radius = tonumber(node:getOption('/graph drawing/node radius')
                                 or self.radius)
    node.pos.x = node_radius * math.cos(i * alpha)
    node.pos.y = node_radius * math.sin(i * alpha)
  end
   
% In pgflibrarygraphdrawing.circle.code.tex
\pgfgddeclareforwardedkeys{/graph drawing}{
  node radius/.node parameter=number
}
\end{codeexample}
\pgfgddeclareforwardedkeys{/graph drawing}{
  node radius/.node parameter=number
}

\begin{codeexample}[]
\tikz \graph [circle layout]
  { a -> b -> c -> d [node radius=2cm] -> e -> a };
\end{codeexample}

Here is the complete code of the final algorithm:
\begin{codeexample}[code only]
-- File SimpleDemo.lua

local SimpleDemo = pgf.gd.new_algorithm_class {
  graph_parameters = { radius = "radius [number]" }
}

function SimpleDemo:run()
  local alpha = (2 * math.pi) / #self.graph.nodes
  for i,node in ipairs(self.graph.nodes) do
    -- the interesting part...
    local node_radius = tonumber(node:getOption('/graph drawing/node radius')
                                 or self.radius)
    node.pos.x = node_radius * math.cos(i * alpha)
    node.pos.y = node_radius * math.sin(i * alpha)
  end
end
   
return SimpleDemo  
\end{codeexample}

\begin{codeexample}[code only]
% File pgflibrarygraphdrawing.circle.code.tex
  
\pgfgddeclarealgorithmkey
  {circle layout}
  {circle layout}
  {algorithm=SimpleDemo}

\pgfgddeclareforwardedkeys{/graph drawing}{
  radius/.graph parameter=number,
  radius/.parameter initial=1cm,
  node radius/.node parameter=number
}
\end{codeexample}




\subsection{Lua Layer: Pre- and Postprocessing}

A number of tasks in graph drawing can be performed independently of
the actual algorithm used. For instance, many algorithms require that
the graph is connected. In this case, unconnected input graphs first
need to be decomposed into their connected components, which should
then be processed independently. Such a step would be
\emph{preprocessing} step. Similarly, once a graph has been laid out
by an algorithm, it often still needs to be shifted around to its
anchoring position. This step is the same for any algorithm and can be
done in a \emph{postprocessing} step.

It turns out that some pre- or postprocessing steps make sense for
certain algorithms, but not for other algorithms. For this reason, an
algorithm can specify which steps should (not) be performed by setting
certain attributes in the algorithm's class. Usually, these attributes
will be set using the |properties| key in the declaration of the
algorithm's class.

In the following, we describe which steps are performed and which keys
influence them.


\subsubsection{Preprocessing}

The following preprocessing steps are performed for every graph:
\begin{enumerate}
\item If the |works_only_on_connected_graphs| property is set, the
  connected components of the graph will first be computed.
\item For each component or, if the property is not set, for the whole
  graph, a new algorithm object is created.
\item The |run| method is then called for each component,
  \emph{unless} the size of the component is |1|. If, however, the
  |run_also_for_single_node| property is set, the algorithm is even
  invoked for a 1-node graph.
\end{enumerate}

\subsubsection{Postprocessing}

Each time the |run| method finishes, the following postprocessing
operations are performed:
\begin{enumerate}
\item The graph is oriented, see
  Section~\ref{subsection-library-graphdrawing-standard-orientation}. A
  graph drawing algorithm can set the |growth_direction| property in
  case the graph has a natural growth direction.
\item The graph is anchored, see
  Section~\ref{subsection-library-graphdrawing-anchoring}. 
\end{enumerate}
The above steps are applied to each connected component individually
if the splitting key has been set.

The components then need to be ``packed'', but this is not yet
implemented.




\subsection{Lua Layer: The Main Classes}

In the following, details of the different main classes that are
useful for graph drawing algorithms are documented.


\subsubsection{The Graph Class}

The class |Graph| is used to represent graphs and contains
references to the nodes and edges stored in a graph.

A graph drawing algorithm gets passed a |Graph| object that represents
the to-be-layouted graph. However, you can also create new graph
objects, for instance to decompose the graph into connected
components. 

To create a new graph, you can use the |copy| method, which creates a 
shallow copy (without coying nodes or edges), and the
|subGraphParent| method, which creates a deep copy of the graph, edge
and node objects starting at a designated parent node. If you need
more control by supplying your own set of already visited nodes, use
the underlying function |subGraph|.

A graph allows you to add and remove nodes and edges via |addNode|,
|addEdge|, |removeNode| and |removeEdge| respectively.  There are also
variants which remove all incident edges on a node removal and
conversely, |deleteNode| and |deleteEdge|.

Nodes can be looked up by name with |findNode|. The more generic
|findNodeIf| allows you to search for a node passing a test
predicate. 

The |walkDepth| and |walkBreadth| methods may be used to get
iterators over all nodes and edges in a depth-first or breadth-first
order (other traversal orders may require a rewrite or extension of the
|walkAux| method).

Positions are represented using the class |Vector|.

The following tasks are typical for manipulating the graph.

\begin{itemize}
\item Iterate over all nodes.
\begin{codeexample}[code only]
for node in table.value_iter(self.graph.nodes) do
   ...
end
\end{codeexample}
\item Get width or height of a node:
\begin{codeexample}[code only]
local width, height = node.width, node.height
\end{codeexample}
\item Get or set the coordinates of a node. The final values of these
  coordinates will be used during as the actual positions of the nodes
  on the page.
\begin{codeexample}[code only]
node.pos.x = node.pos.x + 1
node.pos.y = node.pos.y + 1
\end{codeexample}
\item Iterate over all edges and all nodes of the current edge.
\begin{codeexample}[code only]
for _,edge in ipairs(self.graph.edges) do
   for _,node in ipairs(edge.nodes) do
      ...
   end
end
\end{codeexample}
\item Get the nodes connected by an edge.
\begin{codeexample}[code only]
local nodeLeft = edge.nodes[1]
local nodeRight = edge.nodes[2]
\end{codeexample}
\end{itemize}

%% This file has been generated from the lua sources using LuaDoc.
% To regenerate it call "make genluadoc" in
% doc/generic/pgf/version-for-luatex/en.

\begin{filedescription}{pgflibrarygraphdrawing-graph.lua}


\begin{luacommand}{{Graph:\textunderscore{}\textunderscore{}tostring}()}
Returns a string representation of this graph including all nodes and edges. 


Return value:
\begin{parameterdescription} 
  \item[] Graph as string. 
\end{parameterdescription}


\end{luacommand}
\begin{luacommand}{{Graph:addEdge}(\meta{edge})}
Adds an edge to the graph. 

Parameters:
\begin{parameterdescription}
	\item[\meta{edge}] The edge to be added. 
\end{parameterdescription}



\end{luacommand}
\begin{luacommand}{{Graph:addNode}(\meta{node})}
Adds a node to the graph. 

Parameters:
\begin{parameterdescription}
	\item[\meta{node}] The node to be added. 
\end{parameterdescription}



\end{luacommand}
\begin{luacommand}{{Graph:copy}()}
Creates a shallow copy of a graph.  The nodes and edges of the original graph are not preserved in the copy. 


Return value:
\begin{parameterdescription} 
  \item[] A shallow copy of the graph. 
\end{parameterdescription}


\end{luacommand}
\begin{luacommand}{{Graph:createEdge}(\meta{first\_node},\meta{second\_node},\meta{direction},\meta{edge\_nodes},\meta{options},\meta{tikz\_options})}
Creates and adds a new edge to the graph. 

Parameters:
\begin{parameterdescription}
	\item[\meta{first\_node}] The first node of the new edge.\item[\meta{second\_node}] The second node of the new edge.\item[\meta{direction}] The direction of the new edge. Possible values are \begin{itemize} \item |Edge.UNDIRECTED|, \item |Edge.LEFT|, \item |Edge.RIGHT|, \item |Edge.BOTH| and \item |Edge.NONE| (for invisible edges). \end{itemize}\item[\meta{edge\_nodes}] A string of \tikzname\ edge nodes that needs to be passed back to the \TeX layer unmodified.\item[\meta{options}] The options of the new edge.\item[\meta{tikz\_options}] A table of \tikzname\ options to be used by graph drawing algorithms to treat the edge in special ways. 
\end{parameterdescription}


Return value:
\begin{parameterdescription} 
  \item[] The newly created edge. 
\end{parameterdescription}


\end{luacommand}
\begin{luacommand}{{Graph:deleteEdge}(\meta{edge})}
Like removeEdge, but also removes the edge from its adjacent nodes. 

Parameters:
\begin{parameterdescription}
	\item[\meta{edge}] The edge to be deleted. 
\end{parameterdescription}


Return value:
\begin{parameterdescription} 
  \item[] The removed edge or |nil| if it was not found in the graph. 
\end{parameterdescription}


\end{luacommand}
\begin{luacommand}{{Graph:deleteNode}(\meta{node})}
Like removeNode, but also deletes all adjacent edges of the removed node.  This function also removes the deleted adjacent edges from all neighbours of the removed node. 

Parameters:
\begin{parameterdescription}
	\item[\meta{node}] The node to be deleted together with its adjacent edges. 
\end{parameterdescription}


Return value:
\begin{parameterdescription} 
  \item[] The removed node or |nil| if the node was not found in the graph. 
\end{parameterdescription}


\end{luacommand}
\begin{luacommand}{{Graph:findNode}(\meta{name})}
If possible, looks up the node with the given name in the graph. 

Parameters:
\begin{parameterdescription}
	\item[\meta{name}] Name of the node to look up. 
\end{parameterdescription}


Return value:
\begin{parameterdescription} 
  \item[] The node with the given name or |nil| if it was not found in the graph. 
\end{parameterdescription}


\end{luacommand}
\begin{luacommand}{{Graph:findNodeIf}(\meta{test})}
Looks up the first node for which the function \meta{test} returns |true|. 

Parameters:
\begin{parameterdescription}
	\item[\meta{test}] A function that takes one parameter (a |Node|) and returns |true| or |false|. 
\end{parameterdescription}


Return value:
\begin{parameterdescription} 
  \item[] The first node for which \meta{test} returns |true|. 
\end{parameterdescription}


\end{luacommand}
\begin{luacommand}{{Graph:getOption}(\meta{name})}
Returns the value of the graph option \meta{name}. 

Parameters:
\begin{parameterdescription}
	\item[\meta{name}] Name of the option. 
\end{parameterdescription}


Return value:
\begin{parameterdescription} 
  \item[] The value of the graph option \meta{name} or |nil|. 
\end{parameterdescription}


\end{luacommand}
\begin{luacommand}{{Graph:mergeOptions}(\meta{options})}
Merges the given options into the options of the graph. 

Parameters:
\begin{parameterdescription}
	\item[\meta{options}] The options to be merged. 
\end{parameterdescription}



See also:
\begin{itemize}
	\item[] |table.custom_merge |
\end{itemize}

\end{luacommand}
\begin{luacommand}{{Graph:new}(\meta{values})}
Creates a new graph. 

Parameters:
\begin{parameterdescription}
	\item[\meta{values}] Values to override default graph settings. The following parameters can be set:\par |nodes|: The nodes of the graph.\par |edges|: The edges of the graph.\par |pos|: Initial position of the graph.\par |options|: A table of node options passed over from \tikzname. 
\end{parameterdescription}


Return value:
\begin{parameterdescription} 
  \item[] A newly-allocated graph. 
\end{parameterdescription}


\end{luacommand}
\begin{luacommand}{{Graph:removeEdge}(\meta{edge})}
If possible, removes an edge from the graph and returns it. 

Parameters:
\begin{parameterdescription}
	\item[\meta{edge}] The edge to be removed. 
\end{parameterdescription}


Return value:
\begin{parameterdescription} 
  \item[] The removed edge or |nil| if it was not found in the graph. 
\end{parameterdescription}


\end{luacommand}
\begin{luacommand}{{Graph:removeNode}(\meta{node})}
If possible, removes a node from the graph and returns it. 

Parameters:
\begin{parameterdescription}
	\item[\meta{node}] The node to remove. 
\end{parameterdescription}


Return value:
\begin{parameterdescription} 
  \item[] The removed node or |nil| if it was not found in the graph. 
\end{parameterdescription}


\end{luacommand}
\begin{luacommand}{{Graph:setOption}(\meta{name},\meta{value})}
Sets the graph option \meta{name} to \meta{value}. 

Parameters:
\begin{parameterdescription}
	\item[\meta{name}] Name of the option to be changed.\item[\meta{value}] New value for the graph option \meta{name}. 
\end{parameterdescription}



\end{luacommand}
\begin{luacommand}{{Graph:subGraph}(\meta{root},\meta{graph},\meta{visited})}
Returns a subgraph.  The resulting subgraph begins at the node root, excludes all nodes and edges that are marked as visited. 

Parameters:
\begin{parameterdescription}
	\item[\meta{root}] Root node where the operation starts.\item[\meta{graph}] Result graph object or |nil| if the original graph should be used as the parent graph.\item[\meta{visited}] Set of already visited nodes/edges or |nil|. This set will be modified so make sure not to use a table that you want to remain untouched. 
\end{parameterdescription}



\end{luacommand}
\begin{luacommand}{{Graph:subGraphParent}(\meta{root},\meta{parent},\meta{graph})}
Creates a new subgraph with \meta{parent} marked as visited.  This function can be useful if the graph is a tree structure (and \meta{parent} is the parent node of \meta{root}). 

Parameters:
\begin{parameterdescription}
	\item[\meta{root}] Root node where the operation starts.\item[\meta{parent}] Parent of the recursion step before.\item[\meta{graph}] Result graph object or |nil| if the original graph should be used as the parent graph. 
\end{parameterdescription}



See also:
\begin{itemize}
	\item[] |subGraph |
\end{itemize}

\end{luacommand}
\begin{luacommand}{{Graph:walkAux}(\meta{root},\meta{visited},\meta{remove\_index})}
Auxiliary function to walk a graph. Does nothing if no nodes exist. 

Parameters:
\begin{parameterdescription}
	\item[\meta{root}] The first node to be visited. If nil, chooses some node.\item[\meta{visited}] Set of already visited nodes and edges. |visited[v] == true| indicates that the node or edge |v| has already been visited.\item[\meta{remove\_index}] A numeric value or |nil| that defines the order in which nodes and edges are visited while traversing the graph. |nil| results in queue behavior, |1| in stack behavior. 
\end{parameterdescription}



See also:
\begin{itemize}
	\item[] |walkDepth|\item[] |walkBreadth |
\end{itemize}

\end{luacommand}
\begin{luacommand}{{Graph:walkBreadth}(\meta{root},\meta{visited})}
Returns an iterator to walk the graph in a breadth-first traversal.  The iterator returns all edges and nodes one at a time. In case only the nodes are of interest, a filter function like |iter.filter| can be used to ignore edges. 

Parameters:
\begin{parameterdescription}
	\item[\meta{root}] The first node to be visited. If nil, chooses some node.\item[\meta{visited}] Set of already visited nodes and edges. |visited[v] == true| indicates that the node or edge |v| has already been visited. 
\end{parameterdescription}



See also:
\begin{itemize}
	\item[] |iter.filter |
\end{itemize}

\end{luacommand}
\begin{luacommand}{{Graph:walkDepth}(\meta{root},\meta{visited})}
Returns an iterator to walk the graph in a depth-first traversal.  The iterator returns all edges and nodes one at a time. In case only the nodes are of interest, a filter function like |iter.filter| can be used to ignore edges. 

Parameters:
\begin{parameterdescription}
	\item[\meta{root}] The first node to be visited. If nil, chooses some node.\item[\meta{visited}] Set of already visited nodes and edges. |visited[v] == true| indicates that the node or edge |v| has already been visited. 
\end{parameterdescription}



See also:
\begin{itemize}
	\item[] |iter.filter |
\end{itemize}

\end{luacommand}

\end{filedescription}
%
%The following module simplifies the traversal of graphs:
%
%\input{generated/pgflibrarygraphdrawing-traversal-helpers}



\subsubsection{Nodes}

Nodes serve as direct representations of the \TeX\ level nodes and
include information about incident edges, the calculated position and
the \TeX\ box used.  Typically one'll use its methods to navigate
through the graph or to add and remove edges in an intermediary graph.
Using the information from the \TeX\ side, this class is also able to
provide layout information, i.e. the dimensions of the corresponding
\TeX\ box.

%% This file has been generated from the lua sources using LuaDoc.
% To regenerate it call "make genluadoc" in
% doc/generic/pgf/version-for-luatex/en.

\begin{filedescription}{pgflibrarygraphdrawing-node.lua}


\begin{luacommand}{{Node:\textunderscore{}\textunderscore{}eq}(\meta{object})}
Compares two nodes by name.

Parameters:
\begin{parameterdescription}
	\item[\meta{object}] The node to be compared to self
\end{parameterdescription}


Return value:
\begin{itemize} \item[] True if self is equal to object. \end{itemize}


\end{luacommand}\begin{luacommand}{{Node:\textunderscore{}\textunderscore{}tostring}()}
Returns a formated string representation of the node.


Return value:
\begin{itemize} \item[] String represenation of the node. \end{itemize}


\end{luacommand}\begin{luacommand}{{Node:addEdge}(\meta{edge})}
Adds new Edge to the Node.

Parameters:
\begin{parameterdescription}
	\item[\meta{edge}] The edge to be added.
\end{parameterdescription}



\end{luacommand}\begin{luacommand}{{Node:copy}()}
Creates a shallow copy of a node.


Return value:
\begin{itemize} \item[] Copy of the node. \end{itemize}


\end{luacommand}\begin{luacommand}{{Node:degree}()}
Computes the number of neighbour nodes.


Return value:
\begin{itemize} \item[] Number of neighbours. \end{itemize}


\end{luacommand}\begin{luacommand}{{Node:getEdges}()}
Gets all Edges of the node.


Return value:
\begin{itemize} \item[] The edges of the node as a table. \end{itemize}


\end{luacommand}\begin{luacommand}{{Node:getOption}(\meta{name})}
Returns the value of option name or nil.

Parameters:
\begin{parameterdescription}
	\item[\meta{name}] Name of the option.
\end{parameterdescription}


Return value:
\begin{itemize} \item[] The stored value of the option or nil. \end{itemize}


\end{luacommand}\begin{luacommand}{{Node:getTexHeight}()}
Computes the Heigth of the Node.


Return value:
\begin{itemize} \item[] Height of the Node. \end{itemize}


\end{luacommand}\begin{luacommand}{{Node:getTexWidth}()}
Computes the Width of the Node.


Return value:
\begin{itemize} \item[] Width of the Node. \end{itemize}


\end{luacommand}\begin{luacommand}{{Node:mergeOptions}(\meta{options})}
Merges options.

Parameters:
\begin{parameterdescription}
	\item[\meta{options}] The options to be merged.
\end{parameterdescription}



See also:
\begin{itemize}
	\item[] |mergeTable|
\end{itemize}

\end{luacommand}\begin{luacommand}{{Node:new}(\meta{values})}
Creates a new node.

Parameters:
\begin{parameterdescription}
	\item[\meta{values}] Values (e.g. position) to be merged with the default-metatable of a node
\end{parameterdescription}


Return value:
\begin{itemize} \item[] A newly allocated node object. \end{itemize}


\end{luacommand}\begin{luacommand}{{Node:removeEdge}(\meta{edge})}
Removes an edge from the node.

Parameters:
\begin{parameterdescription}
	\item[\meta{edge}] The edge to remove.
\end{parameterdescription}



\end{luacommand}\begin{luacommand}{{Node:setOption}(\meta{name},\meta{value})}
Sets the option name to value.

Parameters:
\begin{parameterdescription}
	\item[\meta{name}] Name of the option to be set.\item[\meta{value}] Value for the option defined by name.
\end{parameterdescription}



\end{luacommand}
\end{filedescription}


\subsubsection{The Edge Class}

|Edge| objects contain references to incident nodes, including the
possibility to create hyperedges with more than two nodes for an edge.
Edges can be undirected or directed (denoted by the constants
|Edge.UNDIRECTED| or |Edge.LEFT|, |Edge.RIGHT|, |Edge.BOTH| and
|Edge.NONE| for invisible edges, see |Interface:drawEdge|). 

%% This file has been generated from the lua sources using LuaDoc.
% To regenerate it call "make genluadoc" in
% doc/generic/pgf/version-for-luatex/en.

\begin{filedescription}{pgflibrarygraphdrawing-edge.lua}


\begin{luacommand}{{Edge:\textunderscore{}\textunderscore{}eq}(\meta{other})}
Returns whether or not the two edges have the same adjacent nodes. 

Parameters:
\begin{parameterdescription}
	\item[\meta{other}] Another edge to compare with. 
\end{parameterdescription}


Return value:
\begin{parameterdescription} 
  \item[] |true| if the two edges have exactly the same adjacent nodes. 
\end{parameterdescription}


\end{luacommand}
\begin{luacommand}{{Edge:\textunderscore{}\textunderscore{}tostring}()}
Returns a readable string representation of the edge. 


Return value:
\begin{parameterdescription} 
  \item[] String representation of the edge. 
\end{parameterdescription}


\end{luacommand}
\begin{luacommand}{{Edge:addNode}(\meta{node})}
If possible, adds a node to the edge. 

Parameters:
\begin{parameterdescription}
	\item[\meta{node}] The node to be added to the edge. 
\end{parameterdescription}



\end{luacommand}
\begin{luacommand}{{Edge:containsNode}(\meta{node})}
Returns whether or not a node is adjacent to the edge. 

Parameters:
\begin{parameterdescription}
	\item[\meta{node}] The node to check. 
\end{parameterdescription}


Return value:
\begin{parameterdescription} 
  \item[] |true| if the node is adjacent to the edge. |false| otherwise. 
\end{parameterdescription}


\end{luacommand}
\begin{luacommand}{{Edge:copy}()}
Copies an edge (preventing accidental use).  The nodes of the edge are not preserved and have to be added to the copy manually if necessary. 


Return value:
\begin{parameterdescription} 
  \item[] Shallow copy of the edge. 
\end{parameterdescription}


\end{luacommand}
\begin{luacommand}{{Edge:getDegree}()}
Counts the nodes on this edge. 


Return value:
\begin{parameterdescription} 
  \item[] The number of nodes on the edge. 
\end{parameterdescription}


\end{luacommand}
\begin{luacommand}{{Edge:getNeighbour}(\meta{node})}
Gets first neighbour of the node (disregarding hyperedges). 

Parameters:
\begin{parameterdescription}
	\item[\meta{node}] The node which first neighbour should be returned. 
\end{parameterdescription}


Return value:
\begin{parameterdescription} 
  \item[] The first neighbour of the node. 
\end{parameterdescription}


\end{luacommand}
\begin{luacommand}{{Edge:getNeighbours}(\meta{node})}
Returns all neighbours of a node adjacent to the edge.  The edge direction is not taken into account, so this method always returns all neighbours even if called on a directed edge. 

Parameters:
\begin{parameterdescription}
	\item[\meta{node}] A node. Typically but not necessarily adjacent to the edge. If the node is not an intermediate or end point of the edge, an empty array is returned. 
\end{parameterdescription}


Return value:
\begin{parameterdescription} 
  \item[] An array of nodes that are adjacent to the input node via the edge the method is called on. 
\end{parameterdescription}


\end{luacommand}
\begin{luacommand}{{Edge:getNodes}()}
Returns all nodes of the edge.  Instead of calling |edge:getNodes()| the nodes can alternatively be accessed directly with |edge.nodes|. 


Return value:
\begin{parameterdescription} 
  \item[] All edges of the node. 
\end{parameterdescription}


\end{luacommand}
\begin{luacommand}{{Edge:getOption}(\meta{name})}
Returns the value of the edge option \meta{name}. 

Parameters:
\begin{parameterdescription}
	\item[\meta{name}] Name of the option. 
\end{parameterdescription}


Return value:
\begin{parameterdescription} 
  \item[] The value of the edge option \meta{name} or |nil|. 
\end{parameterdescription}


\end{luacommand}
\begin{luacommand}{{Edge:isHead}(\meta{node},\meta{ignore\_reversed})}
Checks whether a node is the head of the edge. Does not work for hyperedges.  This method only works for edges with two adjacent nodes.  For undirected edges or edges that point into both directions, the result will always be true. Directed edges may be reversed internally, so their head and tail might be switched. Whether or not this internal reversal is handled by this method can be specified with the optional second \meta{ignore\_reversed} parameter which is |false| by default. 

Parameters:
\begin{parameterdescription}
	\item[\meta{node}] The node to check.\item[\meta{ignore\_reversed}] Optional parameter. Set this to true if reversed edges should not be considered reversed for this method call. 
\end{parameterdescription}


Return value:
\begin{parameterdescription} 
  \item[] True if the node is the head of the edge. 
\end{parameterdescription}


\end{luacommand}
\begin{luacommand}{{Edge:isHyperedge}()}
Returns whether or not the edge is a hyperedge.  A hyperedge is an edge with more than two adjacent nodes. 


Return value:
\begin{parameterdescription} 
  \item[] |true| if the edge is a hyperedge. |false| otherwise. 
\end{parameterdescription}


\end{luacommand}
\begin{luacommand}{{Edge:isTail}(\meta{node},\meta{ignore\_reversed})}
Checks whether a node is the tail of the edge. Does not work for hyperedges.  This method only works for edges with two adjacent nodes.  For undirected edges or edges that point into both directions, the result will always be true.  Directed edges may be reversed internally, so their head and tail might be switched. Whether or not this internal reversal is handled by this method can be specified with the optional second \meta{ignore\_reversed} parameter which is |false| by default. 

Parameters:
\begin{parameterdescription}
	\item[\meta{node}] The node to check.\item[\meta{ignore\_reversed}] Optional parameter. Set this to true if reversed edges should not be considered reversed for this method call. 
\end{parameterdescription}


Return value:
\begin{parameterdescription} 
  \item[] True if the node is the tail of the edge. 
\end{parameterdescription}


\end{luacommand}
\begin{luacommand}{{Edge:new}(\meta{values})}
Creates an edge between nodes of a graph. 

Parameters:
\begin{parameterdescription}
	\item[\meta{values}] Values to override default edge settings. The following parameters can be set:\par |nodes|: TODO \par |edge_nodes|: TODO \par |options|: TODO \par |tikz_options|: TODO \par |direction|: TODO \par |bend_points|: TODO \par |bend_nodes|: TODO \par |reversed|: TODO \par 
\end{parameterdescription}


Return value:
\begin{parameterdescription} 
  \item[] A newly-allocated edge. 
\end{parameterdescription}


\end{luacommand}
\begin{luacommand}{{Edge:setOption}(\meta{name},\meta{value})}
Sets the edge option \meta{name} to \meta{value}. 

Parameters:
\begin{parameterdescription}
	\item[\meta{name}] Name of the option to be changed.\item[\meta{value}] New value for the edge option \meta{name}. 
\end{parameterdescription}



\end{luacommand}

\end{filedescription}


\subsubsection{Positions and Vectors}

TT: More documentation is needed here!

%% This file has been generated from the lua sources using LuaDoc.
% To regenerate it call "make genluadoc" in
% doc/generic/pgf/version-for-luatex/en.

\begin{filedescription}{pgflibrarygraphdrawing-position.lua}


\begin{luacommand}{{Position.calcCoordsTo}(\meta{posFrom},\meta{posTo})}
Returns a vector between two positions.

Parameters:
\begin{parameterdescription}
	\item[\meta{posFrom}] Position A.\item[\meta{posTo}] Position B.
\end{parameterdescription}


Return value:
\begin{parameterdescription} 
  \item[] x- and y-coordinates of the vector between posFrom and posTo.
\end{parameterdescription}


\end{luacommand}
\begin{luacommand}{{Position:\textunderscore{}\textunderscore{}tostring}()}
Returns a readable string representation of the position.


Return value:
\begin{parameterdescription} 
  \item[] string representation of the position.
\end{parameterdescription}


\end{luacommand}
\begin{luacommand}{{Position:copy}()}
Creates a copy of this position object.


Return value:
\begin{parameterdescription} 
  \item[] Copy of the position.
\end{parameterdescription}


\end{luacommand}
\begin{luacommand}{{Position:equals}(\meta{pos})}
Returns a boolean value whether the object is equal to the given position.


Return value:
\begin{parameterdescription} 
  \item[] true if the position is equal to the given position pos.
\end{parameterdescription}


\end{luacommand}
\begin{luacommand}{{Position:getAbsCoordinates}(\meta{x},\meta{y})}
Computes absolute coordinates of a position.

Parameters:
\begin{parameterdescription}
	\item[\meta{x}] Just used internally for recrusion.\item[\meta{y}] Just used internally for recrusion.
\end{parameterdescription}


Return value:
\begin{parameterdescription} 
  \item[] Absolute position.
\end{parameterdescription}


\end{luacommand}
\begin{luacommand}{{Position:isAbsPosition}()}
Determines if the position is absolute.


Return value:
\begin{parameterdescription} 
  \item[] True if the position is absolute, else false.
\end{parameterdescription}


\end{luacommand}
\begin{luacommand}{{Position:new}(\meta{values})}
Represents a relative postion.

Parameters:
\begin{parameterdescription}
	\item[\meta{values}] Values (e.g. x- and y-coordinate) to be merged with the default-metatable of a position.
\end{parameterdescription}


Return value:
\begin{parameterdescription} 
  \item[] A new position object.
\end{parameterdescription}


\end{luacommand}
\begin{luacommand}{{Position:relateTo}(\meta{pos},\meta{keepAbsPosition})}
Relates a position to the given position.

Parameters:
\begin{parameterdescription}
	\item[\meta{pos}] The relative position.\item[\meta{keepAbsPosition}] If true, the coordinates of the position are computed in the relation to the given position pos.
\end{parameterdescription}



\end{luacommand}

\end{filedescription}
%% This file has been generated from the lua sources using LuaDoc.
% To regenerate it call "make genluadoc" in
% doc/generic/pgf/version-for-luatex/en.

\begin{filedescription}{pgflibrarygraphdrawing-vector.lua}


\begin{luacommand}{{Vector:copy}()}
Creates a copy of the vector that holds the same elements as the original. 


Return value:
\begin{parameterdescription} 
  \item[] A newly-allocated copy of the vector holding exactly the same elements. 
\end{parameterdescription}


\end{luacommand}
\begin{luacommand}{{Vector:dividedBy}(\meta{other})}
Performs a vector division and returns the result in a new vector.  The possible origins of the vector operands are resolved and are dropped in the result vector. 

Parameters:
\begin{parameterdescription}
	\item[\meta{other}] Vector to divide by. 
\end{parameterdescription}


Return value:
\begin{parameterdescription} 
  \item[] A new vector with the result of the division. 
\end{parameterdescription}


\end{luacommand}
\begin{luacommand}{{Vector:dividedByScalar}(\meta{scalar})}
Divides a vector by a scalar value and returns the result in a new vector.  The possible origin of the vector is resolved and is dropped in the result vector. 

Parameters:
\begin{parameterdescription}
	\item[\meta{scalar}] Scalar value to divide the vector by. 
\end{parameterdescription}


Return value:
\begin{parameterdescription} 
  \item[] A new vector with the result of the division. 
\end{parameterdescription}


\end{luacommand}
\begin{luacommand}{{Vector:dotProduct}(\meta{other})}
Performs the dot product of two vectors and returns the result in a new vector.  The possible origins of the vector operands are resolved during the compuation. 

Parameters:
\begin{parameterdescription}
	\item[\meta{other}] Vector to perform the dot product with. 
\end{parameterdescription}


Return value:
\begin{parameterdescription} 
  \item[] A new vector with the result of the dot product. 
\end{parameterdescription}


\end{luacommand}
\begin{luacommand}{{Vector:get}(\meta{index})}
Returns the element at the given \meta{index}. 


Return value:
\begin{parameterdescription} 
  \item[] The element at the given \meta{index}. 
\end{parameterdescription}


\end{luacommand}
\begin{luacommand}{{Vector:getOrigin}()}
Gets the origin of the vector. 


Return value:
\begin{parameterdescription} 
  \item[] Origin of the vector or |nil| if none is set. 
\end{parameterdescription}


\end{luacommand}
\begin{luacommand}{{Vector:limit}(\meta{limit\_function})}
Limits all elements of the vector in-place. 

Parameters:
\begin{parameterdescription}
	\item[\meta{limit\_function}] A function that is called for each index/element pair. It is supposed to return minimum and maximum values for the element. The element is then clamped to these values. 
\end{parameterdescription}



\end{luacommand}
\begin{luacommand}{{Vector:minus}(\meta{other})}
Subtracts two vectors and returns the result in a new vector. 

Parameters:
\begin{parameterdescription}
	\item[\meta{other}] Vector to subtract. If this vector is defined relative to an origin, then that origin is resolved when computing the subtraction of the two vectors. The result becomes |self + other.origin + other|. The origin of |self| is preserved. 
\end{parameterdescription}


Return value:
\begin{parameterdescription} 
  \item[] A new vector with the result of the subtraction. 
\end{parameterdescription}


\end{luacommand}
\begin{luacommand}{{Vector:minusScalar}(\meta{scalar})}
Subtracts a scalar value from a vector and returns the result in a new vector. 

Parameters:
\begin{parameterdescription}
	\item[\meta{scalar}] Scalar value to subtract from all elements. 
\end{parameterdescription}


Return value:
\begin{parameterdescription} 
  \item[] A new vector with the result of the subtraction. 
\end{parameterdescription}


\end{luacommand}
\begin{luacommand}{{Vector:new}(\meta{n},\meta{fill\_function},\meta{origin})}
Creates a new vector with \meta{n} values using an optional \meta{fill\_function}. 

Parameters:
\begin{parameterdescription}
	\item[\meta{n}] The number of elements of the vector.\item[\meta{fill\_function}] Optional function that takes a number between 1 and \meta{n} and is expected to return a value for the corresponding element of the vector. If omitted, all elements of the vector will be initialized with 0.\item[\meta{origin}] Optional origin vector. 
\end{parameterdescription}


Return value:
\begin{parameterdescription} 
  \item[] A newly-allocated vector with \meta{n} elements. 
\end{parameterdescription}


\end{luacommand}
\begin{luacommand}{{Vector:norm}()}
Computes the Euclidean norm of the vector. 


Return value:
\begin{parameterdescription} 
  \item[] The Euclidean norm of the vector. 
\end{parameterdescription}


\end{luacommand}
\begin{luacommand}{{Vector:normalized}()}
Normalizes the vector and returns the result in a new vector.  The possible origin of the vector is resolved during the computation and is dropped in the result vector. 


Return value:
\begin{parameterdescription} 
  \item[] Normalized version of the original vector. 
\end{parameterdescription}


\end{luacommand}
\begin{luacommand}{{Vector:plus}(\meta{other})}
Performs a vector addition and returns the result in a new vector. 

Parameters:
\begin{parameterdescription}
	\item[\meta{other}] The vector to add. If this vector is defined relative to an origin, then that origin is resolved when computing the sum of the two vectors. The sum becomes |self + other.origin + other|. The origin of |self| is preserved. 
\end{parameterdescription}


Return value:
\begin{parameterdescription} 
  \item[] A new vector with the result of the addition. 
\end{parameterdescription}


\end{luacommand}
\begin{luacommand}{{Vector:plusScalar}(\meta{scalar})}
Performs an addition with a scalar value and returns the result in a new vector.  The scalar value is added to all elements of the vector. 

Parameters:
\begin{parameterdescription}
	\item[\meta{scalar}] Scalar value to add to all elements. 
\end{parameterdescription}


Return value:
\begin{parameterdescription} 
  \item[] A new vector with the result of the addition. 
\end{parameterdescription}


\end{luacommand}
\begin{luacommand}{{Vector:reset}()}
Resets all vector elements to 0 in-place.  This does not reset the origin vector. 



\end{luacommand}
\begin{luacommand}{{Vector:set}(\meta{index},\meta{value})}
Changes the element at the given \meta{index}. 

Parameters:
\begin{parameterdescription}
	\item[\meta{index}] The index of the element to change.\item[\meta{value}] New value of the element. 
\end{parameterdescription}



\end{luacommand}
\begin{luacommand}{{Vector:setOrigin}(\meta{origin},\meta{preserve\_values})}
Sets the origin of the vector. 

Parameters:
\begin{parameterdescription}
	\item[\meta{origin}] Vector to use as the origin.\item[\meta{preserve\_values}] Optional flag. If set to |true|, the origin will be set and the current elements of the vector will be changed so that the sum of the origin and the new element values is equal to the old values. 
\end{parameterdescription}



\end{luacommand}
\begin{luacommand}{{Vector:timesScalar}(\meta{scalar})}
Multiplies a vector by a scalar value and returns the result in a new vector.  The possible origin of the vector is resolved and is dropped in the result vector. 

Parameters:
\begin{parameterdescription}
	\item[\meta{scalar}] Scalar value to multiply the vector with. 
\end{parameterdescription}


Return value:
\begin{parameterdescription} 
  \item[] A new vector with the result of the multiplication. 
\end{parameterdescription}


\end{luacommand}
\begin{luacommand}{{Vector:update}(\meta{update\_function})}
Updates the values of the vector in-place. 

Parameters:
\begin{parameterdescription}
	\item[\meta{update\_function}] A function that is called for each element of the vector. The elements are replaced by the values returned from this function. 
\end{parameterdescription}



\end{luacommand}
\begin{luacommand}{{Vector:x}()}
Convenience method that returns the first element of the vector.  The origin vector is not resolved in this function call. 


Return value:
\begin{parameterdescription} 
  \item[] The first element of the vector. 
\end{parameterdescription}


\end{luacommand}
\begin{luacommand}{{Vector:y}()}
Convenience method that returns the second element of the vector.  The origin vector is not resolved in this function call. 


Return value:
\begin{parameterdescription} 
  \item[] The second element of the vector. 
\end{parameterdescription}


\end{luacommand}

\end{filedescription}


\subsubsection{The Interface and System Classes}

The class |Interface| is the main entry point in Lua. Every
communication from \TeX\ to Lua is done here. It provides methods to
create graphs, add nodes and edges to graphs, and finally to invoke the
selected algorithm. The |Interface| class manages the stack of
graphs. When the |newGraph()| function is called, it generates a new graph
object and pushes it on the graph stack. The methods |addNode()| and
|addEdge()| are called for each node and each edge, creating the
actual Lua objects and adding them to the current graph. 

After adding nodes and edges, when the scope ends, the interface
invokes the actual algorithm to layout the graph. This is done in the
|drawGraph()| function. The next step is to put the nodes back in the
\TeX\ output stream. This is invoked by the |finishGraph()| method. 

%% This file has been generated from the lua sources using LuaDoc.
% To regenerate it call "make genluadoc" in
% doc/generic/pgf/version-for-luatex/en.

\begin{filedescription}{pgflibrarygraphdrawing-interface.lua}


\begin{luacommand}{{Interface:addEdge}(\meta{from},\meta{to},\meta{direction},\meta{edgenodes},\meta{options},\meta{tikzoptions})}
Adds an edge from one node to another by name.  Both parameters are node names and have to exist before an edge can be created between them. 

Parameters:
\begin{parameterdescription}
	\item[\meta{from}] Name of the node the edge begins at.\item[\meta{to}] Name of the node the edge ends at.\item[\meta{direction}] Direction of the edge (e.g. |--| for an undirected edge or |->| for a directed edge from the first to the second node).\item[\meta{edgenodes}] A string for \tikzname\ to generate the edge label nodes later. Needs to be passed back to TikZ unmodified.\item[\meta{options}] A string of |{key}{value}| pairs of edge options that are relevant to graph drawing algorithms.\item[\meta{tikzoptions}] A string of |{key}{value}| pairs that need to be passed back to \tikzname\ unmodified. 
\end{parameterdescription}



See also:
\begin{itemize}
	\item[] |addNode |
\end{itemize}

\end{luacommand}\begin{luacommand}{{Interface:addNode}(\meta{name},\meta{xMin},\meta{yMin},\meta{xMax},\meta{yMax},\meta{options})}
Adds a new node to the graph.  The options string of |{key}{value}| pairs is parsed and assigned to the node. Graph drawing algorithms may use these options to treat the node in special ways. 

Parameters:
\begin{parameterdescription}
	\item[\meta{name}] Name of the node.\item[\meta{xMin}] Minimum x point of the bouding box.\item[\meta{yMin}] Minimum y point of the bouding box.\item[\meta{xMax}] Maximum x point of the bouding box.\item[\meta{yMax}] Maximum y point of the bouding box.\item[\meta{options}] Options for the node. 
\end{parameterdescription}



\end{luacommand}\begin{luacommand}{{Interface:drawEdge}(\meta{edge})}
Passes an edge back to the \TeX\ layer.  Edges with a direction of |Edge.NONE| are skipped and not passed back to \TeX. 

Parameters:
\begin{parameterdescription}
	\item[\meta{edge}] The edge to pass back to the \TeX\ layer. 
\end{parameterdescription}



\end{luacommand}\begin{luacommand}{{Interface:drawGraph}()}
Arranges the current graph using the specified algorithm.  The algorithm is derived from the graph options and is loaded on demand from the corresponding algorithm file. For a fictitious algorithm |simple| this file is per convention called |pgflibrarygraphdrawing-algorithms-simple.lua|. It is required to define at least one function as an entry point to the algorithm. The name of the function is again predetermined as |drawGraphAlgorithm_simple|. When a graph is to be layed out, this function is called with the graph as its only parameter. 



\end{luacommand}\begin{luacommand}{{Interface:drawNode}(\meta{node})}
Passes a node back to the \TeX\ layer. 

Parameters:
\begin{parameterdescription}
	\item[\meta{node}] The node to pass back to the \TeX\ layer. 
\end{parameterdescription}



\end{luacommand}\begin{luacommand}{{Interface:finishGraph}()}
Passes the current graph back to the \TeX\ layer and removes it from the stack. 



\end{luacommand}\begin{luacommand}{{Interface:getOption}(\meta{name})}
Returns the value of the graph option \meta{name}. 

Parameters:
\begin{parameterdescription}
	\item[\meta{name}] Name of the option. 
\end{parameterdescription}


Return value:
\begin{itemize} \item[] The value of the \meta{name} option or |nil|.  \end{itemize}


\end{luacommand}\begin{luacommand}{{Interface:loadAlgorithm}(\meta{name})}
Attempts to load the algorithm with the given \meta{name}.  This function tries to look up the corresponding algorithm file |pgflibrarygraphdrawing-algorithms-name.lua| and attempts to look up the main function for calling the algorithm. 

Parameters:
\begin{parameterdescription}
	\item[\meta{name}] Name of the algorithm. 
\end{parameterdescription}


Return value:
\begin{itemize} \item[] The algorithm function or nil.  \end{itemize}


\end{luacommand}\begin{luacommand}{{Interface:newGraph}(\meta{options})}
Creates a new graph and adds it to the graph stack.  The options string consisting of |{key}{value}| pairs is parsed and assigned to the graph. These options are used to configure the different graph drawing algorithms shipped with \tikzname. 

Parameters:
\begin{parameterdescription}
	\item[\meta{options}] A string containing |{key}{value}| pairs of \tikzname\ options. 
\end{parameterdescription}



See also:
\begin{itemize}
	\item[] |finishGraph |
\end{itemize}

\end{luacommand}\begin{luacommand}{{Interface:setOption}(\meta{name},\meta{value})}
Sets the graph option \meta{name} to \meta{value}. Only affects the current graph. 

Parameters:
\begin{parameterdescription}
	\item[\meta{name}] The name of the option to set.\item[\meta{value}] New value for the option. 
\end{parameterdescription}



\end{luacommand}
\end{filedescription}

Communication with \TeX\ on a basic layer is done in the |Sys|
class. The |beginShipout()| function opens a new scope in \pgfname\
to put all graph drawing nodes into. This prevents other graph objects
outside the graph drawing scope from referencing these nodes. The
|endShipout()| method closes the scope. Nodes and edges are put in the
output stream by the methods |putTeXBox()| and |putEdge()|, which
invoke callbacks to \TeX. 


\subsubsection{Support Classes and Functions}

Most classes in the framework (including the module objects) implement
the |__tostring| method, meaning that you can get a somewhat useful
string representation of the object via the standard |tostring|
function.

%% This file has been generated from the lua sources using LuaDoc.
% To regenerate it call "make genluadoc" in
% doc/generic/pgf/version-for-luatex/en.

\begin{filedescription}{pgflibrarygraphdrawing-helper.lua}


\begin{luacommand}{{parseBraces}(\meta{str},\meta{default})}
Parses a braced list of {key}{value} pairs and returns a table mapping keys to values.



\end{luacommand}

\end{filedescription}
%\input{generated/pgflibrarygraphdrawing-table-helpers}
%\input{generated/pgflibrarygraphdrawing-iter-helpers}



% This file has been generated from the lua sources using LuaDoc.
% To regenerate it call "make genluadoc" in
% doc/generic/pgf/version-for-luatex/en.

\begin{filedescription}{pgflibrarygraphdrawing-vector.lua}


\begin{luacommand}{{Vector:copy}()}
Creates a copy of the vector that holds the same elements as the original. 


Return value:
\begin{parameterdescription} 
  \item[] A newly-allocated copy of the vector holding exactly the same elements. 
\end{parameterdescription}


\end{luacommand}
\begin{luacommand}{{Vector:dividedBy}(\meta{other})}
Performs a vector division and returns the result in a new vector.  The possible origins of the vector operands are resolved and are dropped in the result vector. 

Parameters:
\begin{parameterdescription}
	\item[\meta{other}] Vector to divide by. 
\end{parameterdescription}


Return value:
\begin{parameterdescription} 
  \item[] A new vector with the result of the division. 
\end{parameterdescription}


\end{luacommand}
\begin{luacommand}{{Vector:dividedByScalar}(\meta{scalar})}
Divides a vector by a scalar value and returns the result in a new vector.  The possible origin of the vector is resolved and is dropped in the result vector. 

Parameters:
\begin{parameterdescription}
	\item[\meta{scalar}] Scalar value to divide the vector by. 
\end{parameterdescription}


Return value:
\begin{parameterdescription} 
  \item[] A new vector with the result of the division. 
\end{parameterdescription}


\end{luacommand}
\begin{luacommand}{{Vector:dotProduct}(\meta{other})}
Performs the dot product of two vectors and returns the result in a new vector.  The possible origins of the vector operands are resolved during the compuation. 

Parameters:
\begin{parameterdescription}
	\item[\meta{other}] Vector to perform the dot product with. 
\end{parameterdescription}


Return value:
\begin{parameterdescription} 
  \item[] A new vector with the result of the dot product. 
\end{parameterdescription}


\end{luacommand}
\begin{luacommand}{{Vector:get}(\meta{index})}
Returns the element at the given \meta{index}. 


Return value:
\begin{parameterdescription} 
  \item[] The element at the given \meta{index}. 
\end{parameterdescription}


\end{luacommand}
\begin{luacommand}{{Vector:getOrigin}()}
Gets the origin of the vector. 


Return value:
\begin{parameterdescription} 
  \item[] Origin of the vector or |nil| if none is set. 
\end{parameterdescription}


\end{luacommand}
\begin{luacommand}{{Vector:limit}(\meta{limit\_function})}
Limits all elements of the vector in-place. 

Parameters:
\begin{parameterdescription}
	\item[\meta{limit\_function}] A function that is called for each index/element pair. It is supposed to return minimum and maximum values for the element. The element is then clamped to these values. 
\end{parameterdescription}



\end{luacommand}
\begin{luacommand}{{Vector:minus}(\meta{other})}
Subtracts two vectors and returns the result in a new vector. 

Parameters:
\begin{parameterdescription}
	\item[\meta{other}] Vector to subtract. If this vector is defined relative to an origin, then that origin is resolved when computing the subtraction of the two vectors. The result becomes |self + other.origin + other|. The origin of |self| is preserved. 
\end{parameterdescription}


Return value:
\begin{parameterdescription} 
  \item[] A new vector with the result of the subtraction. 
\end{parameterdescription}


\end{luacommand}
\begin{luacommand}{{Vector:minusScalar}(\meta{scalar})}
Subtracts a scalar value from a vector and returns the result in a new vector. 

Parameters:
\begin{parameterdescription}
	\item[\meta{scalar}] Scalar value to subtract from all elements. 
\end{parameterdescription}


Return value:
\begin{parameterdescription} 
  \item[] A new vector with the result of the subtraction. 
\end{parameterdescription}


\end{luacommand}
\begin{luacommand}{{Vector:new}(\meta{n},\meta{fill\_function},\meta{origin})}
Creates a new vector with \meta{n} values using an optional \meta{fill\_function}. 

Parameters:
\begin{parameterdescription}
	\item[\meta{n}] The number of elements of the vector.\item[\meta{fill\_function}] Optional function that takes a number between 1 and \meta{n} and is expected to return a value for the corresponding element of the vector. If omitted, all elements of the vector will be initialized with 0.\item[\meta{origin}] Optional origin vector. 
\end{parameterdescription}


Return value:
\begin{parameterdescription} 
  \item[] A newly-allocated vector with \meta{n} elements. 
\end{parameterdescription}


\end{luacommand}
\begin{luacommand}{{Vector:norm}()}
Computes the Euclidean norm of the vector. 


Return value:
\begin{parameterdescription} 
  \item[] The Euclidean norm of the vector. 
\end{parameterdescription}


\end{luacommand}
\begin{luacommand}{{Vector:normalized}()}
Normalizes the vector and returns the result in a new vector.  The possible origin of the vector is resolved during the computation and is dropped in the result vector. 


Return value:
\begin{parameterdescription} 
  \item[] Normalized version of the original vector. 
\end{parameterdescription}


\end{luacommand}
\begin{luacommand}{{Vector:plus}(\meta{other})}
Performs a vector addition and returns the result in a new vector. 

Parameters:
\begin{parameterdescription}
	\item[\meta{other}] The vector to add. If this vector is defined relative to an origin, then that origin is resolved when computing the sum of the two vectors. The sum becomes |self + other.origin + other|. The origin of |self| is preserved. 
\end{parameterdescription}


Return value:
\begin{parameterdescription} 
  \item[] A new vector with the result of the addition. 
\end{parameterdescription}


\end{luacommand}
\begin{luacommand}{{Vector:plusScalar}(\meta{scalar})}
Performs an addition with a scalar value and returns the result in a new vector.  The scalar value is added to all elements of the vector. 

Parameters:
\begin{parameterdescription}
	\item[\meta{scalar}] Scalar value to add to all elements. 
\end{parameterdescription}


Return value:
\begin{parameterdescription} 
  \item[] A new vector with the result of the addition. 
\end{parameterdescription}


\end{luacommand}
\begin{luacommand}{{Vector:reset}()}
Resets all vector elements to 0 in-place.  This does not reset the origin vector. 



\end{luacommand}
\begin{luacommand}{{Vector:set}(\meta{index},\meta{value})}
Changes the element at the given \meta{index}. 

Parameters:
\begin{parameterdescription}
	\item[\meta{index}] The index of the element to change.\item[\meta{value}] New value of the element. 
\end{parameterdescription}



\end{luacommand}
\begin{luacommand}{{Vector:setOrigin}(\meta{origin},\meta{preserve\_values})}
Sets the origin of the vector. 

Parameters:
\begin{parameterdescription}
	\item[\meta{origin}] Vector to use as the origin.\item[\meta{preserve\_values}] Optional flag. If set to |true|, the origin will be set and the current elements of the vector will be changed so that the sum of the origin and the new element values is equal to the old values. 
\end{parameterdescription}



\end{luacommand}
\begin{luacommand}{{Vector:timesScalar}(\meta{scalar})}
Multiplies a vector by a scalar value and returns the result in a new vector.  The possible origin of the vector is resolved and is dropped in the result vector. 

Parameters:
\begin{parameterdescription}
	\item[\meta{scalar}] Scalar value to multiply the vector with. 
\end{parameterdescription}


Return value:
\begin{parameterdescription} 
  \item[] A new vector with the result of the multiplication. 
\end{parameterdescription}


\end{luacommand}
\begin{luacommand}{{Vector:update}(\meta{update\_function})}
Updates the values of the vector in-place. 

Parameters:
\begin{parameterdescription}
	\item[\meta{update\_function}] A function that is called for each element of the vector. The elements are replaced by the values returned from this function. 
\end{parameterdescription}



\end{luacommand}
\begin{luacommand}{{Vector:x}()}
Convenience method that returns the first element of the vector.  The origin vector is not resolved in this function call. 


Return value:
\begin{parameterdescription} 
  \item[] The first element of the vector. 
\end{parameterdescription}


\end{luacommand}
\begin{luacommand}{{Vector:y}()}
Convenience method that returns the second element of the vector.  The origin vector is not resolved in this function call. 


Return value:
\begin{parameterdescription} 
  \item[] The second element of the vector. 
\end{parameterdescription}


\end{luacommand}

\end{filedescription}
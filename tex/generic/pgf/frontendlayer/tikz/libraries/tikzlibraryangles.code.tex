% Copyright 2019 by Till Tantau
% Copyright 2019 by Kroum Tzanev
%
% This file may be distributed and/or modified
%
% 1. under the LaTeX Project Public License and/or
% 2. under the GNU Public License.
%
% See the file doc/generic/pgf/licenses/LICENSE for more details.

\ProvidesFileRCS{tikzlibraryangles.code.tex}


\tikzset{
  pics/angle/.style = {
    setup code  = \tikz@lib@angle@parse#1\pgf@stop,
    code = {},
    background code = \tikz@lib@angle@background#1\pgf@stop,
    foreground code = \tikz@lib@angle@foreground#1\pgf@stop,
  },
  pics/right angle/.style = {
    setup code  = \tikz@lib@angle@parse#1\pgf@stop,
    code = {},
    background code = \tikz@lib@rightangle@background#1\pgf@stop,
    foreground code = \tikz@lib@rightangle@foreground#1\pgf@stop,
  },
  pics/angle/.default=A--B--C,
  angle eccentricity/.initial=.6,
  angle radius/.initial=5mm,
}%

\def\tikz@lib@angle@background#1--#2--#3\pgf@stop{%
  \path [name prefix ..] [pic actions, draw=none] (#2)
    -- ++(\tikz@start@angle@temp:\tikz@lib@angle@rad pt)
    arc [start angle=\tikz@start@angle@temp, end
    angle=\tikz@end@angle@temp, radius=\tikz@lib@angle@rad pt] -- cycle;
}%

\def\tikz@lib@angle@foreground#1--#2--#3\pgf@stop{%
  \path [name prefix ..] [pic actions, fill=none, shade=none]
  ([shift={(\tikz@start@angle@temp:\tikz@lib@angle@rad pt)}]#2)
    arc [start angle=\tikz@start@angle@temp, end
    angle=\tikz@end@angle@temp, radius=\tikz@lib@angle@rad pt];
  \ifx\tikzpictext\relax\else%
    \def\pgf@temp{\node()[name prefix
      ..,at={([shift={({.5*\tikz@start@angle@temp+.5*\tikz@end@angle@temp}:\pgfkeysvalueof{/tikz/angle
            eccentricity}*\tikz@lib@angle@rad pt)}]#2)}]}
    \expandafter\pgf@temp\expandafter[\tikzpictextoptions]{\tikzpictext};%
  \fi
}%

\def\tikz@lib@rightangle@background#1--#2--#3\pgf@stop{%
  \path [name prefix ..] [pic actions, draw=none] (#2)
    -- ++(\tikz@start@angle@temp:\tikz@lib@angle@rad pt)
    -- ++(\tikz@end@angle@temp:\tikz@lib@angle@rad pt)
    -- ++(\tikz@start@angle@temp:-\tikz@lib@angle@rad pt)
    -- cycle;
}%

\def\tikz@lib@rightangle@foreground#1--#2--#3\pgf@stop{%
  \path [name prefix ..] [pic actions, fill=none, shade=none]
  ([shift={(\tikz@start@angle@temp:\tikz@lib@angle@rad pt)}]#2)
  -- ++(\tikz@end@angle@temp:\tikz@lib@angle@rad pt)
  -- ++(\tikz@start@angle@temp:-\tikz@lib@angle@rad pt);
  \ifx\tikzpictext\relax\else%
    \def\pgf@temp{\node()[name prefix
      ..,at={([shift={({.5*\tikz@start@angle@temp+.5*\tikz@end@angle@temp}:\pgfkeysvalueof{/tikz/angle
            eccentricity}*1.4142136*\tikz@lib@angle@rad pt)}]#2)}]}
    \expandafter\pgf@temp\expandafter[\tikzpictextoptions]{\tikzpictext};%
  \fi
}%

\def\tikz@lib@angle@parse#1--#2--#3\pgf@stop{%
  % Compute radius:
  \pgfmathsetmacro\tikz@lib@angle@rad{\pgfkeysvalueof{/tikz/angle radius}}%
  \ifdim\tikz@lib@angle@rad pt>0pt\else\def\tikz@lib@angle@rad{12}\fi
  % Compute first coordinate:
  \tikz@scan@@absolute\pgf@process(#2)%
  \pgf@xa=\pgf@x
  \pgf@ya=\pgf@y
  \tikz@scan@@absolute\pgf@process(#1)%
  \pgf@xb=\pgf@x
  \pgf@yb=\pgf@y
  \tikz@scan@@absolute\pgf@process(#3)%
  \pgf@xc=\pgf@x
  \pgf@yc=\pgf@y
  \advance\pgf@xb by-\pgf@xa
  \advance\pgf@yb by-\pgf@ya
  \advance\pgf@xc by-\pgf@xa
  \advance\pgf@yc by-\pgf@ya
  \pgfmathatantwo@{\pgf@sys@tonumber\pgf@yb}{\pgf@sys@tonumber\pgf@xb}%
  \let\tikz@start@angle@temp\pgfmathresult
  \pgfmathatantwo@{\pgf@sys@tonumber\pgf@yc}{\pgf@sys@tonumber\pgf@xc}%
  \let\tikz@end@angle@temp\pgfmathresult
  \ifdim\tikz@end@angle@temp pt<\tikz@start@angle@temp pt
    \pgfmathsubtract@{\tikz@start@angle@temp}{360}%
    \let\tikz@start@angle@temp\pgfmathresult
  \fi
}



\endinput

\documentclass[a4paper]{ltxdoc}
%% Copyright 2006 by Till Tantau
%
% This file may be distributed and/or modified
%
% 1. under the LaTeX Project Public License and/or
% 2. under the GNU Free Documentation License.
%
% See the file doc/generic/pgf/licenses/LICENSE for more details.


\usepackage[dvipdfm]{graphics}
% This will tell everyone which driver we are using.

\usepackage[dvipdfm]{hyperref}

% Make sure to compile this using latex + dvipdfm.

%% Copyright 2006 by Till Tantau
%
% This file may be distributed and/or modified
%
% 1. under the LaTeX Project Public License and/or
% 2. under the GNU Free Documentation License.
%
% See the file doc/generic/pgf/licenses/LICENSE for more details.


% pgf version is defined in \pgfversion in file
% generic/pgf/utilities/pgfrcs.code.tex

\def\xcolorversion{2.00}

\usepackage[version=latest]{pgf}

\usepackage{xkeyval,calc,listings,tikz,fp}

\usepackage{hyperref}
\hypersetup{%
	colorlinks=false, % use true to enable colors below:
	linkcolor=blue,%red,
	filecolor=blue,%magenta,
	urlcolor=blue,%cyan,
	citecolor=blue,
	pdfborder=0 0 0,
}

% We need lots of libraries...
\usetikzlibrary{
  arrows,
  arrows.spaced,
  arrows.meta,
  bending,
  babel,
  calc,
  fit,
  patterns,
  plotmarks,
  shapes.geometric,
  shapes.misc,
  shapes.symbols,
  shapes.arrows,
  shapes.callouts,
  shapes.multipart,
  shapes.gates.logic.US,
  shapes.gates.logic.IEC,
  circuits.logic.US,
  circuits.logic.IEC,
  circuits.logic.CDH,
  circuits.ee.IEC,
  datavisualization,
  datavisualization.polar,
  datavisualization.formats.functions,
  er,
  automata,
  backgrounds,
  chains,
  topaths,
  trees,
  petri,
  mindmap,
  matrix,
  calendar,
  folding,
  fadings,
  shadings,
  spy,
  through,
  turtle,
  positioning,
  scopes,
  decorations.fractals,
  decorations.shapes,
  decorations.text,
  decorations.pathmorphing,
  decorations.pathreplacing,
  decorations.footprints,
  decorations.markings,
  shadows,
  lindenmayersystems,
  intersections,
  fixedpointarithmetic,
  fpu,
  svg.path,
  external,
  graphs,
  graphs.standard,
  quotes,
  math,
  angles
}

\usepackage{ifluatex}
\newif\ifgdccodebasic
\newif\ifgdccodeogdf

\ifluatex

  \usetikzlibrary{graphdrawing}
  \usegdlibrary{trees,circular,layered,examples,force,phylogenetics,routing}

  % Test whether C code is available:
  \directlua{
    if pcall(require,'pgf_gd_examples_c_SimpleDemoC') then
      tex.print('\string\\gdccodebasictrue')
    end
  }

  \ifgdccodebasic
    \usegdlibrary{pgf_gd_examples_c_SimpleDemoC,pgf_gd_examples_c_SimpleDemoCPlusPlus}

    % Test whether OGDF code is available:
    \directlua{
      if pcall(require,'pgf_gd_ogdf_c_SimpleDemoOGDF') then
        tex.print('\string\\gdccodeogdftrue')
      end
    }

    \ifgdccodeogdf
      \usegdlibrary{pgf_gd_ogdf_c_SimpleDemoOGDF,ogdf}
    \fi
    
  \fi
  
\fi

\def\LuaTeX{Lua\TeX}%



\iffalse
%\iftrue
  \tikzexternalize[
    mode=list only,export=true,% simply skips EVERY picture -> good for debugging the text.
  ]{pgfmanual}

  \tikzifexternalizing{%
    \pgfkeys{/pdflinks/codeexample links=false}%
  }{}%
\fi


\usepackage[a4paper,left=2.25cm,right=2.25cm,top=2.5cm,bottom=2.5cm,nohead]{geometry}
\usepackage{amsmath,amssymb}
\usepackage{xxcolor}
\usepackage{pifont}
\usepackage{makeidx}

% Fontenc (new beginning 2014, let's see, what it breaks...):
\usepackage[T1]{fontenc}

\ifluatex
%  \usepackage[no-math]{fontspec}
%  \usepackage{luatextra}

  % TT: I have commented luatextra since it loads fontspec, which
  % currently breaks "$\mathrm{\Omega}$" (nothing is
  % shown). Also, fontspec loads so much stuff, I'm not really sure
  % any of it is needed...

  % \filedescription is defined in expl3, required by fontspec,
  % required by luatextra. Needs to be \relaxed since
  % pgfmanual-en-macros.tex defines an environment named filedescription
  \let\filedescription\relax
  \usepackage[utf8]{luainputenc}
\else 
  \usepackage[utf8]{inputenc}
\fi
\usepackage{amsmath}

\graphicspath{{../../images/}}              % TODOsp: under windows this would go up 2 directories, but the file is only one directory up
% Copyright 2006 by Till Tantau
%
% This file may be distributed and/or modified
%
% 1. under the LaTeX Project Public License and/or
% 2. under the GNU Free Documentation License.
%
% See the file doc/generic/pgf/licenses/LICENSE for more details.

% $Header$


\providecommand\href[2]{\texttt{#1}}


\colorlet{examplefill}{yellow!80!black}
\definecolor{graphicbackground}{rgb}{0.96,0.96,0.8}
\definecolor{codebackground}{rgb}{0.8,0.8,1}

\newenvironment{pgfmanualentry}{\list{}{\leftmargin=2em\itemindent-\leftmargin\def\makelabel##1{\hss##1}}}{\endlist}
\newcommand\pgfmanualentryheadline[1]{\itemsep=0pt\parskip=0pt\item\strut{#1}\par\topsep=0pt}
\newcommand\pgfmanualbody{\parskip3pt}



\newenvironment{pgflayout}[1]{
  \begin{pgfmanualentry}
    \pgfmanualentryheadline{\texttt{\string\pgfpagesuselayout\char`\{\declare{#1}\char`\}}\oarg{options}}
    \index{#1@\protect\texttt{#1} layout}%
    \index{Page layouts!#1@\protect\texttt{#1}}%
    \pgfmanualbody
}
{
  \end{pgfmanualentry}
}


\newenvironment{command}[1]{
  \begin{pgfmanualentry}
    \extractcommand#1\@@
    \pgfmanualbody
}
{
  \end{pgfmanualentry}
}

%% MW: START MATH MACROS
\def\mvar#1{{\rmfamily\textit{#1}}}

\makeatletter

\def\extractmathfunctionname#1{\extractmathfunctionname@#1(,)\tmpa\tmpb}
\def\extractmathfunctionname@#1(#2)#3\tmpb{\def\mathname{#1}}

\def\extractmathoperatorname{\begingroup\def\mvar##1{}\def\ {}\extractmathoperatorname@}
\def\extractmathoperatorname@#1{\xdef\mathname{#1}\endgroup}

\makeatother
	
\newenvironment{math-function}[1]{
	\begin{pgfmanualentry}
		\extractmathfunctionname{#1}
		\pgfmanualentryheadline{\texttt{#1}}%
		\index{\mathname @\protect\texttt{\mathname} math function}%
		\index{Math functions!\mathname @\protect\texttt{\mathname}}
		\pgfmanualbody
}
{
	\end{pgfmanualentry}
}

\newenvironment{math-operator}[1]{	
	\begin{pgfmanualentry}
		\extractmathoperatorname{#1}
		\pgfmanualentryheadline{\texttt{#1}}%
		\index{\mathname @\protect\texttt{\mathname} math operator}%
		\index{Math operators!\mathname @\protect\texttt{\mathname}}
    	\pgfmanualbody
}
{%
	\end{pgfmanualentry}
}

\newenvironment{math-constant}[1]{
	\begin{pgfmanualentry}
		\pgfmanualentryheadline{\texttt{#1}}%
		\index{#1@\protect\texttt{#1} math constant}%
		\index{Math constants!#1@\protect\texttt{#1}}
		\pgfmanualbody
}
{
	\end{pgfmanualentry}
}
\def\calcname{\textsc{calc}}
%% MW: END MATH MACROS


\def\extractcommand#1#2\@@{%
  \pgfmanualentryheadline{\declare{\texttt{\string#1}}#2}%
  \removeats{#1}%
  \index{\strippedat @\protect\myprintocmmand{\strippedat}}}


% \begin{environment}{{name}\marg{arguments}}
\renewenvironment{environment}[1]{
  \begin{pgfmanualentry}
    \extractenvironement#1\@@
    \pgfmanualbody
}
{
  \end{pgfmanualentry}
}

\def\extractenvironement#1#2\@@{%
  \pgfmanualentryheadline{{\ttfamily\char`\\begin\char`\{\declare{#1}\char`\}}#2}%
  \pgfmanualentryheadline{{\ttfamily\ \ }\meta{environment contents}}%
  \pgfmanualentryheadline{{\ttfamily\char`\\end\char`\{\declare{#1}\char`\}}}%
  \index{#1@\protect\texttt{#1} environment}%
  \index{Environments!#1@\protect\texttt{#1}}}


\newenvironment{plainenvironment}[1]{
  \begin{pgfmanualentry}
    \extractplainenvironement#1\@@
    \pgfmanualbody
}
{
  \end{pgfmanualentry}
}

\def\extractplainenvironement#1#2\@@{%
  \pgfmanualentryheadline{{\ttfamily\declare{\char`\\#1}}#2}%
  \pgfmanualentryheadline{{\ttfamily\ \ }\meta{environment contents}}%
  \pgfmanualentryheadline{{\ttfamily\declare{\char`\\end#1}}}%
  \index{#1@\protect\texttt{#1} environment}%
  \index{Environments!#1@\protect\texttt{#1}}}


\newenvironment{contextenvironment}[1]{
  \begin{pgfmanualentry}
    \extractcontextenvironement#1\@@
    \pgfmanualbody
}
{
  \end{pgfmanualentry}
}

\def\extractcontextenvironement#1#2\@@{%
  \pgfmanualentryheadline{{\ttfamily\declare{\char`\\start#1}}#2}%
  \pgfmanualentryheadline{{\ttfamily\ \ }\meta{environment contents}}%
  \pgfmanualentryheadline{{\ttfamily\declare{\char`\\stop#1}}}%
  \index{#1@\protect\texttt{#1} environment}%
  \index{Environments!#1@\protect\texttt{#1}}}


\newenvironment{shape}[1]{
  \begin{pgfmanualentry}
  	\pgfmanualentryheadline{Shape {\ttfamily\declare{#1}}}%
    \index{#1@\protect\texttt{#1} shape}%
    \index{Shapes!#1@\protect\texttt{#1}}
    \pgfmanualbody
}
{
  \end{pgfmanualentry}
}


\newenvironment{handler}[1]{
  \begin{pgfmanualentry}
    \extracthandler#1\@nil%
    \pgfmanualbody
}
{
  \end{pgfmanualentry}
}

\def\gobble#1{}
\def\extracthandler#1#2\@nil{%
  \pgfmanualentryheadline{Key handler \meta{key}{\ttfamily/\declare{#1}}#2}%
  \index{\gobble#1@\protect\texttt{#1} handler}%
  \index{Key handlers!#1@\protect\texttt{#1}}
}


\makeatletter


\newenvironment{stylekey}[1]{
  \begin{pgfmanualentry}
    \def\extrakeytext{style, }
    \extractkey#1\@nil%
    \pgfmanualbody
}
{
  \end{pgfmanualentry}
}

\def\choicesep{$\vert$}%
\def\choicearg#1{\texttt{#1}}

\newif\iffirstchoice

% \mchoice{choice1,choice2,choice3}
\newcommand\mchoice[1]{%
	\begingroup
	\firstchoicetrue
	\foreach \mchoice@ in {#1} {%
		\iffirstchoice
			\global\firstchoicefalse
		\else
			\choicesep
		\fi
		\choicearg{\mchoice@}%
	}%
	\endgroup
}%

% \begin{key}{/path/x=value}
% \begin{key}{/path/x=value (initially XXX)}
% \begin{key}{/path/x=value (default XXX)}
\newenvironment{key}[1]{
  \begin{pgfmanualentry}
    \def\extrakeytext{}
    %\def\altpath{\emph{\color{gray}or}}%
    \extractkey#1\@nil%
    \pgfmanualbody
}
{
  \end{pgfmanualentry}
}

\def\extractkey#1\@nil{%
  \pgfutil@in@={#1}%
  \ifpgfutil@in@%
    \extractkeyequal#1\@nil
  \else%
    \pgfutil@in@{(initial}{#1}%
    \ifpgfutil@in@%
      \extractequalinitial#1\@nil%
    \else
      \pgfmanualentryheadline{{\ttfamily\declare{#1}}\hfill(\extrakeytext no value)}%
      \def\mykey{#1}%
      \def\mypath{}%
      \def\myname{}%
      \firsttimetrue%
      \decompose#1/\nil%
    \fi
  \fi%
}

\def\extractkeyequal#1=#2\@nil{%
  \pgfutil@in@{(default}{#2}%
  \ifpgfutil@in@%
    \extractdefault{#1}#2\@nil%
  \else%
    \pgfutil@in@{(initial}{#2}%
    \ifpgfutil@in@%
      \extractinitial{#1}#2\@nil%
    \else
      \pgfmanualentryheadline{{\ttfamily\declare{#1}=}#2\hfill(\extrakeytext no default)}%
    \fi%
  \fi%
  \def\mykey{#1}%
  \def\mypath{}%
  \def\myname{}%
  \firsttimetrue%
  \decompose#1/\nil%
}

\def\extractdefault#1#2(default #3)\@nil{%
  \pgfmanualentryheadline{{\ttfamily\declare{#1}\opt{=}}\opt{#2}\hfill (\extrakeytext default {\ttfamily#3})}%
}

\def\extractinitial#1#2(initially #3)\@nil{%
  \pgfmanualentryheadline{{\ttfamily\declare{#1}=}#2\hfill (\extrakeytext no default, initially {\ttfamily#3})}%
}

\def\extractequalinitial#1 (initially #2)\@nil{%
  \pgfmanualentryheadline{{\ttfamily\declare{#1}}\hfill (\extrakeytext initially {\ttfamily#2})}%
  \def\mykey{#1}%
  \def\mypath{}%
  \def\myname{}%
  \firsttimetrue%
  \decompose#1/\nil%
}

% Introduces a key alias '/#1/<name of current key>'
% to be used inside of \begin{key} ... \end{key}
\def\keyalias#1{\vspace{-3pt}\item{\small alias {\ttfamily/#1/\myname}}\vspace{-2pt}\par}

\newif\iffirsttime

\makeatother

\def\decompose/#1/#2\nil{%
  \def\test{#2}%
  \ifx\test\empty%
    % aha.
    \index{#1@\protect\texttt{#1} key}%
    \index{\mypath#1@\protect\texttt{#1}}%
    \def\myname{#1}%
  \else%
    \iffirsttime
      \def\mypath{#1@\protect\texttt{/#1/}!}%
      \firsttimefalse
    \else
      \expandafter\def\expandafter\mypath\expandafter{\mypath#1@\protect\texttt{#1/}!}%
    \fi
    \def\firsttime{}
    \decompose/#2\nil%
  \fi%
}

\def\indexkey#1{%
  \def\mypath{}%
  \decompose#1/\nil%
}

\newenvironment{predefinedmethod}[1]{
  \begin{pgfmanualentry}
    \extractpredefinedmethod#1\@nil
    \pgfmanualbody
}
{
  \end{pgfmanualentry}
}
\def\extractpredefinedmethod#1(#2)\@nil{%
  \pgfmanualentryheadline{Method \declare{\ttfamily #1}\texttt(#2\texttt) \hfill(predefined for all classes)}
  \index{#1@\protect\texttt{#1} method}%
  \index{Methods!#1@\protect\texttt{#1}}
}


\newenvironment{ooclass}[1]{
  \begin{pgfmanualentry}
    \def\currentclass{#1}
    \pgfmanualentryheadline{Class \declare{\texttt{#1}}}
    \index{#1@\protect\texttt{#1} class}%
    \index{Class #1@Class \protect\texttt{#1}}%
    \index{Classes!#1@\protect\texttt{#1}}
    \pgfmanualbody
}
{
  \end{pgfmanualentry}
}

\newenvironment{method}[1]{
  \begin{pgfmanualentry}
    \extractmethod#1\@nil
    \pgfmanualbody
}
{
  \end{pgfmanualentry}
}
\def\extractmethod#1(#2)\@nil{%
  \def\test{#1}
  \ifx\test\currentclass
    \pgfmanualentryheadline{Constructor \declare{\ttfamily #1}\texttt(#2\texttt)}
  \else
    \pgfmanualentryheadline{Method \declare{\ttfamily #1}\texttt(#2\texttt)}
  \fi
  \index{#1@\protect\texttt{#1} method}%
  \index{Methods!#1@\protect\texttt{#1}}
  \index{Class \currentclass!#1@\protect\texttt{#1}}%
}

\newenvironment{attribute}[1]{
  \begin{pgfmanualentry}
    \extractattribute#1\@nil
    \pgfmanualbody
}
{
  \end{pgfmanualentry}
}
\def\extractattribute#1=#2;\@nil{%
  \def\test{#2}%
  \ifx\test\@empty
    \pgfmanualentryheadline{Private attribute \declare{\ttfamily #1} \hfill (initially empty)}
  \else
    \pgfmanualentryheadline{Private attribute \declare{\ttfamily #1} \hfill (initially {\ttfamily #2})}
  \fi
  \index{#1@\protect\texttt{#1} attribute}%
  \index{Attributes!#1@\protect\texttt{#1}}
  \index{Class \currentclass!#1@\protect\texttt{#1}}%
}



\newenvironment{predefinednode}[1]{
  \begin{pgfmanualentry}
    \pgfmanualentryheadline{Predefined node {\ttfamily\declare{#1}}}%
    \index{#1@\protect\texttt{#1} node}%
    \index{Predefined node!#1@\protect\texttt{#1}}
    \pgfmanualbody
}
{
  \end{pgfmanualentry}
}

\newenvironment{coordinatesystem}[1]{
  \begin{pgfmanualentry}
    \pgfmanualentryheadline{Coordinate system {\ttfamily\declare{#1}}}%
    \index{#1@\protect\texttt{#1} coordinate system}%
    \index{Coordinate systems!#1@\protect\texttt{#1}}
    \pgfmanualbody
}
{
  \end{pgfmanualentry}
}

\newenvironment{snake}[1]{
  \begin{pgfmanualentry}
    \pgfmanualentryheadline{Snake {\ttfamily\declare{#1}}}%
    \index{#1@\protect\texttt{#1} snake}%
    \index{Snakes!#1@\protect\texttt{#1}}
    \pgfmanualbody
}
{
  \end{pgfmanualentry}
}

\newenvironment{decoration}[1]{
  \begin{pgfmanualentry}
    \pgfmanualentryheadline{Decoration {\ttfamily\declare{#1}}}%
    \index{#1@\protect\texttt{#1} decoration}%
    \index{Decorations!#1@\protect\texttt{#1}}
    \pgfmanualbody
}
{
  \end{pgfmanualentry}
}


\def\pgfmanualbar{\char`\|}
\makeatletter
\newenvironment{pathoperation}[3][]{
  \begin{pgfmanualentry}
    \pgfmanualentryheadline{\textcolor{gray}{{\ttfamily\char`\\path}\
        \ \dots}
      \declare{\texttt{#2}}#3\ \textcolor{gray}{\dots\texttt{;}}}%
    \def\pgfmanualtest{#1}%
    \ifx\pgfmanualtest\@empty%
      \index{#2@\protect\texttt{#2} path operation}%
      \index{Path operations!#2@\protect\texttt{#2}}%
    \fi%
    \pgfmanualbody
}
{
  \end{pgfmanualentry}
}
\makeatother

\def\extractcommand#1#2\@@{%
  \pgfmanualentryheadline{\declare{\texttt{\string#1}}#2}%
  \removeats{#1}%
  \index{\strippedat @\protect\myprintocmmand{\strippedat}}}

\def\doublebs{\texttt{\char`\\\char`\\}}


\newenvironment{package}[1]{
  \begin{pgfmanualentry}
    \pgfmanualentryheadline{{\ttfamily\char`\\usepackage\char`\{\declare{#1}\char`\}\space\space \char`\%\space\space  \LaTeX}}
    \index{#1@\protect\texttt{#1} package}%
    \index{Packages and files!#1@\protect\texttt{#1}}%
    \pgfmanualentryheadline{{\ttfamily\char`\\input \declare{#1}.tex\space\space\space \char`\%\space\space  plain \TeX}}
    \pgfmanualentryheadline{{\ttfamily\char`\\usemodule[\declare{#1}]\space\space \char`\%\space\space  Con\TeX t}}
    \pgfmanualbody
}
{
  \end{pgfmanualentry}
}


\newenvironment{pgfmodule}[1]{
  \begin{pgfmanualentry}
    \pgfmanualentryheadline{{\ttfamily\char`\\usepgfmodule\char`\{\declare{#1}\char`\}\space\space\space
        \char`\%\space\space  \LaTeX\space and plain \TeX\space and pure pgf}}
    \index{#1@\protect\texttt{#1} module}%
    \index{Modules!#1@\protect\texttt{#1}}%
    \pgfmanualentryheadline{{\ttfamily\char`\\usepgfmodule[\declare{#1}]\space\space \char`\%\space\space  Con\TeX t\space and pure pgf}}
    \pgfmanualbody
}
{
  \end{pgfmanualentry}
}

\newenvironment{pgflibrary}[1]{
  \begin{pgfmanualentry}
    \pgfmanualentryheadline{{\ttfamily\char`\\usepgflibrary\char`\{\declare{#1}\char`\}\space\space\space
        \char`\%\space\space  \LaTeX\space and plain \TeX\space and pure pgf}}
    \index{#1@\protect\texttt{#1} library}%
    \index{Libraries!#1@\protect\texttt{#1}}%
    \pgfmanualentryheadline{{\ttfamily\char`\\usepgflibrary[\declare{#1}]\space\space \char`\%\space\space  Con\TeX t\space and pure pgf}}
    \pgfmanualentryheadline{{\ttfamily\char`\\usetikzlibrary\char`\{\declare{#1}\char`\}\space\space
        \char`\%\space\space  \LaTeX\space and plain \TeX\space when using \tikzname}}
    \pgfmanualentryheadline{{\ttfamily\char`\\usetikzlibrary[\declare{#1}]\space
        \char`\%\space\space  Con\TeX t\space when using \tikzname}}
    \pgfmanualbody
}
{
  \end{pgfmanualentry}
}

\newenvironment{tikzlibrary}[1]{
  \begin{pgfmanualentry}
    \pgfmanualentryheadline{{\ttfamily\char`\\usetikzlibrary\char`\{\declare{#1}\char`\}\space\space \char`\%\space\space  \LaTeX\space and plain \TeX}}
    \index{#1@\protect\texttt{#1} library}%
    \index{Libraries!#1@\protect\texttt{#1}}%
    \pgfmanualentryheadline{{\ttfamily\char`\\usetikzlibrary[\declare{#1}]\space \char`\%\space\space Con\TeX t}}
    \pgfmanualbody
}
{
  \end{pgfmanualentry}
}



\newenvironment{filedescription}[1]{
  \begin{pgfmanualentry}
    \pgfmanualentryheadline{File {\ttfamily\declare{#1}}}%
    \index{#1@\protect\texttt{#1} file}%
    \index{Packages and files!#1@\protect\texttt{#1}}%
    \pgfmanualbody
}
{
  \end{pgfmanualentry}
}


\newenvironment{packageoption}[1]{
  \begin{pgfmanualentry}
    \pgfmanualentryheadline{{\ttfamily\char`\\usepackage[\declare{#1}]\char`\{pgf\char`\}}}
    \index{#1@\protect\texttt{#1} package option}%
    \index{Package options for \textsc{pgf}!#1@\protect\texttt{#1}}%
    \pgfmanualbody
}
{
  \end{pgfmanualentry}
}



\newcommand\opt[1]{{\color{black!50!green}#1}}
\newcommand\ooarg[1]{{\ttfamily[}\meta{#1}{\ttfamily]}}

\def\opt{\afterassignment\pgfmanualopt\let\next=}
\def\pgfmanualopt{\ifx\next\bgroup\bgroup\color{black!50!green}\else{\color{black!50!green}\next}\fi}



\def\beamer{\textsc{beamer}}
\def\pdf{\textsc{pdf}}
\def\eps{\texttt{eps}}
\def\pgfname{\textsc{pgf}}
\def\tikzname{Ti\emph{k}Z}
\def\pstricks{\textsc{pstricks}}
\def\prosper{\textsc{prosper}}
\def\seminar{\textsc{seminar}}
\def\texpower{\textsc{texpower}}
\def\foils{\textsc{foils}}

{
  \makeatletter
  \global\let\myempty=\@empty
  \global\let\mygobble=\@gobble
  \catcode`\@=12
  \gdef\getridofats#1@#2\relax{%
    \def\getridtest{#2}%
    \ifx\getridtest\myempty%
      \expandafter\def\expandafter\strippedat\expandafter{\strippedat#1}
    \else%
      \expandafter\def\expandafter\strippedat\expandafter{\strippedat#1\protect\printanat}
      \getridofats#2\relax%
    \fi%
  }

  \gdef\removeats#1{%
    \let\strippedat\myempty%
    \edef\strippedtext{\stripcommand#1}%
    \expandafter\getridofats\strippedtext @\relax%
  }
  
  \gdef\stripcommand#1{\expandafter\mygobble\string#1}
}

\def\printanat{\char`\@}

\def\declare{\afterassignment\pgfmanualdeclare\let\next=}
\def\pgfmanualdeclare{\ifx\next\bgroup\bgroup\color{red!75!black}\else{\color{red!75!black}\next}\fi}


\let\textoken=\command
\let\endtextoken=\endcommand

\def\myprintocmmand#1{\texttt{\char`\\#1}}

\def\example{\par\smallskip\noindent\textit{Example: }}
\def\themeauthor{\par\smallskip\noindent\textit{Theme author: }}


\def\indexoption#1{%
  \index{#1@\protect\texttt{#1} option}%
  \index{Graphic options and styles!#1@\protect\texttt{#1}}%
}

\def\itemcalendaroption#1{\item \declare{\texttt{#1}}%
  \index{#1@\protect\texttt{#1} date test}%
  \index{Date tests!#1@\protect\texttt{#1}}%
}



\def\class#1{\list{}{\leftmargin=2em\itemindent-\leftmargin\def\makelabel##1{\hss##1}}%
\extractclass#1@\par\topsep=0pt}
\def\endclass{\endlist}
\def\extractclass#1#2@{%
\item{{{\ttfamily\char`\\documentclass}#2{\ttfamily\char`\{\declare{#1}\char`\}}}}%
  \index{#1@\protect\texttt{#1} class}%
  \index{Classes!#1@\protect\texttt{#1}}}

\def\partname{Part}

\makeatletter
\def\index@prologue{\section*{Index}\addcontentsline{toc}{section}{Index}
  This index only contains automatically generated entries. A good
  index should also contain carefully selected keywords. This index is
  not a good index.
  \bigskip
}
\c@IndexColumns=2
  \def\theindex{\@restonecoltrue
    \columnseprule \z@  \columnsep 29\p@
    \twocolumn[\index@prologue]%
       \parindent -30pt
       \columnsep 15pt
       \parskip 0pt plus 1pt
       \leftskip 30pt
       \rightskip 0pt plus 2cm
       \small
       \def\@idxitem{\par}%
    \let\item\@idxitem \ignorespaces}
  \def\endtheindex{\onecolumn}
\def\noindexing{\let\index=\@gobble}



\newcommand\symarrow[1]{
  \index{#1@\protect\texttt{#1} arrow tip}%
  \index{Arrow tips!#1@\protect\texttt{#1}}
  \texttt{#1}& yields thick  
  \begin{tikzpicture}[arrows={#1-#1},thick,baseline]
    \useasboundingbox (0pt,-0.5ex) rectangle (1cm,2ex);
    \draw (0pt,.5ex) -- (1cm,.5ex);
  \end{tikzpicture} and thin
  \begin{tikzpicture}[arrows={#1-#1},thin,baseline]
    \useasboundingbox (0pt,-0.5ex) rectangle (1cm,2ex);
    \draw (0pt,.5ex) -- (1cm,.5ex);
  \end{tikzpicture}
}
\newcommand\symarrowdouble[1]{
  \index{#1@\protect\texttt{#1} arrow tip}%
  \index{Arrow tips!#1@\protect\texttt{#1}}
  \texttt{#1}& yields thick  
  \begin{tikzpicture}[arrows={#1-#1},thick,baseline]
    \useasboundingbox (0pt,-0.5ex) rectangle (1cm,2ex);
    \draw (0pt,.5ex) -- (1cm,.5ex);
  \end{tikzpicture}
  and thin
  \begin{tikzpicture}[arrows={#1-#1},thin,baseline]
    \useasboundingbox (0pt,-0.5ex) rectangle (1cm,2ex);
    \draw (0pt,.5ex) -- (1cm,.5ex);
  \end{tikzpicture}, double 
  \begin{tikzpicture}[arrows={#1-#1},thick,baseline]
    \useasboundingbox (0pt,-0.5ex) rectangle (1cm,2ex);
    \draw[double,double equal sign distance] (0pt,.5ex) -- (1cm,.5ex);
  \end{tikzpicture} and 
  \begin{tikzpicture}[arrows={#1-#1},thin,baseline]
    \useasboundingbox (0pt,-0.5ex) rectangle (1cm,2ex);
    \draw[double,double equal sign distance] (0pt,.5ex) -- (1cm,.5ex);
  \end{tikzpicture}
}

\newcommand\sarrow[2]{
  \index{#1@\protect\texttt{#1} arrow tip}%
  \index{Arrow tips!#1@\protect\texttt{#1}}
  \index{#2@\protect\texttt{#2} arrow tip}%
  \index{Arrow tips!#2@\protect\texttt{#2}}
  \texttt{#1-#2}& yields thick  
  \begin{tikzpicture}[arrows={#1-#2},thick,baseline]
    \useasboundingbox (0pt,-0.5ex) rectangle (1cm,2ex);
    \draw (0pt,.5ex) -- (1cm,.5ex);
  \end{tikzpicture} and thin
  \begin{tikzpicture}[arrows={#1-#2},thin,baseline]
    \useasboundingbox (0pt,-0.5ex) rectangle (1cm,2ex);
    \draw (0pt,.5ex) -- (1cm,.5ex);
  \end{tikzpicture}
}

\newcommand\sarrowdouble[2]{
  \index{#1@\protect\texttt{#1} arrow tip}%
  \index{Arrow tips!#1@\protect\texttt{#1}}
  \index{#2@\protect\texttt{#2} arrow tip}%
  \index{Arrow tips!#2@\protect\texttt{#2}}
  \texttt{#1-#2}& yields thick  
  \begin{tikzpicture}[arrows={#1-#2},thick,baseline]
    \useasboundingbox (0pt,-0.5ex) rectangle (1cm,2ex);
    \draw (0pt,.5ex) -- (1cm,.5ex);
  \end{tikzpicture} and thin
  \begin{tikzpicture}[arrows={#1-#2},thin,baseline]
    \useasboundingbox (0pt,-0.5ex) rectangle (1cm,2ex);
    \draw (0pt,.5ex) -- (1cm,.5ex);
  \end{tikzpicture}, double 
  \begin{tikzpicture}[arrows={#1-#2},thick,baseline]
    \useasboundingbox (0pt,-0.5ex) rectangle (1cm,2ex);
    \draw[double,double equal sign distance] (0pt,.5ex) -- (1cm,.5ex);
  \end{tikzpicture} and 
  \begin{tikzpicture}[arrows={#1-#2},thin,baseline]
    \useasboundingbox (0pt,-0.5ex) rectangle (1cm,2ex);
    \draw[double,double equal sign distance] (0pt,.5ex) -- (1cm,.5ex);
  \end{tikzpicture}
}

\newcommand\carrow[1]{
  \index{#1@\protect\texttt{#1} arrow tip}%
  \index{Arrow tips!#1@\protect\texttt{#1}}
  \texttt{#1}& yields for line width 1ex
  \begin{tikzpicture}[arrows={#1-#1},line width=1ex,baseline]
    \useasboundingbox (0pt,-0.5ex) rectangle (1.5cm,2ex);
    \draw (0pt,.5ex) -- (1.5cm,.5ex);
  \end{tikzpicture}
}
\def\myvbar{\char`\|}
\newcommand\plotmarkentry[1]{%
  \index{#1@\protect\texttt{#1} plot mark}%
  \index{Plot marks!#1@\protect\texttt{#1}}
  \texttt{\char`\\pgfuseplotmark\char`\{\declare{#1}\char`\}} &
  \tikz\draw[color=black!25] plot[mark=#1,mark options={fill=examplefill,draw=black}] coordinates{(0,0) (.5,0.2) (1,0) (1.5,0.2)};\\
}
\newcommand\plotmarkentrytikz[1]{%
  \index{#1@\protect\texttt{#1} plot mark}%
  \index{Plot marks!#1@\protect\texttt{#1}}
  \texttt{mark=\declare{#1}} & \tikz\draw[color=black!25]
  plot[mark=#1,mark options={fill=examplefill,draw=black}] 
    coordinates {(0,0) (.5,0.2) (1,0) (1.5,0.2)};\\
}



\ifx\scantokens\@undefined
  \PackageError{pgfmanual-macros}{You need to use extended latex
    (elatex) or (pdfelatex) to process this document}{}
\fi

\begingroup
\catcode`|=0
\catcode`[= 1
\catcode`]=2
\catcode`\{=12
\catcode `\}=12
\catcode`\\=12 |gdef|find@example#1\end{codeexample}[|endofcodeexample[#1]]
|endgroup

% define \returntospace.
%
% It should define NEWLINE as {}, spaces and tabs as \space.
\begingroup
\catcode`\^=7
\catcode`\^^M=13
\catcode`\^^I=13
\catcode`\ =13%
\gdef\returntospace{\catcode`\ =13\def {\space}\catcode`\^^I=13\def^^I{\space}\catcode`\^^M=13\def^^M{}}%
\endgroup

\begingroup
\catcode`\%=13
\catcode`\^^M=13
\gdef\commenthandler{\catcode`\%=13\def%{\@gobble@till@return}}
\gdef\@gobble@till@return#1^^M{}
\gdef\@gobble@till@return@ignore#1^^M{\ignorespaces}
\gdef\typesetcomment{\catcode`\%=13\def%{\@typeset@till@return}}
\gdef\@typeset@till@return#1^^M{{\def%{\char`\%}\textsl{\char`\%#1}}\par}
\endgroup

% Define tab-implementation functions
%   \codeexample@tabinit@replacementchars@
% and
%   \codeexample@tabinit@catcode@
%
% They should ONLY be used in case that tab replacement is active.
%
% This here is merely a preparation step.
%
% Idea:
% \codeexample@tabinit@catcode@ will make TAB active
% and
% \codeexample@tabinit@replacementchars@ will insert as many spaces as
% /codeexample/tabsize contains.
{
\catcode`\^^I=13
% ATTENTION: do NOT use tabs in these definitions!!
\gdef\codeexample@tabinit@replacementchars@{%
 \begingroup
 \count0=\pgfkeysvalueof{/codeexample/tabsize}\relax
 \toks0={}%
 \loop
 \ifnum\count0>0
  \advance\count0 by-1
  \toks0=\expandafter{\the\toks0\ }%
 \repeat
 \xdef\codeexample@tabinit@replacementchars@@{\the\toks0}%
 \endgroup
 \let^^I=\codeexample@tabinit@replacementchars@@
}%
\gdef\codeexample@tabinit@catcode@{\catcode`\^^I=13}%
}%

% Called after any options have been set. It assigns
%   \codeexample@tabinit@catcode
% and
%   \codeexample@tabinit@replacementchars
% which are used inside of 
%\begin{codeexample}
% ...
%\end{codeexample}
%
% \codeexample@tabinit@catcode  is either \relax or it makes tab
% active.
%
% \codeexample@tabinit@replacementchars is either \relax or it inserts
% a proper replacement sequence for tabs (as many spaces as
% configured)
\def\codeexample@tabinit{%
	\ifnum\pgfkeysvalueof{/codeexample/tabsize}=0\relax
		\let\codeexample@tabinit@replacementchars=\relax
		\let\codeexample@tabinit@catcode=\relax
	\else
		\let\codeexample@tabinit@catcode=\codeexample@tabinit@catcode@
		\let\codeexample@tabinit@replacementchars=\codeexample@tabinit@replacementchars@
	\fi
}

\pgfqkeys{/codeexample}{%
	width/.code=	{\setlength\codeexamplewidth{#1}},
	graphic/.code=	{\colorlet{graphicbackground}{#1}},
	code/.code=	{\colorlet{codebackground}{#1}},
	execute code/.is if=code@execute,
	code only/.code=	{\code@executefalse},
	pre/.code=	{\def\code@pre{#1}},
	post/.code=	{\def\code@post{#1}},
	vbox/.code=	{\def\code@pre{\vbox\bgroup\setlength{\hsize}{\linewidth-6pt}}\def\code@post{\egroup}},
	ignorespaces/.code=	{\let\@gobble@till@return=\@gobble@till@return@ignore},
	leave comments/.code=	{\def\code@catcode@hook{\catcode`\%=12}\let\commenthandler=\relax\let\typesetcomment=\relax},
	tabsize/.initial=0,% FIXME : this here is merely used for indentation. It is just a TAB REPLACEMENT.
	every codeexample/.style={width=4cm+7pt},
}

\def\code@pre{}
\def\code@post{}
\def\code@catcode@hook{}

\newdimen\codeexamplewidth
\newif\ifcode@execute
\newbox\codeexamplebox
\def\codeexample[#1]{%
  \begingroup%
  \code@executetrue
  \pgfqkeys{/codeexample}{every codeexample,#1}%
  \codeexample@tabinit% assigns \codeexample@tabinit@[catcode,replacementchars]
  \parindent0pt
  \begingroup%
  \par%
  \medskip%
  \let\do\@makeother%
  \dospecials%
  \obeylines%
  \@vobeyspaces%
  \catcode`\%=13%
  \catcode`\^^M=13%
  \code@catcode@hook%
  \codeexample@tabinit@catcode
  \relax%
  \find@example}
\def\endofcodeexample#1{%
  \endgroup%
  \ifcode@execute%
    \setbox\codeexamplebox=\hbox{%
      {%
        {%
          \returntospace%
          \commenthandler%
          \xdef\code@temp{#1}% removes returns and comments
        }%
        \colorbox{graphicbackground}{\color{black}\ignorespaces%
          \code@pre\expandafter\scantokens\expandafter{\code@temp\ignorespaces}\code@post\ignorespaces}%
      }%
    }%
    \ifdim\wd\codeexamplebox>\codeexamplewidth%
      \def\code@start{\par}%
      \def\code@flushstart{}\def\code@flushend{}%
      \def\code@mid{\parskip2pt\par\noindent}%
      \def\code@width{\linewidth-6pt}%
      \def\code@end{}%
    \else%
      \def\code@start{%
        \linewidth=\textwidth%
        \parshape \@ne 0pt \linewidth
        \leavevmode%
        \hbox\bgroup}%
      \def\code@flushstart{\hfill}%
      \def\code@flushend{\hbox{}}%
      \def\code@mid{\hskip6pt}%
      \def\code@width{\linewidth-12pt-\codeexamplewidth}%
      \def\code@end{\egroup}%
    \fi%
    \code@start%
    \noindent%
    \begin{minipage}[t]{\codeexamplewidth}\raggedright
      \hrule width0pt%
      \footnotesize\vskip-1em%
      \code@flushstart\box\codeexamplebox\code@flushend%
      \vskip-1ex
      \leavevmode%
    \end{minipage}%
  \else%
    \def\code@mid{\par}
    \def\code@width{\linewidth-6pt}
    \def\code@end{}
  \fi%
  \code@mid%  
  \colorbox{codebackground}{%
    \begin{minipage}[t]{\code@width}%
      {%
        \let\do\@makeother
        \dospecials
        \frenchspacing\@vobeyspaces
        \normalfont\ttfamily\footnotesize
        \typesetcomment%
		\codeexample@tabinit@replacementchars
        \@tempswafalse
        \def\par{%
          \if@tempswa
          \leavevmode \null \@@par\penalty\interlinepenalty
          \else
          \@tempswatrue
          \ifhmode\@@par\penalty\interlinepenalty\fi
          \fi}%
        \obeylines
        \everypar \expandafter{\the\everypar \unpenalty}%
        #1}
    \end{minipage}}%
  \code@end%
  \par%
  \medskip
  \end{codeexample}
}

\def\endcodeexample{\endgroup}


\makeatother


%%% Local Variables: 
%%% mode: latex
%%% TeX-master: "beameruserguide"
%%% End: 
    % TODOsp: same here

\makeindex

\makeatletter
\renewcommand*\l@section[2]{%
  \ifnum \c@tocdepth >\z@
    \addpenalty\@secpenalty
    \addvspace{1.0em \@plus\p@}%
    \setlength\@tempdima{2.5em}%
    \begingroup
      \parindent \z@ \rightskip \@pnumwidth
      \parfillskip -\@pnumwidth
      \leavevmode \bfseries
      \advance\leftskip\@tempdima
      \hskip -\leftskip
      #1\nobreak\hfil \nobreak\hb@xt@\@pnumwidth{\hss #2}\par
    \endgroup
  \fi}
\renewcommand*\l@subsection{\@dottedtocline{2}{2.5em}{3.3em}}
\renewcommand*\l@subsubsection{\@dottedtocline{3}{5.8em}{4.2em}}
\def\@pnumwidth{2.2em}
\makeatother

%\includeonly{pgfmanual-en-library-profiler}

% Global styles:
\tikzset{
  every plot/.style={prefix=plots/pgf-},
  shape example/.style={
    color=black!30,
    draw,
    fill=yellow!30,
    line width=.5cm,
    inner xsep=2.5cm,
    inner ysep=0.5cm}
}

\index{Options for graphics|see{Graphic options and styles}}
\index{Styles for graphics|see{Graphic options and styles}}
\index{Options for packages|see{Package options}}
\index{Handlers for keys|see{Key handlers}}
\index{File|see{Packages and files}}
\index{Layout|see{Page layout}}
\index{Node|see{Predefined node}}
\index{Data formats|see{Formats}}


%%% Local Variables:
%%% mode: latex
%%% TeX-master: "~/pgf/doc/generic/pgf/version-for-luatex/en/pgfmanual"
%%% coding: iso-latin-1-unix
%%% End:

\usepackage[version=latest]{pgf}
\usepackage{xkeyval,calc,listings,tikz,fp}
\usepackage{makeidx}
\makeindex
\usepackage{hyperref}
\hypersetup{%
        colorlinks=true,
        linkcolor=blue,
        filecolor=blue,
        urlcolor=blue,
        citecolor=blue,
        pdfborder=0 0 0,
}
\makeatletter          % see https://tex.stackexchange.com/q/33946
% Copyright 2006 by Till Tantau
%
% This file may be distributed and/or modified
%
% 1. under the LaTeX Project Public License and/or
% 2. under the GNU Free Documentation License.
%
% See the file doc/generic/pgf/licenses/LICENSE for more details.

\documentclass[a4paper]{ltxdoc}

% pgf version is defined in \pgfversion in file
% generic/pgf/utilities/pgfrcs.code.tex 

% Copyright 2006 by Till Tantau
%
% This file may be distributed and/or modified
%
% 1. under the LaTeX Project Public License and/or
% 2. under the GNU Free Documentation License.
%
% See the file doc/generic/pgf/licenses/LICENSE for more details.

\usepackage[hyphens]{url}

\usepackage[xetex]{graphics}
% This will tell everyone which driver we are using.

\usepackage[xetex]{hyperref}

% Make sure to compile this using xelatex + xdvipdfmx.

% Copyright 2006 by Till Tantau
%
% This file may be distributed and/or modified
%
% 1. under the LaTeX Project Public License and/or
% 2. under the GNU Free Documentation License.
%
% See the file doc/generic/pgf/licenses/LICENSE for more details.

% Copyright 2006 by Till Tantau
%
% This file may be distributed and/or modified
%
% 1. under the LaTeX Project Public License and/or
% 2. under the GNU Free Documentation License.
%
% See the file doc/generic/pgf/licenses/LICENSE for more details.


% pgf version is defined in \pgfversion in file
% generic/pgf/utilities/pgfrcs.code.tex

\def\xcolorversion{2.00}

\usepackage[version=latest]{pgf}

\usepackage{xkeyval,calc,listings,tikz,fp}

\usepackage{hyperref}
\hypersetup{%
	colorlinks=false, % use true to enable colors below:
	linkcolor=blue,%red,
	filecolor=blue,%magenta,
	urlcolor=blue,%cyan,
	citecolor=blue,
	pdfborder=0 0 0,
}

% We need lots of libraries...
\usetikzlibrary{
  arrows,
  arrows.spaced,
  arrows.meta,
  bending,
  babel,
  calc,
  fit,
  patterns,
  plotmarks,
  shapes.geometric,
  shapes.misc,
  shapes.symbols,
  shapes.arrows,
  shapes.callouts,
  shapes.multipart,
  shapes.gates.logic.US,
  shapes.gates.logic.IEC,
  circuits.logic.US,
  circuits.logic.IEC,
  circuits.logic.CDH,
  circuits.ee.IEC,
  datavisualization,
  datavisualization.polar,
  datavisualization.formats.functions,
  er,
  automata,
  backgrounds,
  chains,
  topaths,
  trees,
  petri,
  mindmap,
  matrix,
  calendar,
  folding,
  fadings,
  shadings,
  spy,
  through,
  turtle,
  positioning,
  scopes,
  decorations.fractals,
  decorations.shapes,
  decorations.text,
  decorations.pathmorphing,
  decorations.pathreplacing,
  decorations.footprints,
  decorations.markings,
  shadows,
  lindenmayersystems,
  intersections,
  fixedpointarithmetic,
  fpu,
  svg.path,
  external,
  graphs,
  graphs.standard,
  quotes,
  math,
  angles
}

\usepackage{ifluatex}
\newif\ifgdccodebasic
\newif\ifgdccodeogdf

\ifluatex

  \usetikzlibrary{graphdrawing}
  \usegdlibrary{trees,circular,layered,examples,force,phylogenetics,routing}

  % Test whether C code is available:
  \directlua{
    if pcall(require,'pgf_gd_examples_c_SimpleDemoC') then
      tex.print('\string\\gdccodebasictrue')
    end
  }

  \ifgdccodebasic
    \usegdlibrary{pgf_gd_examples_c_SimpleDemoC,pgf_gd_examples_c_SimpleDemoCPlusPlus}

    % Test whether OGDF code is available:
    \directlua{
      if pcall(require,'pgf_gd_ogdf_c_SimpleDemoOGDF') then
        tex.print('\string\\gdccodeogdftrue')
      end
    }

    \ifgdccodeogdf
      \usegdlibrary{pgf_gd_ogdf_c_SimpleDemoOGDF,ogdf}
    \fi
    
  \fi
  
\fi

\def\LuaTeX{Lua\TeX}%



\iffalse
%\iftrue
  \tikzexternalize[
    mode=list only,export=true,% simply skips EVERY picture -> good for debugging the text.
  ]{pgfmanual}

  \tikzifexternalizing{%
    \pgfkeys{/pdflinks/codeexample links=false}%
  }{}%
\fi


\usepackage[a4paper,left=2.25cm,right=2.25cm,top=2.5cm,bottom=2.5cm,nohead]{geometry}
\usepackage{amsmath,amssymb}
\usepackage{xxcolor}
\usepackage{pifont}
\usepackage{makeidx}

% Fontenc (new beginning 2014, let's see, what it breaks...):
\usepackage[T1]{fontenc}

\ifluatex
%  \usepackage[no-math]{fontspec}
%  \usepackage{luatextra}

  % TT: I have commented luatextra since it loads fontspec, which
  % currently breaks "$\mathrm{\Omega}$" (nothing is
  % shown). Also, fontspec loads so much stuff, I'm not really sure
  % any of it is needed...

  % \filedescription is defined in expl3, required by fontspec,
  % required by luatextra. Needs to be \relaxed since
  % pgfmanual-en-macros.tex defines an environment named filedescription
  \let\filedescription\relax
  \usepackage[utf8]{luainputenc}
\else 
  \usepackage[utf8]{inputenc}
\fi
\usepackage{amsmath}

\graphicspath{{../../images/}}              % TODOsp: under windows this would go up 2 directories, but the file is only one directory up
\input{../../macros/pgfmanual-en-macros}    % TODOsp: same here

\makeindex

\makeatletter
\renewcommand*\l@section[2]{%
  \ifnum \c@tocdepth >\z@
    \addpenalty\@secpenalty
    \addvspace{1.0em \@plus\p@}%
    \setlength\@tempdima{2.5em}%
    \begingroup
      \parindent \z@ \rightskip \@pnumwidth
      \parfillskip -\@pnumwidth
      \leavevmode \bfseries
      \advance\leftskip\@tempdima
      \hskip -\leftskip
      #1\nobreak\hfil \nobreak\hb@xt@\@pnumwidth{\hss #2}\par
    \endgroup
  \fi}
\renewcommand*\l@subsection{\@dottedtocline{2}{2.5em}{3.3em}}
\renewcommand*\l@subsubsection{\@dottedtocline{3}{5.8em}{4.2em}}
\def\@pnumwidth{2.2em}
\makeatother

%\includeonly{pgfmanual-en-library-profiler}

% Global styles:
\tikzset{
  every plot/.style={prefix=plots/pgf-},
  shape example/.style={
    color=black!30,
    draw,
    fill=yellow!30,
    line width=.5cm,
    inner xsep=2.5cm,
    inner ysep=0.5cm}
}

\index{Options for graphics|see{Graphic options and styles}}
\index{Styles for graphics|see{Graphic options and styles}}
\index{Options for packages|see{Package options}}
\index{Handlers for keys|see{Key handlers}}
\index{File|see{Packages and files}}
\index{Layout|see{Page layout}}
\index{Node|see{Predefined node}}
\index{Data formats|see{Formats}}


%%% Local Variables:
%%% mode: latex
%%% TeX-master: "~/pgf/doc/generic/pgf/version-for-luatex/en/pgfmanual"
%%% coding: iso-latin-1-unix
%%% End:

% Copyright 2019 by Till Tantau
%
% This file may be distributed and/or modified
%
% 1. under the LaTeX Project Public License and/or
% 2. under the GNU Free Documentation License.
%
% See the file doc/generic/pgf/licenses/LICENSE for more details.


\begin{document}

% The titlepage

\pgfmathsetseed{1}
\newbox\mybox
{
  \parindent0pt
  \null
  \colorlet{mintgreen}{green!50!black!50}

  \thispagestyle{empty}
  \vskip3cm
  \vfill
  \hfil
  \begin{tikzpicture}[overlay]
    \coordinate (front) at (0,0);
    \coordinate (horizon) at (0,.31\paperheight);
    \coordinate (bottom) at (0,-.6\paperheight);
    \coordinate (sky) at (0,.57\paperheight);
    \coordinate (left) at (-.51\paperwidth,0);
    \coordinate (right) at (.51\paperwidth,0);

    \shade [bottom color=blue!30!black!10,top color=blue!30!black!50]
      ([yshift=-5mm]horizon -|  left) rectangle (sky -| right);
    \shade [bottom color=black!70!green!25,top color=black!70!green!10]
      (front -| left) -- (horizon -| left)
      decorate [decoration=random steps] { -- (horizon -| right) }
      -- (front -| right) -- cycle;
    \shade [top color=black!70!green!25,bottom color=black!25]
      ([yshift=-5mm-1pt]front -| left) rectangle ([yshift=1pt]front -| right);
    \fill [black!25] (bottom -| left) rectangle ([yshift=-5mm]front -| right);

    \def\nodeshadowed[#1]#2;{\node[scale=2,above,#1]{\global\setbox\mybox=\hbox{#2}\copy\mybox};
      \node[scale=2,above,#1,yscale=-1,scope fading=south,opacity=0.4]{\box\mybox};}

    \nodeshadowed [at={(-5,5  )},yslant=0.05] {\Huge Ti\textcolor{orange}{\emph{k}}Z};
    \nodeshadowed [at={( 0,5.3)}] {\huge \textcolor{mintgreen}{\&}};
    \nodeshadowed [at={( 5,5  )},yslant=-0.05] {\Huge \textsc{PGF}};
    \nodeshadowed [at={( 0,2  )}] {Manual for Version \pgftypesetversion};

    \foreach \where in {-9cm,9cm}
    {\nodeshadowed [at={(\where,5cm)}] {
    % TODO: Nesting tikzpictures is NOT supported
    \tikz \draw [green!20!black, rotate=90]
    [l-system={rule set={F -> FF-[-F+F]+[+F-F]}, axiom=F, order=4,
      step=2pt, randomize step percent=50, angle=30, randomize angle percent=5}]
    lindenmayer system;};}

    \foreach \i in {0.5,0.6,...,2}
      \fill [white,decoration=Koch snowflake,opacity=.9]
            [shift=(horizon),shift={(rand*11,rnd*7)},scale=\i]
            [double copy shadow={opacity=0.2,shadow xshift=0pt,shadow
              yshift=3*\i pt,fill=white,draw=none}]
        decorate {
          decorate {
            decorate {
              (0,0) -- ++(60:1) -- ++(-60:1) -- cycle
            }
          }
        };

  \node (left text) [text width=.5\paperwidth-2cm,below right,at={(-.5\paperwidth+1cm,-1.5cm)}]
  {
    \fontencoding{T1}
    \fontfamily{pcr}
    \def\textbraceleft{\char`\{}
    \def\textbraceright{\char`\}}
    \def\textbackslash{\char`\\}
    \begin{lstlisting}[basicstyle=\scriptsize\color{black},
                       keywordstyle=\bfseries\color{white},
                       identifierstyle=\bfseries\color{black},
                       keywords={tikzpicture,shade,fill,draw,path,node},
                       literate={-}{{-}}1]
\begin{tikzpicture}
  \coordinate (front) at (0,0);
  \coordinate (horizon) at (0,.31\paperheight);
  \coordinate (bottom) at (0,-.6\paperheight);
  \coordinate (sky) at (0,.57\paperheight);
  \coordinate (left) at (-.51\paperwidth,0);
  \coordinate (right) at (.51\paperwidth,0);

  \shade [bottom color=white,
          top color=blue!30!black!50]
              ([yshift=-5mm]horizon -|  left)
    rectangle (sky -| right);

  \shade [bottom color=black!70!green!25,
          top color=black!70!green!10]
    (front -| left) -- (horizon -| left)
    decorate [decoration=random steps] {
      -- (horizon -| right)  }
    -- (front -| right) -- cycle;

  \shade [top color=black!70!green!25,
         bottom color=black!25]
              ([yshift=-5mm-1pt]front -| left)
    rectangle ([yshift=1pt]front -| right);

  \fill [black!25]
              (bottom -| left)
    rectangle ([yshift=-5mm]front -| right);

  \def\nodeshadowed[#1]#2;{
    \node[scale=2,above,#1]{
      \global\setbox\mybox=\hbox{#2}
      \copy\mybox};
    \node[scale=2,above,#1,yscale=-1,
          scope fading=south,opacity=0.4]{\box\mybox};
  }
\end{lstlisting}
};

  \node (right text) [text width=.5\paperwidth-2cm,below right,at={(1cm,-1.5cm)}]
  {
    \fontencoding{T1}
    \fontfamily{pcr}
    \def\textbraceleft{\char`\{}
    \def\textbraceright{\char`\}}
    \def\textbackslash{\char`\\}
    \begin{lstlisting}[basicstyle=\scriptsize\color{black},
                       keywordstyle=\bfseries\color{white},
                       identifierstyle=\bfseries\color{black},
                       keywords={tikzpicture,shade,fill,draw,path,node},
                       literate={-}{{-}}1]
  \nodeshadowed [at={(-5,8  )},yslant=0.05]
    {\Huge Ti\textcolor{orange}{\emph{k}}Z};
  \nodeshadowed [at={( 0,8.3)}]
    {\huge \textcolor{green!50!black!50}{\&}};
  \nodeshadowed [at={( 5,8  )},yslant=-0.05]
    {\Huge \textsc{PGF}};
  \nodeshadowed [at={( 0,5  )}]
    {Manual for Version \pgftypesetversion};

  \foreach \where in {-9cm,9cm} {
    \nodeshadowed [at={(\where,5cm)}] { \tikz
      \draw [green!20!black, rotate=90,
             l-system={rule set={F -> FF-[-F+F]+[+F-F]},
               axiom=F, order=4,step=2pt,
               randomize step percent=50, angle=30,
               randomize angle percent=5}] l-system; }}

  \foreach \i in {0.5,0.6,...,2}
    \fill
      [white,opacity=\i/2,
       decoration=Koch snowflake,
       shift=(horizon),shift={(rand*11,rnd*7)},
       scale=\i,double copy shadow={
         opacity=0.2,shadow xshift=0pt,
         shadow yshift=3*\i pt,fill=white,draw=none}]
      decorate {
        decorate {
          decorate {
            (0,0)- ++(60:1) -- ++(-60:1) -- cycle
          } } };

   \node (left text) ...
   \node (right text) ...

   \fill [decorate,decoration={footprints,foot of=gnome},
          opacity=.5,brown]        (rand*8,-rnd*10)
     to [out=rand*180,in=rand*180] (rand*8,-rnd*10);
\end{tikzpicture}
  \end{lstlisting}
  };

  \fill [decorate,decoration=footprints,
         decoration={footprints,foot of=gnome},
         opacity=.5,brown]        (rand*8,-rnd*10)
    to [out=rand*180,in=rand*180] (rand*8,-rnd*10);
\end{tikzpicture}
\vfill
\vbox{}
\clearpage
}

{
  \vbox{}
  \vskip0pt plus 1fill
  Für meinen Vater, damit er noch viele schöne \TeX-Graphiken
  erschaffen kann.
  \vskip1em
  \hfill\emph{Till}
  \vskip0pt plus 3fill

  \parindent=0pt
  Copyright 2007 to 2013 by Till Tantau

  \medskip
  Permission is granted to copy, distribute and/or modify \emph{the
  documentation} under the terms of the \textsc{gnu} Free Documentation
  License, Version 1.2 or any later version published by the Free Software
  Foundation; with no Invariant Sections, no Front-Cover Texts, and no
  Back-Cover Texts. A copy of the license is included in the section entitled
  \textsc{gnu} Free Documentation License.

  \medskip
  Permission is granted to copy, distribute and/or modify \emph{the code of the
  package} under the terms of the \textsc{gnu} Public License, Version 2 or any
  later version published by the Free Software Foundation. A copy of the
  license is included in the section entitled \textsc{gnu} Public License.

  \medskip
  Permission is also granted to distribute and/or modify \emph{both the
  documentation and the code} under the conditions of the LaTeX Project Public
  License, either version 1.3 of this license or (at your option) any later
  version. A copy of the license is included in the section entitled \LaTeX\
  Project Public License.

  \vbox{}
  \clearpage
}


\title{\bfseries The \tikzname\ and {\Large PGF} Packages\\
  \large Manual for version \pgfversion\\[1mm]
\large\href{https://github.com/pgf-tikz/pgf}{\texttt{https://github.com/pgf-tikz/pgf}}}
\author{Till Tantau\footnote{Editor of this documentation. Parts of
    this documentation have been written by other authors as indicated
    in these parts or chapters and in Section~\ref{section-authors}.}\\
  \normalsize Institut für Theoretische Informatik\\[-1mm]
  \normalsize Universität zu Lübeck}

\maketitle
\label{table-of-contents}

\tableofcontents

\clearpage


\include{pgfmanual-en-introduction}



\part{Tutorials and Guidelines}

{\Large \emph{by Till Tantau}}

\bigskip
\noindent To help you get started with \tikzname, instead of a long
installation and configuration section, this manual starts with tutorials. They
explain all the basic and some of the more advanced features of the system,
without going into all the details. This part also contains some guidelines on
how you should proceed when creating graphics using \tikzname.

\vskip3cm

\begin{codeexample}[graphic=white,width=0pt]
\tikz \draw[thick,rounded corners=8pt]
  (0,0) -- (0,2) -- (1,3.25) -- (2,2) -- (2,0) -- (0,2) -- (2,2) -- (0,0) -- (2,0);
\end{codeexample}


\include{pgfmanual-en-tutorial}
\include{pgfmanual-en-tutorial-nodes}
\include{pgfmanual-en-tutorial-Euclid}
\include{pgfmanual-en-tutorial-chains}
\include{pgfmanual-en-tutorial-map}
\include{pgfmanual-en-guidelines}



\part{Installation and Configuration}

{\Large \emph{by Till Tantau}}


\bigskip
\noindent This part explains how the system is installed. Typically, someone
has already done so for your system, so this part can be skipped; but if this
is not the case and you are the poor fellow who has to do the installation,
read the present part.


\vskip1cm

\begin{codeexample}[graphic=white,preamble={\usetikzlibrary{arrows.meta,automata,positioning,shadows}}]
\begin{tikzpicture}[->,>={Stealth[round]},shorten >=1pt,auto,node distance=2.8cm,on grid,semithick,
                    every state/.style={fill=red,draw=none,circular drop shadow,text=white}]

  \node[initial,state] (A)                    {$q_a$};
  \node[state]         (B) [above right=of A] {$q_b$};
  \node[state]         (D) [below right=of A] {$q_d$};
  \node[state]         (C) [below right=of B] {$q_c$};
  \node[state]         (E) [below=of D]       {$q_e$};

  \path (A) edge              node {0,1,L} (B)
            edge              node {1,1,R} (C)
        (B) edge [loop above] node {1,1,L} (B)
            edge              node {0,1,L} (C)
        (C) edge              node {0,1,L} (D)
            edge [bend left]  node {1,0,R} (E)
        (D) edge [loop below] node {1,1,R} (D)
            edge              node {0,1,R} (A)
        (E) edge [bend left]  node {1,0,R} (A);

   \node [right=1cm,text width=8cm] at (C)
   {
     The current candidate for the busy beaver for five states. It is
     presumed that this Turing machine writes a maximum number of
     $1$'s before halting among all Turing machines with five states
     and the tape alphabet $\{0, 1\}$. Proving this conjecture is an
     open research problem.
   };
\end{tikzpicture}
\end{codeexample}


\include{pgfmanual-en-installation}
\include{pgfmanual-en-license}
\include{pgfmanual-en-drivers}



\part{Ti\emph{k}Z ist \emph{kein} Zeichenprogramm}
\label{part-tikz}

{\Large \emph{by Till Tantau}}


\bigskip
\noindent
\vskip3cm
\begin{codeexample}[graphic=white,preamble={\usetikzlibrary{angles,calc,quotes}}]
\begin{tikzpicture}[angle radius=.75cm]

  \node (A) at (-2,0)     [red,left]   {$A$};
  \node (B) at ( 3,.5)    [red,right]  {$B$};
  \node (C) at (-2,2)     [blue,left]  {$C$};
  \node (D) at ( 3,2.5)   [blue,right] {$D$};
  \node (E) at (60:-5mm)  [below]      {$E$};
  \node (F) at (60:3.5cm) [above]      {$F$};

  \coordinate (X) at (intersection cs:first line={(A)--(B)}, second line={(E)--(F)});
  \coordinate (Y) at (intersection cs:first line={(C)--(D)}, second line={(E)--(F)});

  \path
    (A) edge [red, thick]  (B)
    (C) edge [blue, thick] (D)
    (E) edge [thick]       (F)
      pic ["$\alpha$", draw, fill=yellow]   {angle = F--X--A}
      pic ["$\beta$",  draw, fill=green!30] {angle = B--X--F}
      pic ["$\gamma$", draw, fill=yellow]   {angle = E--Y--D}
      pic ["$\delta$", draw, fill=green!30] {angle = C--Y--E};

  \node at ($ (D)!.5!(B) $) [right=1cm,text width=6cm,rounded corners,fill=red!20,inner sep=1ex]
    {
      When we assume that $\color{red}AB$ and $\color{blue}CD$ are
      parallel, i.\,e., ${\color{red}AB} \mathbin{\|} \color{blue}CD$,
      then $\alpha = \gamma$ and $\beta = \delta$.
    };
\end{tikzpicture}
\end{codeexample}


\include{pgfmanual-en-tikz-design}
\include{pgfmanual-en-tikz-scopes}
\include{pgfmanual-en-tikz-coordinates}
\include{pgfmanual-en-tikz-paths}
\include{pgfmanual-en-tikz-actions}
\include{pgfmanual-en-tikz-arrows}
\include{pgfmanual-en-tikz-shapes}
\include{pgfmanual-en-tikz-pics}
\include{pgfmanual-en-tikz-graphs}
\include{pgfmanual-en-tikz-matrices}
\include{pgfmanual-en-tikz-trees}
\include{pgfmanual-en-tikz-plots}
\include{pgfmanual-en-tikz-transparency}
\include{pgfmanual-en-tikz-decorations}
\include{pgfmanual-en-tikz-transformations}
\include{pgfmanual-en-tikz-animations}



\part{Graph Drawing}
\label{part-gd}

{\Large \emph{by Till Tantau et al.}}

\bigskip
\noindent
\emph{Graph drawing algorithms} do the tough work of computing a layout of a
graph for you. \tikzname\ comes with powerful such algorithms, but you can also
implement new algorithms in the Lua programming language. \vskip1cm

\ifluatex
\begin{codeexample}[
    graphic=white,
    preamble={\usetikzlibrary{arrows.meta,graphs,graphdrawing}
\usegdlibrary{layered}}]
\tikz [nodes={text height=.7em, text depth=.2em,
              draw=black!20, thick, fill=white, font=\footnotesize},
       >={Stealth[round,sep]}, rounded corners, semithick]
  \graph [layered layout, level distance=1cm, sibling sep=.5em, sibling distance=1cm] {
    "5th Edition" -> { "6th Edition", "PWB 1.0" };
    "6th Edition" -> { "LSX" [>child anchor=45],  "1 BSD", "Mini Unix", "Wollongong", "Interdata" };
    "Interdata" -> { "Unix/TS 3.0", "PWB 2.0", "7th Edition" };
    "7th Edition" -> { "8th Edition", "32V", "V7M", "Ultrix-11", "Xenix", "UniPlus+" };
    "V7M" -> "Ultrix-11";
    "8th Edition" -> "9th Edition";
    "1 BSD" -> "2 BSD" -> "2.8 BSD" -> { "Ultrix-11", "2.9 BSD" };
    "32V" -> "3 BSD" -> "4 BSD" -> "4.1 BSD" -> { "4.2 BSD", "2.8 BSD", "8th Edition" };
    "4.2 BSD" -> { "4.3 BSD", "Ultrix-32" };
    "PWB 1.0" -> { "PWB 1.2" -> "PWB 2.0", "USG 1.0" -> { "CB Unix 1", "USG 2.0" }};
    "CB Unix 1" -> "CB Unix 2" -> "CB Unix 3" -> { "Unix/TS++", "PDP-11 Sys V" };
    { "USG 2.0" -> "USG 3.0", "PWB 2.0", "Unix/TS 1.0" } -> "Unix/TS 3.0";
    { "Unix/TS++", "CB Unix 3", "Unix/TS 3.0" } -> "TS 4.0" -> "System V.0" -> "System V.2" -> "System V.3";
  };
\end{codeexample}

\else
    You need to use Lua\TeX\ to typeset this part of the manual (and, also, to
    use algorithmic graph drawing).
\fi


\include{pgfmanual-en-gd-overview}
\include{pgfmanual-en-gd-usage-tikz}
\include{pgfmanual-en-gd-usage-pgf}
\include{pgfmanual-en-gd-trees}
\include{pgfmanual-en-gd-layered}
\include{pgfmanual-en-gd-force}
\include{pgfmanual-en-gd-circular}
\include{pgfmanual-en-gd-phylogenetics}
\include{pgfmanual-en-gd-edge-routing}
%
% XXX : disabled because of
% 1. compile-time dependencies which are hard to resolve
% 2. it is "hardly usable anyway" (TT)
%\include{pgfmanual-en-gd-ogdf}
\include{pgfmanual-en-gd-algorithm-layer}
\include{pgfmanual-en-gd-algorithms-in-c}
\include{pgfmanual-en-gd-display-layer}
\include{pgfmanual-en-gd-binding-layer}



\part{Libraries}
\label{part-libraries}

{\Large \emph{by Till Tantau}}


\bigskip
\noindent
In this part the library packages are documented. They provide additional
predefined graphic objects like new arrow heads or new plot marks, but
sometimes also extensions of the basic \pgfname\ or \tikzname\ system. The
libraries are not loaded by default since many users will not need them.

\medskip
\noindent
\begin{codeexample}[graphic=white,preamble={\usetikzlibrary{arrows,trees}}]
\tikzset{
  ld/.style={level distance=#1},lw/.style={line width=#1},
  level 1/.style={ld=4.5mm, trunk,          lw=1ex ,sibling angle=60},
  level 2/.style={ld=3.5mm, trunk!80!leaf a,lw=.8ex,sibling angle=56},
  level 3/.style={ld=2.75mm,trunk!60!leaf a,lw=.6ex,sibling angle=52},
  level 4/.style={ld=2mm,   trunk!40!leaf a,lw=.4ex,sibling angle=48},
  level 5/.style={ld=1mm,   trunk!20!leaf a,lw=.3ex,sibling angle=44},
  level 6/.style={ld=1.75mm,leaf a,         lw=.2ex,sibling angle=40},
}
\pgfarrowsdeclare{leaf}{leaf}
  {\pgfarrowsleftextend{-2pt} \pgfarrowsrightextend{1pt}}
{
  \pgfpathmoveto{\pgfpoint{-2pt}{0pt}}
  \pgfpatharc{150}{30}{1.8pt}
  \pgfpatharc{-30}{-150}{1.8pt}
  \pgfusepathqfill
}

\newcommand{\logo}[5]
{
  \colorlet{border}{#1}
  \colorlet{trunk}{#2}
  \colorlet{leaf a}{#3}
  \colorlet{leaf b}{#4}
  \begin{tikzpicture}
    \scriptsize\scshape
    \draw[border,line width=1ex,yshift=.3cm,
          yscale=1.45,xscale=1.05,looseness=1.42]
      (1,0) to [out=90, in=0]    (0,1)  to [out=180,in=90]  (-1,0)
            to [out=-90,in=-180] (0,-1) to [out=0,  in=-90] (1,0) -- cycle;

    \coordinate (root) [grow cyclic,rotate=90]
    child {
      child [line cap=round] foreach \a in {0,1} {
        child foreach \b in {0,1} {
          child foreach \c in {0,1} {
            child foreach \d in {0,1} {
              child foreach \leafcolor in {leaf a,leaf b}
                { edge from parent [color=\leafcolor,-#5] }
        } } }
      } edge from parent [shorten >=-1pt,serif cm-,line cap=butt]
    };

    \node [align=center,below] at (0pt,-.5ex)
    { \textcolor{border}{T}heoretical \\ \textcolor{border}{C}omputer \\
      \textcolor{border}{S}cience };
  \end{tikzpicture}
}
\begin{minipage}{3cm}
  \logo{green!80!black}{green!25!black}{green}{green!80}{leaf}\\
  \logo{green!50!black}{black}{green!80!black}{red!80!green}{leaf}\\
  \logo{red!75!black}{red!25!black}{red!75!black}{orange}{leaf}\\
  \logo{black!50}{black}{black!50}{black!25}{}
\end{minipage}
\end{codeexample}


\include{pgfmanual-en-library-3d}
\include{pgfmanual-en-library-angles}
\include{pgfmanual-en-library-arrows}
\include{pgfmanual-en-library-automata}
\include{pgfmanual-en-library-babel}
\include{pgfmanual-en-library-backgrounds}
\include{pgfmanual-en-library-bbox}
\include{pgfmanual-en-library-calc}
\include{pgfmanual-en-library-calendar}
\include{pgfmanual-en-library-chains}
\include{pgfmanual-en-library-circuits}
\include{pgfmanual-en-library-decorations}
\include{pgfmanual-en-library-er}
\include{pgfmanual-en-library-external}
\include{pgfmanual-en-library-fadings}
\include{pgfmanual-en-library-fit}
\include{pgfmanual-en-library-fixedpoint}
\include{pgfmanual-en-library-fpu}
\include{pgfmanual-en-library-lsystems}
\include{pgfmanual-en-library-math}
\include{pgfmanual-en-library-matrices}
\include{pgfmanual-en-library-mindmaps}
\include{pgfmanual-en-library-folding}
\include{pgfmanual-en-library-patterns}
\include{pgfmanual-en-library-perspective}
\include{pgfmanual-en-library-petri}
\include{pgfmanual-en-library-plot-handlers}
\include{pgfmanual-en-library-plot-marks}
\include{pgfmanual-en-library-profiler}
\include{pgfmanual-en-library-rdf}
\include{pgfmanual-en-library-shadings}
\include{pgfmanual-en-library-shadows}
\include{pgfmanual-en-library-shapes}
\include{pgfmanual-en-library-spy}
\include{pgfmanual-en-library-svg-path}
\include{pgfmanual-en-library-edges}
\include{pgfmanual-en-library-through}
\include{pgfmanual-en-library-trees}
\include{pgfmanual-en-library-turtle}
\include{pgfmanual-en-library-views}



\part{Data Visualization}
\label{part-dv}

{\Large \emph{by Till Tantau}}

\bigskip
\noindent

\begin{codeexample}[graphic=white,preamble={\usetikzlibrary{datavisualization.formats.functions}}]
\tikz \datavisualization [scientific axes=clean]
[
  visualize as smooth line=Gaussian,
  Gaussian={pin in data={text={$e^{-x^2}$},when=x is 1}}
]
data [format=function] {
  var x : interval [-7:7] samples 51;
  func y = exp(-\value x*\value x);
}
[
  visualize as scatter,
  legend={south east outside},
  scatter={
    style={mark=*,mark size=1.4pt},
    label in legend={text={
        $\sum_{i=1}^{10} x_i$, where $x_i \sim U(-1,1) $}}}
]
data [format=function] {
  var i : interval [0:1] samples 20;
  func y = 0;
  func x = (rand + rand + rand + rand + rand +
            rand + rand + rand + rand + rand);
};
\end{codeexample}


\include{pgfmanual-en-dv-introduction}
\include{pgfmanual-en-dv-main}
\include{pgfmanual-en-dv-formats}
\include{pgfmanual-en-dv-axes}
\include{pgfmanual-en-dv-visualizers}
\include{pgfmanual-en-dv-stylesheets}
\include{pgfmanual-en-dv-polar}
\include{pgfmanual-en-dv-backend}



\part{Utilities}
\label{part-utilities}

{\Large \emph{by Till Tantau}}


\bigskip
\noindent
The utility packages are not directly involved in creating graphics, but you
may find them useful nonetheless. All of them either directly depend on
\pgfname\ or they are designed to work well together with \pgfname\ even though
they can be used in a stand-alone way.

\vskip2cm
\medskip
\noindent
\begin{codeexample}[graphic=white]
\begin{tikzpicture}[scale=2]
  \shade[top color=blue,bottom color=gray!50] (0,0) parabola (1.5,2.25) |- (0,0);
  \draw (1.05cm,2pt) node[above] {$\displaystyle\int_0^{3/2} \!\!x^2\mathrm{d}x$};

  \draw[help lines] (0,0) grid (3.9,3.9)
       [step=0.25cm]      (1,2) grid +(1,1);

  \draw[->] (-0.2,0) -- (4,0) node[right] {$x$};
  \draw[->] (0,-0.2) -- (0,4) node[above] {$f(x)$};

  \foreach \x/\xtext in {1/1, 1.5/1\frac{1}{2}, 2/2, 3/3}
    \draw[shift={(\x,0)}] (0pt,2pt) -- (0pt,-2pt) node[below] {$\xtext$};

  \foreach \y/\ytext in {1/1, 2/2, 2.25/2\frac{1}{4}, 3/3}
    \draw[shift={(0,\y)}] (2pt,0pt) -- (-2pt,0pt) node[left] {$\ytext$};

  \draw (-.5,.25) parabola bend (0,0) (2,4) node[below right] {$x^2$};
\end{tikzpicture}
\end{codeexample}


\include{pgfmanual-en-pgfkeys}
\include{pgfmanual-en-pgffor}
\include{pgfmanual-en-pgfcalendar}
\include{pgfmanual-en-pages}
\include{pgfmanual-en-xxcolor}
\include{pgfmanual-en-module-parser}



\part{Mathematical and Object-Oriented Engines}

{\Large \emph{by Mark Wibrow and Till Tantau}}


\bigskip
\noindent
\pgfname\ comes with two useful engines: One for doing mathematics, one for
doing object-oriented programming. Both engines can be used independently of
the main \pgfname.

The job of the mathematical engine is to support mathematical operations like
addition, subtraction, multiplication and division, using both integers and
non-integers, but also functions such as square-roots, sine, cosine, and
generate pseudo-random numbers. Mostly, you will use the mathematical
facilities of \pgfname\ indirectly, namely when you write a coordinate like
|(5cm*3,6cm/4)|, but the mathematical engine can also be used independently of
\pgfname\ and \tikzname.

The job of the object-oriented engine is to support simple object-oriented
programming in \TeX. It allows the definition of \emph{classes} (without
inheritance), \emph{methods}, \emph{attributes} and \emph{objects}.

\vskip1cm
\begin{codeexample}[graphic=white]
\pgfmathsetseed{1}
\foreach \col in {black,red,green,blue}
{
  \begin{tikzpicture}[x=10pt,y=10pt,ultra thick,baseline,line cap=round]
    \coordinate (current point) at (0,0);
    \coordinate (old velocity) at (0,0);
    \coordinate (new velocity) at (rand,rand);

    \foreach \i in {0,1,...,100}
    {
      \draw[\col!\i] (current point)
      .. controls ++([scale=-1]old velocity) and
                  ++(new velocity) .. ++(rand,rand)
         coordinate (current point);
      \coordinate (old velocity) at (new velocity);
      \coordinate (new velocity) at (rand,rand);
    }
  \end{tikzpicture}
}
\end{codeexample}


\include{pgfmanual-en-math-design}
\include{pgfmanual-en-math-parsing}
\include{pgfmanual-en-math-commands}
\include{pgfmanual-en-math-algorithms}
\include{pgfmanual-en-math-numberprinting}
\include{pgfmanual-en-oo}



\part{The Basic Layer}

{\Large \emph{by Till Tantau}}


\bigskip
\noindent
\vskip1cm
\begin{codeexample}[graphic=white]
\begin{tikzpicture}
  \draw[gray,very thin] (-1.9,-1.9) grid (2.9,3.9)
          [step=0.25cm] (-1,-1) grid (1,1);
  \draw[blue] (1,-2.1) -- (1,4.1); % asymptote

  \draw[->] (-2,0) -- (3,0) node[right] {$x(t)$};
  \draw[->] (0,-2) -- (0,4) node[above] {$y(t)$};

  \foreach \pos in {-1,2}
    \draw[shift={(\pos,0)}] (0pt,2pt) -- (0pt,-2pt) node[below] {$\pos$};

  \foreach \pos in {-1,1,2,3}
    \draw[shift={(0,\pos)}] (2pt,0pt) -- (-2pt,0pt) node[left] {$\pos$};

  \fill (0,0) circle (0.064cm);
  \draw[thick,parametric,domain=0.4:1.5,samples=200]
    % The plot is reparameterised such that there are more samples
    % near the center.
    plot[id=asymptotic-example] function{(t*t*t)*sin(1/(t*t*t)),(t*t*t)*cos(1/(t*t*t))}
    node[right] {$\bigl(x(t),y(t)\bigr) = (t\sin \frac{1}{t}, t\cos \frac{1}{t})$};

  \fill[red] (0.63662,0) circle (2pt)
    node [below right,fill=white,yshift=-4pt] {$(\frac{2}{\pi},0)$};
\end{tikzpicture}
\end{codeexample}


\include{pgfmanual-en-base-design}
\include{pgfmanual-en-base-scopes}
\include{pgfmanual-en-base-points}
\include{pgfmanual-en-base-paths}
\include{pgfmanual-en-base-decorations}
\include{pgfmanual-en-base-actions}
\include{pgfmanual-en-base-arrows}
\include{pgfmanual-en-base-nodes}
\include{pgfmanual-en-base-matrices}
\include{pgfmanual-en-base-transformations}
\include{pgfmanual-en-base-patterns}
\include{pgfmanual-en-base-images}
\include{pgfmanual-en-base-external}
\include{pgfmanual-en-base-plots}
\include{pgfmanual-en-base-layers}
\include{pgfmanual-en-base-shadings}
\include{pgfmanual-en-base-transparency}
\include{pgfmanual-en-base-animations}
\include{pgfmanual-en-base-internalregisters}
\include{pgfmanual-en-base-quick}



\part{The System Layer}
\label{part-system}

{\Large \emph{by Till Tantau}}


\bigskip
\noindent
This part describes the low-level interface of \pgfname, called the
\emph{system layer}. This interface provides a complete abstraction of the
internals of the underlying drivers.

Unless you intend to port \pgfname\ to another driver or unless you intend to
write your own optimized frontend, you need not read this part.

In the following it is assumed that you are familiar with the basic workings of
the |graphics| package and that you know what \TeX-drivers are and how they
work.

\vskip1cm
\begin{codeexample}[graphic=white]
\begin{tikzpicture}
  [shorten >=1pt,->,
   vertex/.style={circle,fill=black!25,minimum size=17pt,inner sep=0pt}]

  \foreach \name/\x in {s/1, 2/2, 3/3, 4/4, 15/11, 16/12, 17/13, 18/14, 19/15, t/16}
    \node[vertex] (G-\name) at (\x,0) {$\name$};

  \foreach \name/\angle/\text in {P-1/234/5, P-2/162/6, P-3/90/7, P-4/18/8, P-5/-54/9}
    \node[vertex,xshift=6cm,yshift=.5cm] (\name) at (\angle:1cm) {$\text$};

  \foreach \name/\angle/\text in {Q-1/234/10, Q-2/162/11, Q-3/90/12, Q-4/18/13, Q-5/-54/14}
    \node[vertex,xshift=9cm,yshift=.5cm] (\name) at (\angle:1cm) {$\text$};

  \foreach \from/\to in {s/2,2/3,3/4,3/4,15/16,16/17,17/18,18/19,19/t}
    \draw (G-\from) -- (G-\to);

  \foreach \from/\to in {1/2,2/3,3/4,4/5,5/1,1/3,2/4,3/5,4/1,5/2}
    { \draw (P-\from) -- (P-\to); \draw (Q-\from) -- (Q-\to); }

  \draw (G-3) .. controls +(-30:2cm) and +(-150:1cm) .. (Q-1);
  \draw (Q-5) -- (G-15);
\end{tikzpicture}
\end{codeexample}


\include{pgfmanual-en-pgfsys-overview}
\include{pgfmanual-en-pgfsys-commands}
\include{pgfmanual-en-pgfsys-paths}
\include{pgfmanual-en-pgfsys-protocol}
\include{pgfmanual-en-pgfsys-animations}



\part{References and Index}

\vskip1cm
\begin{codeexample}[graphic=white]
\begin{tikzpicture}
  \draw[line width=0.3cm,color=red!30,line cap=round,line join=round] (0,0)--(2,0)--(2,5);
  \draw[help lines] (-2.5,-2.5) grid (5.5,7.5);
  \draw[very thick] (1,-1)--(-1,-1)--(-1,1)--(0,1)--(0,0)--
    (1,0)--(1,-1)--(3,-1)--(3,2)--(2,2)--(2,3)--(3,3)--
    (3,5)--(1,5)--(1,4)--(0,4)--(0,6)--(1,6)--(1,5)
    (3,3)--(4,3)--(4,5)--(3,5)--(3,6)
    (3,-1)--(4,-1);
  \draw[below left] (0,0) node(s){$s$};
  \draw[below left] (2,5) node(t){$t$};
  \fill (0,0) circle (0.06cm) (2,5) circle (0.06cm);
  \draw[->,rounded corners=0.2cm,shorten >=2pt]
    (1.5,0.5)-- ++(0,-1)-- ++(1,0)-- ++(0,2)-- ++(-1,0)-- ++(0,2)-- ++(1,0)--
    ++(0,1)-- ++(-1,0)-- ++(0,-1)-- ++(-2,0)-- ++(0,3)-- ++(2,0)-- ++(0,-1)--
    ++(1,0)-- ++(0,1)-- ++(1,0)-- ++(0,-1)-- ++(1,0)-- ++(0,-3)-- ++(-2,0)--
    ++(1,0)-- ++(0,-3)-- ++(1,0)-- ++(0,-1)-- ++(-6,0)-- ++(0,3)-- ++(2,0)--
    ++(0,-1)-- ++(1,0);
\end{tikzpicture}
\end{codeexample}

\printindex

%\typeout{Examples: \the\codeexamplecount}%
\end{document}


%%% Local Variables:
%%% mode: latex
%%% TeX-master: "~/pgf/doc/generic/pgf/version-for-luatex/en/pgfmanual"
%%% coding: iso-latin-1-unix
%%% End:



%%% Local Variables:
%%% mode: latex
%%% End:

 % 
\makeatother           % 
% Copyright 2006 by Till Tantau
%
% This file may be distributed and/or modified
%
% 1. under the LaTeX Project Public License and/or
% 2. under the GNU Free Documentation License.
%
% See the file doc/generic/pgf/licenses/LICENSE for more details.

% $Header$


\providecommand\href[2]{\texttt{#1}}


\colorlet{examplefill}{yellow!80!black}
\definecolor{graphicbackground}{rgb}{0.96,0.96,0.8}
\definecolor{codebackground}{rgb}{0.8,0.8,1}

\newenvironment{pgfmanualentry}{\list{}{\leftmargin=2em\itemindent-\leftmargin\def\makelabel##1{\hss##1}}}{\endlist}
\newcommand\pgfmanualentryheadline[1]{\itemsep=0pt\parskip=0pt\item\strut{#1}\par\topsep=0pt}
\newcommand\pgfmanualbody{\parskip3pt}



\newenvironment{pgflayout}[1]{
  \begin{pgfmanualentry}
    \pgfmanualentryheadline{\texttt{\string\pgfpagesuselayout\char`\{\declare{#1}\char`\}}\oarg{options}}
    \index{#1@\protect\texttt{#1} layout}%
    \index{Page layouts!#1@\protect\texttt{#1}}%
    \pgfmanualbody
}
{
  \end{pgfmanualentry}
}


\newenvironment{command}[1]{
  \begin{pgfmanualentry}
    \extractcommand#1\@@
    \pgfmanualbody
}
{
  \end{pgfmanualentry}
}

%% MW: START MATH MACROS
\def\mvar#1{{\rmfamily\textit{#1}}}

\makeatletter

\def\extractmathfunctionname#1{\extractmathfunctionname@#1(,)\tmpa\tmpb}
\def\extractmathfunctionname@#1(#2)#3\tmpb{\def\mathname{#1}}

\def\extractmathoperatorname{\begingroup\def\mvar##1{}\def\ {}\extractmathoperatorname@}
\def\extractmathoperatorname@#1{\xdef\mathname{#1}\endgroup}

\makeatother
	
\newenvironment{math-function}[1]{
	\begin{pgfmanualentry}
		\extractmathfunctionname{#1}
		\pgfmanualentryheadline{\texttt{#1}}%
		\index{\mathname @\protect\texttt{\mathname} math function}%
		\index{Math functions!\mathname @\protect\texttt{\mathname}}
		\pgfmanualbody
}
{
	\end{pgfmanualentry}
}

\newenvironment{math-operator}[1]{	
	\begin{pgfmanualentry}
		\extractmathoperatorname{#1}
		\pgfmanualentryheadline{\texttt{#1}}%
		\index{\mathname @\protect\texttt{\mathname} math operator}%
		\index{Math operators!\mathname @\protect\texttt{\mathname}}
    	\pgfmanualbody
}
{%
	\end{pgfmanualentry}
}

\newenvironment{math-constant}[1]{
	\begin{pgfmanualentry}
		\pgfmanualentryheadline{\texttt{#1}}%
		\index{#1@\protect\texttt{#1} math constant}%
		\index{Math constants!#1@\protect\texttt{#1}}
		\pgfmanualbody
}
{
	\end{pgfmanualentry}
}
\def\calcname{\textsc{calc}}
%% MW: END MATH MACROS


\def\extractcommand#1#2\@@{%
  \pgfmanualentryheadline{\declare{\texttt{\string#1}}#2}%
  \removeats{#1}%
  \index{\strippedat @\protect\myprintocmmand{\strippedat}}}


% \begin{environment}{{name}\marg{arguments}}
\renewenvironment{environment}[1]{
  \begin{pgfmanualentry}
    \extractenvironement#1\@@
    \pgfmanualbody
}
{
  \end{pgfmanualentry}
}

\def\extractenvironement#1#2\@@{%
  \pgfmanualentryheadline{{\ttfamily\char`\\begin\char`\{\declare{#1}\char`\}}#2}%
  \pgfmanualentryheadline{{\ttfamily\ \ }\meta{environment contents}}%
  \pgfmanualentryheadline{{\ttfamily\char`\\end\char`\{\declare{#1}\char`\}}}%
  \index{#1@\protect\texttt{#1} environment}%
  \index{Environments!#1@\protect\texttt{#1}}}


\newenvironment{plainenvironment}[1]{
  \begin{pgfmanualentry}
    \extractplainenvironement#1\@@
    \pgfmanualbody
}
{
  \end{pgfmanualentry}
}

\def\extractplainenvironement#1#2\@@{%
  \pgfmanualentryheadline{{\ttfamily\declare{\char`\\#1}}#2}%
  \pgfmanualentryheadline{{\ttfamily\ \ }\meta{environment contents}}%
  \pgfmanualentryheadline{{\ttfamily\declare{\char`\\end#1}}}%
  \index{#1@\protect\texttt{#1} environment}%
  \index{Environments!#1@\protect\texttt{#1}}}


\newenvironment{contextenvironment}[1]{
  \begin{pgfmanualentry}
    \extractcontextenvironement#1\@@
    \pgfmanualbody
}
{
  \end{pgfmanualentry}
}

\def\extractcontextenvironement#1#2\@@{%
  \pgfmanualentryheadline{{\ttfamily\declare{\char`\\start#1}}#2}%
  \pgfmanualentryheadline{{\ttfamily\ \ }\meta{environment contents}}%
  \pgfmanualentryheadline{{\ttfamily\declare{\char`\\stop#1}}}%
  \index{#1@\protect\texttt{#1} environment}%
  \index{Environments!#1@\protect\texttt{#1}}}


\newenvironment{shape}[1]{
  \begin{pgfmanualentry}
  	\pgfmanualentryheadline{Shape {\ttfamily\declare{#1}}}%
    \index{#1@\protect\texttt{#1} shape}%
    \index{Shapes!#1@\protect\texttt{#1}}
    \pgfmanualbody
}
{
  \end{pgfmanualentry}
}


\newenvironment{handler}[1]{
  \begin{pgfmanualentry}
    \extracthandler#1\@nil%
    \pgfmanualbody
}
{
  \end{pgfmanualentry}
}

\def\gobble#1{}
\def\extracthandler#1#2\@nil{%
  \pgfmanualentryheadline{Key handler \meta{key}{\ttfamily/\declare{#1}}#2}%
  \index{\gobble#1@\protect\texttt{#1} handler}%
  \index{Key handlers!#1@\protect\texttt{#1}}
}


\makeatletter


\newenvironment{stylekey}[1]{
  \begin{pgfmanualentry}
    \def\extrakeytext{style, }
    \extractkey#1\@nil%
    \pgfmanualbody
}
{
  \end{pgfmanualentry}
}

\def\choicesep{$\vert$}%
\def\choicearg#1{\texttt{#1}}

\newif\iffirstchoice

% \mchoice{choice1,choice2,choice3}
\newcommand\mchoice[1]{%
	\begingroup
	\firstchoicetrue
	\foreach \mchoice@ in {#1} {%
		\iffirstchoice
			\global\firstchoicefalse
		\else
			\choicesep
		\fi
		\choicearg{\mchoice@}%
	}%
	\endgroup
}%

% \begin{key}{/path/x=value}
% \begin{key}{/path/x=value (initially XXX)}
% \begin{key}{/path/x=value (default XXX)}
\newenvironment{key}[1]{
  \begin{pgfmanualentry}
    \def\extrakeytext{}
    %\def\altpath{\emph{\color{gray}or}}%
    \extractkey#1\@nil%
    \pgfmanualbody
}
{
  \end{pgfmanualentry}
}

\def\extractkey#1\@nil{%
  \pgfutil@in@={#1}%
  \ifpgfutil@in@%
    \extractkeyequal#1\@nil
  \else%
    \pgfutil@in@{(initial}{#1}%
    \ifpgfutil@in@%
      \extractequalinitial#1\@nil%
    \else
      \pgfmanualentryheadline{{\ttfamily\declare{#1}}\hfill(\extrakeytext no value)}%
      \def\mykey{#1}%
      \def\mypath{}%
      \def\myname{}%
      \firsttimetrue%
      \decompose#1/\nil%
    \fi
  \fi%
}

\def\extractkeyequal#1=#2\@nil{%
  \pgfutil@in@{(default}{#2}%
  \ifpgfutil@in@%
    \extractdefault{#1}#2\@nil%
  \else%
    \pgfutil@in@{(initial}{#2}%
    \ifpgfutil@in@%
      \extractinitial{#1}#2\@nil%
    \else
      \pgfmanualentryheadline{{\ttfamily\declare{#1}=}#2\hfill(\extrakeytext no default)}%
    \fi%
  \fi%
  \def\mykey{#1}%
  \def\mypath{}%
  \def\myname{}%
  \firsttimetrue%
  \decompose#1/\nil%
}

\def\extractdefault#1#2(default #3)\@nil{%
  \pgfmanualentryheadline{{\ttfamily\declare{#1}\opt{=}}\opt{#2}\hfill (\extrakeytext default {\ttfamily#3})}%
}

\def\extractinitial#1#2(initially #3)\@nil{%
  \pgfmanualentryheadline{{\ttfamily\declare{#1}=}#2\hfill (\extrakeytext no default, initially {\ttfamily#3})}%
}

\def\extractequalinitial#1 (initially #2)\@nil{%
  \pgfmanualentryheadline{{\ttfamily\declare{#1}}\hfill (\extrakeytext initially {\ttfamily#2})}%
  \def\mykey{#1}%
  \def\mypath{}%
  \def\myname{}%
  \firsttimetrue%
  \decompose#1/\nil%
}

% Introduces a key alias '/#1/<name of current key>'
% to be used inside of \begin{key} ... \end{key}
\def\keyalias#1{\vspace{-3pt}\item{\small alias {\ttfamily/#1/\myname}}\vspace{-2pt}\par}

\newif\iffirsttime

\makeatother

\def\decompose/#1/#2\nil{%
  \def\test{#2}%
  \ifx\test\empty%
    % aha.
    \index{#1@\protect\texttt{#1} key}%
    \index{\mypath#1@\protect\texttt{#1}}%
    \def\myname{#1}%
  \else%
    \iffirsttime
      \def\mypath{#1@\protect\texttt{/#1/}!}%
      \firsttimefalse
    \else
      \expandafter\def\expandafter\mypath\expandafter{\mypath#1@\protect\texttt{#1/}!}%
    \fi
    \def\firsttime{}
    \decompose/#2\nil%
  \fi%
}

\def\indexkey#1{%
  \def\mypath{}%
  \decompose#1/\nil%
}

\newenvironment{predefinedmethod}[1]{
  \begin{pgfmanualentry}
    \extractpredefinedmethod#1\@nil
    \pgfmanualbody
}
{
  \end{pgfmanualentry}
}
\def\extractpredefinedmethod#1(#2)\@nil{%
  \pgfmanualentryheadline{Method \declare{\ttfamily #1}\texttt(#2\texttt) \hfill(predefined for all classes)}
  \index{#1@\protect\texttt{#1} method}%
  \index{Methods!#1@\protect\texttt{#1}}
}


\newenvironment{ooclass}[1]{
  \begin{pgfmanualentry}
    \def\currentclass{#1}
    \pgfmanualentryheadline{Class \declare{\texttt{#1}}}
    \index{#1@\protect\texttt{#1} class}%
    \index{Class #1@Class \protect\texttt{#1}}%
    \index{Classes!#1@\protect\texttt{#1}}
    \pgfmanualbody
}
{
  \end{pgfmanualentry}
}

\newenvironment{method}[1]{
  \begin{pgfmanualentry}
    \extractmethod#1\@nil
    \pgfmanualbody
}
{
  \end{pgfmanualentry}
}
\def\extractmethod#1(#2)\@nil{%
  \def\test{#1}
  \ifx\test\currentclass
    \pgfmanualentryheadline{Constructor \declare{\ttfamily #1}\texttt(#2\texttt)}
  \else
    \pgfmanualentryheadline{Method \declare{\ttfamily #1}\texttt(#2\texttt)}
  \fi
  \index{#1@\protect\texttt{#1} method}%
  \index{Methods!#1@\protect\texttt{#1}}
  \index{Class \currentclass!#1@\protect\texttt{#1}}%
}

\newenvironment{attribute}[1]{
  \begin{pgfmanualentry}
    \extractattribute#1\@nil
    \pgfmanualbody
}
{
  \end{pgfmanualentry}
}
\def\extractattribute#1=#2;\@nil{%
  \def\test{#2}%
  \ifx\test\@empty
    \pgfmanualentryheadline{Private attribute \declare{\ttfamily #1} \hfill (initially empty)}
  \else
    \pgfmanualentryheadline{Private attribute \declare{\ttfamily #1} \hfill (initially {\ttfamily #2})}
  \fi
  \index{#1@\protect\texttt{#1} attribute}%
  \index{Attributes!#1@\protect\texttt{#1}}
  \index{Class \currentclass!#1@\protect\texttt{#1}}%
}



\newenvironment{predefinednode}[1]{
  \begin{pgfmanualentry}
    \pgfmanualentryheadline{Predefined node {\ttfamily\declare{#1}}}%
    \index{#1@\protect\texttt{#1} node}%
    \index{Predefined node!#1@\protect\texttt{#1}}
    \pgfmanualbody
}
{
  \end{pgfmanualentry}
}

\newenvironment{coordinatesystem}[1]{
  \begin{pgfmanualentry}
    \pgfmanualentryheadline{Coordinate system {\ttfamily\declare{#1}}}%
    \index{#1@\protect\texttt{#1} coordinate system}%
    \index{Coordinate systems!#1@\protect\texttt{#1}}
    \pgfmanualbody
}
{
  \end{pgfmanualentry}
}

\newenvironment{snake}[1]{
  \begin{pgfmanualentry}
    \pgfmanualentryheadline{Snake {\ttfamily\declare{#1}}}%
    \index{#1@\protect\texttt{#1} snake}%
    \index{Snakes!#1@\protect\texttt{#1}}
    \pgfmanualbody
}
{
  \end{pgfmanualentry}
}

\newenvironment{decoration}[1]{
  \begin{pgfmanualentry}
    \pgfmanualentryheadline{Decoration {\ttfamily\declare{#1}}}%
    \index{#1@\protect\texttt{#1} decoration}%
    \index{Decorations!#1@\protect\texttt{#1}}
    \pgfmanualbody
}
{
  \end{pgfmanualentry}
}


\def\pgfmanualbar{\char`\|}
\makeatletter
\newenvironment{pathoperation}[3][]{
  \begin{pgfmanualentry}
    \pgfmanualentryheadline{\textcolor{gray}{{\ttfamily\char`\\path}\
        \ \dots}
      \declare{\texttt{#2}}#3\ \textcolor{gray}{\dots\texttt{;}}}%
    \def\pgfmanualtest{#1}%
    \ifx\pgfmanualtest\@empty%
      \index{#2@\protect\texttt{#2} path operation}%
      \index{Path operations!#2@\protect\texttt{#2}}%
    \fi%
    \pgfmanualbody
}
{
  \end{pgfmanualentry}
}
\makeatother

\def\extractcommand#1#2\@@{%
  \pgfmanualentryheadline{\declare{\texttt{\string#1}}#2}%
  \removeats{#1}%
  \index{\strippedat @\protect\myprintocmmand{\strippedat}}}

\def\doublebs{\texttt{\char`\\\char`\\}}


\newenvironment{package}[1]{
  \begin{pgfmanualentry}
    \pgfmanualentryheadline{{\ttfamily\char`\\usepackage\char`\{\declare{#1}\char`\}\space\space \char`\%\space\space  \LaTeX}}
    \index{#1@\protect\texttt{#1} package}%
    \index{Packages and files!#1@\protect\texttt{#1}}%
    \pgfmanualentryheadline{{\ttfamily\char`\\input \declare{#1}.tex\space\space\space \char`\%\space\space  plain \TeX}}
    \pgfmanualentryheadline{{\ttfamily\char`\\usemodule[\declare{#1}]\space\space \char`\%\space\space  Con\TeX t}}
    \pgfmanualbody
}
{
  \end{pgfmanualentry}
}


\newenvironment{pgfmodule}[1]{
  \begin{pgfmanualentry}
    \pgfmanualentryheadline{{\ttfamily\char`\\usepgfmodule\char`\{\declare{#1}\char`\}\space\space\space
        \char`\%\space\space  \LaTeX\space and plain \TeX\space and pure pgf}}
    \index{#1@\protect\texttt{#1} module}%
    \index{Modules!#1@\protect\texttt{#1}}%
    \pgfmanualentryheadline{{\ttfamily\char`\\usepgfmodule[\declare{#1}]\space\space \char`\%\space\space  Con\TeX t\space and pure pgf}}
    \pgfmanualbody
}
{
  \end{pgfmanualentry}
}

\newenvironment{pgflibrary}[1]{
  \begin{pgfmanualentry}
    \pgfmanualentryheadline{{\ttfamily\char`\\usepgflibrary\char`\{\declare{#1}\char`\}\space\space\space
        \char`\%\space\space  \LaTeX\space and plain \TeX\space and pure pgf}}
    \index{#1@\protect\texttt{#1} library}%
    \index{Libraries!#1@\protect\texttt{#1}}%
    \pgfmanualentryheadline{{\ttfamily\char`\\usepgflibrary[\declare{#1}]\space\space \char`\%\space\space  Con\TeX t\space and pure pgf}}
    \pgfmanualentryheadline{{\ttfamily\char`\\usetikzlibrary\char`\{\declare{#1}\char`\}\space\space
        \char`\%\space\space  \LaTeX\space and plain \TeX\space when using \tikzname}}
    \pgfmanualentryheadline{{\ttfamily\char`\\usetikzlibrary[\declare{#1}]\space
        \char`\%\space\space  Con\TeX t\space when using \tikzname}}
    \pgfmanualbody
}
{
  \end{pgfmanualentry}
}

\newenvironment{tikzlibrary}[1]{
  \begin{pgfmanualentry}
    \pgfmanualentryheadline{{\ttfamily\char`\\usetikzlibrary\char`\{\declare{#1}\char`\}\space\space \char`\%\space\space  \LaTeX\space and plain \TeX}}
    \index{#1@\protect\texttt{#1} library}%
    \index{Libraries!#1@\protect\texttt{#1}}%
    \pgfmanualentryheadline{{\ttfamily\char`\\usetikzlibrary[\declare{#1}]\space \char`\%\space\space Con\TeX t}}
    \pgfmanualbody
}
{
  \end{pgfmanualentry}
}



\newenvironment{filedescription}[1]{
  \begin{pgfmanualentry}
    \pgfmanualentryheadline{File {\ttfamily\declare{#1}}}%
    \index{#1@\protect\texttt{#1} file}%
    \index{Packages and files!#1@\protect\texttt{#1}}%
    \pgfmanualbody
}
{
  \end{pgfmanualentry}
}


\newenvironment{packageoption}[1]{
  \begin{pgfmanualentry}
    \pgfmanualentryheadline{{\ttfamily\char`\\usepackage[\declare{#1}]\char`\{pgf\char`\}}}
    \index{#1@\protect\texttt{#1} package option}%
    \index{Package options for \textsc{pgf}!#1@\protect\texttt{#1}}%
    \pgfmanualbody
}
{
  \end{pgfmanualentry}
}



\newcommand\opt[1]{{\color{black!50!green}#1}}
\newcommand\ooarg[1]{{\ttfamily[}\meta{#1}{\ttfamily]}}

\def\opt{\afterassignment\pgfmanualopt\let\next=}
\def\pgfmanualopt{\ifx\next\bgroup\bgroup\color{black!50!green}\else{\color{black!50!green}\next}\fi}



\def\beamer{\textsc{beamer}}
\def\pdf{\textsc{pdf}}
\def\eps{\texttt{eps}}
\def\pgfname{\textsc{pgf}}
\def\tikzname{Ti\emph{k}Z}
\def\pstricks{\textsc{pstricks}}
\def\prosper{\textsc{prosper}}
\def\seminar{\textsc{seminar}}
\def\texpower{\textsc{texpower}}
\def\foils{\textsc{foils}}

{
  \makeatletter
  \global\let\myempty=\@empty
  \global\let\mygobble=\@gobble
  \catcode`\@=12
  \gdef\getridofats#1@#2\relax{%
    \def\getridtest{#2}%
    \ifx\getridtest\myempty%
      \expandafter\def\expandafter\strippedat\expandafter{\strippedat#1}
    \else%
      \expandafter\def\expandafter\strippedat\expandafter{\strippedat#1\protect\printanat}
      \getridofats#2\relax%
    \fi%
  }

  \gdef\removeats#1{%
    \let\strippedat\myempty%
    \edef\strippedtext{\stripcommand#1}%
    \expandafter\getridofats\strippedtext @\relax%
  }
  
  \gdef\stripcommand#1{\expandafter\mygobble\string#1}
}

\def\printanat{\char`\@}

\def\declare{\afterassignment\pgfmanualdeclare\let\next=}
\def\pgfmanualdeclare{\ifx\next\bgroup\bgroup\color{red!75!black}\else{\color{red!75!black}\next}\fi}


\let\textoken=\command
\let\endtextoken=\endcommand

\def\myprintocmmand#1{\texttt{\char`\\#1}}

\def\example{\par\smallskip\noindent\textit{Example: }}
\def\themeauthor{\par\smallskip\noindent\textit{Theme author: }}


\def\indexoption#1{%
  \index{#1@\protect\texttt{#1} option}%
  \index{Graphic options and styles!#1@\protect\texttt{#1}}%
}

\def\itemcalendaroption#1{\item \declare{\texttt{#1}}%
  \index{#1@\protect\texttt{#1} date test}%
  \index{Date tests!#1@\protect\texttt{#1}}%
}



\def\class#1{\list{}{\leftmargin=2em\itemindent-\leftmargin\def\makelabel##1{\hss##1}}%
\extractclass#1@\par\topsep=0pt}
\def\endclass{\endlist}
\def\extractclass#1#2@{%
\item{{{\ttfamily\char`\\documentclass}#2{\ttfamily\char`\{\declare{#1}\char`\}}}}%
  \index{#1@\protect\texttt{#1} class}%
  \index{Classes!#1@\protect\texttt{#1}}}

\def\partname{Part}

\makeatletter
\def\index@prologue{\section*{Index}\addcontentsline{toc}{section}{Index}
  This index only contains automatically generated entries. A good
  index should also contain carefully selected keywords. This index is
  not a good index.
  \bigskip
}
\c@IndexColumns=2
  \def\theindex{\@restonecoltrue
    \columnseprule \z@  \columnsep 29\p@
    \twocolumn[\index@prologue]%
       \parindent -30pt
       \columnsep 15pt
       \parskip 0pt plus 1pt
       \leftskip 30pt
       \rightskip 0pt plus 2cm
       \small
       \def\@idxitem{\par}%
    \let\item\@idxitem \ignorespaces}
  \def\endtheindex{\onecolumn}
\def\noindexing{\let\index=\@gobble}



\newcommand\symarrow[1]{
  \index{#1@\protect\texttt{#1} arrow tip}%
  \index{Arrow tips!#1@\protect\texttt{#1}}
  \texttt{#1}& yields thick  
  \begin{tikzpicture}[arrows={#1-#1},thick,baseline]
    \useasboundingbox (0pt,-0.5ex) rectangle (1cm,2ex);
    \draw (0pt,.5ex) -- (1cm,.5ex);
  \end{tikzpicture} and thin
  \begin{tikzpicture}[arrows={#1-#1},thin,baseline]
    \useasboundingbox (0pt,-0.5ex) rectangle (1cm,2ex);
    \draw (0pt,.5ex) -- (1cm,.5ex);
  \end{tikzpicture}
}
\newcommand\symarrowdouble[1]{
  \index{#1@\protect\texttt{#1} arrow tip}%
  \index{Arrow tips!#1@\protect\texttt{#1}}
  \texttt{#1}& yields thick  
  \begin{tikzpicture}[arrows={#1-#1},thick,baseline]
    \useasboundingbox (0pt,-0.5ex) rectangle (1cm,2ex);
    \draw (0pt,.5ex) -- (1cm,.5ex);
  \end{tikzpicture}
  and thin
  \begin{tikzpicture}[arrows={#1-#1},thin,baseline]
    \useasboundingbox (0pt,-0.5ex) rectangle (1cm,2ex);
    \draw (0pt,.5ex) -- (1cm,.5ex);
  \end{tikzpicture}, double 
  \begin{tikzpicture}[arrows={#1-#1},thick,baseline]
    \useasboundingbox (0pt,-0.5ex) rectangle (1cm,2ex);
    \draw[double,double equal sign distance] (0pt,.5ex) -- (1cm,.5ex);
  \end{tikzpicture} and 
  \begin{tikzpicture}[arrows={#1-#1},thin,baseline]
    \useasboundingbox (0pt,-0.5ex) rectangle (1cm,2ex);
    \draw[double,double equal sign distance] (0pt,.5ex) -- (1cm,.5ex);
  \end{tikzpicture}
}

\newcommand\sarrow[2]{
  \index{#1@\protect\texttt{#1} arrow tip}%
  \index{Arrow tips!#1@\protect\texttt{#1}}
  \index{#2@\protect\texttt{#2} arrow tip}%
  \index{Arrow tips!#2@\protect\texttt{#2}}
  \texttt{#1-#2}& yields thick  
  \begin{tikzpicture}[arrows={#1-#2},thick,baseline]
    \useasboundingbox (0pt,-0.5ex) rectangle (1cm,2ex);
    \draw (0pt,.5ex) -- (1cm,.5ex);
  \end{tikzpicture} and thin
  \begin{tikzpicture}[arrows={#1-#2},thin,baseline]
    \useasboundingbox (0pt,-0.5ex) rectangle (1cm,2ex);
    \draw (0pt,.5ex) -- (1cm,.5ex);
  \end{tikzpicture}
}

\newcommand\sarrowdouble[2]{
  \index{#1@\protect\texttt{#1} arrow tip}%
  \index{Arrow tips!#1@\protect\texttt{#1}}
  \index{#2@\protect\texttt{#2} arrow tip}%
  \index{Arrow tips!#2@\protect\texttt{#2}}
  \texttt{#1-#2}& yields thick  
  \begin{tikzpicture}[arrows={#1-#2},thick,baseline]
    \useasboundingbox (0pt,-0.5ex) rectangle (1cm,2ex);
    \draw (0pt,.5ex) -- (1cm,.5ex);
  \end{tikzpicture} and thin
  \begin{tikzpicture}[arrows={#1-#2},thin,baseline]
    \useasboundingbox (0pt,-0.5ex) rectangle (1cm,2ex);
    \draw (0pt,.5ex) -- (1cm,.5ex);
  \end{tikzpicture}, double 
  \begin{tikzpicture}[arrows={#1-#2},thick,baseline]
    \useasboundingbox (0pt,-0.5ex) rectangle (1cm,2ex);
    \draw[double,double equal sign distance] (0pt,.5ex) -- (1cm,.5ex);
  \end{tikzpicture} and 
  \begin{tikzpicture}[arrows={#1-#2},thin,baseline]
    \useasboundingbox (0pt,-0.5ex) rectangle (1cm,2ex);
    \draw[double,double equal sign distance] (0pt,.5ex) -- (1cm,.5ex);
  \end{tikzpicture}
}

\newcommand\carrow[1]{
  \index{#1@\protect\texttt{#1} arrow tip}%
  \index{Arrow tips!#1@\protect\texttt{#1}}
  \texttt{#1}& yields for line width 1ex
  \begin{tikzpicture}[arrows={#1-#1},line width=1ex,baseline]
    \useasboundingbox (0pt,-0.5ex) rectangle (1.5cm,2ex);
    \draw (0pt,.5ex) -- (1.5cm,.5ex);
  \end{tikzpicture}
}
\def\myvbar{\char`\|}
\newcommand\plotmarkentry[1]{%
  \index{#1@\protect\texttt{#1} plot mark}%
  \index{Plot marks!#1@\protect\texttt{#1}}
  \texttt{\char`\\pgfuseplotmark\char`\{\declare{#1}\char`\}} &
  \tikz\draw[color=black!25] plot[mark=#1,mark options={fill=examplefill,draw=black}] coordinates{(0,0) (.5,0.2) (1,0) (1.5,0.2)};\\
}
\newcommand\plotmarkentrytikz[1]{%
  \index{#1@\protect\texttt{#1} plot mark}%
  \index{Plot marks!#1@\protect\texttt{#1}}
  \texttt{mark=\declare{#1}} & \tikz\draw[color=black!25]
  plot[mark=#1,mark options={fill=examplefill,draw=black}] 
    coordinates {(0,0) (.5,0.2) (1,0) (1.5,0.2)};\\
}



\ifx\scantokens\@undefined
  \PackageError{pgfmanual-macros}{You need to use extended latex
    (elatex) or (pdfelatex) to process this document}{}
\fi

\begingroup
\catcode`|=0
\catcode`[= 1
\catcode`]=2
\catcode`\{=12
\catcode `\}=12
\catcode`\\=12 |gdef|find@example#1\end{codeexample}[|endofcodeexample[#1]]
|endgroup

% define \returntospace.
%
% It should define NEWLINE as {}, spaces and tabs as \space.
\begingroup
\catcode`\^=7
\catcode`\^^M=13
\catcode`\^^I=13
\catcode`\ =13%
\gdef\returntospace{\catcode`\ =13\def {\space}\catcode`\^^I=13\def^^I{\space}\catcode`\^^M=13\def^^M{}}%
\endgroup

\begingroup
\catcode`\%=13
\catcode`\^^M=13
\gdef\commenthandler{\catcode`\%=13\def%{\@gobble@till@return}}
\gdef\@gobble@till@return#1^^M{}
\gdef\@gobble@till@return@ignore#1^^M{\ignorespaces}
\gdef\typesetcomment{\catcode`\%=13\def%{\@typeset@till@return}}
\gdef\@typeset@till@return#1^^M{{\def%{\char`\%}\textsl{\char`\%#1}}\par}
\endgroup

% Define tab-implementation functions
%   \codeexample@tabinit@replacementchars@
% and
%   \codeexample@tabinit@catcode@
%
% They should ONLY be used in case that tab replacement is active.
%
% This here is merely a preparation step.
%
% Idea:
% \codeexample@tabinit@catcode@ will make TAB active
% and
% \codeexample@tabinit@replacementchars@ will insert as many spaces as
% /codeexample/tabsize contains.
{
\catcode`\^^I=13
% ATTENTION: do NOT use tabs in these definitions!!
\gdef\codeexample@tabinit@replacementchars@{%
 \begingroup
 \count0=\pgfkeysvalueof{/codeexample/tabsize}\relax
 \toks0={}%
 \loop
 \ifnum\count0>0
  \advance\count0 by-1
  \toks0=\expandafter{\the\toks0\ }%
 \repeat
 \xdef\codeexample@tabinit@replacementchars@@{\the\toks0}%
 \endgroup
 \let^^I=\codeexample@tabinit@replacementchars@@
}%
\gdef\codeexample@tabinit@catcode@{\catcode`\^^I=13}%
}%

% Called after any options have been set. It assigns
%   \codeexample@tabinit@catcode
% and
%   \codeexample@tabinit@replacementchars
% which are used inside of 
%\begin{codeexample}
% ...
%\end{codeexample}
%
% \codeexample@tabinit@catcode  is either \relax or it makes tab
% active.
%
% \codeexample@tabinit@replacementchars is either \relax or it inserts
% a proper replacement sequence for tabs (as many spaces as
% configured)
\def\codeexample@tabinit{%
	\ifnum\pgfkeysvalueof{/codeexample/tabsize}=0\relax
		\let\codeexample@tabinit@replacementchars=\relax
		\let\codeexample@tabinit@catcode=\relax
	\else
		\let\codeexample@tabinit@catcode=\codeexample@tabinit@catcode@
		\let\codeexample@tabinit@replacementchars=\codeexample@tabinit@replacementchars@
	\fi
}

\pgfqkeys{/codeexample}{%
	width/.code=	{\setlength\codeexamplewidth{#1}},
	graphic/.code=	{\colorlet{graphicbackground}{#1}},
	code/.code=	{\colorlet{codebackground}{#1}},
	execute code/.is if=code@execute,
	code only/.code=	{\code@executefalse},
	pre/.code=	{\def\code@pre{#1}},
	post/.code=	{\def\code@post{#1}},
	vbox/.code=	{\def\code@pre{\vbox\bgroup\setlength{\hsize}{\linewidth-6pt}}\def\code@post{\egroup}},
	ignorespaces/.code=	{\let\@gobble@till@return=\@gobble@till@return@ignore},
	leave comments/.code=	{\def\code@catcode@hook{\catcode`\%=12}\let\commenthandler=\relax\let\typesetcomment=\relax},
	tabsize/.initial=0,% FIXME : this here is merely used for indentation. It is just a TAB REPLACEMENT.
	every codeexample/.style={width=4cm+7pt},
}

\def\code@pre{}
\def\code@post{}
\def\code@catcode@hook{}

\newdimen\codeexamplewidth
\newif\ifcode@execute
\newbox\codeexamplebox
\def\codeexample[#1]{%
  \begingroup%
  \code@executetrue
  \pgfqkeys{/codeexample}{every codeexample,#1}%
  \codeexample@tabinit% assigns \codeexample@tabinit@[catcode,replacementchars]
  \parindent0pt
  \begingroup%
  \par%
  \medskip%
  \let\do\@makeother%
  \dospecials%
  \obeylines%
  \@vobeyspaces%
  \catcode`\%=13%
  \catcode`\^^M=13%
  \code@catcode@hook%
  \codeexample@tabinit@catcode
  \relax%
  \find@example}
\def\endofcodeexample#1{%
  \endgroup%
  \ifcode@execute%
    \setbox\codeexamplebox=\hbox{%
      {%
        {%
          \returntospace%
          \commenthandler%
          \xdef\code@temp{#1}% removes returns and comments
        }%
        \colorbox{graphicbackground}{\color{black}\ignorespaces%
          \code@pre\expandafter\scantokens\expandafter{\code@temp\ignorespaces}\code@post\ignorespaces}%
      }%
    }%
    \ifdim\wd\codeexamplebox>\codeexamplewidth%
      \def\code@start{\par}%
      \def\code@flushstart{}\def\code@flushend{}%
      \def\code@mid{\parskip2pt\par\noindent}%
      \def\code@width{\linewidth-6pt}%
      \def\code@end{}%
    \else%
      \def\code@start{%
        \linewidth=\textwidth%
        \parshape \@ne 0pt \linewidth
        \leavevmode%
        \hbox\bgroup}%
      \def\code@flushstart{\hfill}%
      \def\code@flushend{\hbox{}}%
      \def\code@mid{\hskip6pt}%
      \def\code@width{\linewidth-12pt-\codeexamplewidth}%
      \def\code@end{\egroup}%
    \fi%
    \code@start%
    \noindent%
    \begin{minipage}[t]{\codeexamplewidth}\raggedright
      \hrule width0pt%
      \footnotesize\vskip-1em%
      \code@flushstart\box\codeexamplebox\code@flushend%
      \vskip-1ex
      \leavevmode%
    \end{minipage}%
  \else%
    \def\code@mid{\par}
    \def\code@width{\linewidth-6pt}
    \def\code@end{}
  \fi%
  \code@mid%  
  \colorbox{codebackground}{%
    \begin{minipage}[t]{\code@width}%
      {%
        \let\do\@makeother
        \dospecials
        \frenchspacing\@vobeyspaces
        \normalfont\ttfamily\footnotesize
        \typesetcomment%
		\codeexample@tabinit@replacementchars
        \@tempswafalse
        \def\par{%
          \if@tempswa
          \leavevmode \null \@@par\penalty\interlinepenalty
          \else
          \@tempswatrue
          \ifhmode\@@par\penalty\interlinepenalty\fi
          \fi}%
        \obeylines
        \everypar \expandafter{\the\everypar \unpenalty}%
        #1}
    \end{minipage}}%
  \code@end%
  \par%
  \medskip
  \end{codeexample}
}

\def\endcodeexample{\endgroup}


\makeatother


%%% Local Variables: 
%%% mode: latex
%%% TeX-master: "beameruserguide"
%%% End: 
 % link from
% /usr/local/texlive/2019/texmf-dist/doc/generic/pgf/macros/pgfmanual-en-macros.tex
% or the equivalent on your installation
\def\pgfautoxrefs{1}
\usetikzlibrary{3dtools}
\begin{document}
\section{3D Tools}
\begin{tikzlibrary}{3dtools}
    This library provides additional tools to create 3d--like pictures.
\end{tikzlibrary}

TikZ has the |3d| and |tpp| libraries which deal with the projections of
three--dimensional drawings. This library provides some means to manipulate
the coordinates. It supports linear combinations of vectors, vector and scalar
products.

\noindent\textbf{Note:} Hopefully this library is only temporary and its
contents will be absorbed in slightly extended versions of the |3d| and |calc|
libraries.

\subsection{Coordinate computations}
\label{sec:3DCoordinateComputations}


The |3dtools| library has some options and styles for coordinate computations.
\begin{key}{/tikz/3d parse}
        Parses and expression and inserts the result in form of a coordinate.
\end{key}
\begin{key}{/tikz/3d coordinate}
        Allow one to define a 3d coordinate from other coordinates.
\end{key}
Both keys support both symbolic and explicit coordinates but for the explicit
ones one needs additional braces.

\begin{codeexample}[width=6cm]
\begin{tikzpicture}
 \path (1,2,3) coordinate (A) 
  (2,3,-1) coordinate (B) 
  (-1,-2,1) coordinate (C)
  [3d parse={0.25*(1,2,3)x(B)}] 
  	coordinate(D)
  [3d parse={0.25*(C)x(B)}] 
  	coordinate(E);
 \path foreach \X in {A,...,E} 
 {(\X) node[fill,inner sep=1pt,
 label=above:$\X$]{}};
\end{tikzpicture}
\end{codeexample}

\begin{codeexample}[width=6cm]
\begin{tikzpicture}
 \path (1,2,3) coordinate (A) 
  (2,3,-1) coordinate (B) 
  (-1,-2,1) coordinate (C)
  [3d coordinate={(D)=0.25*(1,2,3)x(B)},
  3d coordinate={(E)=0.25*(C)x(B)}];
 \path foreach \X in {A,...,E} 
 {(\X) node[fill,inner sep=1pt,
 label=above:$\X$]{}};
\end{tikzpicture}
\end{codeexample}

The library comes also with a function |\pgfmathtdparse| that allows one to parse 3d
expressions. The supported vector operations are |+| (addition $+$), |-|
(subtraction $-$), |*| (multiplication of the vector by a scalar), |x|
(vector product $\times$) and |o| (scalar product).

\begin{command}{\pgfmathtdparse{\marg{x}}}
   Parses 3d expressions.
\end{command}


\begin{codeexample}[]
\pgfmathtdparse{(1,0,0)x(0,1,0)}\pgfmathresult
\end{codeexample}

In order to pretty-print the result one may want to use |\pgfmathprintvector|,
and use the math function |TD| for parsing.

\begin{command}{\pgfmathprintvector\marg{x}}
   Pretty-prints vectors.
\end{command}

\begin{codeexample}[width=6.5cm]
\pgfmathparse{TD("0.2*(A)
-0.3*(B)+0.6*(C)")}%
$0.2\,\vec A-0.3\,\vec B+0.6\,\vec C
=(\pgfmathprintvector\pgfmathresult)$
\end{codeexample}

The alert reader may wonder why this works, i.e.\ how would \tikzname\ ``know''
what the coordinates $A$, $B$ and $C$ are. It works because the coordinates in
\tikzname\ are global, so they get remembered from the above example.

\begin{codeexample}[width=5.2cm]
\pgfmathparse{TD("(1,0,0)x(0,1,0)")}%
$(1,0,0)^T\times(0,1,0)^T=
(\pgfmathprintvector\pgfmathresult)^T$
\end{codeexample}


\begin{codeexample}[width=5.2cm]
\pgfmathparse{TD("(A)o(B)")}%
$\vec A\cdot \vec B=
\pgfmathprintnumber\pgfmathresult$
\end{codeexample}



\end{document}

\tdplotsetmaincoords{70}{110} 
\begin{tikzpicture}
 \begin{scope}[local bounding box=tests,tdplot_main_coords]
 % to work with this library, you need to define the cordinate
 % with \path (<x>,<y>,<z>) coordinate (<name>);
  \path (0,0,0) coordinate (O) 
  (1,2,3) coordinate (A) 
  (2,3,-1) coordinate (B) 
  (-1,-2,1) coordinate (C)
  % you can use 3d parse (clumsy)
  [3d parse={0.25*(A)x(B)}] coordinate(D)
  % you can use 3d coordinate to define a new coordinate from existing ones
  [3d coordinate={(E)=0.2*(A)-0.3*(B)+0.6*(C)}] 
  [3d coordinate={(H)=0.2*(A)-0.3*(B)+0.6*(C)}]; 
  \draw (A) -- (B) -- (C) -- (D) -- (E) -- cycle; 
 \end{scope}
 %\RawCoord yields the components
 \edef\tempD{\RawCoord(D)} 
 \edef\tempE{\RawCoord(E)} 
 \edef\tempH{\RawCoord(H)} 
 \node[below right,align=left] at (tests.south west) 
  {$(D)=\tempD$,\\ $(E)=\tempE$,\\ $(H)=\tempH$}; 
\end{tikzpicture} 

\noindent% clumsy parser
$\tdparse{(A)+0.3*(B)>(A)+0.3(B)}=(\pgfmathresult)$

\noindent% parsing inside \pgfmathparse. You need to wrap the argument in "..."
\pgfmathparse{TD("0.2*(A)-0.3*(B)+0.6*(C)")}%
$0.2\,\vec A-0.3\,\vec B+0.6\vec C=(\pgfmathresult)$

%one can parse with the same parser vector products
\noindent\pgfmathparse{TD("0.5*(A)x(B)")}%
$0.5\,\vec A\times\vec B=(\pgfmathresult)$
%(note, however, that something like (A)x(B)x(C) does NOT work)

%as well as scalar products
\noindent\pgfmathparse{TD("(A)+(C)o(B)")}%
$\left(\begin{array}{@{}c@{}}1\\ 0\\ 0\end{array}\right)$
%(note, however, that + and - have higher precedence than o)\end{document}
\endinput
